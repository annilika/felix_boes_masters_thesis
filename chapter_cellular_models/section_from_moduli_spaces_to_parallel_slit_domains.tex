\section{The Bundle \texorpdfstring{$\Harmr$}{Hmgn[r]}}
\label{cellular_models:from_moduli_spaces_to_parallel_slit_domains}
\label{page:potential_function_parallel}%
\index{potential function}
\symbolindex[u]{$u$}{A potential function defined on a Riemann surface.}{Page \pageref{page:potential_function_parallel}}

We start off with the construction of the bundle $\Harmr$ for a fixed moduli space $\Modspc$ and an ordered partition $(r_1, \ldots, r_n)$ of $r = r_1 + \ldots + r_n$.
Consider a point $[F, \mc P, \mc Q, \mc X]$ in the moduli space $\Modspc$.
For positive real numbers $B_1^-, \ldots, B_m^-$ and $B_1^+, \ldots, B_n^+$ that satisfy the residue equation $\sum_{i=1}^{m} B_i^- - \sum_{j=1}^n B_j^+ = 0$,
and complex numbers $a_{jk}$ where $j = 1, \ldots, n$ while $k = 1, \ldots, r_j-1$, there is a {\bfseries potential function} $u \colon F \to \ov{\mathbb R}$, i.e.\ 
\begin{enumerate}
    \item $u|_{F - (\mc P \cup \mc Q)}$ is harmonic;
    \item in a chart domain $V_i$ around $P_i$ with coordinates $z = x + \sqrt{-1}\cdot y$ such that $z(P_i) = 0$
        \[
            u(z) = B_i^- \log|z| + \phi_i \mspc{with}{20} \phi_i \colon V_i \to \mathbb R \text{ harmonic};
        \]
    \item in a chart domain $W_j$ around $Q_j$ with coordinates $z = x + \sqrt{-1}\cdot y$ such that $z(Q_j) = 0$ and $(T_{Q_j}z) (X_j) = \frac{\del}{\del x} \in T_0\mathbb C$
        \[
            u(z) = \Re \left( \frac{1}{z^{r_j}} + \sum_{k=1}^{r_j-1} \frac{a_{jk}}{z^k} \right) - B_j^+ \log|z| + \psi_j \mspc{with}{20} \psi_j \colon W_j \to \mathbb R \text{ harmonic} \,.
        \]
\end{enumerate}
Such a potential function is unique up to an additive constant $D_0$, i.e.\
it is uniquely defined by one real, $n+m-1$ positive and $r-n$ complex parameters.
A proof of this classical result can be found in many sources, for example \cite{Koch199105}.
The condition $\sum_{i=1}^{m} B_i^- - \sum_{j=1}^n B_j^+ = 0$ is imposed by the Residue Theorem
(the complex differential $\del u$ is a meromorphic 1-form, so the sum of its residues vanishes).


Following \cite[Section 3.2]{Boedigheimer19901}, we use the flow of $-\grad_u$, the gradient field of steepest descent of a potential function $u$,
in order to construct the (directed) critical graph $\mc K_0 \subset F$.
This may be seen as a 1-skeleton of $F$.
Before going on, let us look at two very simple examples.
In Figure \ref{cellular_models:from_moduli_spaces_to_parallel_slit_domains:re_z} we sketch the flow lines of the harmonic function $\Re(z) = x$ and 
\begin{figure}[ht]
    \centering
    \def\svgwidth{.3\columnwidth}
    \input{pictures/flow_Re_z_sphere.pdf_tex}
    \hfill
    \def\svgwidth{.3\columnwidth}
    \input{pictures/flow_Re_z_at_zero.pdf_tex}
    \hfill
    \def\svgwidth{.3\columnwidth}
    \input{pictures/flow_Re_z_at_infinity.pdf_tex}
    \caption{\label{cellular_models:from_moduli_spaces_to_parallel_slit_domains:re_z}The gradient flow of $\Re(z)$ on the sphere $\mathbb S^2$ and in charts around zero and infinity.}
\end{figure}
Figure \ref{cellular_models:from_moduli_spaces_to_parallel_slit_domains:re_z_squared} pictures the flow lines of $\Re(z^2) = x^2 - y^2$.
\begin{figure}[ht]
    \centering
    \def\svgwidth{.3\columnwidth}
    \input{pictures/flow_Re_z_squared_sphere.pdf_tex}
    \hfill
    \def\svgwidth{.3\columnwidth}
    \input{pictures/flow_Re_z_squared_at_zero.pdf_tex}
    \hfill
    \def\svgwidth{.3\columnwidth}
    \input{pictures/flow_Re_z_squared_at_infinity.pdf_tex}
    \caption{\label{cellular_models:from_moduli_spaces_to_parallel_slit_domains:re_z_squared}The gradient flow of $\Re(z^2)$ on the sphere $\mathbb S^2$ and in charts around zero and infinity.}
\end{figure}
Geometrically speaking, the conditions on the potential function $u$ are as follows.
At every point $Q_j$, the dominating term is the pole $\Re(\frac{1}{z^{r_j}})$ whereas
the terms $\Re ( \frac{a_{jk}}{z^k} )$ for $k = r_j - 1, \ldots, 1$ and the logarithmic term $B^+_j\log|z|$ have no influence on the qualitative picture.
In Figures \ref{cellular_models:from_moduli_spaces_to_parallel_slit_domains:re_z} or \ref{cellular_models:from_moduli_spaces_to_parallel_slit_domains:re_z_squared},
the chart around infinity pictures the stream lines near a pole of order $r_j = 1$ or $r_j = 2$.
At a point $P_j$ the gradient flow has a sink as pictured in Figure \ref{cellular_models:from_moduli_spaces_to_parallel_slit_domains:re_log_z}.
\begin{figure}[ht]
    \centering
    \def\svgwidth{.3\columnwidth}
    \input{pictures/flow_log_z_at_zero.pdf_tex}
    \caption{\label{cellular_models:from_moduli_spaces_to_parallel_slit_domains:re_log_z}The gradient flow of $\Re(\log(z)) = \log|z|$ near zero.}
\end{figure}
The coefficients $B_i^-$ respectively $B_j^+$ indicate the magnitude of the logarithmic sink respectively source.
Following the stream lines, one will end up either in one of the singularities $\mc P \cup \mc Q$ or in a critical point of $u$.
At a critical point, $u$ has the form $\Re(z^k)$ for some $k \ge 2$.
The critical graph of $u$ consists of all critical flow lines that start in such critical points. %
\label{page:stagnation_point_parallel}%
\index{stagnation point}
\index{stagnation point!set of stagnation points}
\symbolindex[s]{$S$}{A critical point of a given potential function $u$}{Page \pageref{page:stagnation_point_parallel}}
\symbolindex[s]{$\mc S$}{The set of stagnation points of a given potential function $u$}{Page \pageref{page:stagnation_point_parallel}}
A point $S \in F$ at which $\grad_u$ vanishes is called {\bfseries stagnation point} and we denote the {\bfseries set of stagnation points} by $\mc S$ --- they are the critical points of $u$.
\label{page:critical_graph_parallel}%
\index{critical graph}
\index{critical edge}
\symbolindex[k]{$\mc K_0$}{The critical graph of a given potential function $u$}{Page \pageref{page:stagnation_point_parallel}}
The vertices of $\mc K_0$ are $V\mc K_0 = \mc S \cup \mc P \cup \mc Q$ and the edges $\gamma \in E\mc K_0$ are specific segments of the gradient flow of $u$:
For two vertices $S_0 \in \mc S$ and $S_1 \in V\mc K_0$ with $t_0 = u(S_0) < u(S_1) = t_1$ (as points in $\ov {\mathbb R}$),
a smooth curve $\gamma \colon [t_0, t_1] \to F$ starting in $S_0$ and ending in $S_1$ is called {\bfseries critical edge} if it satisfies the conditions
\begin{enumerate}
    \item $u(\gamma(t)) = t$ and
    \item $\dot{\gamma}(t) = -\grad_u(\gamma(t)) \neq 0$ for $t_0 < t < t_1$.
\end{enumerate}
In particular, curves must not traverse through critical points.

Both potential functions in Figures \ref{cellular_models:from_moduli_spaces_to_parallel_slit_domains:re_z} and
\ref{cellular_models:from_moduli_spaces_to_parallel_slit_domains:re_z_squared} have exactly one singularity namely at infinity. 
The harmonic function $\Re(z)$ has no stagnation points and $\Re(z^2)$ has exactly one.
In Figure \ref{cellular_models:from_moduli_spaces_to_parallel_slit_domains:flow_with_one_puncture}, we picture the flow of a potential function $u$
having one dipole $Q$ of simple order, one puncture $P$ and one stagnation point $S$.
\begin{figure}[ht]
    \centering
    \def\svgwidth{.9\columnwidth}
    \input{pictures/flow_with_one_puncture.pdf_tex}
    \caption{\label{cellular_models:from_moduli_spaces_to_parallel_slit_domains:flow_with_one_puncture}The gradient flow of a potential function which has exactly one dipole $Q$ of simple order, one puncture $P$ and one stagnation point $S$.
        We have $g=0$, $m=1$ and $n=1$.}
\end{figure}

\label{page:basin_parallel}%
\index{basin}
Dissecting $F$ along the critical graph $\mc K_0$ yields $r$ open, contractible%
\footnote{%
    Using the flow lines, each $F_j$ is contracted to an equipotential line $l_i$.
    Observe that every potential line $l_i$ is contractible since $\mc Q$ and $\mc P$ are removed.
} components $F_1, \ldots, F_r$ called {\bfseries basins}.
These basins are ordered.
The poles $\mc Q$ are enumerated and at every pole $Q_j$, the tangent vector $X_j$ points into the first since distinguished component,
whereas the others are numbered following counter-clockwise around $Q_j$.
In Figure \ref{cellular_models:from_moduli_spaces_to_parallel_slit_domains:flow_with_three_basins} we illustrate the stream lines of $\Re(z^3)$ on $\mathbb S^2$ around infinity.
The critical flow is stressed and the basins are denoted by $F_1$,$F_2$ or $F_3$.
\begin{figure}[ht]
    \centering
    \def\svgwidth{.4\columnwidth}
    \input{pictures/flow_with_three_basins.pdf_tex}
    \caption{\label{cellular_models:from_moduli_spaces_to_parallel_slit_domains:flow_with_three_basins}Dissecting $\mathbb S^2$ along the critical graph of $\Re(z^3)$ yields three basins.}
\end{figure}

From the topological point of view, we obtain a cell decomposition.
This means that the homotopy type of $F$, which is just its genus, can be reconstructed from glueing the disc-shaped components $F_j$.
In order to keep track of the given complex structure, we follow \cite{Boedigheimer19901}.
On every sub-surface $F_j$, the harmonic function $u$ admits a harmonic conjugate $v_j$, unique up to an integration constant, by letting
\[
    v_j(\zeta) = \int_{\zeta_o}^\zeta \frac{\del u}{\del x}dy - \frac{\del u}{\del y}dx \mspc{for some}{20} \zeta_0 \in F_j \,.
\]
The holomorphic function $w_j = u + \sqrt{-1}\cdot v_j$ maps the basin $F_j$ injectively onto an open domain in $\mathbb C$ and
its image is obviously the entire plane with finitely many slits removed,
each slit running from some point horizontally to the left all the way to infinity.
Changing the integration constant $D_j$ corresponds to a translation parallel to the imaginary axis.

Let us go back to the examples pictured in Figures \ref{cellular_models:from_moduli_spaces_to_parallel_slit_domains:re_z_squared} and
\ref{cellular_models:from_moduli_spaces_to_parallel_slit_domains:flow_with_one_puncture}.
Dissecting the sphere along the critical flow lines of $u = \Re(z^2)$ yields exactly two components $F_1  = \{ z \in \mathbb S^2 - \infty \mid \Re(z) > 0\}$ and $F_2 = \{ z \in \mathbb S^2 - \infty \mid \Re(z) < 0\}$.
The tangent vector $X$ points into the first component $F_1$.
For the other example, consider Figure \ref{cellular_models:from_moduli_spaces_to_parallel_slit_domains:flow_with_one_puncture_with_equipotential_lines}
where we added three dashed lines on which $u$ is constant --- so-called equipotential lines --- and sketch the relevant clipping under the biholomorphic function.
The critical graph is stressed.
\begin{figure}[ht]
    \centering
    \def\svgwidth{.72\columnwidth}
    \raisebox{-0.5\height}{\input{pictures/flow_with_one_puncture_with_equipotential_lines.pdf_tex}}
    \hfill
    \def\svgwidth{.26\columnwidth}
    \raisebox{-0.5\height}{\input{pictures/flow_with_one_puncture_slit_picture.pdf_tex}}
    \caption{
        \label{cellular_models:from_moduli_spaces_to_parallel_slit_domains:flow_with_one_puncture_with_equipotential_lines}
        The gradient flow of a potential function and the slit picture.
        We have $g=0$, $m=1$ and $n=1$.}
\end{figure}
In Figure \ref{cellular_models:from_moduli_spaces_to_parallel_slit_domains:flow_with_one_puncture_with_equipotential_lines_with_help} on Page \pageref{page:huge_picture_flow},
we picture an enlarged version of Figure \ref{cellular_models:from_moduli_spaces_to_parallel_slit_domains:flow_with_one_puncture_with_equipotential_lines}
with enumerated equipotential lines in order to provide more guidance.

Let $\Harmr$ be the space of all $(F, \mc P, \mc Q, \mc X, w)$ with $w = (u, v_1, \ldots, v_r)$, where
$u$ is a globally defined potential function as declared above and each $v_j$ is a harmonic conjugates of $u$, defined only on $F_j$.
There is a projection
\[
    \Harmr \xr{\simeq} \Modspc
\]
with contractible fibres,
namely the space of all $(a_{jk}, B^-_i, B^+_j, D_0, \ldots, D_r)$ subject only to the residue equation.
More precisely, it is the open affine half-space $\mathbb C^{r-n} \times \mathbb R^{m+n-1}_{>0} \times \mathbb R^{r+1}$.
This bundle would be trivial if we required both $(r_1, \ldots, r_n) = (1, \ldots, 1)$ and non-permutable punctures $\mc P$.

Its real dimension is readily computed, as $2(r-n) + m+n-1 + 1$ real parameters correspond to a choice of $u$ and
every harmonic conjugate $v_j$ is unique up to an additive constant, adding $r$ real dimensions:
\[
    \dim \Harmr = \dim \Modspc + (2(r-n) + m+n-1 + 1) + r = 6g-6 + 3m + 3n + 3r = 3h \,.
\]

We end this section with the following remark.
There are situation in which we want to think of surfaces with boundary or, in terms of slit picutes, of the relevant clipping as seen in Figure \ref{cellular_models:from_moduli_spaces_to_parallel_slit_domains:flow_with_one_puncture_with_equipotential_lines}.
This is achieved by removing the critical graph as well as specific discs around the poles and punctures.
Each disc around a puncture $P_j \in \mc P$ is bound by an equipotential circle beeing a closed curve on which $u$ is constant,
whereas each disc around a pole $Q_i \in \mc Q$ is bounded by a circle consisting of flow lines or equipotential lines, see Figure \ref{cellular_models:from_moduli_spaces_to_parallel_slit_domains:flow_with_three_basins_and_cut}.
Observe that this process can be reversed without loss of information as the basins are ordered.
\begin{figure}[ht]
    \centering
    \def\svgwidth{.4\columnwidth}
    \input{pictures/flow_with_three_basins_and_cut.pdf_tex}
    \caption{\label{cellular_models:from_moduli_spaces_to_parallel_slit_domains:flow_with_three_basins_and_cut}%
        The red matter is removed and the thick red lines are seen as the boundary curves of the resulting surface.}
\end{figure}

\begin{sidewaysfigure}[hbtp]
    \centering
    \def\svgwidth{\columnwidth}
    \input{pictures/flow_with_one_puncture_with_equipotential_lines_with_help.pdf_tex}
    
    \vspace{.5cm}
    
    \def\svgwidth{.4\columnwidth}
    \input{pictures/flow_with_one_puncture_slit_picture_with_help.pdf_tex}
    \caption{
        \label{page:huge_picture_flow}
        \label{cellular_models:from_moduli_spaces_to_parallel_slit_domains:flow_with_one_puncture_with_equipotential_lines_with_help}%
        The gradient flow of a potential function and the slit picture.
        The dashed curves are equipotential lines and we have $g=0$, $m=1$ and $n=1$.}
\end{sidewaysfigure}

