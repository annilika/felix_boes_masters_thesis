\subsection{Vertical and Horizontal Faces [H]}
\label{cellular_models:parallel:faces}

\begin{defi}
\label{cellular_models:parallel:dv}
\index{cell!vertical face of a parallel cell}
\symbolindex[d]{$d'_j(\Sigma)$}{The $j\Th$ vertical face of a parallel cell $\Sigma$}{Definition \ref{cellular_models:parallel:dv}}
Let $\Sigma = \homogq$ be an arbitrary cell of bidegree $(p,q)$.
The $j^{\text{th}}$ {\bf vertical face} of $\Sigma$ is obtained by removing the $j^{\text{th}}$ permutation, where $0 \le j \le q$:
\[
    d'_j \homogq = (\sigma_q: \ldots: \widehat{\sigma_j}: \ldots: \sigma_0)
\]
and this translates into the inhomogeneous notation as follows.
\[
    d'_j(\tau_q \mid \ldots \mid \tau_1) =
        \begin{cases}
            (\tau_q \mid \ldots \mid \tau_2) & j = 1 \\
            (\tau_q \mid \ldots \mid \tau_i \tau_{j-1} \mid \ldots \mid \tau_1) & 1 < j < q \\
            (\tau_{q-1} \mid \ldots \mid \tau_1) & j = q
        \end{cases}
\]
\end{defi}

Imagining an inner cell as in Figure \ref{cellular_models:parallel:homogeneous_glueing}, we collapse the $j^{\text{th}}$ vertical stripe of the corresponding parallel slit domain.
Collapsing the $i^{\text{th}}$ horizontal stripe corresponds to a face map in the multisimplex $\Delta^{p_1} \times \ldots \times \Delta^{p_r}$.
For our techniques, it is convenient to group these face maps using the corresponding partition $[p]$.
Consequently, we will speak of the $i^{\text{th}}$ horizontal face where $i \in [p]$.
Before going into the details, one might take a look at Figure \ref{cellular_models:parallel:comparison_face_operators}.
\begin{figure}[ht]
\centering
\incgfx{pictures/cellular_para_comparison_face_operators}
\caption{\label{cellular_models:parallel:comparison_face_operators}The vertical and horizontal face operators.}
\end{figure}

\begin{defi}
\label{cellular_models:parallel:dh}
\index{cell!horizontal face of a parallel cell}
\symbolindex[d]{$d''_i(\Sigma)$}{The $i\Th$ horizontal face of a parallel cell $\Sigma$}{Definition \ref{cellular_models:parallel:dh}}
Let $i \in [p]$.
The $i^{\text{th}}$ {\bf horizontal face} of $\Sigma$ is
\[ 
    d''_i(\Sigma) = (D_i(\sigma_q), \dotsc, D_i(\sigma_0))\,,
\]
where $D_i \colon \Symgrp_{[p]} \to \Symgrp_{[p-1]}$ removes the symbol $i$ from its cycle in $\sigma$ (recall Definition \ref{notation:sym_grp:face}).
\end{defi}

We usually omit the simplicial degeneracy and face maps since they only rename the symbols used.
Hence we write
\[
   D_i(\sigma) = (i\ \sigma(i)) \cdot \sigma \mspc{or}{20} D_i( \sigma) = \sigma \cdot (\sigma^{-1}(i)\ i) \,.
\]
From this observation, we can easily derive a formula for the inverse of $D_i(\sigma)$, since
\[
    D_i(\sigma)^{-1} = ((i\ \sigma(i)) \cdot \sigma)^{-1} = \sigma^{-1} \cdot (i\ \sigma(i)) = D_i(\sigma^{-1}).
\]

The next proposition reformulates the definition of the horizontal faces for the inhomogeneous notation.
Using Figure \ref{cellular_models:parallel:cell_comparison_notations} it is not hard to come up with the right idea.

\begin{prop}
\label{cellular_models:parallel:prop_dh}
Let $\Sigma = \inhomq$ be an inner cell with homogeneous representation $\homogq$ and let $0_k < i < p_k$ for some $k$.
Then the $i^{\text{th}}$ horizontal face is
\[ 
    d''_i(\Sigma) = (\tau''_q \mid \ldots \mid \tau''_1)\,, 
\]
where
\[
    \tau''_k = D_i(\ \tau_k \cdot (i\  \sigma_{k-1}(i))\ )  \mspc{for}{20} 1 \leq k \leq q \,.
\]
In particular
\[
    \tau_k'' = D_i(\tau_k) \mspc{if}{20} i \notin \supp(\tau_k) \,.
\]
\end{prop}

\begin{proof}
For readibility, write $\rho_k = \tau_k \cdot (i\  \sigma_{k-1}(i))$.
First note that
\[ 
    d''_i(\Sigma) = (D_i(\sigma_q): \dotsc: D_i(\sigma_0)) = (D_i(\sigma_q) \cdot D_i(\sigma_{q-1})^{-1} \mid \ldots \mid D_i(\sigma_1) \cdot D_i(\sigma_0)^{-1})\,.
\]
Hence it suffices to show that
\[
    D_i(\sigma_k) \cdot D_i(\sigma_{k-1})^{-1} = D_i (\rho_k)
\]
holds for each $k = 1, \dotsc, q$. 
We have
\begin{align*}
D_i(\rho_k)
    &= (i\  \rho_k(i)) \cdot \rho_k \\
    \intertext{and using $\tau_k = \sigma_k \cdot \sigma_{k-1}^{-1}$ its clear that $\rho_k(i) = \sigma_k(i)$, so we are left with}
    &= (i\  \sigma_k(i)) \cdot \sigma_k \cdot \sigma_{k-1}^{-1} \cdot (i\  \sigma_{k-1}(i))\\
    &= D_i(\sigma_k) \cdot D_i(\sigma_{k-1})^{-1}\,.
\end{align*}
\end{proof}
