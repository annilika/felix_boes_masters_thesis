\subsection{The Dual of \texorpdfstring{$\kappa$}{kappa}}
\label{cellular_models:dual_ehrenfried:kappa_dual}
\symbolindex[k]{$\kappa^\ast$}{The dual of $\kappa$}{Subsection \ref{cellular_models:dual_ehrenfried:kappa_dual}}
The map $\pi^\ast$ is the canonical inclusion and we understood the horizontal coboundary operator $(\del'')^\ast$ in terms of coboundary traces via Proposition \ref{cellular_models:dual_ehrenfried:cob_tr_equals_cob}.
It remains to gain some insights on the dual of $\kappa$.

\begin{defi}
    \label{cellular_models:dual_ehrenfried:kappa_dual_sequences}
    \index{kappa!kappa star sequence}
    \symbolindex[k]{$\kappa^\ast_J$}{The summand $\kappa^\ast_J = \mueta^\ast_{j_1} \circ \ldots \circ \mueta^\ast_{j_k}$ with $J = (j_1, \ldots, j_k)$.}{Definition \ref{cellular_models:dual_ehrenfried:kappa_dual_sequences}}
    \symbolindex[1]{$\mueta^\ast$ or $\mueta^\ast_j$}{The dual of $\mueta$ or $\mueta_j$}{Definition \ref{cellular_models:dual_ehrenfried:kappa_dual_sequences}}
    \symbolindex[l]{$\Lambda^\ast_h$}{Parametrizes the $\kappa^\ast$-squences of length $h$.}{Definition \ref{cellular_models:dual_ehrenfried:kappa_dual_sequences}}
    Let $a$ and $b$ be positive integers.
    Denote $J_a^b = (b,b-1, \ldots, a+1, a)$ for $a \le b$ and let $J_a^b = ()$ be the empty sequence for $a > b$.
    The set of {\bfseries $\kappa^\ast$-sequences} is
    \[
        \Lambda^\ast_1 \{ () \} \mspc{and by concatenation}{20} \Lambda^\ast_{n+1} = \Lambda^\ast_n.\{ J_1^n, \ldots, J_n^n, J_{n+1}^n \} \,.
    \]
    For a $\kappa^\ast$-sequence $J = (i_1, \ldots, i_k)$ we set
    \[
        \kappa^\ast_J = \mueta^\ast_{i_1} \circ \ldots \circ \mueta^\ast_{i_k} \,.
    \]
\end{defi}

\begin{lem}
    \label{cellular_models:dual_ehrenfried:formula_for_kappa_dual}
    The map $\kappa^\ast$ is the alternating sum of all $\kappa^\ast$-sequences:
    \[
        \kappa^\ast_h = \sum_{(i_1, \ldots, i_k) \in \Lambda^\ast_h} (-1)^k \kappa^\ast_{(i_1, \ldots, i_k)} \,.
    \]
\end{lem}

\begin{proof}
    This is just the dual statement of Lemma \ref{cellular_models:ehrenfried:formula_for_kappa}.
\end{proof}

To get our hands on $\kappa^\ast$, we have to understand $\mueta_j^\ast$.
From the definition of the factorization map and $\kappa^\ast$,
it suffices to examine the image of a cell $(\tau_2 \mid \tau_1)$ of bidegree $(p,2)$ with $p = 2,3,4$ under $\mueta = \mueta_1$.

\begin{lem}
    \label{cellular_models:dual_ehrenfried:f_dual_of_a_cell}
    We have
    \begin{align}
        \tag{1.1} \mueta^\ast \left( \vierzweizelle{2}{1}{4}{3} \right) & = \vierzweizelle{2}{1}{4}{3} + \vierzweizelle{4}{3}{2}{1} \\
        \tag{1.2} \mueta^\ast \left( \vierzweizelle{3}{1}{4}{2} \right) & = \vierzweizelle{3}{1}{4}{2} + \vierzweizelle{4}{2}{3}{1} \\
        \tag{1.3} \mueta^\ast \left( \vierzweizelle{3}{2}{4}{1} \right) & = \vierzweizelle{3}{2}{4}{1} + \vierzweizelle{4}{1}{3}{2} \\
        \tag{2.1} \mueta^\ast \left( \dreizweizelle{2}{1}{3}{2} \right) & = \dreizweizelle{2}{1}{3}{2} + \dreizweizelle{3}{2}{3}{1} + \dreizweizelle{3}{1}{2}{1} \\
        \tag{2.2} \mueta^\ast \left( \dreizweizelle{2}{1}{3}{1} \right) & = \dreizweizelle{2}{1}{3}{1} + \dreizweizelle{3}{2}{2}{1} + \dreizweizelle{3}{1}{3}{2} \\
        \intertext{and for every $\Sigma$ not listed above we have}
        \tag{3}   \mueta^\ast (\Sigma) &= 0
    \end{align}
\end{lem}

\begin{proof}
    This is follows directly from a case-by-case analysis of $\mueta(\tau_2 \mid \tau_1)$ for all inner cells of bidegree $(p,2)$ with $p = 2,3,4$.
\end{proof}

\begin{lem}
    \label{cellular_models:dual_ehrenfried:f_dual_vanishes_at_monotonous_spot}
    Let $\Sigma = \inhomq$ be an inner cell with $\hgt(\tau_{j+1}) > \hgt(\tau_j)$ for some $q > j \ge 1$.
    Then
    \[
        \mueta^\ast_j(\Sigma) = 0 \,.
    \]
\end{lem}

\begin{proof}
    This follows immediately from the definition of the factorization map or from the lemma above.
\end{proof}

\begin{defi}
    \label{cellular_models:dual_ehrenfried:relevant_kappa_dual_sequences}
    \index{kappa!relevant kappa star sequence}
    \index{kappa!irrelevent kappa star sequence}
    \symbolindex[r]{$R^\Sigma$}{The set of relevant $\kappa^\ast$-sequences.}{Definition \ref{cellular_models:dual_ehrenfried:relevant_kappa_dual_sequences}}
    Let $\Sigma$ be a top dimensional cell.
    A $\kappa^\ast$-sequence $I \in \Lambda^\ast$ is {\bfseries relevant} if $\kappa^\ast_I(\Sigma) \neq 0$ and {\bfseries irrelevant} else.
    The set of relevant $\kappa^\ast$-sequences with respect to $\Sigma$ is denoted by $R^\Sigma$.
\end{defi}

The next lemma will become handy in the study of homology operations see Chapter \ref{homology_operations}.
It states that every relevant $\kappa^\ast$-sequences of a cell pictured in Figure \ref{cellular_models:dual_ehrenfried:motivate_lemma_for_multiplication}
is (up to a canonical shift) the concaternation of $\kappa^\ast$-sequences of $\Sigma'$ and $\Sigma''$
\begin{figure}[ht]
\centering
\incgfx{pictures/cellular_ehrenfried_dual_motivate_lemma_for_multiplication}
\caption{\label{cellular_models:dual_ehrenfried:motivate_lemma_for_multiplication}This cell might be seen as the product of $\Sigma'$ and $\Sigma''$ (see Definition \ref{homology_operations:parallel_patching_slit_pics:mu}).}
\end{figure}

\begin{lem}
    \label{cellular_models:dual_ehrenfried:relevant_kappa_dual_sequences_of_blocks}
    Consider a top dimensional cell $\Sigma = (\tau_{t+q} \mid \ldots \mid \tau_{t+1} \mid \tau_t \mid \ldots \mid \tau_1)$ with
    \[
        \supp(\tau_t, \ldots, \tau_1) \subseteq \{ \ul 1, \ldots, \ul s \} \mspc{and}{20} \supp(\tau_{t+q}, \ldots, \tau_{t+1}) \subseteq \{ \ul{s+1}, \ldots, \ul{s+p} \} \,.
    \]
    Then, the set of relevant $\kappa^\ast$-sequences with respect to $\Sigma$ is the concatenation
    \[
        R^\Sigma = R^{(\tau_t \mid \ldots \mid \tau_1)}.S_tR^{(\tau_{t+q} \mid \ldots \mid \tau_{t+1})} \,,
    \]
    where $S_t$ is defined on $\kappa^\ast$-sequences to be $S_t( i_1, \ldots, i_k ) = (t+i_1, \ldots, t+i_k)$.
\end{lem}

\begin{proof}
    The supports of $\tau_{t+q}, \ldots, \tau_{t+1}$ and $\tau_t, \ldots, \tau_1$ satisfy the inequality
    \[
         \min\supp(\tau_{t+q}, \ldots, \tau_{t+1})  > \max\supp(\tau_t, \ldots, \tau_1)
    \]
    and so does every term of $\kappa^\ast_I (\Sigma)$ for $I$ a relevant $\kappa^\ast$-sequence (compare Lemma \ref{cellular_models:dual_ehrenfried:f_dual_of_a_cell}).
    Then, by Lemma \ref{cellular_models:dual_ehrenfried:f_dual_vanishes_at_monotonous_spot}, there is not a single $\kappa^\ast$-sequence $(i_1, \ldots, i_k)$ with $i_j = t$ for some $j$.
    Therefore every relevant $\kappa^\ast$-sequence $I \in R^\Sigma$ is the concatenation of two relevant $\kappa^\ast$-sequences $I \in R^{(\tau_t \mid \ldots \mid \tau_1)}.S_tR^{(\tau_{t+q} \mid \ldots \mid \tau_{t+1})}$ and vice versa.
\end{proof}
