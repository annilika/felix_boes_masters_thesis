\section{The Bundle \texorpdfstring{$\HarmRad$}{{H\_g(m, n)}}}
\label{cellular:radial_bundle}

In this section, we want to outline the construction of the bundle $\HarmRad$ over the moduli space $\ModspcRad$.
For further details, see \cite{Boedigheimer2006}.

Let $[F, \mathcal C^+, \mathcal C^-, \mathcal P] \in \ModspcRad$ be a point of the moduli space, using the same notation as in the introduction (Section \ref{cellular_models:introduction}).
In order to describe the fiber over this point, we proceed as follows.
By classical potential theory, e.g. \cite[Theorem I.25]{opac-b1083717}, there exists a \textbf{harmonic potential} $u \colon F \to \R$
without any singularities and with all critical points in the interior of $F$.
The potential $u$ is uniquely determined by the complex structure and by the conditions that 
\begin{enumerate}
 \item on each boundary curve $C^+_k$ and $C^-_k$, $u$ is constant and non-negative, and
 \item for each outer boundary curve $C^+_k$, the constant value is $0$.
\end{enumerate}
Thereby, we can only choose the constant value of $u$ on one kind of boundary curves.
Here, it will be on the outgoing ones.
On the incoming boundaries, the potential $u$ yields constants $c_k > 0$ such that $u(C^-_k) = c_k$.

Similar as in the parallel case, we construct the unstable critical graph $\mathcal K_0$ of the negative gradient flow $- grad_u$.
Again calling the zeroes $\mathcal S$ of the gradient flow \textbf{stagnation points},
note that each flow line leaving a stagnation point $S$ either goes to another stagnation point or to a point $Q^+ \in \mathcal C^+$ in the outer boundary.
These points shall be called \textbf{cut points}, and the set of all cut points is denoted by $\mathcal Q^+$.
In Figure $\ref{trousers}$, we see an example for a surface with $n = 2$ incoming boundaries, $m = 1$ outgoing boundaries and $g = 0$.
Some lines of the gradient flow are indicated in blue, whereas the unstable flow lines, which are used to build up the critical graph, are drawn bold.
For reasons of clarity, only the critical flow line is drawn on the backside of the surface.

\begin{figure}[ht]
    \centering
    \def\svgwidth{.7\columnwidth}
    \input{pictures/intro_trousers.pdf_tex}
    \caption{
        \label{trousers}
        The gradient flow of a potential function on a surface with $n = 2$, $m = 1$ and $g = 0$.}
\end{figure}

Since $u$ is locally the real part of a holomorphic function,
the stagnation points $S \in \mathcal S$ are saddle points of some index $-2h \leq \ind(S) \leq -1$.
The sum of these indices has to equal the Euler characteristic $\chi(F)= -h$, 
thus we can conclude that there are at most $h$ stagnation points.

The vertices of the unstable critical graph are the points in $VK_0 = S \cup \mathcal Q^+$.
The (directed) edges of the unstable critical graph correspond to the (directed) unstable flow lines only.
It is possible that $\mathcal K_0$ is empty, namely when $F$ is an annulus.
Note that every component of the complement of the critical graph in $F$ contains exactly one boundary curve.
Hence, we can write $F_1, \dotsc, F_n$ for the components of $F \backslash \mathcal K_0$, which we also call \textbf{basins}.
Since the gradient vector field does not have any singularities,
we obtain a deformation retraction of $F_k$ onto $C^-_k$ by running the flow lines backwards.

In Figure \ref{trousers_flat}, our surface from Figure \ref{trousers} is looked at from above and dissected along the unstable critical graph,
yielding one basin for each of the two incoming boundary curves.

\begin{figure}[ht]
    \centering
    \def\svgwidth{.7\columnwidth}
    \input{pictures/intro_trousers_flat.pdf_tex}
    \caption{
        \label{trousers_flat}
        The surface with $n = 2$, $m = 1$ and $g = 0$ of Figure \ref{trousers} looked at from above, dissected along the unstable critical graph.}
\end{figure}

On each basin $F_k$, the harmonic function $u_k = u \mid_{F_k} \colon F_k \to \R$ is the real part of a holomorphic function
\[
   w_k = u_k + i v_k \colon F_k \to \C\,,
\]
where $v_k$ is a harmonic conjugate of $u_k$.
The function $v_k$ is only defined up to integer multiples of $2\pi i$,
but after this it is unique up to an additive constant $d_k$.
This we fix soon.
Thus, the function
\[
 W_k(z) = \exp(-w_k(z)) = \exp(-u_k(z)) \exp(-i v_k(z)) \colon F_k \to \subset \C
\]
is well defined and maps $F_k$ injectively into an annulus $\A_k$. 
By this, the modulus is determined by $u_k$ and the angle by $v_k$. 
Since $\exp(-u(z))$ equals $1$ when restricted to any outer boundary curve $C^+_l$ incident to $F_k$ 
and $\rho_k := \exp(-c_k) < 1$ when restricted to $C^-_k$,
the image of $F_k$ under $W_k$ is contained in an annulus $\A_k$ with outer radius $1$ and inner radius $\rho_k < 1$.
The additive constant $d_k$ in the definition of the harmonic conjugate $v_k$ of $u_k$ 
can be chosen such that the marked point $P_k$ on the incoming boundary curve $Q^-_k$ is mapped to the real point $(\rho_k, 0)$ of the annulus.
The image of $F_k \subset \A_k$ consists of the entire annulus, 
where finitely many slits from the outer boundary towards the center of the annulus are missing.
Remembering that the surface originally was glued together along these missing slits,
we can reconstruct the surface $F$ from the image of the basins $F_k$ on the annuli $\A_k$.
A more detailed description of these so-called radial slit domains follows in Section \ref{cellular_models:radial}.

We are now ready to finish the description of the bundle $\HarmRad$.
Let $\HarmRad$ be the space of all 
\[
   [F, \mathcal C^+, \mathcal C^-, \mathcal P, w]\,.
\] 
As above, $[F, \mathcal C^+, \mathcal C^-, \mathcal P] \in \ModspcRad$ is a point in the moduli space
and $w = (u, (v_k)_{k = 1, \dotsc, n})$ with
$u \colon F \to \R$ being the harmonic potential defined on the whole surface $F$,
and the functions $v_k \colon F_k \to \R$ being locally defined harmonic conjugates of $u$ on the basins $F_k$, for $k = 1, \dotsc, n$.
There is a projection
\[
  \HarmRad \xr{\cong} \ModspcRad\,, [F, \mathcal C^+, \mathcal C^-, \mathcal P, w] \mapsto [F, \mathcal C^+, \mathcal C^-, \mathcal P]\,,
\]
 with trivial fibres since there are no free parameters in the choice of $u$ and $v_k$.
Thus, the dimension of $\HarmRad$ equals
\[
 \dim(\HarmRad) = \dim(\ModspcRad) = 3h + n = 6g - 6 + 3 m + 4 n\,.
\]
