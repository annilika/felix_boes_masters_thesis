\subsection{The Parallel Slit Complex [B]}
\label{cellular_models:parallel:bicomplex}
Fixing the genus $g$, the number of punctures $m$, the number of boundary curves $n$ and the ordered partition $r = r_1 + \ldots + r_n$,
we are ready to define the parallel slit complex $P = P(h,m;r_1, \ldots, r_n)$.

\begin{defi}
\label{cellular_models:parallel:parallel_slit_complex}
\index{parallel slit complex}
\index{cell!degenerate}
\index{cell!non-degenerate}
Let $P_{p,q}$ be freely generated by all cells $\Sigma$ on $r$ levels of bidegree $(p,q)$ such that the conditions
\begin{enumerate}
    \item $N(\Sigma) \le h$,
    \item $m(\Sigma) \le m$,
    \item $n(\Sigma) \le n$,
    \item $\Sigma$ is connected and
    \item the levels of $\Sigma$ are ordered ascendingly with respect to $(r_1, \ldots, r_n)$.
\end{enumerate}
are fulfilled.
A cell $\Sigma \in P$ is said to be {\bf non-degenerate} with respect to the moduli space $\Modspc[n]$ and the partition $(r_1, \ldots, r_n)$
if it is a connected inner cell and has exactly $n$ boundary cycles, $m$ punctures and norm $h$.
All other cells in $P$ are called {\bf degenerate}.
Observe that cells $\Sigma \notin P$ are neither degenerate nor non-degenerate with respect to $\Modspc[n]$ and $(r_1, \ldots, r_n)$.

The {\bf vertical boundary operator} is the alternating sum of the vertical faces
\[
    \del' = \sum_{i=0}^{q}(-1)^i d'_i
\]
and the {\bf horizontal boundary operator} is the alternating sum of the horizontal faces
\[
    \del'' = \sum_{j=0}^{p}(-1)^j d''_j \,.
\]
The double complex $(P(h,m;r_1, \ldots, r_n), \del', \del'')$ is the {\bf parallel slit complex} with respect to the moduli space $\Modspc[n]$ and the partition $r = r_1 + \ldots + r_n$.
\end{defi}

\begin{rem}
Recall that the horizontal boundary operator is the alternating sum over all face maps in $\Delta^{p_1} \times \ldots \times \Delta^{p_r}$.
Hence $P$ is indeed a semi-multisimplicial complex although we are mostly concerned with the associated bicomplex, which we denote by $P$ as well.
\end{rem}

\begin{rem}
    Observe that every face $d'_0(\Sigma)$, $d'_q(\Sigma)$, $d''_{\ul 0_k}(\Sigma)$ and $d''_{\ul p_k}(\Sigma)$ of a non-degenerate cell $\Sigma \in P_{p,q}$ is degenerate.
    Observe further that all faces of a degenerate cell remain degenerate.
\end{rem}

Summing up the construction, we obtain the following theorem.
\begin{thm}
    The parallel slit complex $P$ is a semi-multisimplicial complex and the degenerate cells consitute a subcomplex $P'$.
    The space of parallel slit domains $\Parr$ is the complement of $|P'|$ inside $|P|$.
\end{thm}

Recall the construction of the Hilber uniformization in Section \ref{cellular_models:introduction}
\[
    \mathcal H \colon \Harmr \xhr{} |P| \,.
\]
Its corestriction to $\Parr = |P| - |P'|$ is a homeomorphism due to \cite{Boedigheimer19901}.
Therfore, $\Parr \simeq \Modspc$ serves as a good model for the corresponding mapping class group $\Gamma_{g,n}^m$:
\begin{thm}
    \label{cellular_models:parallel:parallel_slit_complex_serves_as_model}
    The space of parallel slit domains $\Parr = |P| - |P'|$ is a manifold of dimension $3h$ in the finite, semi-bisimplical complex $(P,P')$.
    By Poincaré duality
    \[
        H_\ast( \Modspc; \mathbb Z ) = H_\ast( \Parr; \mathbb Z ) \cong H^{3h-\ast}( P, P'; \mathcal O)
    \]
    where $\mathcal O$ are the orientation coefficients.
\end{thm}