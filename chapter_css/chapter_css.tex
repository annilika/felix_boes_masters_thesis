\chapter{Cluster Spectral Sequence}
\label{chapter_css}
We assume that the reader is familiar with spectral sequences.
There are several introductions to the theory of spectral sequences and we recommend working through \cite[Chapter 5]{Weibel1995} or \cite[Chapter 9]{Spanier199412}.

As before, we discuss the parallel and radial slit complex $P(h,m;r_1, \ldots, r_n)$ and $R(h,m,n)$ at once.
In order to have a compact notation, we concentrate on the parallel case and
we keep $g$, $n$, $m$ and $(r_1, \ldots, r_n)$ fixed.
\label{page:PP}%
\symbolindex[p]{$\PP_{\bullet,\bullet}$}{The abbriviation $\PP_{\bullet,\bullet}(h,m,; r_1, \ldots, r_n) = (P / P')_{\bullet, \bullet}$ with $P_{\bullet, \bullet} =  P_{\bullet,\bullet}(h,m,; r_1, \ldots, r_n)$.}{Page \pageref{page:PP}}
The double complex is denoted by $P_{\bullet,\bullet} = P_{\bullet,\bullet}(h,m,; r_1, \ldots, r_n)$,
the relative bicomplex associated with a fixed $P_{\bullet,\bullet}(h,m,; r_1, \ldots, r_n)$ is $\PP_{\bullet, \bullet} = (P / P')_{\bullet, \bullet}$
and the corresponding Ehrenfried complex is $\E = \E_\bullet(h,m; r_1, \ldots, r_n)$.

In this chapter, we follow \cite{Boedigheimer201314}.
We define a filtration on the bicomplex $\PP_{\bullet, \bullet}$ inducing a filtration on the Ehrenfried complex $\E_\bullet$.
In both cases, we obtain a first quadrant spectral sequence which collapses at the second page.

\section{The Cluster-Filtration on \texorpdfstring{$\PP_{\bullet, \bullet}$}{P} and \texorpdfstring{$\E_\bullet$}{E}}
The upcoming filtration is inspired by the following observation.
For a surface $F$ and a potential function $u$, we obtain the critical graph $\mc K_0$ on $F$.
Observe that $\mc K_0$ is connected.
Removing all poles and punctures from $\mc K_0$ yields a possibly disconnected graph $\mc K_-$ with $c$ connected components.
We imagine the horizontal and vertical face operators, which were defined on slit domains, as follows.
The critical flow lines and the equipotential lines which run through the stagnation points indicate vertical and horizontal stripes.
Each vertical or horizontal face operator collapses its corresponding stripe.
Let us assume that the surface, which results from a single collapse, is non-degenerate in order to study how $\mc K_-$ is altered.
It is readily seen that vertical faces leave the number of connected components of $\mc K_-$ fixed since vertical faces only collapse edges inside $\mc K_-$.
The horizontal face operator collapses a horizontal stripe $l$ along an equipotential line.
Therefore, it identifies the edges of $\mc K_-$ which form the upper margin of $l$ with the edges of $\mc K_-$ that form the lower margin of $l$.
The number of connected components of the new graph is therefore at most $c$ and at least $c-1$.
Keeping this in mind, following definitions and lemmata are straightforward.
\begin{defi}
    \label{css:cluster_relation}
    \index{cluster!index cluster}
    \index{cluster!cluster number}
    \symbolindex[2]{$\equicl$}{The equivalence relation used to count the clusters of a given cell.}{Definition \ref{css:cluster_relation}}
    \symbolindex[c]{$c(\Sigma)$}{The cluster number of a given cell $\Sigma$.}{Definition \ref{css:cluster_relation}}
    Consider a cell $ \Sigma = \homogq = \inhomq \in \PP_{p,q}$ with respect to $[p] = \{ \ul0_1, \ldots, \ul p_1, \ldots, \ul 0_r, \ldots, \ul p_r\}$. 
    On $[p]$, we declare a relation $\equicl$ as follows.
    \begin{center}
        $i \equicl i'$ \hspace{10pt} if \hspace{10pt} $i$ and $i'$ are in the same cycle of some $\tau_j$. 
    \end{center}
    The transitive closure of $\equicl$ is an equivalence relation.
    Equivalence classes are called {\bf (index) cluster} of $\Sigma$.
    The number $c(\Sigma) = c$ of equivalence classes is called the {\bf cluster number} of $\Sigma$ and we set $c(0) = 0$. 
\end{defi}

\begin{rem}
    Obviously, we have $1 \leq c(\Sigma) \leq h$ for every generator $\Sigma$.
\end{rem}

\begin{defi}
    \label{css:cluster_filtration_of_PP}
    \index{cluster!cluster filtration of the double complex}
    \symbolindex[F]{$F_c\PP$}{The cluster filtration of the double complex $\PP$.}{Definition \ref{css:cluster_filtration_of_PP}}
    The modules of the bicomplex $\PP$ are filtred as follows.
    For $c = 1, \ldots, h$, let
    \[
        F_c\PP_{p,q} = \langle  \Sigma \mspc{with}{10} c(\Sigma) \leq c \rangle \,.
    \]
    This filtration of the chain modules is a filtration of the chain complex by the next lemma. 
\end{defi}

\begin{lem}
    \label{css:c_induces_a_filtration}
    For a generator $\Sigma \in \PP_{p,q}$ we have
    \begin{enumerate}
        \item \label{css:c_induces_a_filtration_vertical} $c( d'_j(\Sigma) ) = c(\Sigma)$ for all $j$ and
        \item \label{css:c_induces_a_filtration_horizontal} $c( \Sigma ) -1 \leq c( d''_i( \Sigma ) ) \leq c(\Sigma)$ for all $i \in [p]$,
    \end{enumerate}
    if the faces are non-degenerate and therefore generators.
\end{lem}
\begin{proof}
    The index set $[p]$ of a non-degenerate vertical face $d'_j(\Sigma)$ agrees with the index set of $\Sigma$.
    In fact, the equivalence relation $\equicl$ coincides on these index sets.
    
    The index set of a horizontal face $d''_{\ul i_k}(\Sigma)$ is reduced by one, namely --- in the inhomogeneous notation --- by identifying $\ul i_k$ with $\ul{i+1}_k$.
    This changes the number of clusters if and only if $\ul i_k \not\equicl \ul{i+1}_k$, and obviously we have at most one cluster less.
    
    Thus $c(\del(\Sigma)) \leq c(\Sigma)$ and the filtration is a filtration of a chain complex. 
\end{proof}

The definition of the equivalence relation $\equicl$ on the index set $[p]$ of a cell $\Sigma$ is valid for all non-degenerate cells $\Sigma$.
In particular, we have a cluster number $c(\Sigma)$ defined for the generators of the Ehrenfried complex.
We need to see how it behaves under the boundary operator $\del_\E$.
Recall that $\E$ is a quasi-isomorphic direct summand\footnote{%
    To be precise, the Ehrenfried complex is, up to a shift in the homological degree, identified with a direct summand.
    The inclusion induces an isomorphism in homology.
} of the total complex of $\PP$.
The projection $\pi$ onto the top dimensional monotone cells is just the projection onto this summand.
The inverse of $\pi$ is $\kappa$, compare Section \ref{cellular_models:ehrenfried}.

\begin{defi}
    \label{css:cluster_filtration_of_E}
    \index{cluster!cluster filtration of the Ehrenfried complex}
    \symbolindex[F]{$F_c\E$}{The cluster filtration of the Ehrenfried complex.}{Definition \ref{css:cluster_filtration_of_E}}
    The modules of $\E$ are filtred as follows.
    For $c = 1, \ldots, h$, let
    \[
        F_c\E_p = \langle  \Sigma \mspc{with}{10} c(\Sigma) \leq c \rangle \,.
    \]
\end{defi}

This filtration of the chain modules is a filtration of the chain complex by the next lemma since $\del_\E = \pi \circ \del'' \circ \kappa$.

\begin{lem}
    \label{css:c_induces_a_filtration_ehr}
    For a generator $\Sigma \in \E_p$ we have
    \begin{enumerate}
        \item \label{css:c_induces_a_filtration_pi} $c( \pi (\Sigma) ) = c( \Sigma)$ and
        \item \label{css:c_induces_a_filtration_kappa} $c( \kappa( \Sigma ) ) = c( \Sigma )$.
    \end{enumerate}
\end{lem}
This lemma is an immediate consequence of Section \ref{cellular_models:ehrenfried} as $\E$ is a direct summand of $Tot(\PP)$.
However, we give another proof.
\begin{proof}
    The projection $\pi$ preserves the cluster number if $\Sigma$ is monotone.
    
    Recall that $\kappa$ is the alternating sum of all $\kappa$-sequences
    \[
        \kappa_{(j_1, \ldots, j_k)} = \mueta_{j_1} \circ \ldots \circ \mueta_{j_k}
    \]
    and $\mueta_l$ is the composition $\eta\mu$ of the multiplication $\mu$ with the factorability structure $\eta$,
    applied to $\tau_{j+1} | \tau_j$ in the $h$-tuple $\Sigma = \inhom$.
    Since $d'_j( \mueta_l( \Sigma )) = d'_j( \Sigma )$ for all $j = 1, \ldots, h-1$ 
    it follows from \ref{css:c_induces_a_filtration_vertical} in Lemma \ref{css:c_induces_a_filtration},
    that $c(\mueta_l( \Sigma))$ is either zero or equal to the cluster number of $\Sigma$.
    Thus the same is true for iterations of these $\mueta_l$ for various $l$. 
    In the linear combination $\kappa( \Sigma )$ all non-zero terms therefore have the same cluster number as $\Sigma$.
    Since $\Sigma$ itself is such a term, claim \ref{css:c_induces_a_filtration_kappa} follows.
\end{proof}

\begin{prop}
    \label{css:filtration_degree_boundary}
    Let $\Sigma$ be a generator in $\PP$ or $\E$ and denote the respective boundary operator by $\del = \del_{\PP}$ or $\del_\E$.
    The cluster number of every non-vanishing term $\tilde\Sigma$ in $\del(\Sigma)$ satisfies
    \[
        c( \Sigma ) -1 \leq c( \tilde\Sigma ) \leq c(\Sigma) \,.
    \]
\end{prop}
\begin{proof}
    The claim is an immediate consequence of Lemmata \ref{css:c_induces_a_filtration} and \ref{css:c_induces_a_filtration_ehr} since $\del_\E = \pi \circ \del'' \circ \kappa$.
\end{proof}

\section{The Cluster Spectral Sequence for \texorpdfstring{$\PP_{\bullet, \bullet}$}{P} and \texorpdfstring{$\E_\bullet$}{E}}
\label{css:cluster_spectral_sequence_chapter}%
\index{cluster!cluster spectral sequence}
Throughout this section, we fix a ring $A$.
Consequently, we treat $\PP$ and $\E$ as complexes over $A$.

\begin{prop}
    \label{css:cluster_spectral_sequence}
    \symbolindex[e]{$E^0_{k,c}(\PP)$}{The cluster spectral sequence of the double complex.}{Proposition \ref{css:cluster_spectral_sequence}}
    \symbolindex[e]{$E^0_{p,c}(\E)$}{The cluster spectral sequence of the Ehrenfried complex.}{Proposition \ref{css:cluster_spectral_sequence}}
    Let $g$, $n$, $m$ and $(r_1, \ldots, r_n)$ be given and set $h = 2g-2+m+n+r_1 + \ldots + r_n$.
    There are two first quadrant spectral sequences
    \[
        E^0_{k,c}(\PP) = \bigoplus_{p+q=k}\left[F_c\PP_{p,q}(h,m;r_1, \ldots, r_n) / F_{c-1}\PP_{p,q}(h,m;r_1, \ldots, r_n) \right]
    \]
    converging towards
    \[
        E^0_{k,c}(\PP) \Rightarrow H_{k+c}( \PP_{\bullet, \bullet}(h,m;r_1, \ldots, r_n) ; A )
    \]
    and
    \[
        E^0_{p,c}(\E) = F_c\E_p(h,m;r_1, \ldots, r_n) / F_{c-1}\E_p(h,m;r_1, \ldots, r_n)
    \]
    converging towards
    \[
        E^0_{p,c}(\E) \Rightarrow H_{p+c}( \E_\bullet(h,m;r_1, \ldots, r_n); A ) \,.
    \]
    Both spectral sequences collapse at the second page.
\end{prop}
\begin{proof}
    The existence of both spectral sequences is evident.
    Both complexes $\PP$ and $\E$ are bounded, so the associated spectral sequence is first quadrant and convergent.
    The only non-trivial differentials are page zero and one by Proposition \ref{css:filtration_degree_boundary}, see Lemma \ref{css:collapsing_ss}.
\end{proof}
\begin{rem}
If $A$ is a field, then we have
\[
    H_\ast( \E_\bullet(h,m;r_1, \ldots, r_n); A ) = \bigoplus_{p+c=\ast} E^2_{p,c}(\E) \,.
\]
This is one foundation of our computer-aided computations.
\end{rem}

\begin{lem}
    \label{css:collapsing_ss}
    Consider a chain complex $(C, \del)$ with filtration $F_cC$ and assume $\del$ decreases the filtration degree by at most $s$.
    Then, the associated spectral sequence collapses at $E^{s+1}$.
\end{lem}
\begin{proof}
    In order to prove the convergence theorem for reasonable filtred chain complexes (c.f.\ \cite[Chapter 9, Theorem 2]{Spanier199412}) one finds
    \[
        Z^r_{p,c} = \{ x \in F_pC_{p+c} \mid \del x \in F_{p-r}C_{p+c-1} \}
    \]
    and
    \[
        E^r_{p,c} = Z^r_{p,c} / \big( Z^{r-1}_{p-1,c+1} + \del(Z^{r-1}_{p+r-1,c-r+2})\big) \,.
    \]
    The $r\Th$ differential $d^r_{p,c} \colon E^r_{p,c} \to E^r_{p-r,c+r-1}$ is induced by $\del$ since
    \[
        \del(Z^r_{p,c}) \subseteq Z^r_{p-r,c+r-1} \mspc{and}{20} \del\big( Z^{r-1}_{p-1,c+1} + \del(Z^{r-1}_{p+r-1,c-r+2})\big) \subseteq \del(Z^{r-1}_{p-1,c+1}) \,.
    \]
    For $r \ge s+1$, we assumed $\del(Z^r_{p,c}) = 0$.
    We conclude $d^r = 0$ and $E^\infty = E^{s+1}$.
\end{proof}

\section{The Cluster Spectral Sequence in Terms of Matrices}
\label{css:section_matrix_version}
Let us study the differentials of the spectral sequence associated with the Ehrenfried complex.
The presented arguments can be applied to $\PP$ as well.

The Ehrenfried complex is a based chain complex and the cluster-filtration has a remarkable effect on the transformation matrices.
We exploit this fact in our computer program, compare Section \ref{program}.
The filtration of $\E$ is induced by a filtration of the bases elements.
For each degree $p$, we regroup the basis elements with identical cluster number and order the groups ascendingly.
The boundary operator reduces the filtration degree by at most one by Proposition \ref{css:filtration_degree_boundary}.
The $p\Th$ transformation matrix is therefore a block matrix.
\[ 
    \begin{pmatrix}
        d^0 & d^1 \\
            & d^0   & d^1 \\
            &       & d^0   & d^1 \\
            &       &       &     & \ddots
    \end{pmatrix}
\]    
The submatrices $d^0$ or $d^1$ correspond to the differentials of the zeroth respectively first page.
Observe that the second term $E^2$ is given by
\[
    \ker( d^1|_{\ker(d^0)}) / \left[ \img(d^0) + \img(d^1|_{\ker(d^0)}) \right] \,.
\]
For actual computation, it is worthwile to detect the homology via determining and diagonalizing the transformation matrix block by block.
