\section{A Survey on the Stable and Unstable (Co-)Homology}
First of all, we recall the definition of the mapping class group.
Consider the space $\Diff_{g,n}^{m}$ of orientation-preserving diffeomorphisms on a surface of genus $g$, leaving its $n$ boundary curves pointwise fixed while permuting $m$ selected points.
Paths in $\Diff_{g,n}^{m}$ are isotopies and $\Gamma_{g,n}^m = \pi_0(\Diff_{g,n}^m)$ is the group of path components.
Analogously, the mapping class group $\Gamma_g^\bullet(m,n)$ is the group of path components of the space of diffeomorphisms on a surface of genus $g$ leaving the $n$ incoming boundary curves pointwise fixed while permuting the outgoing $m$ boundary curves.
The group structure is induced by the composition of diffeomorphisms.
The mapping class groups $\Gamma_{g,n}^m$ and $\Gamma_g^\bullet(m,n)$ are known to be isomorphic.

\paragraph{Stable (Co-)Homology}
We begin with a revision of the stable cohomology of $\Gamma_{g,n} = \Gamma_{g,n}^0$.
Glueing a pair of pants along one or two boundary curves of a given oriented surface induces a group homomorphism
$\varphi_g \colon \Gamma_{g,n} \to \Gamma_{g,n+1}$, respectively $\psi_g \colon \Gamma_{g, n+1} \to \Gamma_{g+1,n}$ on the mapping class groups,
by extending the diffeomorphisms in question via the identity.
If the surface has exactly one boundary curve, glueing in a disc induces a homomorphism $\vartheta \colon \Gamma_{g,1} \to \Gamma_{g,0}$.
Due to \cite{Harer1985}, the mapping class groups $\Gamma_{g,n}$ with $n \ge 1$ are homologically stable.
Including several improvements concerning the degree of stabilization we have
\begin{thm*}[Harer]
    Let $g \ge 0$ and $n \ge 1$.
    The induced map
    \[
        \varphi_\ast \colon H_\ast( \Gamma_{g,n}; \mathbb Z ) \to H_\ast( \Gamma_{g,n+1}; \mathbb Z)
    \]
    is an injection for all $\ast$ and an isomorphism for $\ast \le \frac{2}{3}g$.
    The induced map
    \[
        \psi_\ast \colon H_\ast( \Gamma_{g,n+1}; \mathbb Z ) \to H_\ast( \Gamma_{g+1,n}; \mathbb Z)
    \]
    is a surjection for $\ast \le \frac{2}{3}g + \frac{1}{3}$ and an isomorphism for $\ast \le \frac{2}{3}g - \frac{2}{3}$.
    The induced map
    \[
        \vartheta_\ast \colon H_\ast( \Gamma_{g,1}; \mathbb Z ) \to H_\ast( \Gamma_{g,0}; \mathbb Z)
    \]
    is a surjection for $\ast \le \frac{2}{3}g + 1$ and an isomorphism for $\ast \le \frac{2}{3}g$.
\end{thm*}
\label{page:stabilization_map}%
A proof including the mentioned improvements can be found in \cite{Wahl2012}.

The composition $\psi_g \varphi_g \colon \Gamma_{g,1} \to \Gamma_{g+1,1}$ is injective and $\Gamma_{\infty,1} = \cup_{g=1}^\infty \Gamma_{g,1}$ is the stable mapping class group.
We obviously obtain $\varinjlim H_\ast( \Gamma_{g,1}; \mathbb Z ) \cong H_\ast(\Gamma_{\infty,1})$.
\begin{thm*}[Mumford's Conjecture (Madsen--Weiss \cite{MadenWeiss2007})]
    The rational cohomology of the stable mapping class group is a polynomial algebra
    \[
        H^\ast( \Gamma_{\infty,1}; \mathbb Q ) \cong \mathbb Q[ \kappa_1, \kappa_2, \ldots ]
    \]
    in the Mumford--Morita--Miller classes $\kappa_i$ living in degree $2i$.
\end{thm*}

\paragraph{Unstable Homology}
In contrast to the stable picture, very little is known about the unstable one, i.e., the homology or cohomology of single moduli spaces.
Note that for a class in degree say $2$ to be stable, we have to go to $g \ge 4$.

Before reviewing $\Modspc$, consider the moduli space $\tildeModspc$ of Riemann surfaces of genus $g$ where both the boundary curves and punctures are pointwise fixed.
For single degrees $\ast = 1,2,3$, there are results known for almost all $g$.
Based on the works of Mumford \cite{Mumford1967} and Powell \cite{Powell1978} the first integral homology is known to be
\[
    H_1(\tildeModspc; \mathbb Z) \cong
        \begin{cases}
            \mathbb Z / 10      & g=2 \\
            0                   & g\ge3
        \end{cases} \,.
\]
A proof of this version can be found in Korkmaz--Stipsicz \cite{KorkmazStipsicz2003}.
Moreover, \cite{KorkmazStipsicz2003} improves a theorem by Harer \cite[Theorem 0.a]{Harer1991}:
\[
    H_2(\tildeModspc; \mathbb Z) \cong \mathbb Z^{m+1} \mspc{for}{20} g \ge 4 \,.
\]
The third rational homology vanishes due to \cite[Theorem 0.b]{Harer1991}:
\[
    H_3(\mathfrak{M}_{g,n}^0; \mathbb Q) = 0 \mspc{for}{20} g \ge 6 \,.
\]

In case of no punctures but permutable boundary, the first integral homology is known due to Korkmaz--McCarthy \cite[Theorem 3.13]{KorkmazMcCarthy2000}.
Denoting the corresponding moduli space by $\mathfrak M_{g,(n)}^0$ they show
\[
    H_1( \mathfrak M_{g,(n)}^0; \mathbb Z) \cong
    \begin{cases}
        \mathbb Z / 10                          & g=1, n=0,1 \\
        \mathbb Z / 12 \oplus \mathbb Z/ 2      & g=1, n\ge2 \\
        \mathbb Z / 10                          & g=2, n=0,1 \\
        \mathbb Z / 10 \oplus \mathbb Z/ 2      & g=2, n\ge2 \\
        0                                       & g=3, n=0,1 \\
        \mathbb Z / 2                           & g=3, n\ge2 \\
    \end{cases} \,.
\]

In this thesis we study the moduli space $\Modspc$.
For $g=0$ and $n=1$, the integral homology of the moduli space $\mathfrak{M}_{0,1}^m$ coincides with the well-known group homology of the braid group on $m$ strings.
Besides that, there are some scattered computations for low $g$ and $n$.

\paragraph{Slit models}
In \cite{Boedigheimer19901} Bödigheimer provides the space of parallel slit domains $\Par_{g,n}^m$, which is homeomorphic to an affine bundle over $\Modspc$ via the Hilbert uniformization.
It is a manifold and an open subspace of a finite semi-multisimplicial space $P$ making it possible
(1) to compute the homology of the moduli spaces via Poincaré duality and
(2) to define an operad structure by the action of the little cubes operad on the family of moduli spacess $\Modspc$;
this induces an action of the Dyer-Lashof algebra on their homology.
Exploiting this model, Ehrenfried could completly compute the integral homology for $g=2$ and $n=1$, compare \cite{Ehrenfried1997}.
This is, up to date and apart from $g=0$ and $g=1$, the only moduli space whose integral homology is known. 
His result is reproduced in the following tabl.
\[
    H_\ast( \mathfrak{M}_{2,1}^0; \mathbb Z) \cong
        \begin{cases}
            \mathbb Z                           & \ast = 0\\
            \mathbb Z/10                        & \ast = 1\\
            \mathbb Z/2                         & \ast = 2\\
            \mathbb Z \oplus \mathbb Z/2        & \ast = 3\\
            \mathbb Z/6                         & \ast = 4\\
            0                                   & \ast \ge 5
        \end{cases}
\]
Later, Godin obtained the same results with different methods, compare \cite{Godin2007}.
For $g=3$ and $m=1$, Wang computed the $p$-torsion for many primes in \cite{Wang201102}.
We will describe her results in detail, see below.

We mentioned above a complex $P$ with a subcomplex $P'$ such that $P - P' = \Par$.
The double complex associated with $P$ admits an explicit combinatorial decribtion.
However, the number of cells prevents (even computer-aided) calculations exceeding $h = 5$ where $h = 2g-2+m+2n$.
To demonstrate this, we list the number of cells in bidegree $(p,q)$ for $g=1$ and $m=3$ (see Figure \ref{introduction:number_cells_g_1_m_3_n_1}).
\begin{figure}[ht]
    \centering
    \begin{tabular}{|r||r|r|r|r|r|r|r|}
        \hline
        $q=5$ & 640 & 12425 & 74610 & 202825 & 278600 & 189000 & 50400 \\ \hline
        $q=4$ & 800 & 18500 & 122700 & 357280 & 516880 & 365400 & 100800 \\ \hline
        $q=3$ & 240 & 7425 & 57375 & 185220 & 289380 & 217350 & 63000 \\ \hline
        $q=2$ & 10 & 650 & 6800 & 26600 & 47740 & 39900 & 12600 \\ \hline
        $q=1$ & 0 & 0 & 35 & 315 & 910 & 1050 & 420 \\ \hline \hline
              & $p=4$ & $p=5$ & $p=6$ & $p=7$ & $p=8$ & $p=9$ & $p=10$ \\ \hline
    \end{tabular}
    \caption{\label{introduction:number_cells_g_1_m_3_n_1}The number of cells of the bicomplex for $\mathfrak M_{1,1}^3$.}
\end{figure}
Due to \cite{Visy201011}, the vertical homology of $(P,P')$ is always concentrated in its top row being of degree $q=h$.
The resulting chain complex, called Ehrenfried complex, is considerably smaller, compare Figure \ref{introduction:cells_ehr_g_1_m_3_n_1}.
\begin{figure}[ht]
    \centering
    \begin{tabular}{|r|r|r|r|r|r|r|}
        \hline
        70 & 700 & 2520 & 4480 & 4270 & 2100 & 420 \\ \hline \hline
        $p=4$ & $p=5$ & $p=6$ & $p=7$ & $p=8$ & $p=9$ & $p=10$ \\ \hline
    \end{tabular}
    \caption{\label{introduction:cells_ehr_g_1_m_3_n_1}The number of cells of the Ehrenfried complex for $\mathfrak M_{1,1}^3$.}
\end{figure}
These insights make it possible to perform several computations for $h \le 6$.
In \cite{Wang201102}, Wang computes the elementary divisors modulo $p^{k_p}$ of the differentials in this Ehrenfried complex for
$p^{k_p} = 2^6$, $3^4$, $5^3$, $7^2$, $11^2$, $13^2$, $17$, $19$ and $23$.
Observe that there might be undetected $p$-torsion of the form $\mathbb Z / p^k \mathbb Z$ in case
(1) $p$ a prime greater then $23$ and $k \ge 1$ or
(2) $p$ a prime at most $23$ and $k > k_p$.
Besides $H_0(\Modspc;\mathbb Z) = \mathbb Z$,
we have $H_1(\mathfrak{M}_{3,1}^0;\mathbb Z) = 0$ due to \cite{Powell1978} and $H_2(\mathfrak{M}_{3,1}^0;\mathbb Z) = \mathbb Z \oplus \mathbb Z / 2 \mathbb Z$ due to \cite{Sakasai2012}.
For $2g+m=6$ and $n=1$, the remaining free summands where unkown until this point in time.
Using a new spectral sequene, we provide the free parts by computing the rational homology.
This, in turn, allows for $g=3$ and $n=1$ to conclude, that Wang had indeed discovered all $p$-torsion for $p \le 23$.
\begin{thm*}[Bödigheimer, Powell, Sakasai, Wang, B., H.]
    Let $k_2 = 6$, $k_3 = 4$, $k_5 = 3$, $k_7 = k_{11} = k_{13} = 2$, $k_{17} = k_{19} = k_{23} = 1$ and $k_p = 0$ for $p > 23$ prime.
    The integral homology of the moduli spaces $\mathfrak{M}_{3,1}^0$, $\mathfrak{M}_{2,1}^2$ or $\mathfrak{M}_{1,1}^6$ is given by the following tables,
    where $\bldots$ denotes in the first case possible $p$-torsion for primes $p > 23$, 
    and in the other two cases possible $p$-torsion of the form $\mathbb Z / p^k \mathbb Z$ for $p$ any prime and $k > k_p$.
    \\[2pt]
    \noindent The integral homology of the moduli space $\mathfrak{M}_{3,1}^0$ is
    \[
        H_\ast( \mathfrak{M}_{3,1}^0; \mathbb Z ) \cong 
            \begin{cases}
                \mathbb Z           & \ast = 0\\
                0                   & \ast = 1\\
                \mathbb Z \oplus \mathbb Z / 2                                  & \ast = 2\\
                \mathbb Z \oplus \mathbb Z/2 \oplus \mathbb Z/3 \oplus \mathbb Z/4 \oplus \mathbb Z/7 \oplus \bldots & \ast = 3\\
                (\mathbb Z/2)^2 \oplus (\mathbb Z/3)^2  \oplus \bldots          & \ast = 4\\
                \mathbb Z \oplus \mathbb Z/2 \oplus \mathbb Z/3  \oplus \bldots & \ast = 5\\
                \mathbb Z \oplus (\mathbb Z/2)^3  \oplus \bldots                & \ast = 6\\
                \mathbb Z / 2  \oplus \bldots   & \ast = 7\\
                0  \oplus \bldots               & \ast = 8\\
                \mathbb Z  \oplus \bldots       & \ast = 9\\
                0                               & \ast \ge 10\\
            \end{cases}\,.
    \]
    \\[2pt]
    \noindent The integral homology of $\mathfrak M_{2,1}^2$ is
    \[
        H_\ast( \mathfrak{M}_{2,1}^2; \mathbb Z ) \cong 
        \begin{cases}
            \mathbb Z           & \ast = 0\\
            (\mathbb Z/2)^2 \oplus \mathbb Z/5 \oplus \bldots    & \ast = 1\\
            \mathbb Z \oplus (\mathbb Z/2)^2 \oplus \bldots      & \ast = 2\\
            \mathbb Z^3 \oplus (\mathbb Z/2)^4 \oplus \bldots    & \ast = 3\\
            \mathbb Z \oplus (\mathbb Z/2)^5 \oplus (\mathbb Z/3)^3 \oplus \bldots       & \ast = 4\\
            \mathbb Z^2 \oplus (\mathbb Z/2)^4 \oplus \mathbb Z/3 \oplus \bldots         & \ast = 5\\
            \mathbb Z^2 \oplus (\mathbb Z/2)^3 \oplus \bldots    & \ast = 6\\
            \mathbb Z/2 \oplus \bldots                           & \ast = 7\\
            0                   & \ast \ge 8\\
        \end{cases} \,.
    \]
    
    \noindent The integral homology of $\mathfrak M_{1,1}^4$ is
    \[
        H_\ast( \mathfrak{M}_{1,1}^4; \mathbb Z ) \cong 
        \begin{cases}
            \mathbb Z           & \ast = 0\\
            \mathbb Z \oplus \mathbb Z/2 \oplus \bldots          & \ast = 1\\
            (\mathbb Z/2)^3 \oplus \bldots                       & \ast = 2\\
            \mathbb Z^2 \oplus (\mathbb Z/2)^3 \oplus \bldots    & \ast = 3\\
            \mathbb Z^3 \oplus (\mathbb Z/2)^2 \oplus \bldots    & \ast = 4\\
            \mathbb Z^2 \oplus \mathbb Z/2 \oplus \bldots        & \ast = 5\\
            \mathbb Z \oplus \bldots     & \ast = 6\\
            0                           & \ast \ge 7\\
        \end{cases} \,.
    \]
\end{thm*}
One might conjecture that the undetermined torsion $\bldots$ is trivial in all cases.

In \cite{Mehner201112}, Mehner provides a computer program that computes the integral and $\mathbb F_2$ homology of single moduli spaces for $n=1$, $g \le 2$.
Moreover, he implements simplicial versions of the Dyer-Lashof operations introduced in \cite{Boedigheimer19902} and obtaines some of the generators of the respectively homology via operations.
