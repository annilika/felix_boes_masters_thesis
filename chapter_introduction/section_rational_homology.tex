\section{The Rational Homology of the Moduli Spaces in Short Form}
\label{introduction:more_rational_homology}
In this section, we present a short form of our computations with coefficients in the rationals.
Some of the results were already known, compare the discussion above.
We include them anyways.
The number of boundary components is always $n=1$.
All cluster spectral sequences with coefficients in $\mathbb Q$ and $\mathbb F_2$ are found in Section \ref{program:results}.

\paragraph{The case \texorpdfstring{$g=0$}{g=0}:}
The moduli space $\mathfrak M_{0,1}^m$ is the classifying space of the braid group on $m$ strings.
Its homology is understood.
We have
\[
    H_\ast(\mathfrak M_{0,1}^m; \mathbb Q) =
        \begin{cases}
            \mathbb Q   & \ast = 0,1 \; and\; \ast \le m \\
            0           & else
        \end{cases}
\]

\paragraph{The case \texorpdfstring{$g=1$}{g=1}:}
For $m = 0,1,2,3,4,5$, the rational homology of $\mathfrak M_{1,1}^m$ is given by the following tables.
\begin{center}
    \begin{tabular}{|r|r|r|}
        \hline
        \multicolumn{3}{|c|}{$H_p( \mathfrak{M}_{1,1}^0; \mathbb Q )$} \\ \hline
        $p=0$&$p=1$&$p\ge3$\\ \hline \hline
        $\mathbb Q$&$\mathbb Q$&$0$\\ \hline
    \end{tabular}
    
    \vspace{2.5ex}
    
    \begin{tabular}{|r|r|r|}
        \hline
        \multicolumn{3}{|c|}{$H_p( \mathfrak{M}_{1,1}^1; \mathbb Q )$} \\ \hline
        $p=0$&$p=1$&$p\ge3$\\ \hline \hline
        $\mathbb Q$&$\mathbb Q$&$0$\\ \hline
    \end{tabular}
    
    \vspace{2.5ex}
    
    \begin{tabular}{|r|r|r|}
        \hline
        \multicolumn{3}{|c|}{$H_p( \mathfrak{M}_{1,1}^2; \mathbb Q )$} \\ \hline
        $p=0$&$p=1$&$p\ge3$\\ \hline \hline
        $\mathbb Q$&$\mathbb Q$&$0$\\ \hline
    \end{tabular}
    
    \vspace{2.5ex}
    
    \begin{tabular}{|r|r|r|r|r|r|r|}
        \hline
        \multicolumn{7}{|c|}{$H_p( \mathfrak{M}_{1,1}^3; \mathbb Q )$} \\ \hline
        $p=0$&$p=1$&$p=2$&$p=3$&$p=4$&$p=5$&$p\ge6$\\ \hline \hline
        $\mathbb Q$&$\mathbb Q$&$0$&$\mathbb Q$&$\mathbb Q$&$\mathbb Q^2$&$0$\\ \hline
    \end{tabular}
    
    \vspace{2.5ex}
    
    \begin{tabular}{|r|r|r|r|r|r|r|r|}
        \hline
        \multicolumn{8}{|c|}{$H_p( \mathfrak{M}_{1,1}^4; \mathbb Q )$} \\ \hline
        $p=0$&$p=1$&$p=2$&$p=3$&$p=4$&$p=5$&$p=6$&$p\ge7$\\ \hline \hline
        $\mathbb Q$&$\mathbb Q$&$0$&$\mathbb Q^2$&$\mathbb Q^3$&$\mathbb Q^2$&$\mathbb Q$&$0$\\ \hline
    \end{tabular}
    
    \vspace{2.5ex}
    
    \begin{tabular}{|r|r|r|r|r|r|r|r|r|r|}
        \hline
        \multicolumn{10}{|c|}{$H_p( \mathfrak{M}_{1,1}^5; \mathbb Q )$} \\ \hline
        $p=0$&$p=1$&$p=2$&$p=3$&$p=4$&$p=5$&$p=6$&$p=7$&$p=8$&$p\ge9$\\ \hline \hline
        $\mathbb Q$&$?$&$?$&$?$&$?$&$?$&$?$&$\mathbb Q^4$&$\mathbb Q^2$&$0$\\ \hline
    \end{tabular}
\end{center}

\paragraph{The case \texorpdfstring{$g=2$}{g=2}:}
For $m = 0,1,2$, the rational homology of $\mathfrak M_{2,1}^m$ is given by the following tables.
\begin{center}
    \begin{tabular}{|r|r|r|r|r|}
        \hline
        \multicolumn{5}{|c|}{$H_p( \mathfrak{M}_{2,1}^0; \mathbb Q )$} \\ \hline
        $p=0$&$p=1$&$p=2$&$p=3$&$p\ge4$\\ \hline \hline
        $\mathbb Q$&$0$&$0$&$\mathbb Q$&$0$\\ \hline
    \end{tabular}
    
    \vspace{2.5ex}
    
    \begin{tabular}{|r|r|r|r|r|r|r|r|}
        \hline
        \multicolumn{8}{|c|}{$H_p( \mathfrak{M}_{2,1}^1; \mathbb Q )$} \\ \hline
        $p=0$&$p=1$&$p=2$&$p=3$&$p=4$&$p=5$&$p=6$&$p\ge7$\\ \hline \hline
        $\mathbb Q$&$0$&$\mathbb Q$&$\mathbb Q^2$&$0$&$\mathbb Q$&$\mathbb Q$&$0$\\ \hline
    \end{tabular}
    
    \vspace{2.5ex}
    
    \begin{tabular}{|r|r|r|r|r|r|r|r|}
        \hline
        \multicolumn{8}{|c|}{$H_p( \mathfrak{M}_{2,1}^2; \mathbb Q )$} \\ \hline
        $p=0$&$p=1$&$p=2$&$p=3$&$p=4$&$p=5$&$p=6$&$p\ge7$\\ \hline \hline
        $\mathbb Q$&$0$&$\mathbb Q$&$\mathbb Q^3$&$\mathbb Q$&$\mathbb Q^2$&$\mathbb Q^2$&$0$\\ \hline
    \end{tabular}
\end{center}

\paragraph{The case \texorpdfstring{$g=3$}{g=3}:}
The rational homology of $\mathfrak M_{3,1}^0$ is given by the following table.
\begin{center}
    \begin{tabular}{|r|r|r|r|r|r|r|r|r|r|r|}
        \hline
        \multicolumn{11}{|c|}{$H_p( \mathfrak{M}_{3,1}^0; \mathbb Q )$} \\ \hline
        $p=0$&$p=1$&$p=2$&$p=3$&$p=4$&$p=5$&$p=6$&$p=7$&$p=8$&$p=9$&$p\ge10$\\ \hline \hline
        $\mathbb Q$&$0$&$\mathbb Q$&$\mathbb Q$&$0$&$\mathbb Q$&$\mathbb Q$&$0$&$0$&$\mathbb Q$&$0$\\ \hline
    \end{tabular}
\end{center}
