\chapter{Introduction}
In this thesis, we study two families of moduli spaces:

\noindent
(1) the moduli spaces $\Modspc[n]$ of Riemann surfaces of genus $g \ge 0$ with $m \ge 0$ (permutable) punctures and $n \ge 1$ boundary curves and

\noindent
(2) the moduli spaces $\ModspcRad$ of Riemann surfaces of genus $g \ge 0$ with $n \ge 1$ incoming and $m \ge 1$ outgoing boundary curves (the moduli space of cobordisms) and
with an extra marked point on each of the boundary incoming curves.

The latter are important for string topology or conformal field theories; the former (but for $n=m=0$) are the classical moduli space from algebraic geometry or complex analysis.
For our techniques to work, we always need $n \ge 1$ in case (1) and $n,m \ge 1$ in case (2).

Under these assumptions of non-empty boundary, the moduli spaces $\Modspc$ and $\ModspcRad$ are manifolds of dimension $6g-6+2m+4n$ respectively $6g-6+3m+3n$.
They are orientable for $m < 2$.
Moreover, they are homotopy equivalent to the classifying spaces $B\Gamma_{g,n}^m$ respectively $B\Gamma^\bullet_g(m,n)$ of the mapping class groups $\Gamma_{g,n}^m$ respectively $\Gamma^\bullet_g(m,n)$.

In this introduction, we review the stable and unstable (co-)homology of the moduli spaces and present our results.
At the end, we explain the organization of our thesis.

\section{A Survey on the Stable and Unstable (Co-)Homology [B]}
First of all, we recall the definition of the mapping class group.
Consider the space $\Diff_{g,n}^{m}$ of orientation-preserving diffeomorphisms on a surface of genus $g$, leaving its $n$ boundary curves pointwise fixed while permuting $m$ selected points.
Paths in $\Diff_{g,n}^{m}$ are isotopies and $\Gamma_{g,n}^m = \pi_0(\Diff_{g,n}^m)$ is the group of path components.
Analogously, the mapping class group $\Gamma_g^\bullet(m,n)$ is the group of path components of the space of diffeomorphisms on a surface of genus $g$ leaving the $n$ incoming boundary curves pointwise fixed while permuting the outgoing $m$ boundary curves.
The group structure is induced by the composition of diffeomorphisms.
The mapping class groups $\Gamma_{g,n}^m$ and $\Gamma_g^\bullet(m,n)$ are known to be isomorphic.

\paragraph{Stable (Co-)Homology}
We begin with a revision of the stable cohomology of $\Gamma_{g,n} = \Gamma_{g,n}^0$.
Glueing a pair of pants along one or two boundary curves of a given oriented surface induces a group homomorphism
$\varphi_g \colon \Gamma_{g,n} \to \Gamma_{g,n+1}$, respectively $\psi_g \colon \Gamma_{g, n+1} \to \Gamma_{g+1,n}$ on the mapping class groups,
by extending the diffeomorphisms in question via the identity.
If the surface has exactly one boundary curve, glueing in a disc induces a homomorphism $\vartheta \colon \Gamma_{g,1} \to \Gamma_{g,0}$.
Due to \cite{Harer1985}, the mapping class groups $\Gamma_{g,n}$ with $n \ge 1$ are homologically stable.
Including several improvements concerning the degree of stabilization we have
\begin{thm*}[Harer]
    Let $g \ge 0$ and $n \ge 1$.
    The induced map
    \[
        \varphi_\ast \colon H_\ast( \Gamma_{g,n}; \mathbb Z ) \to H_\ast( \Gamma_{g,n+1}; \mathbb Z)
    \]
    is an injection for all $\ast$ and an isomorphism for $\ast \le \frac{2}{3}g$.
    The induced map
    \[
        \psi_\ast \colon H_\ast( \Gamma_{g,n+1}; \mathbb Z ) \to H_\ast( \Gamma_{g+1,n}; \mathbb Z)
    \]
    is a surjection for $\ast \le \frac{2}{3}g + \frac{1}{3}$ and an isomorphism for $\ast \le \frac{2}{3}g - \frac{2}{3}$.
    The induced map
    \[
        \vartheta_\ast \colon H_\ast( \Gamma_{g,1}; \mathbb Z ) \to H_\ast( \Gamma_{g,0}; \mathbb Z)
    \]
    is a surjection for $\ast \le \frac{2}{3}g + 1$ and an isomorphism for $\ast \le \frac{2}{3}g$.
\end{thm*}
\label{page:stabilization_map}%
A proof including the mentioned improvements can be found in \cite{Wahl2012}.

The composition $\psi_g \varphi_g \colon \Gamma_{g,1} \to \Gamma_{g+1,1}$ is injective and $\Gamma_{\infty,1} = \cup_{g=1}^\infty \Gamma_{g,1}$ is the stable mapping class group.
We obviously obtain $\varinjlim H_\ast( \Gamma_{g,1}; \mathbb Z ) \cong H_\ast(\Gamma_{\infty,1})$.
\begin{thm*}[Mumford's Conjecture (Madsen--Weiss \cite{MadenWeiss2007})]
    The rational cohomology of the stable mapping class group is a polynomial algebra
    \[
        H^\ast( \Gamma_{\infty,1}; \mathbb Q ) \cong \mathbb Q[ \kappa_1, \kappa_2, \ldots ]
    \]
    in the Mumford--Morita--Miller classes $\kappa_i$ living in degree $2i$.
\end{thm*}

\paragraph{Unstable Homology}
In contrast to the stable picture, very little is known about the unstable one, i.e., the homology or cohomology of single moduli spaces.
Note that for a class in degree say $2$ to be stable, we have to go to $g \ge 4$.

Before reviewing $\Modspc$, consider the moduli space $\tildeModspc$ of Riemann surfaces of genus $g$ where both the boundary curves and punctures are pointwise fixed.
For single degrees $\ast = 1,2,3$, there are results known for almost all $g$.
Based on the works of Mumford \cite{Mumford1967} and Powell \cite{Powell1978} the first integral homology is known to be
\[
    H_1(\tildeModspc; \mathbb Z) \cong
        \begin{cases}
            \mathbb Z / 10      & g=2 \\
            0                   & g\ge3
        \end{cases} \,.
\]
A proof of this version can be found in Korkmaz--Stipsicz \cite{KorkmazStipsicz2003}.
Moreover, \cite{KorkmazStipsicz2003} improves a theorem by Harer \cite[Theorem 0.a]{Harer1991}:
\[
    H_2(\tildeModspc; \mathbb Z) \cong \mathbb Z^{m+1} \mspc{for}{20} g \ge 4 \,.
\]
The third rational homology vanishes due to \cite[Theorem 0.b]{Harer1991}:
\[
    H_3(\mathfrak{M}_{g,n}^0; \mathbb Q) = 0 \mspc{for}{20} g \ge 6 \,.
\]

In case of no punctures but permutable boundary, the first integral homology is known due to Korkmaz--McCarthy \cite[Theorem 3.13]{KorkmazMcCarthy2000}.
Denoting the corresponding moduli space by $\mathfrak M_{g,(n)}^0$ they show
\[
    H_1( \mathfrak M_{g,(n)}^0; \mathbb Z) \cong
    \begin{cases}
        \mathbb Z / 10                          & g=1, n=0,1 \\
        \mathbb Z / 12 \oplus \mathbb Z/ 2      & g=1, n\ge2 \\
        \mathbb Z / 10                          & g=2, n=0,1 \\
        \mathbb Z / 10 \oplus \mathbb Z/ 2      & g=2, n\ge2 \\
        0                                       & g=3, n=0,1 \\
        \mathbb Z / 2                           & g=3, n\ge2 \\
    \end{cases} \,.
\]

In this thesis we study the moduli space $\Modspc$.
For $g=0$ and $n=1$, the integral homology of the moduli space $\mathfrak{M}_{0,1}^m$ coincides with the well-known group homology of the braid group on $m$ strings.
Besides that, there are some scattered computations for low $g$ and $n$.

\paragraph{Slit models}
In \cite{Boedigheimer19901} Bödigheimer provides the space of parallel slit domains $\Par_{g,n}^m$, which is homeomorphic to an affine bundle over $\Modspc$ via the Hilbert uniformization.
It is a manifold and an open subspace of a finite semi-multisimplicial space $P$ making it possible
(1) to compute the homology of the moduli spaces via Poincaré duality and
(2) to define an operad structure by the action of the little cubes operad on the family of moduli spacess $\Modspc$;
this induces an action of the Dyer-Lashof algebra on their homology.
Exploiting this model, Ehrenfried could completly compute the integral homology for $g=2$ and $n=1$, compare \cite{Ehrenfried1997}.
This is, up to date and apart from $g=0$ and $g=1$, the only moduli space whose integral homology is known. 
His result is reproduced in the following tabl.
\[
    H_\ast( \mathfrak{M}_{2,1}^0; \mathbb Z) \cong
        \begin{cases}
            \mathbb Z                           & \ast = 0\\
            \mathbb Z/10                        & \ast = 1\\
            \mathbb Z/2                         & \ast = 2\\
            \mathbb Z \oplus \mathbb Z/2        & \ast = 3\\
            \mathbb Z/6                         & \ast = 4\\
            0                                   & \ast \ge 5
        \end{cases}
\]
Later, Godin obtained the same results with different methods, compare \cite{Godin2007}.
For $g=3$ and $m=1$, Wang computed the $p$-torsion for many primes in \cite{Wang201102}.
We will describe her results in detail, see below.

We mentioned above a complex $P$ with a subcomplex $P'$ such that $P - P' = \Par$.
The double complex associated with $P$ admits an explicit combinatorial decribtion.
However, the number of cells prevents (even computer-aided) calculations exceeding $h = 5$ where $h = 2g-2+m+2n$.
To demonstrate this, we list the number of cells in bidegree $(p,q)$ for $g=1$ and $m=3$ (see Figure \ref{introduction:number_cells_g_1_m_3_n_1}).
\begin{figure}[ht]
    \centering
    \begin{tabular}{|r||r|r|r|r|r|r|r|}
        \hline
        $q=5$ & 640 & 12425 & 74610 & 202825 & 278600 & 189000 & 50400 \\ \hline
        $q=4$ & 800 & 18500 & 122700 & 357280 & 516880 & 365400 & 100800 \\ \hline
        $q=3$ & 240 & 7425 & 57375 & 185220 & 289380 & 217350 & 63000 \\ \hline
        $q=2$ & 10 & 650 & 6800 & 26600 & 47740 & 39900 & 12600 \\ \hline
        $q=1$ & 0 & 0 & 35 & 315 & 910 & 1050 & 420 \\ \hline \hline
              & $p=4$ & $p=5$ & $p=6$ & $p=7$ & $p=8$ & $p=9$ & $p=10$ \\ \hline
    \end{tabular}
    \caption{\label{introduction:number_cells_g_1_m_3_n_1}The number of cells of the bicomplex for $\mathfrak M_{1,1}^3$.}
\end{figure}
Due to \cite{Visy201011}, the vertical homology of $(P,P')$ is always concentrated in its top row being of degree $q=h$.
The resulting chain complex, called Ehrenfried complex, is considerably smaller, compare Figure \ref{introduction:cells_ehr_g_1_m_3_n_1}.
\begin{figure}[ht]
    \centering
    \begin{tabular}{|r|r|r|r|r|r|r|}
        \hline
        70 & 700 & 2520 & 4480 & 4270 & 2100 & 420 \\ \hline \hline
        $p=4$ & $p=5$ & $p=6$ & $p=7$ & $p=8$ & $p=9$ & $p=10$ \\ \hline
    \end{tabular}
    \caption{\label{introduction:cells_ehr_g_1_m_3_n_1}The number of cells of the Ehrenfried complex for $\mathfrak M_{1,1}^3$.}
\end{figure}
These insights make it possible to perform several computations for $h \le 6$.
In \cite{Wang201102}, Wang computes the elementary divisors modulo $p^{k_p}$ of the differentials in this Ehrenfried complex for
$p^{k_p} = 2^6$, $3^4$, $5^3$, $7^2$, $11^2$, $13^2$, $17$, $19$ and $23$.
Observe that there might be undetected $p$-torsion of the form $\mathbb Z / p^k \mathbb Z$ in case
(1) $p$ a prime greater then $23$ and $k \ge 1$ or
(2) $p$ a prime at most $23$ and $k > k_p$.
Besides $H_0(\Modspc;\mathbb Z) = \mathbb Z$,
we have $H_1(\mathfrak{M}_{3,1}^0;\mathbb Z) = 0$ due to \cite{Powell1978} and $H_2(\mathfrak{M}_{3,1}^0;\mathbb Z) = \mathbb Z \oplus \mathbb Z / 2 \mathbb Z$ due to \cite{Sakasai2012}.
For $2g+m=6$ and $n=1$, the remaining free summands where unkown until this point in time.
Using a new spectral sequene, we provide the free parts by computing the rational homology.
This, in turn, allows for $g=3$ and $n=1$ to conclude, that Wang had indeed discovered all $p$-torsion for $p \le 23$.
\begin{thm*}[Bödigheimer, Powell, Sakasai, Wang, B., H.]
    Let $k_2 = 6$, $k_3 = 4$, $k_5 = 3$, $k_7 = k_{11} = k_{13} = 2$, $k_{17} = k_{19} = k_{23} = 1$ and $k_p = 0$ for $p > 23$ prime.
    The integral homology of the moduli spaces $\mathfrak{M}_{3,1}^0$, $\mathfrak{M}_{2,1}^2$ or $\mathfrak{M}_{1,1}^6$ is given by the following tables,
    where $\bldots$ denotes in the first case possible $p$-torsion for primes $p > 23$, 
    and in the other two cases possible $p$-torsion of the form $\mathbb Z / p^k \mathbb Z$ for $p$ any prime and $k > k_p$.
    \\[2pt]
    \noindent The integral homology of the moduli space $\mathfrak{M}_{3,1}^0$ is
    \[
        H_\ast( \mathfrak{M}_{3,1}^0; \mathbb Z ) \cong 
            \begin{cases}
                \mathbb Z           & \ast = 0\\
                0                   & \ast = 1\\
                \mathbb Z \oplus \mathbb Z / 2                                  & \ast = 2\\
                \mathbb Z \oplus \mathbb Z/2 \oplus \mathbb Z/3 \oplus \mathbb Z/4 \oplus \mathbb Z/7 \oplus \bldots & \ast = 3\\
                (\mathbb Z/2)^2 \oplus (\mathbb Z/3)^2  \oplus \bldots          & \ast = 4\\
                \mathbb Z \oplus \mathbb Z/2 \oplus \mathbb Z/3  \oplus \bldots & \ast = 5\\
                \mathbb Z \oplus (\mathbb Z/2)^3  \oplus \bldots                & \ast = 6\\
                \mathbb Z / 2  \oplus \bldots   & \ast = 7\\
                0  \oplus \bldots               & \ast = 8\\
                \mathbb Z  \oplus \bldots       & \ast = 9\\
                0                               & \ast \ge 10\\
            \end{cases}\,,
    \]
    \\[2pt]
    \noindent The integral homology of $\mathfrak M_{2,1}^2$ is
    \[
        H_\ast( \mathfrak{M}_{2,1}^2; \mathbb Z ) \cong 
        \begin{cases}
            \mathbb Z           & \ast = 0\\
            (\mathbb Z/2)^2 \oplus \mathbb Z/5 \oplus \bldots    & \ast = 1\\
            \mathbb Z \oplus (\mathbb Z/2)^2 \oplus \bldots      & \ast = 2\\
            \mathbb Z^3 \oplus (\mathbb Z/2)^4 \oplus \bldots    & \ast = 3\\
            \mathbb Z \oplus (\mathbb Z/2)^5 \oplus (\mathbb Z/3)^3 \oplus \bldots       & \ast = 4\\
            \mathbb Z^2 \oplus (\mathbb Z/2)^4 \oplus \mathbb Z/3 \oplus \bldots         & \ast = 5\\
            \mathbb Z^2 \oplus (\mathbb Z/2)^3 \oplus \bldots    & \ast = 6\\
            \mathbb Z/2 \oplus \bldots                           & \ast = 7\\
            0                   & \ast \ge 8\\
        \end{cases}
    \]
    
    \noindent The integral homology of $\mathfrak M_{1,1}^4$ is
    \[
        H_\ast( \mathfrak{M}_{1,1}^4; \mathbb Z ) \cong 
        \begin{cases}
            \mathbb Z           & \ast = 0\\
            \mathbb Z \oplus \mathbb Z/2 \oplus \bldots          & \ast = 1\\
            (\mathbb Z/2)^3 \oplus \bldots                       & \ast = 2\\
            \mathbb Z^2 \oplus (\mathbb Z/2)^3 \oplus \bldots    & \ast = 3\\
            \mathbb Z^3 \oplus (\mathbb Z/2)^2 \oplus \bldots    & \ast = 4\\
            \mathbb Z^2 \oplus \mathbb Z/2 \oplus \bldots        & \ast = 5\\
            \mathbb Z \oplus \bldots     & \ast = 6\\
            0                           & \ast \ge 7\\
        \end{cases}
    \]
\end{thm*}
One might conjecture that the undetermined torsion $\bldots$ is trivial in all cases.

In \cite{Mehner201112}, Mehner provides a computer program that computes the integral and $\mathbb F_2$ homology of single moduli spaces for $n=1$, $g \le 2$.
Moreover, he implements simplicial versions of the Dyer-Lashof operations introduced in \cite{Boedigheimer19902} and obtaines some of the generators of the respectively homology via operations.

\section{Our Results in the Unstable Case}
In our thesis, we obtain several new results.
In this section, we discuss the most important ones.

We review Bödigheimer's models introduced in \cite{Boedigheimer19901} and \cite{Boedigheimer2006}.
We discuss the first model, the space of parallel slit domains $\Parr$ sitting in the semi-multisimplicial parallel slit complex $(P,P')$.
As before, we dissect a given surface using the flow lines of distinguished potential functions with exactly $n$ poles $\mc Q = (Q_1, \ldots, Q_n)$.
Here we permit poles of arbitrary order $r_1, \ldots, r_n \ge 1$ and obtain a parallel slit domain on $r = r_1 + \ldots + r_n$ planes.
The second model is the space of radial slit domains $\Rad$ sitting in the radial slit complex $(R,R')$.
These models are manifolds.
Moreover, they are homotopy equivalent to moduli space $\Parr \simeq \Modspc$ respectively $\Rad \simeq \ModspcRad$.
For both models, we construct the associated Ehrenfried complex $\E$ and show that 
(1) the theorem of Bödigheimer (that the Hilbert uniformization provides a homeomorphism)
as well as the theorem of Visy (that the vertical homology of the corresponding double complex is concentrated in degree $h$) hold for both
the parallel slit complex with arbitrary $n$ and $r = r_1 + \ldots + r_n$ and the radial slit complex.
The following diagram shows the schematic picture of our approach.
The homology of the moduli spaces is determined with the help of several models and the lower line represents both the parallel and radial models.
\[
    \resizebox{\linewidth}{!}{
        \begin{tikzcd}[row sep=10ex, scale=1.3, ampersand replacement=\&]
            \&\&\&\&\&H_\ast(\mathfrak M) \\
            B\Gamma \arrow[<-, shorten <=1ex, shorten >=1ex]{rr}{\simeq} \arrow[dashed, out=25, in=180, shorten <=1ex, shorten >=1ex]{rrrrru}[description]{H_\ast} \&\&
            \mathfrak M \arrow[<-, shorten <=1ex, shorten >=1ex]{rr}[swap]{\text{affine bundle}} \arrow[dashed, shorten <=1ex, shorten >=1ex]{rrru}[description]{H_\ast} \&\&
            \mathfrak H \arrow[shorten <=1ex, shorten >=1ex]{rr}{\cong}[swap]{\text{Hilbert uniformization}} \arrow[dashed, shorten <=1ex, shorten >=1ex]{ru}[description]{H_\ast} \&\&
            \stackbin[\mathfrak R]{\mathfrak P}{\scriptscriptstyle{resp.}} \arrow[<-, shorten <=1ex, shorten >=1ex]{rr}[swap]{\text{Poincaré duality}} \arrow[dashed, shorten <=1ex, shorten >=1ex]{lu}[description]{H_\ast} \&\&
            \stackbin[(R,R')]{(P,P')}{\scriptscriptstyle{resp.}} \arrow[<-, shorten <=1ex, shorten >=1ex]{rr}{\simeq}[swap]{\text{quasi-isomorphic}} \arrow[dashed, shorten <=1ex, shorten >=1ex]{lllu}[description]{H^{\ast-\ldots}} \&\&
            \E \arrow[dashed, out=155, in=0, shorten <=1ex, shorten >=1ex]{lllllu}[description]{H^{\ast-\ldots}}
        \end{tikzcd}
    }
\]
All in all we have:
\begin{thm*}[Bödigheimer, Visy, B., H.]
    The parallel slit complex respectively the radial slit complex is a relative manifold of dimension $6g-6+3m+3n+3r$ respectively $6g-6+3m+4n$.
    The Ehrenfried complex is a quasi-isomorphic direct summand%
    \footnote{%
        To be precise, the Ehrenfried complex is, up to a shift in the homological degree, identified with a direct summand.
        The inclusion induces an isomorphism in homology.
    } of $P/P'$ respectively $R/R'$.
    In particular
    \[
        H_\ast(\Modspc;\Z) \cong H^{3h-\ast}(P,P'; \mathcal O) \cong H^{2h-\ast}(\E; \mathcal O)
    \]
    where $h = 2g-2+m+n+r$ and $\mathcal O$ are the orientation coefficients respectively
    \[
        H_\ast(\ModspcRad;\Z) \cong H^{3h+n-\ast}(R,R'; \mathcal O) \cong H^{2h+n-\ast}(\E; \mathcal O)
    \]
    where $h=2g-2+m+n$ and $\mathcal O$ are the orientation coefficients.
\end{thm*}

In \cite{Boedigheimer201314}, Bödigheimer introduces a filtration of the bicomplex $\PP = P/P'$ respectively $R/R'$.
It is, roughly speaking, given by the number of components of the critical graph associated with the gradient flow of the given potential function.
It induces a filtration of the Ehrenfried complex.
\begin{prop*}[Bödigheimer]
    There are two first quadrant spectral sequences
    \[
        E^0_{k,c}(\PP) = \bigoplus_{p+q=k}\left[F_c\PP_{p,q} / F_{c-1}\PP_{p,q} \right] \Rightarrow H_{k+c}( \PP_{\bullet, \bullet} )
    \]
    and
    \[
        E^0_{p,c}(\E) = F_c\E_p / F_{c-1}\E_p \Rightarrow H_{p+c}( \E_\bullet ) \,.
    \]
    Both spectral sequences collapse at the second page.
\end{prop*}

Implementing the spectral sequence for the Ehrenfried complex in a software project we compute the rational and some $\mathbb F_p$ homology of certain moduli spaces with $h \le 8$.
A short form of the rational results can be found in Section \ref{introduction:more_rational_homology} and
the complete description is presented in Section \ref{program:results}.
In particular, we confirm the rational version of Wang's conjecture.
\begin{thm*}[Bödigheimer, B., H.]
    The rational homology of the moduli space of Riemann surfaces of genus three with one boundary component is
    \[
        \begin{tabular}{|r|r|r|r|r|r|r|r|r|r|r|}
            \hline
            \multicolumn{11}{|c|}{$H_p( \mathfrak{M}_{3,1}^0; \mathbb Q )$} \\ \hline
            $p=0$&$p=1$&$p=2$&$p=3$&$p=4$&$p=5$&$p=6$&$p=7$&$p=8$&$p=9$&$p\ge10$\\ \hline \hline
            $\mathbb Q$&$0$&$\mathbb Q$&$\mathbb Q$&$0$&$\mathbb Q$&$\mathbb Q$&$0$&$0$&$\mathbb Q$&$0$\\ \hline
        \end{tabular}\,.
    \]
    The rational homology of the moduli space of Riemann surfaces of genus two with one boundary component and two permutable punctures is
    \[
        \begin{tabular}{|r|r|r|r|r|r|r|r|r|}
            \hline
            \multicolumn{9}{|c|}{$H_p( \mathfrak{M}_{2,1}^2; \mathbb Q )$} \\ \hline
            $p=0$&$p=1$&$p=2$&$p=3$&$p=4$&$p=5$&$p=6$&$p=7$&$p\ge8$\\ \hline \hline
            $\mathbb Q$&$0$&$\mathbb Q$&$\mathbb Q^3$&$\mathbb Q$&$\mathbb Q^2$&$\mathbb Q^2$&$0$&$0$\\ \hline
        \end{tabular}\,.
    \]
    The rational homology of the moduli space of Riemann surfaces of genus one with one boundary component and four permutable punctures is
    \[
        \begin{tabular}{|r|r|r|r|r|r|r|r|}
            \hline
            \multicolumn{8}{|c|}{$H_p( \mathfrak{M}_{1,1}^4; \mathbb Q )$} \\ \hline
            $p=0$&$p=1$&$p=2$&$p=3$&$p=4$&$p=5$&$p=6$&$p\ge7$\\ \hline \hline
            $\mathbb Q$&$\mathbb Q$&$0$&$\mathbb Q^2$&$\mathbb Q^3$&$\mathbb Q^2$&$\mathbb Q$&$0$\\ \hline
        \end{tabular}\,.
    \]
\end{thm*}

Most of the well-known homology operations on the moduli spaces were constructed via the bicomplexes (see below).
In order to realize them in terms of the dual Ehrenfried complex, we provide an explicit formula for the coboundary operator via so-called coboundary traces:
\begin{prop*}[B., H.]
    The coboundary of a cell $\Sigma \in \E$ of degree $p$ is
    \[
        \del_\E^\ast(\Sigma) = \sum_{i=1}^p (-1)^i \sum_{a \in T_i(\Sigma)} \kappa^\ast( a.\Sigma ) \,.
    \]
\end{prop*}

Using this formula, we discuss some of the well-known homology operations.
Moreover, we classify the cells of a given Ehrenfried complex.
\begin{prop*}[B., H.]
    Every cell in the Ehrenfried complex $\E$ is uniquely obtained as an expansion of a thin cell in $\E$.
\end{prop*}

Bödigheimer's models have a strong connection to configuration spaces.
Roughly speaking, a parallel slit domain $L \in \Parr$ consists of $r = r_1 + \ldots + r_n$ copies of the complex plane with finitely many slits removed,
each slit running from some point horizontally to the left all the way to infinity.
There is a pairing of the slits, subject to several conditions.
It is reasonable to think that the pairing enables us to jump through a given slit to end up at its partner.
The description of a radial slit domain $L \in \Rad$ is similar.
Here we consider paired slits on an annulus each running from some point radially to the outer boundary.
There are various geometric flavoured constructions.
\begin{prop*}[Bödigheimer]
    For every $g \ge 0$, $n \ge 1$, $m \ge 1$ and partition $(r_1, \ldots, r_n)$ of $r = r_1 + \ldots + r_n$,
    there are continous maps
    \[
        par \colon \Rad \to \Parr
    \]
    and
    \[
        rad \colon \Parr \to \Rad \,.
    \]
    The maps are indicated in the following by Figures \ref{intro:parallelization} and \ref{intro:radialization}.
\end{prop*}
\begin{figure}[ht]
    \centering
    \incgfx{pictures/intro_parallelization.pdf}
    \caption{\label{intro:parallelization}The parallelization map with $n=1$ and $r = 3$.}
\end{figure}
\begin{figure}[ht]
    \centering
    \incgfx{pictures/intro_radialization.pdf}
    \caption{\label{intro:radialization}The radialization map.}
\end{figure}
We discuss several homology operations.
One family of operations is induced by the action of little cubes operads, namely
the ordered configuration spaces with respect to the complex plane $\cspc k (\C)$ or the annulus $\cspc k (\A)$, compare \cite{Boedigheimer19902} and \cite{Boedigheimer2006}.
We propose a generalization of the well-known operations on $\Par_{g,n}^m[(1, \ldots, 1)]$ to $\Parr$ for an arbitrary partition $(r_1, \ldots, r_n)$.
There are many generalizations which are all covered by our glueing construction.
Roughly speaking, one has to decide how two surfaces, corresponding to given parallel slit domains $L_1$ and $L_2$, are glued along parts of their boundary and
one has to declare an enumeration of the resulting boundaries.
\begin{defi*}[B., H.]
    The {\bf combinatorial type} $G$ which specifies the glueing construction {\bf depends on} the parameters
    \[
        \mathfrak P(G) = (g_1, g_2, n_1, n_2, m_1, m_2, (r_1^{(1)}, \ldots, r_{n_1}^{(1)}), (r_1^{(2)}, \ldots, r_{n_2}^{(2)}))
    \]
    and {\bf consists of} the following two data.
    \begin{enumerate}
        \item A partial, non-empty matching of the planes of the parallel slit domains in $\Par_{g_1, n_1}^{m_1}(r_1^{(1)}, \ldots, r_{n_1}^{(1)})$ and $\Par_{g_2, n_2}^{m_2}(r_1^{(2)}, \ldots, r_{n_2}^{(2)})$.
    \end{enumerate}
    The size of the matching is denoted by $s(G)$.
    The glueing construction defines a surface of genus $g(G)$ with $m(G) = m_1 + m_2$ punctures and $n(G)$ (yet unordered) boundary curves each consisting of several planes.
    \begin{enumerate}
        \setcounter{enumi}{1}
        \item A partial enumeration of the planes such that each boundary curve belongs to exactly one selected planes.
    \end{enumerate}
    The corresponding ordered configuration is $(r^{(G)}_1, \ldots, r^{(G)}_{n(G)})$.
    The {\bf set of combinatorial types} that specify a glueing construction is denoted by $\mc G$.
\end{defi*}
\begin{prop*}[Bödigheimer]
    For every combinatorial type $G \in \mc G$ with parameters
    \[
        \mathfrak P(G) = (g_1, g_2, n_1, n_2, m_1, m_2, (r_1^{(1)}, \ldots, r_{n_1}^{(1)}), (r_1^{(2)}, \ldots, r_{n_2}^{(2)}))
    \]
    there are homology operations induced by the action of the little cubes operad
    \begin{multline*}
        (\tilde\vartheta_G)_\ast \colon 
            H_i(\cspc 2(\C))^{\oplus s(G)} \otimes
            H_j(\Par_{g_1, n_1}^{m_1}(r_1^{(1)}, \ldots, r_{n_1}^{(1)})) \otimes
            H_k(\Par_{g_2, n_2}^{m_2}(r_1^{(2)}, \ldots, r_{n_2}^{(2)})) \to\\[10pt]
            H_{i+j+k}(\Par_{g(G), n(G)}^{m(G)}(r^{(G)}_1, \ldots, r^{(G)}_{n(G)}))\,.
    \end{multline*}
\end{prop*}

In addition to the parallelization and radialization map mentioned above,
we have the following propositions relating the space of parallel slit domains to the space of radial slit domains.
\begin{prop*}[Bödigheimer]
    The action of the little cubes operad on the space of parallel slit domains extends to an operation
    \[
        \begin{tikzcd}
            \tilde C^k(\A) \times \mathfrak{Par}_{g_1, 1}^{m_1} \times \dotsb \times \mathfrak{Par}_{g_k, 1}^{m_k} \arrow{r}{\tilde \vartheta} & \Radt_{\tilde g}(\tilde m + 1, 1) \\
            \tilde C^k(\C) \times \mathfrak{Par}_{g_1, 1}^{m_1} \times \dotsb \times \mathfrak{Par}_{g_k, 1}^{m_k} \arrow{r}{\tilde \vartheta} \arrow[hookrightarrow]{u}{\iota \times \text{id}} & \mathfrak{Par}_{\tilde g, 1}^{\tilde m} \arrow[hookrightarrow]{u}{rad}
        \end{tikzcd} \,,
   \]
   where $\tilde g = \sum_{i = 1}^{k} g_i$ and $\tilde m = \sum_{i = 1}^k m_i$.
\end{prop*}
\begin{prop*}[Bödigheimer]
    Let $n \ge 1$ and $\Par_n = \coprod_{g,m}\Par_{g,n}^m[(1, \ldots, 1)]$ and $\mathfrak{Rad}_n = \coprod_{g,m}\Rad$.
    There is a right module structure
    \[
        H_\ast(\mathfrak{Rad}_n) \otimes H_\ast(\Par_n) \to H_\ast(\mathfrak{Rad}_n)
    \]
    induced by an action of the little cubes operad.
\end{prop*}
\begin{prop*}[Bödigheimer]
    There is a composition operation
    \[
        \odot \colon \mathfrak{M}^{\bullet \bullet}_g(l, m) \times \mathfrak{M}^{\bullet \bullet}_{g'}(m, n) \to \mathfrak{M}^{\bullet \bullet}_{\tilde g}(l, n)\,, (F, F') \mapsto F \odot F'\,,
    \]
    where $\tilde g = g + g' + m - 1$.
\end{prop*}

Besides operations which are induced by the action of little cubes operads,
we generalize the operations discussed in \cite{Mehner201112} to arbitrary $n$ and $(r_1, \ldots, r_n)$, present the radial multipliciation
\[
    \radmult \colon \Rad \times \Radt_{g'}(m', n) \to \Radt_{\tilde g}(m+m', n)\,,
\]
with $\tilde g = g + g' + n - 1$ and introduce $\alpha \colon \mathfrak{M}_{g,n}^m \to \mathfrak{M}_{g,n}^{m+n}$
inducing a split injective map in homology
\[
    \alpha_\ast \colon H_\ast(\mathfrak{M}_{g,n}^m) \to H_\ast(\mathfrak{M}_{g,n}^{m+n}) \,.
\]
Rotating radial slit domains simultaneously induces the operation of degree one
\[
    rot \colon H_i(\Rad) \to H_{i+1}(\Rad)
\]
with $rot^2 = 0$.
Eventually, we present formulas relating some of the operations.

Furthermore, we provide an ongoing, extendable software project consisting of about 4500 lines of code.
It was used to compute the homology of the moduli spaces for $h \le 8$ and
the first author plans to implement the known operations in order to relate the found generators.
In addition, there are upcoming master students under the supervision of Bödigheimer who will implement further features of both the Ehrenfried complex and its corresponding bicomplex.

A deeper study of the cluster spectral sequence, the relations of generators via homology operations and the interdependencies of these operations outline an ongoing research project.
\section{The Rational Homology of the Moduli Spaces in Short Form}
\label{introduction:more_rational_homology}
In this section, we present a short form of our computations with coefficients in the rationals.
Some of the results were already known, compare the discussion above.
We include them anyways.
The number of boundary components is always $n=1$.
All cluster spectral sequences with coefficients in $\mathbb Q$ and $\mathbb F_2$ are found in Section \ref{program:results}.

\paragraph{The case \texorpdfstring{$g=0$}{g=0}:}
The moduli space $\mathfrak M_{0,1}^m$ is the classifying space of the braid group on $m$ strings.
Its homology is understood.
We have
\[
    H_\ast(\mathfrak M_{0,1}^m; \mathbb Q) =
        \begin{cases}
            \mathbb Q   & \ast = 0,1 \; and\; \ast \le m \\
            0           & else
        \end{cases}
\]

\paragraph{The case \texorpdfstring{$g=1$}{g=1}:}
For $m = 0,1,2,3,4,5$, the rational homology of $\mathfrak M_{1,1}^m$ is given by the following tables.
\begin{center}
    \begin{tabular}{|r|r|r|}
        \hline
        \multicolumn{3}{|c|}{$H_p( \mathfrak{M}_{1,1}^0; \mathbb Q )$} \\ \hline
        $p=0$&$p=1$&$p\ge3$\\ \hline \hline
        $\mathbb Q$&$\mathbb Q$&$0$\\ \hline
    \end{tabular}
    
    \vspace{2.5ex}
    
    \begin{tabular}{|r|r|r|}
        \hline
        \multicolumn{3}{|c|}{$H_p( \mathfrak{M}_{1,1}^1; \mathbb Q )$} \\ \hline
        $p=0$&$p=1$&$p\ge3$\\ \hline \hline
        $\mathbb Q$&$\mathbb Q$&$0$\\ \hline
    \end{tabular}
    
    \vspace{2.5ex}
    
    \begin{tabular}{|r|r|r|}
        \hline
        \multicolumn{3}{|c|}{$H_p( \mathfrak{M}_{1,1}^2; \mathbb Q )$} \\ \hline
        $p=0$&$p=1$&$p\ge3$\\ \hline \hline
        $\mathbb Q$&$\mathbb Q$&$0$\\ \hline
    \end{tabular}
    
    \vspace{2.5ex}
    
    \begin{tabular}{|r|r|r|r|r|r|r|}
        \hline
        \multicolumn{7}{|c|}{$H_p( \mathfrak{M}_{1,1}^3; \mathbb Q )$} \\ \hline
        $p=0$&$p=1$&$p=2$&$p=3$&$p=4$&$p=5$&$p\ge6$\\ \hline \hline
        $\mathbb Q$&$\mathbb Q$&$0$&$\mathbb Q$&$\mathbb Q$&$\mathbb Q^2$&$0$\\ \hline
    \end{tabular}
    
    \vspace{2.5ex}
    
    \begin{tabular}{|r|r|r|r|r|r|r|r|}
        \hline
        \multicolumn{8}{|c|}{$H_p( \mathfrak{M}_{1,1}^4; \mathbb Q )$} \\ \hline
        $p=0$&$p=1$&$p=2$&$p=3$&$p=4$&$p=5$&$p=6$&$p\ge7$\\ \hline \hline
        $\mathbb Q$&$\mathbb Q$&$0$&$\mathbb Q^2$&$\mathbb Q^3$&$\mathbb Q^2$&$\mathbb Q$&$0$\\ \hline
    \end{tabular}
    
    \vspace{2.5ex}
    
    \begin{tabular}{|r|r|r|r|r|r|r|r|r|r|}
        \hline
        \multicolumn{10}{|c|}{$H_p( \mathfrak{M}_{1,1}^5; \mathbb Q )$} \\ \hline
        $p=0$&$p=1$&$p=2$&$p=3$&$p=4$&$p=5$&$p=6$&$p=7$&$p=8$&$p\ge9$\\ \hline \hline
        $\mathbb Q$&$?$&$?$&$?$&$?$&$?$&$?$&$\mathbb Q^4$&$\mathbb Q^2$&$0$\\ \hline
    \end{tabular}
\end{center}

\paragraph{The case \texorpdfstring{$g=2$}{g=2}:}
For $m = 0,1,2$, the rational homology of $\mathfrak M_{2,1}^m$ is given by the following tables.
\begin{center}
    \begin{tabular}{|r|r|r|r|r|}
        \hline
        \multicolumn{5}{|c|}{$H_p( \mathfrak{M}_{2,1}^0; \mathbb Q )$} \\ \hline
        $p=0$&$p=1$&$p=2$&$p=3$&$p\ge4$\\ \hline \hline
        $\mathbb Q$&$0$&$0$&$\mathbb Q$&$0$\\ \hline
    \end{tabular}
    
    \vspace{2.5ex}
    
    \begin{tabular}{|r|r|r|r|r|r|r|r|}
        \hline
        \multicolumn{8}{|c|}{$H_p( \mathfrak{M}_{2,1}^1; \mathbb Q )$} \\ \hline
        $p=0$&$p=1$&$p=2$&$p=3$&$p=4$&$p=5$&$p=6$&$p\ge7$\\ \hline \hline
        $\mathbb Q$&$0$&$\mathbb Q$&$\mathbb Q^2$&$0$&$\mathbb Q$&$\mathbb Q$&$0$\\ \hline
    \end{tabular}
    
    \vspace{2.5ex}
    
    \begin{tabular}{|r|r|r|r|r|r|r|r|}
        \hline
        \multicolumn{8}{|c|}{$H_p( \mathfrak{M}_{2,1}^2; \mathbb Q )$} \\ \hline
        $p=0$&$p=1$&$p=2$&$p=3$&$p=4$&$p=5$&$p=6$&$p\ge7$\\ \hline \hline
        $\mathbb Q$&$0$&$\mathbb Q$&$\mathbb Q^3$&$\mathbb Q$&$\mathbb Q^2$&$\mathbb Q^2$&$0$\\ \hline
    \end{tabular}
\end{center}

\paragraph{The case \texorpdfstring{$g=3$}{g=3}:}
The rational homology of $\mathfrak M_{3,1}^0$ is given by the following table.
\begin{center}
    \begin{tabular}{|r|r|r|r|r|r|r|r|r|r|r|}
        \hline
        \multicolumn{11}{|c|}{$H_p( \mathfrak{M}_{3,1}^0; \mathbb Q )$} \\ \hline
        $p=0$&$p=1$&$p=2$&$p=3$&$p=4$&$p=5$&$p=6$&$p=7$&$p=8$&$p=9$&$p\ge10$\\ \hline \hline
        $\mathbb Q$&$0$&$\mathbb Q$&$\mathbb Q$&$0$&$\mathbb Q$&$\mathbb Q$&$0$&$0$&$\mathbb Q$&$0$\\ \hline
    \end{tabular}
\end{center}

\section{Organization of our Thesis}

Let us sketch the organization of the content of this thesis. The {\bf first chapter} is this introduction.

The {\bf second chapter} provides a detailed description of our models.
Section \ref{cellular_models:introduction} serves as an overview of our approach.
The details are carried out in the subsequent Sections \ref{cellular_models:from_moduli_spaces_to_parallel_slit_domains}-\ref{cellular_models:ehrenfried}.
Having this done, one has all ingredients to make sense of the following diagram.
\[
    \resizebox{\linewidth}{!}{
        \begin{tikzcd}[row sep=7.5ex, scale=1.3, ampersand replacement=\&]
            \&\&\&\&\&H_\ast(\mathfrak M) \\
            B\Gamma \arrow[<-, shorten <=1ex, shorten >=1ex]{rr}{\simeq} \arrow[dashed, out=25, in=180, shorten <=1ex, shorten >=1ex]{rrrrru}[description]{H_\ast} \&\&
            \mathfrak M \arrow[<-, shorten <=1ex, shorten >=1ex]{rr}[swap]{\text{affine bundle}} \arrow[dashed, shorten <=1ex, shorten >=1ex]{rrru}[description]{H_\ast} \&\&
            \mathfrak H \arrow[shorten <=1ex, shorten >=1ex]{rr}{\cong}[swap]{\text{Hilbert uniformization}} \arrow[dashed, shorten <=1ex, shorten >=1ex]{ru}[description]{H_\ast} \&\&
            \stackbin[\mathfrak R]{\mathfrak P}{\scriptscriptstyle{resp.}} \arrow[<-, shorten <=1ex, shorten >=1ex]{rr}[swap]{\text{Poincaré duality}} \arrow[dashed, shorten <=1ex, shorten >=1ex]{lu}[description]{H_\ast} \&\&
            \stackbin[(R,R')]{(P,P')}{\scriptscriptstyle{resp.}} \arrow[<-, shorten <=1ex, shorten >=1ex]{rr}{\simeq}[swap]{\text{quasi-isomorphic}} \arrow[dashed, shorten <=1ex, shorten >=1ex]{lllu}[description]{H^{\ast-\ldots}} \&\&
            \E \arrow[dashed, out=155, in=0, shorten <=1ex, shorten >=1ex]{lllllu}[description]{H^{\ast-\ldots}}
        \end{tikzcd}
    }
\]
The rightmost model is the Ehrenfried complex.
It is a finite chain complex and its differentials admits an explicit description.
The homology of the moduli spaces is computed via its dual.
In Section \ref{cellular_models:dual_ehrenfried}, we provide an explicit formula for its coboundary operator.
Hereby, we begin with an explanation of our geometric intuition in order to make the upcoming definitions, statements and proofs straightforward.

The {\bf third chapter} is a brief introduction to the cluster spectral sequence.
We introduce the cluster filtration of the bicomplex and the Ehrenfried complex and show that the associated spectral sequences collapse at the second page.

The {\bf fourth chapter} covers various homology operations.
Sections \ref{homology_operations:parallel_patching_slit_pics}-\ref{radial_composition} describe operations defined either for $\Par$ or $\mathfrak{Rad}$.
Operations and formulas relating $\Par$ and $\mathfrak{Rad}$ are discussed in Section \ref{homology_operations:comparision_of_par_and_rad}.
In Section \ref{homology_operations:parallel_patching_slit_pics}, we review well-known operations on $\Par$ via little cubes operads and propose a generalization in Section \ref{homology_operations:glueing_construction}.
Operations on $\Par$ which are induced by bundle maps are discussed in Section \ref{homology_operations:operations_par_via_bundles}.
In Section \ref{homology_operations:alpha}, we present the operations $\alpha$.
The radial multiplication is treated in Section \ref{homology_operations:radial_multiplication} and
the composition of two radial slit domains is reviewed in Section \ref{radial_composition}.
The rotation of radial slit domains in introduced in Section \ref{homology_operations:rotation}.

The {\bf fifth chapter} is a brief analysis of the computational complexity of homology calculations via the Ehrenfried complex.
We compute the number of its cells and discuss nearby algorithms used to derive homological data.

The {\bf sixth chapter} provides the documentation of our software project and lists all cluster spectral sequences we computed.

The {\bf Appendix} reviews possibly unkown notation.
\section*{Acknowledgements}

First and foremost, we thank Professor Bödigheimer for endless hours of inspiring discussions and guidance during the completion of our Master's Thesis.
We thank Professor Vygen from the Insititute for Discrete Mathematics and Professor Griebel from the Institute for Numerical Simulation 
for the opportunity to run our computer program on their highly efficient machines.
Furthermore, we thank a lot of people for proof-reading or for their support with carrying out our computer program:
Linus Boes, Mathias Gerdes, Christian Hemminghaus, Johannes Holke, Nils Hoppmann, David Hornshaw, Philipp Ochsendorf, Emanuel Reinecke, Max Schmidt, Jannik Silvanus, Raffael Stenzel and Peter Zaspel.
Our deepest thank goes to our families who supported us warmly during our studies.

\section*{Online Version}
A PDF version of this thesis and the computer program can found at
\begin{center}
    \url{http://felixboes.de}.
\end{center}
