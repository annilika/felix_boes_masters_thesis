\section{Our Results in the Unstable Case}
In our thesis, we obtain several new results.
In this section, we discuss the most important ones.

We review Bödigheimer's models introduced in \cite{Boedigheimer19901} and \cite{Boedigheimer2006}.
We discuss the first model, the space of parallel slit domains $\Parr$ sitting in the semi-multisimplicial parallel slit complex $(P,P')$.
As before, we dissect a given surface using the flow lines of distinguished potential functions with exactly $n$ poles $\mc Q = (Q_1, \ldots, Q_n)$.
Here we permit poles of arbitrary order $r_1, \ldots, r_n \ge 1$ and obtain a parallel slit domain on $r = r_1 + \ldots + r_n$ planes.
The second model is the space of radial slit domains $\Rad$ sitting in the radial slit complex $(R,R')$.
These models are manifolds.
Moreover, they are homotopy equivalent to moduli space $\Parr \simeq \Modspc$ respectively $\Rad \simeq \ModspcRad$.
For both models, we construct the associated Ehrenfried complex $\E$ and show that 
(1) the theorem of Bödigheimer (that the Hilbert uniformization provides a homeomorphism)
as well as the theorem of Visy (that the vertical homology of the corresponding double complex is concentrated in degree $h$) hold for both
the parallel slit complex with arbitrary $n$ and $r = r_1 + \ldots + r_n$ and the radial slit complex.
The following diagram shows the schematic picture of our approach.
The homology of the moduli spaces is determined with the help of several models and the lower line represents both the parallel and radial models.
\[
    \resizebox{\linewidth}{!}{
        \begin{tikzcd}[row sep=10ex, scale=1.3, ampersand replacement=\&]
            \&\&\&\&\&H_\ast(\mathfrak M) \\
            B\Gamma \arrow[<-, shorten <=1ex, shorten >=1ex]{rr}{\simeq} \arrow[dashed, out=25, in=180, shorten <=1ex, shorten >=1ex]{rrrrru}[description]{H_\ast} \&\&
            \mathfrak M \arrow[<-, shorten <=1ex, shorten >=1ex]{rr}[swap]{\text{affine bundle}} \arrow[dashed, shorten <=1ex, shorten >=1ex]{rrru}[description]{H_\ast} \&\&
            \mathfrak H \arrow[shorten <=1ex, shorten >=1ex]{rr}{\cong}[swap]{\text{Hilbert uniformization}} \arrow[dashed, shorten <=1ex, shorten >=1ex]{ru}[description]{H_\ast} \&\&
            \stackbin[\mathfrak R]{\mathfrak P}{\scriptscriptstyle{resp.}} \arrow[<-, shorten <=1ex, shorten >=1ex]{rr}[swap]{\text{Poincaré duality}} \arrow[dashed, shorten <=1ex, shorten >=1ex]{lu}[description]{H_\ast} \&\&
            \stackbin[(R,R')]{(P,P')}{\scriptscriptstyle{resp.}} \arrow[<-, shorten <=1ex, shorten >=1ex]{rr}{\simeq}[swap]{\text{quasi-isomorphic}} \arrow[dashed, shorten <=1ex, shorten >=1ex]{lllu}[description]{H^{\ldots-\ast}} \&\&
            \E \arrow[dashed, out=155, in=0, shorten <=1ex, shorten >=1ex]{lllllu}[description]{H^{\ldots-\ast}}
        \end{tikzcd}
    }
\]
All in all we have:
\begin{thm*}[Bödigheimer, Visy, B., H.]
    The parallel slit complex respectively the radial slit complex is a relative manifold of dimension $6g-6+3m+3n+3r$ respectively $6g-6+3m+4n$.
    The Ehrenfried complex is a quasi-isomorphic direct summand%
    \footnote{%
        To be precise, the Ehrenfried complex is, up to a shift in the homological degree, identified with a direct summand.
        The inclusion induces an isomorphism in homology.
    } of $P/P'$ respectively $R/R'$.
    In particular
    \[
        H_\ast(\Modspc;\Z) \cong H^{3h-\ast}(P,P'; \mathcal O) \cong H^{2h-\ast}(\E; \mathcal O)
    \]
    where $h = 2g-2+m+n+r$ and $\mathcal O$ are the orientation coefficients respectively
    \[
        H_\ast(\ModspcRad;\Z) \cong H^{3h+n-\ast}(R,R'; \mathcal O) \cong H^{2h+n-\ast}(\E; \mathcal O)
    \]
    where $h=2g-2+m+n$ and $\mathcal O$ are the orientation coefficients.
\end{thm*}

In \cite{Boedigheimer201314}, Bödigheimer introduces a filtration of the bicomplex $\PP = P/P'$ respectively $R/R'$.
It is, roughly speaking, given by the number of components of the critical graph associated with the gradient flow of the given potential function.
It induces a filtration of the Ehrenfried complex.
\begin{prop*}[Bödigheimer]
    There are two first quadrant spectral sequences
    \[
        E^0_{k,c}(\PP) = \bigoplus_{p+q=k}\left[F_c\PP_{p,q} / F_{c-1}\PP_{p,q} \right] \Rightarrow H_{k+c}( \PP_{\bullet, \bullet} )
    \]
    and
    \[
        E^0_{p,c}(\E) = F_c\E_p / F_{c-1}\E_p \Rightarrow H_{p+c}( \E_\bullet ) \,.
    \]
    Both spectral sequences collapse at the second page.
\end{prop*}

Implementing the spectral sequence for the Ehrenfried complex in a software project we compute the rational and some $\mathbb F_p$ homology of certain moduli spaces with $h \le 8$.
A short form of the rational results can be found in Section \ref{introduction:more_rational_homology} and
the complete description is presented in Section \ref{program:results}.
In particular, we confirm the rational version of Wang's conjecture.
\begin{thm*}[Bödigheimer, B., H.]
    The rational homology of the moduli space of Riemann surfaces of genus three with one boundary component is
    \[
        \begin{tabular}{|r|r|r|r|r|r|r|r|r|r|r|}
            \hline
            \multicolumn{11}{|c|}{$H_p( \mathfrak{M}_{3,1}^0; \mathbb Q )$} \\ \hline
            $p=0$&$p=1$&$p=2$&$p=3$&$p=4$&$p=5$&$p=6$&$p=7$&$p=8$&$p=9$&$p\ge10$\\ \hline \hline
            $\mathbb Q$&$0$&$\mathbb Q$&$\mathbb Q$&$0$&$\mathbb Q$&$\mathbb Q$&$0$&$0$&$\mathbb Q$&$0$\\ \hline
        \end{tabular}\,.
    \]
    The rational homology of the moduli space of Riemann surfaces of genus two with one boundary component and two permutable punctures is
    \[
        \begin{tabular}{|r|r|r|r|r|r|r|r|r|}
            \hline
            \multicolumn{9}{|c|}{$H_p( \mathfrak{M}_{2,1}^2; \mathbb Q )$} \\ \hline
            $p=0$&$p=1$&$p=2$&$p=3$&$p=4$&$p=5$&$p=6$&$p=7$&$p\ge8$\\ \hline \hline
            $\mathbb Q$&$0$&$\mathbb Q$&$\mathbb Q^3$&$\mathbb Q$&$\mathbb Q^2$&$\mathbb Q^2$&$0$&$0$\\ \hline
        \end{tabular}\,.
    \]
    The rational homology of the moduli space of Riemann surfaces of genus one with one boundary component and four permutable punctures is
    \[
        \begin{tabular}{|r|r|r|r|r|r|r|r|}
            \hline
            \multicolumn{8}{|c|}{$H_p( \mathfrak{M}_{1,1}^4; \mathbb Q )$} \\ \hline
            $p=0$&$p=1$&$p=2$&$p=3$&$p=4$&$p=5$&$p=6$&$p\ge7$\\ \hline \hline
            $\mathbb Q$&$\mathbb Q$&$0$&$\mathbb Q^2$&$\mathbb Q^3$&$\mathbb Q^2$&$\mathbb Q$&$0$\\ \hline
        \end{tabular}\,.
    \]
\end{thm*}

Most of the well-known homology operations on the moduli spaces were constructed via the bicomplexes (see below).
In order to realize them in terms of the dual Ehrenfried complex, we provide an explicit formula for the coboundary operator via so-called coboundary traces:
\begin{prop*}[B., H.]
    The coboundary of a cell $\Sigma \in \E$ of degree $p$ is
    \[
        \del_\E^\ast(\Sigma) = \sum_{i=1}^p (-1)^i \sum_{a \in T_i(\Sigma)} \kappa^\ast( a.\Sigma ) \,.
    \]
\end{prop*}

Using this formula, we discuss some of the well-known homology operations.
Moreover, we classify the cells of a given Ehrenfried complex.
\begin{prop*}[B., H.]
    Every cell in the Ehrenfried complex $\E$ is uniquely obtained as an expansion of a thin cell in $\E$.
\end{prop*}

Bödigheimer's models have a strong connection to configuration spaces.
Roughly speaking, a parallel slit domain $L \in \Parr$ consists of $r = r_1 + \ldots + r_n$ copies of the complex plane with finitely many slits removed,
each slit running from some point horizontally to the left all the way to infinity.
There is a pairing of the slits, subject to several conditions.
It is reasonable to think that the pairing enables us to jump through a given slit to end up at its partner.
The description of a radial slit domain $L \in \Rad$ is similar.
Here we consider paired slits on an annulus each running from some point radially to the outer boundary.
There are various geometric flavoured constructions.
\begin{prop*}[Bödigheimer]
    For every $g \ge 0$, $n \ge 1$, $m \ge 1$ and partition $(r_1, \ldots, r_n)$ of $r = r_1 + \ldots + r_n$,
    there are continous maps
    \[
        par \colon \Rad \to \Parr
    \]
    and
    \[
        rad \colon \Parr \to \Rad \,.
    \]
    The maps are indicated in the following by Figures \ref{intro:parallelization} and \ref{intro:radialization}.
\end{prop*}
\begin{figure}[ht]
    \centering
    \incgfx{pictures/intro_parallelization.pdf}
    \caption{\label{intro:parallelization}The parallelization map with $n=1$ and $r = 3$.}
\end{figure}
\begin{figure}[ht]
    \centering
    \incgfx{pictures/intro_radialization.pdf}
    \caption{\label{intro:radialization}The radialization map.}
\end{figure}
We discuss several homology operations.
One family of operations is induced by the action of little cubes operads, namely
the ordered configuration spaces with respect to the complex plane $\cspc k (\C)$ or the annulus $\cspc k (\A)$, compare \cite{Boedigheimer19902} and \cite{Boedigheimer2006}.
We propose a generalization of the well-known operations on $\Par_{g,n}^m[(1, \ldots, 1)]$ to $\Parr$ for an arbitrary partition $(r_1, \ldots, r_n)$.
There are many generalizations which are all covered by our glueing construction.
Roughly speaking, one has to decide how two surfaces, corresponding to given parallel slit domains $L_1$ and $L_2$, are glued along parts of their boundary and
one has to declare an enumeration of the resulting boundaries.
\begin{defi*}[B., H.]
    The {\bf combinatorial type} $G$ which specifies the glueing construction {\bf depends on} the parameters
    \[
        \mathfrak P(G) = (g_1, g_2, n_1, n_2, m_1, m_2, (r_1^{(1)}, \ldots, r_{n_1}^{(1)}), (r_1^{(2)}, \ldots, r_{n_2}^{(2)}))
    \]
    and {\bf consists of} the following two data.
    \begin{enumerate}
        \item A partial, non-empty matching of the planes of the parallel slit domains in $\Par_{g_1, n_1}^{m_1}(r_1^{(1)}, \ldots, r_{n_1}^{(1)})$ and $\Par_{g_2, n_2}^{m_2}(r_1^{(2)}, \ldots, r_{n_2}^{(2)})$.
    \end{enumerate}
    The size of the matching is denoted by $s(G)$.
    The glueing construction defines a surface of genus $g(G)$ with $m(G) = m_1 + m_2$ punctures and $n(G)$ (yet unordered) boundary curves each consisting of several planes.
    \begin{enumerate}
        \setcounter{enumi}{1}
        \item A partial enumeration of the planes such that each boundary curve belongs to exactly one selected planes.
    \end{enumerate}
    The corresponding ordered configuration is $(r^{(G)}_1, \ldots, r^{(G)}_{n(G)})$.
    The {\bf set of combinatorial types} that specify a glueing construction is denoted by $\mc G$.
\end{defi*}
\begin{prop*}[Bödigheimer]
    For every combinatorial type $G \in \mc G$ with parameters
    \[
        \mathfrak P(G) = (g_1, g_2, n_1, n_2, m_1, m_2, (r_1^{(1)}, \ldots, r_{n_1}^{(1)}), (r_1^{(2)}, \ldots, r_{n_2}^{(2)}))
    \]
    there are homology operations induced by the action of the little cubes operad
    \begin{multline*}
        (\tilde\vartheta_G)_\ast \colon 
            H_i(\cspc 2(\C))^{\oplus s(G)} \otimes
            H_j(\Par_{g_1, n_1}^{m_1}(r_1^{(1)}, \ldots, r_{n_1}^{(1)})) \otimes
            H_k(\Par_{g_2, n_2}^{m_2}(r_1^{(2)}, \ldots, r_{n_2}^{(2)})) \to\\[10pt]
            H_{i+j+k}(\Par_{g(G), n(G)}^{m(G)}(r^{(G)}_1, \ldots, r^{(G)}_{n(G)}))\,.
    \end{multline*}
\end{prop*}

In addition to the parallelization and radialization map mentioned above,
we have the following propositions relating the space of parallel slit domains to the space of radial slit domains.
\begin{prop*}[Bödigheimer]
    The action of the little cubes operad on the space of parallel slit domains extends to an operation
    \[
        \begin{tikzcd}
            \tilde C^k(\A) \times \mathfrak{Par}_{g_1, 1}^{m_1} \times \dotsb \times \mathfrak{Par}_{g_k, 1}^{m_k} \arrow{r}{\tilde \vartheta} & \Radt_{\tilde g}(\tilde m + 1, 1) \\
            \tilde C^k(\C) \times \mathfrak{Par}_{g_1, 1}^{m_1} \times \dotsb \times \mathfrak{Par}_{g_k, 1}^{m_k} \arrow{r}{\tilde \vartheta} \arrow[hookrightarrow]{u}{\iota \times \text{id}} & \mathfrak{Par}_{\tilde g, 1}^{\tilde m} \arrow[hookrightarrow]{u}{rad}
        \end{tikzcd} \,,
   \]
   where $\tilde g = \sum_{i = 1}^{k} g_i$ and $\tilde m = \sum_{i = 1}^k m_i$.
\end{prop*}
\begin{prop*}[Bödigheimer]
    Let $n \ge 1$ and $\Par_n = \coprod_{g,m}\Par_{g,n}^m[(1, \ldots, 1)]$ and $\mathfrak{Rad}_n = \coprod_{g,m}\Rad$.
    There is a right module structure
    \[
        H_\ast(\mathfrak{Rad}_n) \otimes H_\ast(\Par_n) \to H_\ast(\mathfrak{Rad}_n)
    \]
    induced by an action of the little cubes operad.
\end{prop*}
\begin{prop*}[Bödigheimer]
    There is a composition operation
    \[
        \odot \colon \mathfrak{M}^{\bullet \bullet}_g(l, m) \times \mathfrak{M}^{\bullet \bullet}_{g'}(m, n) \to \mathfrak{M}^{\bullet \bullet}_{\tilde g}(l, n)\,, (F, F') \mapsto F \odot F'\,,
    \]
    where $\tilde g = g + g' + m - 1$.
\end{prop*}

Besides operations which are induced by the action of little cubes operads,
we generalize the operations discussed in \cite{Mehner201112} to arbitrary $n$ and $(r_1, \ldots, r_n)$, present the radial multipliciation
\[
    \radmult \colon \Rad \times \Radt_{g'}(m', n) \to \Radt_{\tilde g}(m+m', n)\,,
\]
with $\tilde g = g + g' + n - 1$ and introduce $\alpha \colon \mathfrak{M}_{g,n}^m \to \mathfrak{M}_{g,n}^{m+n}$
inducing a split injective map in homology
\[
    \alpha_\ast \colon H_\ast(\mathfrak{M}_{g,n}^m) \to H_\ast(\mathfrak{M}_{g,n}^{m+n}) \,.
\]
Rotating radial slit domains simultaneously induces the operation of degree one
\[
    rot \colon H_i(\Rad) \to H_{i+1}(\Rad)
\]
with $rot^2 = 0$.
Eventually, we present formulas relating some of the operations.

Furthermore, we provide an ongoing, extendable software project consisting of about 4500 lines of code.
It was used to compute the homology of the moduli spaces for $h \le 8$ and
the first author plans to implement the known operations in order to relate the found generators.
In addition, there are upcoming master students under the supervision of Bödigheimer who will implement further features of both the Ehrenfried complex and its corresponding bicomplex.

A deeper study of the cluster spectral sequence, the relations of generators via homology operations and the interdependencies of these operations outline an ongoing research project.