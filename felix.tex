\documentclass[12pt]{beamer}            % Latex-Beamer Klasse. Um ein Handout zu produzieren reicht es aus, die Option von [12pt] zu [12pt, handout] zu ändern.
\usepackage{etex}

\usepackage{tikz-cd} 

\usefonttheme[onlymath]{serif}          % "Normale" Mathemathik-Schriftart in Latexbeamer verwenden.
\setbeamertemplate{footline}[frame number]    % Seitenzahl unten anzeigen.
\setbeamertemplate{navigation symbols}{}      % Die komische Navigationsleiste ausschalten.

% Pakete
\usepackage[linesnumbered,ruled,vlined]{algorithm2e}    % Algorithmen setzen.
\usepackage{amsmath,amssymb,amsthm,amsfonts,amsbsy,latexsym}    % "Notwendige" AMS-Math Pakete.
\usepackage{array}                      % Bessere Tabellen.
\renewcommand{\arraystretch}{1.15}      % Tabellen bekommen ein wenig mehr Platz.
\usepackage{bbm}                        % Dicke 1.
\usepackage[hypcap]{caption}            % Damit Hyperrefs bei der figure-Umgebung auf die Figure zeigt statt auf die Caption.
\usepackage{diagbox}                    % Diagonale in Tabellen.
\usepackage{enumitem}                   % Zum Ändern der Nummerierungsumgenung 'enumerate'
\setlist[enumerate,1]{label=(\roman*)}  % Aufzählungen sind vom Typ 'Klammer auf; kleine römische Zahl; Klammer zu'
\usepackage[T1]{fontenc}                % Bessere Schrift
\usepackage{ifthen}                     % Zum checken ob Parameter leer sind.
\usepackage[utf8]{inputenc}             % utf8 als Eingabeformat.
\usepackage{lmodern}                    % Bessere Schrift
\usepackage{mathtools}                  % Subscript unter Summen behandeln. Der Befehl lautet \mathclap.
\usepackage{multirow}                   % In Tabellen mehrere Zeilen zu einer machen.

\graphicspath{{pictures/}}              % Pfad in dem die mit Inkscape erstellen Bilder liegen (relativ zum Hauptverzeichnis).


\newcommand{\todo}[1]{{\bf\color{red}-- #1 --}}
\newcommand{\incgfx}[1]{ \IfFileExists{#1}{ \includegraphics{#1} }{ \parbox{5cm}{Could not include your picture. You might want to run 'make pictures'} } }
\def\cpp{{C\nolinebreak[4]\hspace{-.05em}\raisebox{.4ex}{\tiny\bf ++}}}
\def\cppeleven{{C\nolinebreak[4]\hspace{-.05em}\raisebox{.4ex}{\tiny\bf ++}11}}

\newcommand{\blanc}{\mbox{\hspace{3pt}\underline{\ \ }\hspace{3pt}}}    % Blank
\newcommand{\mspc}[2]{ \hspace{#2pt}\text{#1}\hspace{#2pt} }            % Setze Wort in Mathe-Umgebung und lasse dabei Platz.
\newcommand{\xr}[1]{ \xrightarrow{\hspace{6pt}{#1}\hspace{6pt}} }       % Abbildungspfeil nach rechts mit Formel drüber.
\newcommand{\xhr}[1]{ \xhookrightarrow{\hspace{6pt}{#1}\hspace{6pt}} }  % Abbildungspfeil mit Bogen nach rechts mit Formel drüber.
\newcommand{\xdr}[1]{ \xtwoheadrightarrow{\hspace{6pt}{#1}\hspace{6pt}} }       % Abbildungspfeil mit zwei Köpfen nach rechts mit Formel drüber.
\newcommand{\xl}[1]{ \xleftarrow {\hspace{6pt}{#1}\hspace{6pt}} }       % Abbildungspfeil nach links mit Formel drüber.
\renewcommand{\to}{\xr{}}                                               % Abbildungspfeil nach rechts im selben Format wie oben.
\newcommand{\xmr}[1]{ \xmapsto{\hspace{6pt}{#1}\hspace{6pt}} }          % Elementpfeil nach rechts mit Formel drüber.
\renewcommand{\mapsto}{\xmr{}}                                          % Elementpfeil nach rechts im selben Format wie oben.

\newcommand{\B}{\mathbb B}
\newcommand{\bldots}{\raisebox{.5ex}{\fbox{$\ldots$}}}
\DeclareMathOperator{\cl}{cl}
\DeclareMathOperator{\cof}{cf}
\newcommand{\cspc}[1]{{\tilde C^{#1}}}
\DeclareMathOperator{\cyc}{cyc}
\newcommand{\del}{\partial}
\newcommand{\dE}{\hat \partial}
\DeclareMathOperator{\Diff}{Diff}
\newcommand{\dimx}[4]{\dim_{#4}H_{#3-p}(\mathfrak M_{#1,1}^{#2};{#4})}
\newcommand{\dimr}[3]{\dimx{#1}{#2}{#3}{R}}
\newcommand{\dimf}[3]{\dimx{#1}{#2}{#3}{\mathbb F_2}}
\newcommand{\dimq}[3]{\dimx{#1}{#2}{#3}{\mathbb Q}}
\newcommand{\E}{\mathbb E}
\newcommand{\Eins}{\mathbbm{1}}
%\newcommand{\Ehr}[1][n]{\mathbb{E}^m_{g,{#1}}}
\newcommand{\Ehrr}[1][r_1, \ldots, r_n]{\mathbb E(h,m;{#1})}
\newcommand{\equicl}{\sim_{\mathrm{CL}}}
\DeclareMathOperator{\grad}{grad}
\newcommand{\Harm}{\mathfrak{H}^m_{g,n}}
\newcommand{\Harmr}{{\Harm}[(r_1, \ldots, r_n)]}
\newcommand{\mc}[1]{\mathcal{#1}}
\newcommand{\mf}[1]{\mathfrak{#1}}
\DeclareMathOperator{\hgt}{ht}
\DeclareMathOperator{\ind}{ind}
\DeclareMathOperator{\img}{im}
\newcommand{\KK}{\mathbb K}
\newcommand{\MM}{\mathfrak M}
\newcommand{\Modspc}[1][n]{\mathfrak{M}^m_{g,#1}}
\DeclareMathOperator{\ncyc}{ncyc}
\newcommand{\Ncplx}{\mathcal N}
\newcommand{\nth}{\ensuremath{n^{\text{th}}}}
\newcommand{\PP}{\mathbb P}
\newcommand{\Par}[1][]{\mathfrak{Par}#1}
\newcommand{\Parr}[1][n]{\mathfrak{Par}^m_{g,#1}[(r_1, \ldots, r_n)]}
\DeclareMathOperator{\punc}{punc}
\DeclareMathOperator{\supp}{supp}
\newcommand{\SymGr}{\mathfrak S}
\newcommand{\Symgrp}{\mathfrak S}
\newcommand{\SymD}{\SymGr^\Delta}
\newcommand{\Thin}{Thin_{g,n}^m[(r_1, \ldots, r_n)]}
\newcommand{\tildeModspc}{\widetilde{\mathfrak M}_{g,n}^m}
\newcommand{\ov}{\overline}
\newcommand{\ul}[1]{\underline{\smash{#1}}}

\newcommand{\eps}{\varepsilon}
\newcommand{\A}{\mathbb A}
\newcommand{\C}{\mathbb C}
\newcommand{\F}{\mathbb F}
\newcommand{\M}{\mathfrak M}
\newcommand{\N}{\mathbb N}
\newcommand{\Z}{\mathbb Z}
\newcommand{\Q}{\mathbb Q}
\newcommand{\R}{\mathbb R}
\newcommand{\V}{\mathbb V}
\newcommand{\dimrx}[4]{\dim_{#4}H_{#3-p}(\mathfrak M_{#1}(#2, 1);{#4})}
\newcommand{\dimrr}[3]{\dimrx{#1}{#2}{#3}{R}}
\newcommand{\dimrf}[3]{\dimrx{#1}{#2}{#3}{\mathbb F_2}}
\newcommand{\dimrq}[3]{\dimrx{#1}{#2}{#3}{\mathbb Q}}
%\newcommand{\Ehrprog}{\mathbb{E}^m_{g,1}}
\newcommand{\Ehrprog}{\E}
\newcommand{\Modspcpunc}[1][n]{\mathfrak{M}^{m, n}_{g,#1}}
\newcommand{\ModspcRad}[1][n]{\mathfrak{M}^\bullet_g(m, #1)}
\newcommand{\ModspcRadm}[1][m]{\mathfrak{M}^\bullet_g(#1, n)}
\newcommand{\Rcomplex}[1][m]{R^\bullet_g({#1}, n)}
\newcommand{\Pcomplex}[1][m]{P^{#1}_{g,n}}
\newcommand{\Rad}[1][n]{{\mathfrak{Rad}_g(m,#1)}}
\newcommand{\Radt}{\mathfrak{Rad}}
\newcommand{\Part}{\mathfrak{Par}}
\newcommand{\HarmRad}[1][n]{\mathfrak{H}^{\bullet}_g(m, #1)}
\newcommand{\Har}{\mathfrak H}
\DeclareMathOperator{\height}{ht}
\DeclareMathOperator{\id}{id}
\DeclareMathOperator{\rk}{rk}
\DeclareMathOperator{\parmap}{par}
\DeclareMathOperator{\radmap}{rad}
\DeclareMathOperator{\rot}{rot}
\newcommand{\Th}{\ensuremath{^{\text{th}}}}
\newcommand{\nd}{\ensuremath{^{\text{nd}}}}
\newcommand{\complex}{{\bf\color{green} which complex? }}

\newtheorem*{thm*}{Theorem}

% Titel
\author{Felix Boes}
\title{Abschlussvortrag - Teil 2}
\subtitle{}
\date{12.09.2015}

\begin{document}

\begin{frame}
    \only<1>
    {
        \begin{thm*}
        \[
            H_\ast( \mathfrak{M}_{3,1}^0; \mathbb Z ) \cong
            \begin{cases}
                \mathbb Z           & \ast = 0\\
                                   & \ast = 1\\
                \phantom{\mathbb Z \oplus \mathbb Z/2 \oplus \mathbb Z/3 \oplus \mathbb Z/4 \oplus \mathbb Z/7 \oplus \bldots }                                 & \ast = 2\\
                & \ast = 3\\
                & \ast = 4\\
                & \ast = 5\\
                & \ast = 6\\
                & \ast = 7\\
                & \ast = 8\\
                & \ast = 9\\
                0 & \ast \ge 10\\
            \end{cases}
        \]
        \end{thm*}
    }
    \pause
    \only<2>
    {
        \begin{thm*}
        \[
            H_\ast( \mathfrak{M}_{3,1}^0; \mathbb Z ) \cong
            \begin{cases}
                \mathbb Z           & \ast = 0\\
                0                   & \ast = 1\\
                & \ast = 2\\
                \phantom{\mathbb Z \oplus \mathbb Z/2 \oplus \mathbb Z/3 \oplus \mathbb Z/4 \oplus \mathbb Z/7 \oplus \bldots} & \ast = 3\\
                & \ast = 4\\
                & \ast = 5\\
                & \ast = 6\\
                & \ast = 7\\
                & \ast = 8\\
                & \ast = 9\\
                0 & \ast \ge 10\\
            \end{cases}
        \]
        \end{thm*}
    }
    \only<3>
    {
        \begin{thm*}
        \[
            H_\ast( \mathfrak{M}_{3,1}^0; \mathbb Z ) \cong
            \begin{cases}
                \mathbb Z           & \ast = 0\\
                0                   & \ast = 1\\
                \mathbb Z^? \oplus \mathbb Z / 2                                  & \ast = 2\\
                \mathbb Z^? \oplus \mathbb Z/2 \oplus \mathbb Z/3 \oplus \mathbb Z/4 \oplus \mathbb Z/7 \oplus \bldots & \ast = 3\\
                \mathbb Z^? \oplus (\mathbb Z/2)^2 \oplus (\mathbb Z/3)^2  \oplus \bldots          & \ast = 4\\
                \mathbb Z^? \oplus \mathbb Z/2 \oplus \mathbb Z/3  \oplus \bldots & \ast = 5\\
                \mathbb Z^? \oplus (\mathbb Z/2)^3  \oplus \bldots                & \ast = 6\\
                \mathbb Z^? \oplus \mathbb Z/ 2\mathbb Z  \oplus \bldots   & \ast = 7\\
                \mathbb Z^? \oplus \bldots               & \ast = 8\\
                \mathbb Z^?  \oplus \bldots       & \ast = 9\\
                0                               & \ast \ge 10\\
            \end{cases}
        \]
        \end{thm*}
    }
    \only<4>
    {
        \begin{thm*}
        \[
            H_\ast( \mathfrak{M}_{3,1}^0; \mathbb Z ) \cong
            \begin{cases}
                \mathbb Z           & \ast = 0\\
                0                   & \ast = 1\\
                \mathbb Z \oplus \mathbb Z / 2                                  & \ast = 2\\
                \mathbb Z \oplus \mathbb Z/2 \oplus \mathbb Z/3 \oplus \mathbb Z/4 \oplus \mathbb Z/7 \oplus \bldots & \ast = 3\\
                (\mathbb Z/2)^2 \oplus (\mathbb Z/3)^2  \oplus \bldots          & \ast = 4\\
                \mathbb Z \oplus \mathbb Z/2 \oplus \mathbb Z/3  \oplus \bldots & \ast = 5\\
                \mathbb Z \oplus (\mathbb Z/2)^3  \oplus \bldots                & \ast = 6\\
                \mathbb Z / 2  \oplus \bldots   & \ast = 7\\
                0  \oplus \bldots               & \ast = 8\\
                \mathbb Z  \oplus \bldots       & \ast = 9\\
                0                               & \ast \ge 10\\
            \end{cases}
        \]
        \end{thm*}
    }
\end{frame}

\begin{frame}
    \begin{thm*}
        \[
            H_\ast( \mathfrak{M}_{2,1}^2; \mathbb Z ) \cong 
            \begin{cases}
                \mathbb Z           & \ast = 0\\
                (\mathbb Z/2)^2 \oplus \mathbb Z/5 \oplus \bldots    & \ast = 1\\
                \mathbb Z \oplus (\mathbb Z/2)^2 \oplus \bldots      & \ast = 2\\
                \mathbb Z^3 \oplus (\mathbb Z/2)^4 \oplus \bldots    & \ast = 3\\
                \mathbb Z \oplus (\mathbb Z/2)^5 \oplus (\mathbb Z/3)^3 \oplus \bldots       & \ast = 4\\
                \mathbb Z^2 \oplus (\mathbb Z/2)^4 \oplus \mathbb Z/3 \oplus \bldots         & \ast = 5\\
                \mathbb Z^2 \oplus (\mathbb Z/2)^3 \oplus \bldots    & \ast = 6\\
                \mathbb Z/2 \oplus \bldots                           & \ast = 7\\
                0                   & \ast \ge 8\\
            \end{cases}
        \]  
    \end{thm*}
\end{frame}

\begin{frame}
    \begin{thm*}
        \[
            H_\ast( \mathfrak{M}_{1,1}^4; \mathbb Z ) \cong 
            \begin{cases}
                \mathbb Z           & \ast = 0\\
                \mathbb Z \oplus \mathbb Z/2 \oplus \bldots          & \ast = 1\\
                (\mathbb Z/2)^3 \oplus \bldots                       & \ast = 2\\
                \mathbb Z^2 \oplus (\mathbb Z/2)^3 \oplus \bldots    & \ast = 3\\
                \mathbb Z^3 \oplus (\mathbb Z/2)^2 \oplus \bldots    & \ast = 4\\
                \mathbb Z^2 \oplus \mathbb Z/2 \oplus \bldots        & \ast = 5\\
                \mathbb Z \oplus \bldots     & \ast = 6\\
                0                           & \ast \ge 7\\
            \end{cases}
        \]
    \end{thm*}
\end{frame}

\begin{frame}
    \pause
    \def\svgwidth{\columnwidth}
    \input{pictures/gluing_caps.pdf_tex}
\end{frame}

\begin{frame}
    \pause
    \incgfx{pictures/radial_threading}
    \pause
    \[
        \begin{tikzcd}[ampersand replacement=\&]
        \tilde C^k(n\A) \times \mathfrak{Par}_{g_1, 1}^{m_1} \times \dotsb \times \mathfrak{Par}_{g_k, 1}^{m_k} \arrow{r}{\tilde \vartheta} \& \Radt_{\tilde g}(\tilde m + n, 1) \\
        \tilde C^k(n\C) \times \mathfrak{Par}_{g_1, 1}^{m_1} \times \dotsb \times \mathfrak{Par}_{g_k, 1}^{m_k} \arrow{r}{\tilde \vartheta} \arrow[hookrightarrow]{u}{\iota \times id} \& \mathfrak{Par}_{\tilde g, n}^{\tilde m} \arrow[hookrightarrow]{u}{map}
        \end{tikzcd}
   \]
   $\tilde g = \sum_{i = 1}^{k} g_i$ und $\tilde m = \sum_{i = 1}^k m_i$. 
\end{frame}

\begin{frame}
    \pause
    \incgfx{pictures/par_operates_on_rad}
    \pause
    \[
      \radmap(L_1 \cdot L_2) \simeq \radmap(L_1).L_2
    \]
\end{frame}

\begin{frame}
    \incgfx{pictures/rad_compatible}
    \[
      \radmap(L_1 \cdot L_2) \simeq \radmap(L_1).L_2
   \]
\end{frame}


\end{document}