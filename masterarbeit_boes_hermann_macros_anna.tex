\newcommand{\eps}{\varepsilon}
\newcommand{\A}{\mathbb A}
\newcommand{\C}{\mathbb C}
\newcommand{\F}{\mathbb F}
\newcommand{\M}{\mathfrak M}
\newcommand{\N}{\mathbb N}
\newcommand{\Z}{\mathbb Z}
\newcommand{\Q}{\mathbb Q}
\newcommand{\R}{\mathbb R}
\newcommand{\V}{\mathbb V}
\newcommand{\dimrx}[4]{\dim_{#4}H_{#3-p}(\mathfrak M_{#1}(#2, 1);{#4})}
\newcommand{\dimrr}[3]{\dimrx{#1}{#2}{#3}{R}}
\newcommand{\dimrf}[3]{\dimrx{#1}{#2}{#3}{\mathbb F_2}}
\newcommand{\dimrq}[3]{\dimrx{#1}{#2}{#3}{\mathbb Q}}
%\newcommand{\Ehrprog}{\mathbb{E}^m_{g,1}}
\newcommand{\Ehrprog}{\E}
\newcommand{\Modspcpunc}[1][n]{\mathfrak{M}^{m, n}_{g,#1}}
\newcommand{\ModspcRad}[1][n]{\mathfrak{M}^\bullet_g(m, #1)}
\newcommand{\ModspcRadm}[1][m]{\mathfrak{M}^\bullet_g(#1, n)}
\newcommand{\Rcomplex}[1][m]{R^\bullet_g({#1}, n)}
\newcommand{\Pcomplex}[1][m]{P^{#1}_{g,n}}
\newcommand{\Rad}[1][n]{{\mathfrak{Rad}_g(m,#1)}}
\newcommand{\Radt}{\mathfrak{Rad}}
\newcommand{\Part}{\mathfrak{Par}}
\newcommand{\HarmRad}[1][n]{\mathfrak{H}^{\bullet}_g(m, #1)}
\newcommand{\Har}{\mathfrak H}
\DeclareMathOperator{\height}{ht}
\DeclareMathOperator{\id}{id}
\DeclareMathOperator{\rk}{rk}
\DeclareMathOperator{\parmap}{par}
\DeclareMathOperator{\radmap}{rad}
\DeclareMathOperator{\rot}{rot}
\newcommand{\Th}{\ensuremath{^{\text{th}}}}
\newcommand{\nd}{\ensuremath{^{\text{nd}}}}
\newcommand{\complex}{{\bf\color{green} which complex? }}

%Siehe http://tex.stackexchange.com/questions/56079/using-tikz-inside-a-figure-caption
\DeclareRobustCommand\radmult{%
    {%
        \ensuremath{%
            \mathrel{%
                \tikz[baseline={([yshift=-3.5pt]current bounding box.center)}, x=5pt, y=2.5pt, every node/.style={shape=circle, fill=black, inner sep=.8pt}]{%
                    \draw[line width=0.9pt] (5, 0) -- (3, 0);
                    \draw[line width=0.9pt] (2, 0) -- (0, 0);
                    \filldraw (3, 0) circle (1pt);
                    \filldraw (2, 0) circle (1pt);
                }%
            }%
        }%
    }%
}
