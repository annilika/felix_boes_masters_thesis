\makeatletter
\newcommand{\mylabel}[2]{%
   \protected@write \@auxout {}{\string \newlabel {#1}{{#2}{\thepage}{#2}{#1}{}} }%
   \hypertarget{#1}{#2}%
}%
\makeatother

\newcommand{\todo}[1]{{\bf\color{red}-- #1 --}}
\newcommand{\incgfx}[1]{ \IfFileExists{#1}{ \includegraphics{#1} }{ \parbox{5cm}{Could not include your picture. You might want to run 'make pictures'} } }
\def\cpp{{C\nolinebreak[4]\hspace{-.05em}\raisebox{.4ex}{\tiny\bf ++}}}
\def\cppeleven{{C\nolinebreak[4]\hspace{-.05em}\raisebox{.4ex}{\tiny\bf ++}11}}

\newcommand{\blanc}{\mbox{\hspace{3pt}\underline{\ \ }\hspace{3pt}}}    % Blank
\newcommand{\mspc}[2]{ \hspace{#2pt}\text{#1}\hspace{#2pt} }            % Setze Wort in Mathe-Umgebung und lasse dabei Platz.
\newcommand{\xr}[1]{ \xrightarrow{\hspace{6pt}{#1}\hspace{6pt}} }       % Abbildungspfeil nach rechts mit Formel drüber.
\newcommand{\xhr}[1]{ \xhookrightarrow{\hspace{6pt}{#1}\hspace{6pt}} }  % Abbildungspfeil mit Bogen nach rechts mit Formel drüber.
\newcommand{\xdr}[1]{ \xtwoheadrightarrow{\hspace{6pt}{#1}\hspace{6pt}} }       % Abbildungspfeil mit zwei Köpfen nach rechts mit Formel drüber.
\newcommand{\xl}[1]{ \xleftarrow {\hspace{6pt}{#1}\hspace{6pt}} }       % Abbildungspfeil nach links mit Formel drüber.
\renewcommand{\to}{\xr{}}                                               % Abbildungspfeil nach rechts im selben Format wie oben.
\newcommand{\xmr}[1]{ \xmapsto{\hspace{6pt}{#1}\hspace{6pt}} }          % Elementpfeil nach rechts mit Formel drüber.
\renewcommand{\mapsto}{\xmr{}}                                          % Elementpfeil nach rechts im selben Format wie oben.

\newcommand{\B}{\mathbb B}
\newcommand{\bldots}{\raisebox{.5ex}{\fbox{$\ldots$}}}
\DeclareMathOperator{\cl}{cl}
\DeclareMathOperator{\cof}{cf}
\newcommand{\cspc}[1]{{\tilde C^{#1}}}
\DeclareMathOperator{\cyc}{cyc}
\newcommand{\del}{\partial}
\newcommand{\dE}{\hat \partial}
\DeclareMathOperator{\Diff}{Diff}
\newcommand{\dimx}[4]{\dim_{#4}H_{#3-p}(\mathfrak M_{#1,1}^{#2};{#4})}
\newcommand{\dimr}[3]{\dimx{#1}{#2}{#3}{R}}
\newcommand{\dimf}[3]{\dimx{#1}{#2}{#3}{\mathbb F_2}}
\newcommand{\dimq}[3]{\dimx{#1}{#2}{#3}{\mathbb Q}}
\newcommand{\E}{\mathbb E}
\newcommand{\Eins}{\mathbbm{1}}
%\newcommand{\Ehr}[1][n]{\mathbb{E}^m_{g,{#1}}}
\newcommand{\Ehrr}[1][r_1, \ldots, r_n]{\mathbb E(h,m;{#1})}
\newcommand{\equicl}{\sim_{\mathrm{CL}}}
\DeclareMathOperator{\grad}{grad}
\newcommand{\Harm}{\mathfrak{H}^m_{g,n}}
\newcommand{\Harmr}{{\Harm}[(r_1, \ldots, r_n)]}
\newcommand{\mc}[1]{\mathcal{#1}}
\newcommand{\mf}[1]{\mathfrak{#1}}
\DeclareMathOperator{\hgt}{ht}
\DeclareMathOperator{\ind}{ind}
\DeclareMathOperator{\img}{im}
\newcommand{\KK}{\mathbb K}
\newcommand{\MM}{\mathfrak M}
\newcommand{\Modspc}[1][n]{\mathfrak{M}^m_{g,#1}}
\DeclareMathOperator{\ncyc}{ncyc}
\newcommand{\Ncplx}{\mathcal N}
\newcommand{\nth}{\ensuremath{n^{\text{th}}}}
\newcommand{\PP}{\mathbb P}
\newcommand{\Par}[1][]{\mathfrak{Par}#1}
\newcommand{\Parr}[1][n]{\mathfrak{Par}^m_{g,#1}[(r_1, \ldots, r_n)]}
\DeclareMathOperator{\punc}{punc}
\DeclareMathOperator{\supp}{supp}
\newcommand{\SymGr}{\mathfrak S}
\newcommand{\Symgrp}{\mathfrak S}
\newcommand{\SymD}{\SymGr^\Delta}
\newcommand{\Thin}{Thin_{g,n}^m[(r_1, \ldots, r_n)]}
\newcommand{\tildeModspc}{\widetilde{\mathfrak M}_{g,n}^m}
\newcommand{\ov}{\overline}
\newcommand{\ul}[1]{\underline{\smash{#1}}}

\DeclareRobustCommand\updownarrows{%
    {%
        \ensuremath{%
            \raisebox{-.5ex}{\rotatebox{90}{$\scriptstyle\rightleftarrows$}}
        }
    }
}

\newcommand{\tl}[1]{\multicolumn{1}{|r||}{#1}}

% Siehe http://tex.stackexchange.com/questions/56079/using-tikz-inside-a-figure-caption
\DeclareRobustCommand\ikurz{%
    {%
        \ensuremath{%
            \mathrel{%
                \begin{tikzpicture}[baseline=-.75ex]
                    \draw [line width=0.1ex, line cap=round]
                        (0, 0.3ex) -- (1.2ex, 0.3ex);
                    \draw [line width=0.1ex, line cap=round]
                        (0, 0) -- (.5ex, 0);
                \end{tikzpicture}%
            }%
        }%
    }%
}
\DeclareRobustCommand\ilang{%
    {%
        \ensuremath{%
            \mathrel{%
                \begin{tikzpicture}[baseline=-.75ex]
                    \draw [line width=0.1ex, line cap=round]
                        (0, 0) -- (1.2ex, 0);
                    \draw [line width=0.1ex, line cap=round]
                        (0, 0.3ex) -- (.5ex, 0.3ex);
                \end{tikzpicture}%
            }%
        }%
    }%
}

\newcommand{\nzweizelle}[5]{
    {%
        \ensuremath{%
            \ %
            \tikz[baseline={([yshift=-2.5pt]current bounding box.center)}, x=15pt, y=7pt, every node/.style={shape=circle, fill=black, inner sep=.8pt}]{%
                \foreach \y in {1,...,#5}
                {
                    \draw (-0.5, \y) -- (1.5, \y);
                }
                \draw[color=black!50] (-0.5,.7) -- (1.5,.7) -- (1.5, {#5 + .3}) -- (-0.5, {#5 + .3}) -- (-0.5, .7);
                \draw (0,#1) node {} -- (0,#2) node {};
                \draw (1,#3) node {} -- (1,#4) node {};
            }%
            \ %
        }%
    }%
}
\newcommand{\vierzweizelle}[4]{\nzweizelle{#1}{#2}{#3}{#4}{4}}
\newcommand{\dreizweizelle}[4]{\nzweizelle{#1}{#2}{#3}{#4}{3}}
\newcommand{\zweizweizelle}[4]{\nzweizelle{#1}{#2}{#3}{#4}{2}}

\DeclareRobustCommand\mueta{%
    {%
        \ensuremath{%
            \tikz[baseline={([yshift=-3.5pt]current bounding box.center)}, x=5pt, y=2.5pt, every node/.style={shape=circle, fill=black, inner sep=.8pt}]{%
                \draw[line width=.8pt] (1,4) -- (.5,3.2) -- (.5,1.8) -- (1,1);
                \draw[line width=.8pt] (0,4) -- (.5,3.2) -- (.5,1.8) -- (0,1);
            }%
        }%
    }%
}


\makeatletter
\providecommand*{\twoheadrightarrowfill@}{%
  \arrowfill@\relbar\relbar\twoheadrightarrow
}
\providecommand*{\twoheadleftarrowfill@}{%
  \arrowfill@\twoheadleftarrow\relbar\relbar
}
\providecommand*{\xtwoheadrightarrow}[2][]{%
  \ext@arrow 0579\twoheadrightarrowfill@{#1}{#2}%
}
\providecommand*{\xtwoheadleftarrow}[2][]{%
  \ext@arrow 5097\twoheadleftarrowfill@{#1}{#2}%
}
\makeatother

\newcommand{\homog}[1][\sigma]{{\ensuremath{(#1_h : \ldots : #1_0)}}}
\newcommand{\homogq}[1][\sigma]{{\ensuremath{(#1_q : \ldots : #1_0)}}}
\newcommand{\inhom}[1][\tau]{{\ensuremath{(#1_h \mid \ldots \mid #1_1)}}}
\newcommand{\inhomq}[1][\tau]{{\ensuremath{(#1_q \mid \ldots \mid #1_1)}}}


\newcommand{\progclass}[1]{{\ttfamily\color{col_classes}#1}}
\newcommand{\progkeyword}[1]{{\ttfamily\color{col_keywords}#1}}
\newcommand{\progmember}[1]{{\ttfamily\color{col_members}#1}}
\newcommand{\progname}[1]{{\ttfamily#1}}