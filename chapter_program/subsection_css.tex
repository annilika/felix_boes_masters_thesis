\subsection{The Class ClusterSpectralSequence}
\label{chapter_program:kappa:css}

Having described how basis elements of the Ehrenfried complex $\Ehrprog$ are represented by our computer program, 
we will now explain how the cluster spectral sequence associated with $\Ehrprog$ is realized 
and how the methods provided by the class \progclass{Tuple} (compare Subsection \ref{chapter_program:kappa:tuple}) are combined 
to compute its differentials.
For these purposes, we introduce the class \progclass{ClusterSpectralSequence}, 
which is a template class with the class type of the underlying \progclass{ChainComplex} as a template parameter
(compare Subsection \ref{program:libhomology:ChainComplex}).

At first, we are going to describe the class \progclass{CSSBasis} in Subsection \ref{chapter_program:kappa:css:css_basis}, 
which represents the basis of a single module of the cluster spectral sequence. 
Next, we describe the data members of the \progclass{ClusterSpectralSequence} (Subsection \ref{chapter_program:kappa:css:data_members}),
which are most importantly a collection of bases of the modules of the Ehrenfried complex and 
a collection of its differentials.
Thereafter, we explain how the bases (see Subsubsection \ref{chapter_program:kappa:css:gen_bases}) and 
the differentials (see Subsubsection \ref{chapter_program:kappa:css:gen_diff}) are generated, 
and what other methods are provided by the class \progclass{ClusterSpectralSequence} (see Subsubsection \ref{chapter_program:kappa:css:more_methods}). 

\subsubsection{CSSBasis}
\label{chapter_program:kappa:css:css_basis}

Just like the cluster spectral sequence associated with $\Ehrprog$ consists of a finite number of modules, 
our class \progclass{ClusterSpectralSequence} contains a finite number of \progclass{CSSBases}, 
which are structs representing the bases of these modules.

Thus the only data member of the struct \progclass{CSSBasis} is the basis
\begin{lstlisting}
BasisType basis;
\end{lstlisting}
Thereby, \progclass{BasisType} is a map storing all \progclass{Tuples} of this basis, sorted by cluster sizes.
For each cluster size $l$, we organize the corresponding \progclass{Tuples} in an \progname{std::unordered\_set}
with an appropriate hash function that makes it possible to search for basis elements in amortized constant running time.

The struct \progclass{CSSBasis} provides the usual methods for saving and loading as well as the functions
\begin{lstlisting}
uint64_t size( int32_t l ) const;
\end{lstlisting}
returning the number of basis elements with exactly $l$ clusters and 
\begin{lstlisting}
uint64_t total_size() const;
\end{lstlisting}
returning the total number of basis elements.

Since the computation of the differential (compare Subsubsection \ref{chapter_program:kappa:css:gen_diff}) requires 
a unique identification of one basis element among the other basis elements of the same cluster size of one \progclass{CSSBasis}, 
there is a method
\begin{lstlisting}
int64_t id_of( Tuple& t );
\end{lstlisting} 
returning the unique \progname{id} of the given \progclass{Tuple} \progname{t}. 
If \progname{t} is not an element of this \progclass{CSSBasis}, we return \progname{-1} to indicate the failure of the function.

The most important method of the struct \progclass{CSSBasis} is the function
\begin{lstlisting}
uint32_t add_basis_element ( Tuple& t );
\end{lstlisting}

Using the method \progname{num\_clusters} of the class \progclass{Tuple} (see Subsubsection \ref{program:kappa:tuple:basics}),
it inserts the \progclass{Tuple} \progname{t} into the part of the basis corresponding to its number of clusters
and sets the \progname{id} of \progname{t} to the current number of basis elements with exactly $l$ clusters. 
This means that if one builds up a \progclass{MonoBasis} by successively adding basis elements,
all basis elements can be distinguished by their \progname{ids}.

\subsubsection{Data Members}
\label{chapter_program:kappa:css:data_members}

The class \progclass{ClusterSpectralSequence} represents the cluster spectral sequence assoicated with the Ehrenfried complex.
Hence, it contains a collection of \progclass{CSSBases} 
\begin{lstlisting}
std::map< uint32_t, CSSBasis > basis_complex;
\end{lstlisting}
where for each $0 \leq p \leq 2h$, the basis elements of the Ehrenfried complex on the symbols $0, \dotsc, p$ are stored,
and the data member
\begin{lstlisting}
MatrixComplex diff_complex;
\end{lstlisting}
where at each time, the differential needed for computations is stored.

Note that \progclass{MatrixComplex} is a template parameter. 
Depending on the coffficients of the homology one wants to compute,
it can be chosen to be \progclass{ClusterSpectralSequenceQ} or \progclass{ClusterSpectralSequenceZm}.
For coefficients in the field $\mathbb F_2$, we highly recommend to use \progclass{ClusterSpectralSequenceBool} for efficiency reasons.
The genus \progmember{g}, the number of punctures \progmember{m} and the number \progmember{h} of transpositions in a basis tuple 
associated with this \progclass{MonoComplex} are also stored as data members. 
Furthermore, the data member
\begin{lstlisting}
SignConvention sign_conv;
\end{lstlisting}
indicates which sign convention (see also Subsubsection \ref{chapter_program:kappa:compute_css:implementation}) is used to compute the homology of this \progclass{MonoComplex},
and the data member
\begin{lstlisting}
size_t num_threads;
\end{lstlisting}
determines the number of threads using for the construction of the differentials.

\subsubsection{Generating Bases}
\label{chapter_program:kappa:css:gen_bases}

The first step to build up a \progclass{ClusterSpectralSequence} is to call the constructor
\begin{lstlisting}
ClusterSpectralSequence( uint32_t genus, 
                         uint32_t num_punctures, 
                         SignConvention sgn, 
                         uint32_t num_working_threads, 
                         uint32_t num_remaining_threads );
\end{lstlisting}
For an explanation of the parameters \progname{num\_working\_threads} and \progname{num\_remaining\_threads}, 
see Subsubsection \ref{diag_field_implementation}.
In the constructor, \progclass{Diagonalizer} is configured (compare Subsection \ref{program:libhomology:DiagonalizerT}), 
and the bases of all the modules belonging to the \progclass{MonoComplex} are initialized 
via a recursive method we want to explain now. 

Our aim is to enumerate all cells $\Sigma = (\tau_h \mid \ldots \mid \tau_1)$ 
of bidegree $(p, h)$ for all $0 \leq p \leq 2h$ such that
\begin{enumerate}
 \item All $\tau_i$ are non-trivial transpositions on the symbols $min, \dotsc, p$.
 \item Each symbol $min, \dotsc, p$ is permuted non-trivially by at least one $\tau_i$.
 \item $\Sigma$ is monotonous.
\end{enumerate}
Hereby, in the parallel case, $min$ equals $1$, and in the radial case, we want to enumerate all cells for $min = 0$ or $min = 1$.

The enumeration of all these cells works recursively.
Let $\Sigma = (\tau_k \mid \ldots \mid \tau_1)$ be a cell fulfilling the above conditions, 
but with bidegree $(p, k)$ with $k < h$.
Assume we want to find all possibilities to extend such a cell $\Sigma$ to a monotonous tuple $\Sigma'$ of $k+1$ transpositions
by inserting a transposition $\tau_{k + 1}$, perharps using more symbols. 
The following cases occur.

\begin{enumerate}
\item We also use the symbols $min, \dotsc, p$ for $\tau_{k+1}$.
      Since $\Sigma'$ is supposed to be monotonous, 
      $\tau_{k + 1}$ needs to contain the symbol $p$.
      Hence we can set $\tau_{k + 1} := (p, i)$ with $i \in \{min, \dotsc, p-1\}$ chosen arbitrarily.
\item We insert a new row into our parallel slit domain and use the symbols $min, \dotsc, p + 1$ for $\Sigma'$.
      By monotony, $\tau_{k + 1}$ has to contain the highest symbol $p + 1$. 
      But now there are two possibilities:
      \begin{enumerate}
      \item The new row is inserted as a $(p+1)\Th$ row above the old rows. 
            Hence the symbol that is contained in $\tau_{k + 1}$ apart from $p + 1$
            is one of the symbols that were already used for the transpositions $\tau_1, \dotsc, \tau_k$, 
            and we can set $\tau_{k + 1} := (p + 1, i)$ with $i \in \{min, \dotsc, p\}$ chosen arbitrarily.
      \item The new row is inserted as the $i\Th$ row for some $i \in \{min, \dotsc, p\}$  
            and the indices of the old rows $min, \dotsc, p$ are shifted up by one.
            Note that these indices are also shifted up in the transpositions of $\Sigma$.
            By monotony and since all symbols in $\{min, \dotsc, p\}$ have to be covered, 
            the new transposition has to be $\tau_{k + 1} := (p + 1, i)$, 
            meaning that the symbol $p + 1$ is also used by at least one of the transpositions $\tau_1, \dotsc, \tau_k$, 
            and that the symbol $i$ is used by $\tau_{k + 1}$ only.
      \end{enumerate}
\item We insert two rows into our slit domain, using the symbols $\{min, \dotsc, p + 2\}$. 
      Thus the symbols used by the new transposition $\tau_{k + 1}$ do not yet appear in $\Sigma$.
      Therefore $\tau_{k + 1}$ has to contain the symbol $p + 2$ and some other symbol $i \in \{min, \dotsc, p + 1\}$.
      Again, the indices of the rows $i + 1, \dotsc, p$ have to be shifted up by $1$ in the transpositions of $\Sigma$.
\end{enumerate}

We can use these observations to define the recursive method
\begin{lstlisting}
void gen_bases( uint32_t k, uint32_t p, uint32_t min, Tuple& tuple );
\end{lstlisting}
This methods gets as input data a monotonous \progclass{Tuple} consisting of $k$ transpositions 
which contain each of the symbols $min, \dotsc, p$ at least once,
and the minimum symbol the transpositions may contain.
Using the above case distinction, it calls itself recursively with the appropriate parameters
and stores all the transpositions with $h$ transpositions detected thereby in the corresponding basis,
but only if they contain the required number of cycles.

Now we can describe the way the constructor of the \progclass{CSS} sets up its \progmember{basis\_complex}.

\begin{prop}
We can enumerate all basis elements of the parallel Ehrenfried complex
by defining the \progclass{Tuple} $\Sigma = ((2, 1))$ consisting of only one \progclass{Transposition},
and then calling the recursive function \progname{gen\_bases(1, 2, 1, $\Sigma$)}. 

For enumerating all basis elements of the radial Ehrenfried complex,
we additionally need to call the recursive function \progname{gen\_bases(1, 1, 0, $\Sigma'$)} with $\Sigma' = ((1, 0))$.
\begin{proof}

Note that for a given $\Sigma$, the tranpositions arising from the different cases fulfill the above conditions (i) - (iii),
and that they are all distinct.
If we consider two different \progclass{Tuples}, 
the transpositions arising from the case distinction also cannot coincide
since either they have different numbers of transpositions or
their starting sequence of transpositions already differs.
Especially, we don't enumerate \progclass{Tuples} multiple times by the two initial calls in the radial case
since the first family of \progclass{Tuples} does not contain the symbol $0$, and the second family does.

Hence it suffices to show that all monotonous transpositions can be found by our algorithm.
By induction on the number of transpositions, 
we can assume that we have already built up all monotonous tuples with $k$ transpositions.
Let $\Sigma' = (\tau_{k + 1}\mid \dotsc\mid \tau_1)$ be monotonous. 
Then $\tau_{k + 1} = (p+1, i)$ with $i \in \{1, \dotsc, p\}$ by monotony.
The tuple $\Sigma := (\tau_k \mid \dotsc \mid \tau_1)$ is also monotonous, 
but we might have to shift the indices $i, \dotsc, p+1$ down by one
if the symbol $i$ is not contained in the transpositions $\tau_1, \dotsc, \tau_k$.
This yields one of the tuples of $k$ transpositions we have already found.
By reverting the process just described, we can rebuild $\Sigma'$ from $\Sigma$, 
and this case is covered by the above case distinction.
\end{proof}
\end{prop}

\subsubsection{Generating Differentials}
\label{chapter_program:kappa:css:gen_diff}

In Subsubsection \ref{chapter_program:kappa:compute_css:implementation}, we explained how both the $E^1$-term and $E^2$-term are computed.
Recalling Chapter \ref{cellular_models}, the $p\Th$ differential of the Ehrenfried complex is given by the composition $\pi \del''_p \kappa$.
The member function
\begin{lstlisting}
gen_d0( const int32_t p, const int32_t l );
\end{lstlisting}
generates the restriction of the $p\Th$ differential $\del_\E$ to the cells of $\Ehrprog$ with exactly $l$ clusters.
Similarly, the method
\begin{lstlisting}
gen_d1_stage_1( const int32_t p, const int32_t l );
\end{lstlisting}
generates the restriction of the $p\Th$ differential $\del_\E$ from the cells with $l$ clusters to the cells with $l-1$ clusters and
applies all row operations to this restriction that come from the sub-matrix $d^0$ from above.

For runtime reasons, we thereby use yet another formula for the map $\kappa$.
\begin{prop}
    Let $I'$ be the set of tuples of integers $(t_h, \ldots, t_1)$ such that $0 \leq t_q < q$.
    Then the map
    \[
        \{ 0, \ldots, h! - 1\} \to I'
        \mspc{with}{20}
        k \mapsto \left( \left\lfloor \frac{k}{(q-1)!} \right\rfloor \pmod q\right)_q
    \]
    is a bijection.
\end{prop}
\begin{proof}
We show that
\[
    (t_h, \ldots, t_1) \mapsto \sum_{q=2}^h t_q \cdot (q-1)!
\]
defines an inverse of the above map. 
Since both sets have the same cardinality, 
it suffices to show that this map is a right inverse.
For a fixed coordinate $q \in \{1, \dotsc, h\}$, we compute
\[
    \sum_{q=2}^h t_q \cdot (q-1)! = \sum_{q=r}^h t_q \cdot (q-1)! + \sum_{q=2}^{r-1} t_q \cdot (q-1)!\,.
\]
Using induction on $r$, we see that
\[
    \sum_{q=2}^{r-1} t_q \cdot (q-1)! < (r-1)!\,,
\]
since for $r = 1$, the statement is true, and if we assume the statement for $r-1$, we can conclude
\begin{align*}
    \sum_{q=2}^{r} t_q \cdot (q-1)! &= \sum_{q=2}^{r-1} t_q \cdot (q-1)! + t_r (r-1)! \\
                                    &< (r-1)!               + (r-1) (r-1)! \\
                                    &= r!\,.
\end{align*}
From this, we get
\[
    \left\lfloor \frac{\sum_{q=2}^h t_q \cdot (q-1)!}{(r-1)!} \right\rfloor=\left\lfloor \sum_{q=r}^h t_q \cdot \frac{(q-1)!}{(r-1)!} + \sum_{q=2}^{r-1} t_q \cdot \frac{(q-1)!}{(r-1)!} \right\rfloor = \sum_{q=r}^h t_q \cdot \frac{(q-1)!}{(r-1)!}
\]
since the left sum is integral and the right sum vanishes after rounding down.
Taking the remaining term modulo $r$, we get $t_r$ as a result since all summands but the $r\Th$ are zero modulo $r$.
\end{proof}

\subsubsection{More Member Functions}
\label{chapter_program:kappa:css:more_methods}

We also offer class methods to print the \progmember{basis\_complex} and the \progmember{matrix\_complex}, and a method
\begin{lstlisting}
void erase_differential( int32_t p );
\end{lstlisting}
which deletes the \progmember{p}${}\Th$ differential and releases its memory. 
This function is defined because we only want to keep a differential in the memory
as long as we really need it due to storage limitation.