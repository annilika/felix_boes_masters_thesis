\subsection{The Tool \progname{compute\_css}}
\label{chapter_program:kappa:compute_css}

The tool \progname{compute\_css}, which computes the homology of the moduli spaces $\Modspc[1]$ and $\ModspcRad[1]$
by deriving the second term of the corresponding cluster spectral sequence,
is the most important tool supported by the project {\bf kappa}.
In Subsubsection \ref{chapter_program:kappa:compute_css:usage}, we describe 
how one operates this tool, 
while in Subsubsection \ref{chapter_program:kappa:compute_css:implementation},
we explain its implementation. 

\subsubsection{Usage}
\label{chapter_program:kappa:compute_css:usage}

For the computation of the homology of the moduli space $\ModspcRad[1]$, 
one can use the command
\begin{lstlisting}
.\compute_css -g arg -m arg (-q | -s arg)
\end{lstlisting}
Thereby,
\begin{itemize}
\item the parameter \progname{g} is the genus of the moduli space,
\item the parameter \progname{m} is the number of punctures of the moduli space,
\item one can either use the parameter $q$ or the parameter $s$. 
      If $q$ is chosen -- without any argument -- homology with rational coefficients will be computed.
      If the parameter $s$ is set to some positive prime, we will compute homology with coefficients in $\Z/s\Z$.
\end{itemize}
As the output of this command, one obtains a description of the $E^0$, $E^1$ and $E^2$ term of the cluster spectral sequence associated with $\ModspcRad[1]$,
and can read off the homology from the $E^2$ page.  

Recall that, for $m > 0$, the homology of the moduli space $\Modspc[1]$ coincides with the homology of the moduli space $\Modspc[1]$ 
(compare Proposition \ref{cellular_models:comparision_of_the_models:bundles_are_h_equiv}).
For determining the homology of $\Modspc[1]$ for $m = 0$ 
or the dimensions of the modules in the cluster spectral sequence of $\Modspc[1]$, one can set the optional parameter
\begin{itemize}
 \item \progname{parallel}
\end{itemize}
to true.

For instance, the call
\begin{lstlisting}
 ./compute_css -g 1 -m 3 -s 2
\end{lstlisting}
computes the homolgy of the moduli space $\mathfrak{M}^\bullet_1(3, 1)$ with $\Z/2\Z$ coefficients, 
while the call
\begin{lstlisting}
 ./compute_css -g 1 -m 3 -q --parallel 1
\end{lstlisting}
determines the homolgy of the moduli space $\mathfrak{M}^1_{3, 1}$ with rational coefficients.

There are other optional parameters that improve the performance or handling of the program.
\begin{itemize}
\item The optional parameter \progname{t} is the number of threads 
      that are allowed to be used for parallelization,
      which is $1$ by default. 
\item In addition, one can use the optional parameter \progname{num\_remaining\_threads}
      to determine how exactly computations are parallelized. 
      For a detailed explanation, see \ref{diag_field_implementation}.
      The total number of threads used will then be 
      \[
        \text{\progname{num\_threads + num\_remaining\_threads,}}
      \]
      and we recommend to use one third of the total number of threads as remaining threads.
\item The optional parameters \progname{first\_diff} respectively \progname{last\_diff} are the minimal respectively maximal $p \in \N$ 
      for which the homology $H_p(\Modspc[1])$ is supposed to be computed,
      which are $0$ respectively $2h$ by default.
\end{itemize}

If the command \progname{help} is used or if the input is not valid,
instructions for use will be printed to the console.

During the computations, we offer intermediate results and progress bars as console output. 
When the computation of the homology is finished, 
the file 
\[
\text{\progname{compute\_homology\_(parameters)}}
\]
created by the program contains all the intermediate and final results.
Be aware that this means that calls with the same parameters produce files with the same names and hence old files are overwritten.
The intermediate results give the oportunity to abort the computation
and continue it some other time, 
using the parameters \progname{first\_diff} and \progname{last\_diff} to select certain homology groups.

\subsubsection{Implementation Details}
\label{chapter_program:kappa:compute_css:implementation}

The computation of the cluster spectral sequence and the homology starts in the main function of the file 
\begin{lstlisting}
main_compute_css.cpp
\end{lstlisting}
The input parameters described above are stored in the struct
\begin{lstlisting}
SessionConfig;
\end{lstlisting}
which also tests whether the given configuration of parameters is valid, 
outputting the correct usage to the console if not.
Apart from the data members corresponding to the input parameters, 
the struct \progclass{SessionConfig} contains the data member
\begin{lstlisting}
SignConvention sgn_conv;
\end{lstlisting}
it can set on its own in dependence on the parameters for the coefficients.
The \progclass{SignConvention} parameter can have three different values, 
which indicate which signs have to be respected in the computation of the differentials. 
This way, we avoid sign computations whenever we can.
We set the parameter to \progname{no\_signs} if we only compute the homology up to sign, 
which is the case if we use coefficients in $\Z/2\Z$. 
If we have different coefficients and \progname{m} $\geq 2$, the moduli space $\Modspc[1]$ is non-orientable and the sign convention is set to \progname{all\_signs}. 
Otherwise, it equals \progname{no\_orientation\_sign}. 

At the begin of the computation, the constructor of \progclass{ClusterSpectralSequenceT} is called with respect to the parameters provided above.
Thereby, the bases (see Subsubsection \ref{chapter_program:kappa:css:gen_bases}) are generated, with basis elements sorted by their cluster number.
In particular, printing the $E^0$-term to screen is readily done.

The main task is determining the first and second page of the cluster spectral sequence.
Recall Section \ref{css:section_matrix_version}, which discusses the matrix version of the cluster spectral sequence.
The $p\Th$ transposed transformation matrix of $\del_\E$ is a block matrix of the form
\[ 
    \begin{pmatrix}
        d^1 & d^0 \\
            & d^1   & d^0 \\
            &       & d^1   & d^0 \\
            &       &       &       & \ddots
    \end{pmatrix} \,,
\]
if we sort the basis elements by their number of clusters.
The modules of the $E^1$-term are given by
\[
    \ker(d^0) / \img(d^0)\,,
\]
whereas the modules of the $E^2$-term are given by
\[
    \ker(d^1|_{\ker(d^0)}) / \big( \img(d^0) + \img(d^1|_{\ker(d^0)}) \big) \,.
\]
Hence, it suffices to apply all row operations induced by the sub-matrices $d^0$ and proceed with the diagonalization process of
the parts of the (altered) sub-matrices $d^1$ that are in the kernel of the (diagonalized) sub-matrices $d^0$.
In order to save execution time and memory, our program operates on exactly one $d^0$ and one $d^1$ sub matrix at a time:

\begin{algorithm}[H]
\DontPrintSemicolon
\For{$p = 1$ \KwTo $2h$}
{
    \For{$l=1$ \KwTo $p$}
    {
        Construct $d^1_{p,l}$ and apply row operations of $d^0_{p,l-1}$
        
        Forget $d^0_{p,l-1}$
        
        Generate $d^0_{p,l}$
        
        Compute and save kernel and image of $d^0_{p,l}$
        
        Print the homological results to the screen
        
        Save the diagonal of $d^0_{p,l}$ and detect superflous rows of $d^1_{p,l}$
        
        Compute and save the kernel and image of $d^1_{p,l}$
        
        Print the homological results to the screen
        
        Forget $d^1_{p,l}$
    }
}
Print all three pages of the spectral sequence to the screen.
\caption{Computing $E^1$ and $E^2$}
\end{algorithm}

Here, we have $h = 2g + m$ in the parallel case and $h = 2g + m - 1$ in the radial case.

During all computations, the function \progname{compute\_css} generates the previously mentioned intermediate results 
printed out in the console and into the corresponding output file.
Furthermore, it measures the duration of the important steps of the computations 
and also writes them into the console. 