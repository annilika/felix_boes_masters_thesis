\subsection{The Type DiagonalizerT}
\label{program:libhomology:DiagonalizerT}
Given a differential $C_n \xr{\del} C_{n-1}$ of a chain complex, one wants to derive its kernel and image in order to compute the homology of the chain complex.
In our situation, we are given a differential of the type \progclass{MatrixT}, so we want to apply a range of base changes to end up with a matrix where reading off these informations is easy.
These base changes depend heavily on the coefficient ring.
For field coefficients, one can apply Gaussian elimination, but for integral coefficients, one has to work much harder.
Some state of the art algorithms can be found in \cite{Jaeger2003} and \cite{Jaeger2009}.
We emphasize that this is the most time consuming operation (see Chapter \ref{complexity}) and suggest to carry out an algorithm that makes use of concurrency.

The \progclass{DiagonalizerT} is a function object, so you have to provide the method
\begin{lstlisting}
void operator()( MatrixT& matrix );
\end{lstlisting}
that diagonalizes the given matrix.
Afterwards, kernel and torsion of the matrix should be available by the diagonalizer's member functions
\begin{lstlisting}
HomologyT::KernT kern();
HomologyT::TorsT tors();
\end{lstlisting}
where \progclass{HomologyT} is the class we use to store the homology of a chain complex (compare Subsection \ref{program:libhomology:HomologyT}).

Moreover, we are interested in the defect and the rank of the matrix, so you have to provide the following two member functions.
\begin{lstlisting}
uint32_t dfct();
uint32_t rank();
\end{lstlisting}

\label{program:libhomology:DiagonalizerT:DiagonalizerDummy}

There are situations in which one wants to generate a chain complex without computing homological data:
The size of the differentials of the Ehrenfried complex (see Section \ref{cellular_models:ehrenfried}) is enourmous by Proposition \ref{complexity:number_of_mono_cells},
so it is impossible to compute the number of non-vanishing entries per column by hand.
Therefore we offer the template class
\begin{lstlisting}
template < class MatrixT >
class DiagonalizerDummy;
\end{lstlisting}
that does absolutely nothing, so you can use it together with \progclass{HomologyDummy} (see Subsubsection \ref{program:libhomology:HomologyT:HomologyDummy})
as a template parameter for the template class \progclass{ChainComplexT} (see Subsection \ref{program:libhomology:ChainComplex}).

In the following, we shall describe our implementations of the class \progclass{DiagonalizerField}.

\subsection{The Class DiagonalizerField}
\label{program:libhomology:DiagonalizerT:DiagonalizerField}

The \progclass{DiagonalizerField} applies a slightly modified version of the Gaussian elimination to a given matrix.
Note that for computing the homology of a chain complex with field coefficients,
it is sufficient to determine the rank and defect of all its differentials.
Thus, the class \progclass{DiagonalizerField} merely transforms row operations upon the matrix in order to determine its rank,
but does not exchange columns in order to obtain a triangular matrix.
Since computing the rank is equivalent to diagonalizing for our purpose, we refer to this process as diagonalizing nevertheless.
After giving an overview on the usage of this class, we will explain implementation details and runtime results of our parallelized diagonalization algorithm.

\subsubsection{Overview and Usage of DiagonalizerField}

For field coefficients, we offer the following template class.
\begin{lstlisting}
template < class CoefficientT >
class DiagonalizerField;
\end{lstlisting}
Here we assume that \progclass{MatrixT} is given by
\begin{lstlisting}
typedef MatrixField< CoefficientT > MatrixT;
\end{lstlisting}
and it is trivial to alter the class definition in order to allow arbitrary matrices.

The member variable \progname{current\_rank} of the class \progclass{DiagonalizerField} keeps track of the progress of an ongoing computation as
it stores the number of linearly independent rows the algorithm has already found.
Operations on variables of the type \progclass{atomic\_uint} are atomic, i.e. reading, writing, incrementing and so forth is free of race conditions.
We suggest to make use of this feature as follows.
You start two threads, one computes kernel and torsion and the other monitors the progress.
\begin{lstlisting}
ChainComplex< ... > complex;
// Define the differentials of matrix_complex.
// ...

atomic_uint& rank = diagonalizer.current_rank;
measure_duration = Clock(); // Measures duration.

// Set the value of state to 1 if and only if kernel and torsion are computed.
// This is done to terminate the 'monitoring thread'.
atomic_uint state(0);

// Diagonalizing thread.
auto partial_homology_thread = std::async( std::launch::async, [&]()
{
    auto ret = complex.compute_current_kernel_and_torsion( p );
    state = 1;
    return ret;
} );

// Monitoring thread.
auto monitor_thread = std::async( std::launch::async, [&]()
{
    while( state != 1 )
    {
        std::cout << "Diagonalization " << current_rank << "\r";
        std::this_thread::sleep_for( std::chrono::milliseconds( 50 ) );
    }
} );
\end{lstlisting}
The current progress is printed to screen and updated every 50 milliseconds till the computation is done.

\subsubsection{Implementation Details}
\label{diag_field_implementation}
Our key algorithm for computing the rank of a matrix via Gaussian elimination is given by

\begin{algorithm}[H]
\label{rank}
\DontPrintSemicolon
\SetKw{KwCont}{continue}
\KwIn{A matrix $A = (a_{ij})$ with coefficients in a field $\F$}
\KwOut{The rank $\rk(A)$}

Let $R$ be the set of rows of $A$\;
Set $R_r := \emptyset$ \tcp*[f]{Let $R_r$ denote the set of rows contributing to the rank}\;
\ForEach{column $c$}
{
	Let $j \in R \backslash R_r$ be a row index with $a_{jc}$ invertible in $\F$\;
	\If{No such $j$ exists}
	{
		\KwCont\;
	}
	Let $S \subset R \backslash R_r$ be the subset of rows $s \neq j$ with $a_{sc}$ invertible in $\F$\;
	\If{$S \neq \emptyset$}
	{
		Set $R_r := R_r \cup \{j\}$\;
	}
	\ForEach{row $s \in S$}
	{
		Perform \progname{A.row\_operation(j, s, c)}\;
	}
}
\KwRet{$|R|$}

\caption{Rank Computation}

\end{algorithm}

Hereby, \progname{row\_operation} is the member function of the class \progclass{MatrixType} described in Subsubsection
\ref{program:libhomology:MatrixT:optionals}.

A sequential version of this algorithm is implemented as the member function
\begin{lstlisting}
uint32_t diag_field( MatrixType& matrix );
\end{lstlisting}
of \progclass{DiagonalizerField}.
For the parallelized version, the method
\begin{lstlisting}
uint32_t diag_field_parallelized( MatrixType& matrix );
\end{lstlisting}
is used.
Since -- at least for our matrices of type \progclass{MatrixBool} -- a single row operation is performed very fast,
we do not use several threads to parallelize row operations, 
but subdivide the set of row operations such that several row operations are performed simultaneously.

We define two helper classes for parallelizing, which we will explain here roughly,
using the notation from Algorithm \ref{rank}.
The class \progclass{JobQueue} keeps track of all significant data used in Algorithm \ref{rank}.
Obviously, a \progclass{JobQueue} has to know the
\begin{lstlisting}
 MatrixType & matrix;
\end{lstlisting}
which is supposed to be diagonalized, and the column 
\begin{lstlisting}
 size_t col;
\end{lstlisting}
and row
\begin{lstlisting}
 size_t row_1;
\end{lstlisting}
that are currently considered, where, in the notation of Algorithm \ref{rank}, we have \progname{col = $c$} and \progname{row\_1 = $r$}. 
Furthermore, the \progclass{JobQueue} maintains the 
\begin{lstlisting}
 std::vector rows_to_work_at;
\end{lstlisting}
which resembles the set $S$, and the
\begin{lstlisting}
 std::vector remaining_rows;
\end{lstlisting}
consisting of the rows $t$ not yet contributing to the rank for that the entry $a_{tc}$ is not invertible in $\F$, i.e. of $R\backslash (R_r \cup S)$.
Since the \progclass{JobQueue} contains all the information necessary to perform the required row operations for a given column $c$, only two tasks remain:
updating these data members when passing over from one column to the next
and parallelizing the row operations as well as the update.

For each thread used, we create an instantiation of the class \progclass{Worker}, which will not be discussed in this thesis, to perform computations.
Experiments showed that having two different kinds of \progclass{Workers} is more efficient:
Firstly, we define a family of \progclass{Workers} that actually perform the diagonalizing work.
The \progclass{JobQueue} distributes the rows \progname{rows\_to\_work\_at} among these \progclass{Workers} equally.
Afterwards, each of these \progclass{Workers} considers all its assigned rows $s$, 
performs the row operation upon $s$ and marks whether $s$ will also be in the set $S$ for the next column.
This means that the already defined \progclass{Workers} update parts of the arrays \progname{rows\_to\_work\_at} and \progname{remaining\_rows},
and that it remains to update these arrays with respect to the set of \progname{remaining\_rows}.
This is the task the other family of \progclass{Workers} execute,
where the \progname{remaining\_rows} are again distributed equally among the \progclass{Workers} by the \progclass{JobQueue}.

The input parameters \progname{num\_threads} respectively \progname{num\_remaining\_threads} define 
how many threads are occupied for the first respectively second type of \progclass{Workers}, see also Subsection \ref{chapter_program:kappa:compute_css}.
