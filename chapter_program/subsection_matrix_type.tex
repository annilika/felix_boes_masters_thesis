\subsection{The Type MatrixT}
\label{program:libhomology:MatrixT}
\subsubsection{Existing Template Classes}
Before writing your own \progclass{MatrixT} you may want to have a look at the {\bf ublas library} provided by {\bf boost}.
They offer several matrix templates for sparse and dense matrices 
as well as BLAS implementations for numerical computations.
Moreover, our {\bf libhomology} provides the template class
\begin{lstlisting}
template < class CoefficientT >
class MatrixField;
\end{lstlisting}
for exact computations with matrix coefficients in a given field.
In the following, we present an overview of the requirements any implementation of \progclass{MatrixT} has to fulfill, whereas
in Subsection \ref{program:libhomology:MatrixT:MatrixField_for_F_2_and_css} we treat more specialized implementations.

\subsubsection{Requirements on MatrixT}
Your implementation of \progclass{MatrixT} has to meet some requirements.
These are inspired by the {\bf boost ublas library} as we make heavy use of it to compactify implementation details.
We denote the coefficients of the matrix by \progclass{CoefficientT}.
A \progclass{MatrixT} is created as follows:
\begin{lstlisting}
MatrixT ( size_t number_rows, size_t number_cols );
\end{lstlisting}
The coefficient in the $i\Th$ row and the $j\Th$ column is accessed by calling
\begin{lstlisting}
CoefficientT& operator()( size_t i, size_t j );
\end{lstlisting}
The number of rows is 
\begin{lstlisting}
size_t size1() const;
\end{lstlisting}
and the number of columns is
\begin{lstlisting}
size_t size2() const;
\end{lstlisting}

As we intend to save differentials in our project {\bf kappa}, you should provide a method that stores a \progclass{MatrixT} (See Subsection \ref{program:libhomology:serialization}).

\subsubsection{Optional Requirements on MatrixT}
\label{program:libhomology:MatrixT:optionals}
If you provide your own \progclass{MatrixT} with coefficients in a field, you may want to use our class \progclass{DiagonalizerField} (see Subsection \ref{program:libhomology:DiagonalizerT:DiagonalizerField}) to compute rank and torsion of your matrices.
In order to do so, you have to provide the member function
\begin{lstlisting}
void row_operation( size_t row_1, size_t row_2, size_t col );
\end{lstlisting}
that applies a row operation on the matrix, 
i.e. adds the appropriate multiple of \progname{row\_1} to \progname{row\_2} in order to erase the entry \progname{(row\_2, col)} of the matrix.
Our implementation makes use of multithreading, therefore you have to be careful with race conditions.
You have to ensure that row operations for fixed \progname{row\_1} and \progname{col} with varying \progname{row\_2} can be applied concurrently.

\subsection{Special Implementations of MatrixField}
\label{program:libhomology:MatrixT:MatrixField_for_F_2_and_css}
\subsubsection{MatrixField for Coefficients in \texorpdfstring{$\mathbb F_2$}{Z/2Z}}
For coefficients in the field $\mathbb F_2$, our implementation provides siginificant improvements concerning memory and execution duration, see Section \ref{program:runtime}.
Using well-known techniques, we store multiple entries of a row in a single data entity.
Note that, since the only invertible element in $\mathbb F_2$ is $1$, a row operation corresponds to the bitwise \progname{XOR}-instruction.

Using these insights, we provide an implementation called \progclass{MatrixBool}.
It behaves almost like \progclass{MatrixField} but has a few technical limitations (which are unavoidable as these are direct consequences of the enormous performance gain).
E.g.\ for a matrix of type \progclass{MatrixBool}, it is not possible to access its coefficients by reference.
\begin{lstlisting}
bool operator() ( const size_t i, const size_t j );
bool at         ( const size_t i, const size_t j ) const;
\end{lstlisting}
Observe that for our purpose, it suffices to add $1$ to a given entry which is provided by the method
\begin{lstlisting}
void add_entry( const size_t i, const size_t j );
\end{lstlisting}
It should be easy to equip \progclass{MatrixBool} with more member functions if needed.

\subsubsection{MatrixField for the Cluster Spectral Sequence}
In order to exploit the cluster spectral squence, we provide the adapted version \progclass{MatrixFieldCSS} of \progclass{MatrixField} and also \progclass{MatrixBoolCSS} of \progclass{MatrixBool}.
Here, one should think of a spectral sequence that collapses at the second page as a subdivision of the differentials:
The bases are ordered in a way such that the transposed differential $D$ consists of a diagonal of block matrices $d^0$ which respect the filtration degree and below a single second diagonal of block matrices which decrease the filtration degree by one.
\[
    D = 
        \begin{pmatrix}
            d^1 & d^0 \\
                & d^1   & d^0 \\
                &       & d^1   & d^0 \\
                &       &       &       & \ddots
        \end{pmatrix}
\]
Such a matrix is diagonalized as follows.
We construct the first line given by $d^1$ and $d^0$ in the top left corner.
Then we apply row operations to $d^0$ until its image is determined and then apply row operations to the remaning rows of $d^1$ until the image of the first row is fully understood.
Afterwards, we may forget the matrix $d^1$ in the top left corner, construct the matrix $d^1$ of the next line and apply the needed row operations that are due to the matrix $d^0$ from above.
Now we forget the entire first line, construct the next matrix $d^0$ and iterate this process.

During this procedure, we store at most two submatrices of $D$, namely one of type $d^0$ and one of type $d^1$, so our implementation \progclass{MatrixFieldCSS} and \progclass{MatrixBoolCSS} does exactly the same.
We provide two ways to access the two submatrices $d^0$ and $d^1$.
To use the first approach, the method
\begin{lstlisting}
void define_operations( const OperationType );
\end{lstlisting}
defines on which submatrix we are currently working, where \progclass{OperationType} is an enumeration type and set to be \progclass{main\_and\_secondary} to access $d^0$ or \progclass{secondary} to access $d^1$.
Now one calls member functions of \progclass{MatrixFieldCSS} respectively \progclass{MatrixBoolCSS} which have the same name as the member functions of \progclass{MatrixField} respectively \progclass{MatrixBool}.
Let us give a simple example by printing $d^0$ and $d^1$ to the screen.
\begin{lstlisting}
M.define_operations( main_and_secondary );
std::cout << M; \\ prints d^0 to screen.

M.define_operations( secondary );
std::cout << M; \\ prints d^1 to screen.
\end{lstlisting}


In the second approach, one calls a member method corresponding to $d^0$ by adding the prefix \progclass{main\_}, whereas \progclass{sec\_} applies to $d^1$.
The following listing is an example for the member functions \progname{size1} and \progname{size2}.
\begin{lstlisting}
size_t main_size1() const; // Returns the number of rows of d^0.
size_t main_size2() const; // Returns the number of columns of d^0.
size_t sec_size1() const;  // Returns the number of rows of d^1.
size_t sec_size2() const;  // Returns the number of columns of d^1.
\end{lstlisting}


A row operation on $d^0$ clearly affects the submatrix $d^1$ in the same line.
In the algorithm presented above, we apply only those operations to $d^1$ that leave $d^0$ unchanged.
Therefore we provide the following member functions.
\begin{lstlisting}
void row_operation_main_and_secondary
    ( const size_t row_1, const size_t row_2, const size_t col );
void row_operation_secondary
    ( const size_t row_1, const size_t row_2, const size_t col );
\end{lstlisting}





