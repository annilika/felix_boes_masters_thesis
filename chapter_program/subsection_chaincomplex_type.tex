\subsection{The Class ChainComplex [B]}
\label{program:libhomology:ChainComplex}

We start with the description of the members of the template class
\begin{lstlisting}
template< class CoefficientT,
          class MatrixT,
          class DiagonalizerT,
          class HomologyT >
class ChainComplex;
\end{lstlisting}
A \progclass{ChainComplex} is a finite sequence of differentials which we represent by
\begin{lstlisting}
std::map< int32_t, MatrixT > differential;
\end{lstlisting}
It is reasonable to define the following pass-through methods.
You access the \nth differential by calling
\begin{lstlisting}
MatrixT& operator[]( const int32_t n );
\end{lstlisting}
or its \progkeyword{const} counter part
\begin{lstlisting}
const MatrixT& at( const int32_t n ) const;
\end{lstlisting}
You can test whether the \nth differential is defined by checking whether
\begin{lstlisting}
size_t count( const int32_t n ) const;
\end{lstlisting}
evaluates to zero.
The \nth differential is deleted by the following method.
\begin{lstlisting}
void erase( const int32_t n );
\end{lstlisting}
All homology modules are computed by calling
\begin{lstlisting}
HomologyT homology();
\end{lstlisting}
In order to compute the \nth homology, one calls
\begin{lstlisting}
HomologyT homology( const int32_t n );
\end{lstlisting}
If you are interested in the kernel and torsion parts of the $n\Th$ differential, you should call
\begin{lstlisting}
HomologyT compute_kernel_and_torsion( int32_t n );
\end{lstlisting}
These operations consume by far the most time and 
we recommend using a parallelized diagonalization process (compare Subsection \ref{program:libhomology:DiagonalizerT:DiagonalizerField}).

The \progclass{DiagonalizerT} in use, is accessed via
\begin{lstlisting}
      DiagonalizerT& get_diagonalizer();
const DiagonalizerT& get_diagonalizer() const;
\end{lstlisting}

In order to derive the homology of the moduli spaces, we have to handle very large differentials (compare Section \ref{complexity:number_of_mono_cells}).
Thus we work with a single differential at a time, which is accessed via
\begin{lstlisting}
// Access the current differential.
      MatrixT& get_current_differential();
const MatrixT& get_current_differential() const;

// Erases the current differential.
void erase();

// Access the coefficient of the current differential.
CoefficientT& operator() ( const uint32_t row, const uint32_t col );
    
// Return number of rows resp. columns of the current differential
size_t num_rows() const;
size_t num_cols() const;
\end{lstlisting}

