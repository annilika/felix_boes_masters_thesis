\section{The Library Libhomology}
\label{program:libhomology}

With {\bf libhomology} we provide an expandable framework for all kinds of homology computations.
In our context, it is used as the foundation of the program kappa (see Section \ref{chapter_program:kappa}).
In the following we explain use and essential details of its classes.

The template class \progclass{ChainComplex} is the core of {\bf libhomology}.
Here we think of a chain complex as a finite sequence of compatible matrices $D_n$ satisfying $D_{n-1} D_n = 0$ which we call differential.
We do not mention bases.
Given a \progclass{ChainComplex}, our goal is to determine its homology.
Therefore one needs an implementation of the coefficient ring \progclass{CoefficientT} (see Subsection \ref{program:libhomology:CoefficientT}) and the matrix type \progclass{MatrixT} (see Subsection \ref{program:libhomology:MatrixT}) of the differentials.
The class \progclass{HomologyT} (see Subsection \ref{program:libhomology:HomologyT}) specifies the scope of homological information one wants to extract.
These should be derived using the class \progclass{DiagonalizerT} (see Subsection \ref{program:libhomology:DiagonalizerT}) which diagonalizes matrices by applying row or column operations.

The implementations of the classes \progclass{ChainComplex}, \progclass{MatrixT}, \progclass{DiagonalizerT} and \progclass{HomologyT} are interdependent,
so we recommend to skim over the details on the first reading.

\subsection{The Class ChainComplex}
\label{program:libhomology:ChainComplex}

We start with the description of the members of the template class
\begin{lstlisting}
template< class CoefficientT,
          class MatrixT,
          class DiagonalizerT,
          class HomologyT >
class ChainComplex;
\end{lstlisting}
A \progclass{ChainComplex} is a finite sequence of differentials which we represent by
\begin{lstlisting}
std::map< int32_t, MatrixT > differential;
\end{lstlisting}
It is reasonable to define the following pass-through methods.
You access the \nth differential by calling
\begin{lstlisting}
MatrixT& operator[]( const int32_t n );
\end{lstlisting}
or its \progkeyword{const} counter part
\begin{lstlisting}
const MatrixT& at( const int32_t n ) const;
\end{lstlisting}
You can test whether the \nth differential is defined by checking whether
\begin{lstlisting}
size_t count( const int32_t n ) const;
\end{lstlisting}
evaluates to zero.
The \nth differential is deleted by the following method.
\begin{lstlisting}
void erase( const int32_t n );
\end{lstlisting}
All homology modules are computed by calling
\begin{lstlisting}
HomologyT homology();
\end{lstlisting}
In order to compute the \nth homology, one calls
\begin{lstlisting}
HomologyT homology( const int32_t n );
\end{lstlisting}
If you are interested in the kernel and torsion parts of the $n\Th$ differential, you should call
\begin{lstlisting}
HomologyT compute_kernel_and_torsion( int32_t n );
\end{lstlisting}
These operations consume by far the most time and 
we recommend using a parallelized diagonalization process (compare Subsection \ref{program:libhomology:DiagonalizerT:DiagonalizerField}).

The \progclass{DiagonalizerT} in use, is accessed via
\begin{lstlisting}
      DiagonalizerT& get_diagonalizer();
const DiagonalizerT& get_diagonalizer() const;
\end{lstlisting}

In order to derive the homology of the moduli spaces, we have to handle very large differentials (compare Section \ref{complexity:number_of_mono_cells}).
Thus we work with a single differential at a time, which is accessed via
\begin{lstlisting}
// Access the current differential.
      MatrixT& get_current_differential();
const MatrixT& get_current_differential() const;

// Erases the current differential.
void erase();

// Access the coefficient of the current differential.
CoefficientT& operator() ( const uint32_t row, const uint32_t col );
    
// Return number of rows resp. columns of the current differential
size_t num_rows() const;
size_t num_cols() const;
\end{lstlisting}


\subsection{The Type CoefficientT}
\label{program:libhomology:CoefficientT}

The coefficient ring must be represented by a class that meets the requirements discussed in Subsubsection \ref{program:libhomology:CoefficientT:obligatory_reqs}.

\subsubsection{Obligatory Operations for CoefficientT}
\label{program:libhomology:CoefficientT:obligatory_reqs}
Clearly, one has to provide the basic ring operations.
\begin{lstlisting}
CoefficientT& operator= ( const CoefficientT& );        // Assignment
bool          operator==( const CoefficientT& ) const;  // Comparison
bool          operator!=( const CoefficientT&, const CoefficientT& );
CoefficientT  operator- () const;                       // Negation
CoefficientT& operator+=( const CoefficientT& );        // Addition
CoefficientT  operator+ ( const CoefficientT&, const CoefficientT& );
CoefficientT& operator-=( const CoefficientT& );        // Subtraction
CoefficientT  operator- ( const CoefficientT&, const CoefficientT& );
CoefficientT& operator*=( const CoefficientT& );        // Multiplication
CoefficientT  operator* ( const CoefficientT&, const CoefficientT& );
\end{lstlisting}
If the coefficients form a field, we suggest to implement the division operators.
\begin{lstlisting}
CoefficientT& operator/=( const CoefficientT& );
CoefficientT  operator/ ( const CoefficientT&, const CoefficientT& );
\end{lstlisting}

The integers are initial in the category of rings, thus it is reasonable to implement the following methods.
\begin{lstlisting}
CoefficientT& operator= ( const int );                      // Assignement
bool          operator==( const int ) const;                // Comparision
CoefficientT  operator* ( const CoefficientT&, const int ); // Multiplication
\end{lstlisting}

In our project {\bf kappa}, we intend to save differentials so you should provide a method that stores a \progclass{CoefficientT} (See Subsection \ref{program:libhomology:serialization}).

\subsubsection{Coefficients in the Rationals and the Integers Mod \texorpdfstring{$m$}{m}}
\label{program:libhomology:CoefficientT:our_implementation}
We offer the classes \progclass{Q} respectively \progclass{Zm} that represent coefficients in $\mathbb Q$ respectively $\mathbb Z / m \mathbb Z$:
The class \progclass{Q} is defined via the following \progkeyword{typedef}.
\begin{lstlisting}
typedef mpq_class Q;
\end{lstlisting}
The class \progclass{mpq\_class} itself is the \cpp\ variant of the {\bf GMP} type \progclass{mpq\_t}.
Before using the class \progclass{Zm}, you have to call the static member function
\begin{lstlisting}
static void set_modulus(const uint8_t p, const uint8_t e = 1);
\end{lstlisting}
that defines $m = p^e$.
Omitting the call will throw a segmentation fault which is the result of a division by zero.
The following self-explaining member functions might be useful.
\begin{lstlisting}
static void const print_modulus();    
static void const print_inversetable();
static bool       is_field();
\end{lstlisting}

\subsection{The Type MatrixT}
\label{program:libhomology:MatrixT}
\subsubsection{Existing Template Classes}
Before writing your own \progclass{MatrixT} you may want to have a look at the {\bf ublas library} provided by {\bf boost}.
They offer several matrix templates for sparse and dense matrices 
as well as BLAS implementations for numerical computations.
Moreover, our {\bf libhomology} provides the template class
\begin{lstlisting}
template < class CoefficientT >
class MatrixField;
\end{lstlisting}
for exact computations with matrix coefficients in a given field.
In the following, we present an overview of the requirements any implementation of \progclass{MatrixT} has to fulfill, whereas
in Subsection \ref{program:libhomology:MatrixT:MatrixField_for_F_2_and_css} we treat more specialized implementations.

\subsubsection{Requirements on MatrixT}
Your implementation of \progclass{MatrixT} has to meet some requirements.
These are inspired by the {\bf boost ublas library} as we make heavy use of it to compactify implementation details.
We denote the coefficients of the matrix by \progclass{CoefficientT}.
A \progclass{MatrixT} is created as follows:
\begin{lstlisting}
MatrixT ( size_t number_rows, size_t number_cols );
\end{lstlisting}
The coefficient in the $i\Th$ row and the $j\Th$ column is accessed by calling
\begin{lstlisting}
CoefficientT& operator()( size_t i, size_t j );
\end{lstlisting}
The number of rows is 
\begin{lstlisting}
size_t size1() const;
\end{lstlisting}
and the number of columns is
\begin{lstlisting}
size_t size2() const;
\end{lstlisting}

As we intend to save differentials in our project {\bf kappa}, you should provide a method that stores a \progclass{MatrixT} (See Subsection \ref{program:kappa:serialization}).

\subsubsection{Optional Requirements on MatrixT}
\label{program:libhomology:MatrixT:optionals}
If you provide your own \progclass{MatrixT} with coefficients in a field, you may want to use our class \progclass{DiagonalizerField} (see Subsection \ref{program:libhomology:DiagonalizerT:DiagonalizerField}) to compute rank and torsion of your matrices.
In order to do so, you have to provide the member function
\begin{lstlisting}
void row_operation( size_t row_1, size_t row_2, size_t col );
\end{lstlisting}
that applies a row operation on the matrix, 
i.e. adds the appropriate multiple of \progname{row\_1} to \progname{row\_2} in order to erase the entry \progname{(row\_2, col)} of the matrix.
Our implementation makes use of multithreading, therefore you have to be careful with race conditions.
You have to ensure that row operations for fixed \progname{row\_1} and \progname{col} with varying \progname{row\_2} can be applied concurrently.

\subsection{Special Implementations of MatrixField}
\label{program:libhomology:MatrixT:MatrixField_for_F_2_and_css}
\subsubsection{MatrixField for Coefficients in \texorpdfstring{$\mathbb F_2$}{Z/2Z}}
For coefficients in the field $\mathbb F_2$, our implementation provides siginificant improvements concerning memory and execution duration, see Section \ref{program:runtime}.
Using well-known techniques, we store multiple entries of a row in a single data entity.
Note that, since the only invertible element in $\mathbb F_2$ is $1$, a row operation corresponds to the bitwise \progname{XOR}-instruction.

Using these insights, we provide an implementation called \progclass{MatrixBool}.
It behaves almost like \progclass{MatrixField} but has a few technical limitations (which are unavoidable as these are direct consequences of the enormous performance gain).
E.g.\ for a matrix of type \progclass{MatrixBool}, it is not possible to access its coefficients by reference.
\begin{lstlisting}
bool operator() ( const size_t i, const size_t j );
bool at         ( const size_t i, const size_t j ) const;
\end{lstlisting}
Observe that for our purpose, it suffices to add $1$ to a given entry which is provided by the method
\begin{lstlisting}
void add_entry( const size_t i, const size_t j );
\end{lstlisting}
It should be easy to equip \progclass{MatrixBool} with more member functions if needed.

\subsubsection{MatrixField for the Cluster Spectral Sequence}
In order to exploit the cluster spectral squence, we provide the adapted version \progclass{MatrixFieldCSS} of \progclass{MatrixField} and also \progclass{MatrixBoolCSS} of \progclass{MatrixBool}.
Here, one should think of a spectral sequence that collapses at the second page as a subdivision of the differentials:
The bases are ordered in a way such that the transposed differential $D$ consists of a diagonal of block matrices $d^0$ which respect the filtration degree and below a single second diagonal of block matrices which decrease the filtration degree by one.
\[
    D = 
        \begin{pmatrix}
            d^1 & d^0 \\
                & d^1   & d^0 \\
                &       & d^1   & d^0 \\
                &       &       &       & \ddots
        \end{pmatrix}
\]
Such a matrix is diagonalized as follows.
We construct the first line given by $d^1$ and $d^0$ in the top left corner.
Then we apply row operations to $d^0$ until its image is determined and then apply row operations to the remaning rows of $d^1$ until the image of the first row is fully understood.
Afterwards, we may forget the matrix $d^1$ in the top left corner, construct the matrix $d^1$ of the next line and apply the needed row operations that are due to the matrix $d^0$ from above.
Now we forget the entire first line, construct the next matrix $d^0$ and iterate this process.

During this procedure, we store at most two submatrices of $D$, namely one of type $d^0$ and one of type $d^1$, so our implementation \progclass{MatrixFieldCSS} and \progclass{MatrixBoolCSS} does exactly the same.
We provide two ways to access the two submatrices $d^0$ and $d^1$.
To use the first approach, the method
\begin{lstlisting}
void define_operations( const OperationType );
\end{lstlisting}
defines on which submatrix we are currently working, where \progclass{OperationType} is an enumeration type and set to be \progclass{main\_and\_secondary} to access $d^0$ or \progclass{secondary} to access $d^1$.
Now one calls member functions of \progclass{MatrixFieldCSS} respectively \progclass{MatrixBoolCSS} which have the same name as the member functions of \progclass{MatrixField} respectively \progclass{MatrixBool}.
Let us give a simple example by printing $d^0$ and $d^1$ to the screen.
\begin{lstlisting}
M.define_operations( main_and_secondary );
std::cout << M; \\ prints d^0 to screen.

M.define_operations( secondary );
std::cout << M; \\ prints d^1 to screen.
\end{lstlisting}


In the second approach, one calls a member method corresponding to $d^0$ by adding the prefix \progclass{main\_}, whereas \progclass{sec\_} applies to $d^1$.
The following listing is an example for the member functions \progname{size1} and \progname{size2}.
\begin{lstlisting}
size_t main_size1() const; // Returns the number of rows of d^0.
size_t main_size2() const; // Returns the number of columns of d^0.
size_t sec_size1() const;  // Returns the number of rows of d^1.
size_t sec_size2() const;  // Returns the number of columns of d^1.
\end{lstlisting}


A row operation on $d^0$ clearly affects the submatrix $d^1$ in the same line.
In the algorithm presented above, we apply only those operations to $d^1$ that leave $d^0$ unchanged.
Therefore we provide the following member functions.
\begin{lstlisting}
void row_operation_main_and_secondary
    ( const size_t row_1, const size_t row_2, const size_t col );
void row_operation_secondary
    ( const size_t row_1, const size_t row_2, const size_t col );
\end{lstlisting}






\subsection{The Type DiagonalizerT [B,H]}
\label{program:libhomology:DiagonalizerT}
{\bf [B]}
Given a differential $C_n \xr{\del} C_{n-1}$ of a chain complex, one wants to derive its kernel and image in order to compute the homology of the chain complex.
In our situation, we are given a differential of the type \progclass{MatrixT}, so we want to apply a range of base changes to end up with a matrix where reading off these informations is easy.
These base changes depend heavily on the coefficient ring.
For field coefficients, one can apply Gaussian elimination, but for integral coefficients, one has to work much harder.
Some state of the art algorithms can be found in \cite{Jaeger2003} and \cite{Jaeger2009}.
We emphasize that this is the most time consuming operation (see Chapter \ref{complexity}) and suggest to carry out an algorithm that makes use of concurrency.

The \progclass{DiagonalizerT} is a function object, so you have to provide the method
\begin{lstlisting}
void operator()( MatrixT& matrix );
\end{lstlisting}
that diagonalizes the given matrix.
Afterwards, kernel and torsion of the matrix should be available by the diagonalizer's member functions
\begin{lstlisting}
HomologyT::KernT kern();
HomologyT::TorsT tors();
\end{lstlisting}
where \progclass{HomologyT} is the class we use to store the homology of a chain complex (compare Subsection \ref{program:libhomology:HomologyT}).

Moreover, we are interested in the defect and the rank of the matrix, so you have to provide the following two member functions.
\begin{lstlisting}
uint32_t dfct();
uint32_t rank();
\end{lstlisting}

\label{program:libhomology:DiagonalizerT:DiagonalizerDummy}

There are situations in which one wants to generate a chain complex without computing homological data:
The size of the differentials of the Ehrenfried complex (see Section \ref{cellular_models:ehrenfried}) is enourmous by Proposition \ref{complexity:number_of_mono_cells},
so it is impossible to compute the number of non-vanishing entries per column by hand.
Therefore we offer the template class
\begin{lstlisting}
template < class MatrixT >
class DiagonalizerDummy;
\end{lstlisting}
that does absolutely nothing, so you can use it together with \progclass{HomologyDummy} (see Subsubsection \ref{program:libhomology:HomologyT:HomologyDummy})
as a template parameter for the template class \progclass{ChainComplexT} (see Subsection \ref{program:libhomology:ChainComplex}).

In the following, we shall describe our implementations of the class \progclass{DiagonalizerField}.

\subsection{The Class DiagonalizerField [B,H]}
\label{program:libhomology:DiagonalizerT:DiagonalizerField}

The \progclass{DiagonalizerField} applies a slightly modified version of the Gaussian elimination to a given matrix.
Note that for computing the homology of a chain complex with field coefficients,
it is sufficient to determine the rank and defect of all its differentials.
Thus, the class \progclass{DiagonalizerField} merely transforms row operations upon the matrix in order to determine its rank,
but does not exchange columns in order to obtain a triangular matrix.
Since computing the rank is equivalent to diagonalizing for our purpose, we refer to this process as diagonalizing nevertheless.
After giving an overview on the usage of this class, we will explain implementation details and runtime results of our parallelized diagonalization algorithm.

\subsubsection{Overview and Usage of DiagonalizerField [H]}

For field coefficients, we offer the following template class.
\begin{lstlisting}
template < class CoefficientT >
class DiagonalizerField;
\end{lstlisting}
Here we assume that \progclass{MatrixT} is given by
\begin{lstlisting}
typedef MatrixField< CoefficientT > MatrixT;
\end{lstlisting}
and it is trivial to alter the class definition in order to allow arbitrary matrices.

The member variable \progname{current\_rank} of the class \progclass{DiagonalizerField} keeps track of the progress of an ongoing computation as
it stores the number of linearly independent rows the algorithm has already found.
Operations on variables of the type \progclass{atomic\_uint} are atomic, i.e. reading, writing, incrementing and so forth is free of race conditions.
We suggest to make use of this feature as follows.
You start two threads, one computes kernel and torsion and the other monitors the progress.
\begin{lstlisting}
ChainComplex< ... > complex;
// Define the differentials of matrix_complex.
// ...

atomic_uint& rank = diagonalizer.current_rank;
measure_duration = Clock(); // Measures duration.

// Set the value of state to 1 if and only if kernel and torsion are computed.
// This is done to terminate the 'monitoring thread'.
atomic_uint state(0);

// Diagonalizing thread.
auto partial_homology_thread = std::async( std::launch::async, [&]()
{
    auto ret = complex.compute_current_kernel_and_torsion( p );
    state = 1;
    return ret;
} );

// Monitoring thread.
auto monitor_thread = std::async( std::launch::async, [&]()
{
    while( state != 1 )
    {
        std::cout << "Diagonalization " << current_rank << "\r";
        std::this_thread::sleep_for( std::chrono::milliseconds( 50 ) );
    }
} );
\end{lstlisting}
The current progress is printed to screen and updated every 50 milliseconds till the computation is done.

\subsubsection{Implementation Details [H]}
\label{diag_field_implementation}
Our key algorithm for computing the rank of a matrix via Gaussian elimination is given by

\begin{algorithm}[H]
\label{rank}
\DontPrintSemicolon
\SetKw{KwCont}{continue}
\KwIn{A matrix $A = (a_{ij})$ with coefficients in a field $\F$}
\KwOut{The rank $\rk(A)$}

Let $R$ be the set of rows of $A$\;
Set $R_r := \emptyset$ \tcp*[f]{Let $R_r$ denote the set of rows contributing to the rank}\;
\ForEach{column $c$}
{
	Let $j \in R \backslash R_r$ be a row index with $a_{jc}$ invertible in $\F$\;
	\If{No such $j$ exists}
	{
		\KwCont\;
	}
	Let $S \subset R \backslash R_r$ be the subset of rows $s \neq j$ with $a_{sc}$ invertible in $\F$\;
	\If{$S \neq \emptyset$}
	{
		Set $R_r := R_r \cup \{j\}$\;
	}
	\ForEach{row $s \in S$}
	{
		Perform \progname{A.row\_operation(j, s, c)}\;
	}
}
\KwRet{$|R|$}

\caption{Rank Computation}

\end{algorithm}

Hereby, \progname{row\_operation} is the member function of the class \progclass{MatrixType} described in Subsubsection
\ref{program:libhomology:MatrixT:optionals}.

A sequential version of this algorithm is implemented as the member function
\begin{lstlisting}
uint32_t diag_field( MatrixType& matrix );
\end{lstlisting}
of \progclass{DiagonalizerField}.
For the parallelized version, the method
\begin{lstlisting}
uint32_t diag_field_parallelized( MatrixType& matrix );
\end{lstlisting}
is used.
Since -- at least for our matrices of type \progclass{MatrixBool} -- a single row operation is performed very fast,
we do not use several threads to parallelize row operations, 
but subdivide the set of row operations such that several row operations are performed simultaneously.

We define two helper classes for parallelizing, which we will explain here roughly,
using the notation from Algorithm \ref{rank}.
The class \progclass{JobQueue} keeps track of all significant data used in Algorithm \ref{rank}.
Obviously, a \progclass{JobQueue} has to know the
\begin{lstlisting}
 MatrixType & matrix;
\end{lstlisting}
which is supposed to be diagonalized, and the column 
\begin{lstlisting}
 size_t col;
\end{lstlisting}
and row
\begin{lstlisting}
 size_t row_1;
\end{lstlisting}
that are currently considered, where, in the notation of Algorithm \ref{rank}, we have \progname{col = $c$} and \progname{row\_1 = $r$}. 
Furthermore, the \progclass{JobQueue} maintains the 
\begin{lstlisting}
 std::vector rows_to_work_at;
\end{lstlisting}
which resembles the set $S$, and the
\begin{lstlisting}
 std::vector remaining_rows;
\end{lstlisting}
consisting of the rows $t$ not yet contributing to the rank for that the entry $a_{tc}$ is not invertible in $\F$, i.e. of $R\backslash (R_r \cup S)$.
Since the \progclass{JobQueue} contains all the information necessary to perform the required row operations for a given column $c$, only two tasks remain:
updating these data members when passing over from one column to the next
and parallelizing the row operations as well as the update.

For each thread used, we create an instantiation of the class \progclass{Worker}, which will not be discussed in this thesis, to perform computations.
Experiments showed that having two different kinds of \progclass{Workers} is more efficient:
Firstly, we define a family of \progclass{Workers} that actually perform the diagonalizing work.
The \progclass{JobQueue} distributes the rows \progname{rows\_to\_work\_at} among these \progclass{Workers} equally.
Afterwards, each of these \progclass{Workers} considers all its assigned rows $s$, 
performs the row operation upon $s$ and marks whether $s$ will also be in the set $S$ for the next column.
This means that the already defined \progclass{Workers} update parts of the arrays \progname{rows\_to\_work\_at} and \progname{remaining\_rows},
and that it remains to update these arrays with respect to the set of \progname{remaining\_rows}.
This is the task the other family of \progclass{Workers} execute,
where the \progname{remaining\_rows} are again distributed equally among the \progclass{Workers} by the \progclass{JobQueue}.

The input parameters \progname{num\_threads} respectively \progname{num\_remaining\_threads} define 
how many threads are occupied for the first respectively second type of \progclass{Workers}, see also Subsection \ref{chapter_program:kappa:compute_css}.

\subsection{The Type HomologyT}
\label{program:libhomology:HomologyT}
In order to derive the homology of a chain complex, we compute all kernels and images of the differentials, given by transposed transformation matrices.
In our situation, we start with a \progclass{ChainComplex} (see Subsection \ref{program:libhomology:ChainComplex}) that is essentially a finite series of matrices of the type \progclass{MatrixT} (see Subsection \ref{program:libhomology:MatrixT}).
The function object \progclass{DiagonalizerT} (see Subsection \ref{program:libhomology:DiagonalizerT}) applies row and column operations until both kernel and image can be read off.
This data should then be communicated to \progclass{HomologyT}.

\subsubsection{Essential Members}
The type \progclass{HomologyT} requires the following members.
You have to provide the types \progclass{HomologyT}\progname{::}\progclass{KernT} respectively \progclass{HomologyT}\progname{::}\progclass{TorsT} that store the kernel respectively the image of a differential.
This can be achieved by including the following two lines in the \progkeyword{public} section of the class definition.
\begin{lstlisting}
class HomologyT{
public:
    typedef /* ... */ KernT;
    typedef /* ... */ TorsT;
};
\end{lstlisting}
You have to define the following two constructors
\begin{lstlisting}
HomologyT ();
HomologyT ( int32_t n, KernT& k, TorsT& t ); // Sets k and t at the spot n.
\end{lstlisting}
and member functions for storing kernels and images.
\begin{lstlisting}
void set_kern ( int32_t , KernT& );
void set_tors ( int32_t , TorsT& );
\end{lstlisting}
Moreover, we want to print the homology to the screen, thus the class definition has to include the following line.
\begin{lstlisting}
friend std::ostream& operator<< ( std::ostream& , const HomologyT& );
\end{lstlisting}

\subsubsection{The Class HomologyDummy}
\label{program:libhomology:HomologyT:HomologyDummy}
As mentioned in Subsection \ref{program:libhomology:DiagonalizerT:DiagonalizerDummy}, there are situations in which one wants to generate a chain complex without computing homological data.
For this purpose, we offer the class
\begin{lstlisting}
class HomologyDummy
\end{lstlisting}
that does absolutely nothing, so you can use it together with \progclass{DiagonalizerDummy} (see Subsection \ref{program:libhomology:DiagonalizerT:DiagonalizerDummy})
as a template parameter for the template class \progclass{ChainComplexT} (see Subsection \ref{program:libhomology:ChainComplex}).

\subsubsection{The Class HomologyField}
Using field coefficients, the homology modules are all vector spaces.
For those who are only interested in the Betti numbers, we offer the class \progclass{HomologyField}.
Here we store only the dimensions of kernel and image.
The class definition is essentially as follows, where the member functions should be self-explaining.
\begin{lstlisting}
class HomologyField{
public:
    typedef int64_t KernT;
    typedef int64_t TorsT;
    
    HomologyField ();
    HomologyField ( int32_t, KernT, TorsT );
    
    void  set_kern   ( int32_t, KernT );
    void  set_tors   ( int32_t, TorsT );
    KernT get_kern   ( int32_t ) const;
    TorsT get_tors   ( int32_t ) const;
    void  erase_kern ( int32_t );
    void  erase_tors ( int32_t );
    friend std::ostream& operator<< ( std::ostream&, const HomologyField& );
private:
    std::map< int32_t, int64_t > kern_rep;
    std::map< int32_t, int64_t > tors_rep;
};
\end{lstlisting}


\subsection{Serialization}
\label{program:kappa:serialization}

The transformation of data and objects of a running computer program into storable information (which can be saved on a hard disk) is called serialization.
In our project, we want to generate basis elements and differentials to save them for later use, i.e.\ we want to serialize them.
Therefore, we provide the general purpose template functions
\begin{lstlisting}
template < class StorableType >
void save_to_file_bz2
    ( StorableType& t, std::string filename, bool print_duration=true );
\end{lstlisting}
and
\begin{lstlisting}
template < class StorableType >
void load_from_file_bz2
    ( StorableType& t, std::string filename, bool print_duration=true);
\end{lstlisting}
Both functions, use the \progname{bzip2} algorithm for fast file compresseion, so we produce much smaller files.

\subsubsection{Storable Types}
A type can be handled by the above template functions, if
it is defined by the \cppeleven\ standard (such as \progclass{int} or \progclass{std}::\progclass{vector}\progname{<} \progclass{char} \progname{>}) or the {\bf boost \cpp library}.
Otherwise your class has to meet the following conditions.
First of all, it has to be a friend of \progclass{boost}\progname{::}\progclass{serialization}\progname{::}\progclass{access}, i.e.\ its class description includes the following line.
\begin{lstlisting}
class my_class
{
    ...
    friend class boost::serialization::access;
    ...
};
\end{lstlisting}
There are two ways of proceeding from here.
Most commonly, the process of saving and loading is the same.
In this situation one defines the template member function
\begin{lstlisting}
template < class Archive >
void serialize( Archive &ar, const unsigned int version );
\end{lstlisting}
whereas the template parameter \progclass{Archive} is filled in by {\bf boost} (and you do not need to know what it is).
Now saving and loading is achieved via the binary operator \progname{\&} applied to \progname{ar} and every member variable we want to save and load.
In the example below, we tell the serialization function, to save and load two out of three member variables.
\begin{lstlisting}
class my_class
{
    ...
    int keep_this;
    int this_is_also_important;
    int useless;
    
    friend class boost::serialization::access;
    template < class Archive >
    void serialize( Archive &ar, const unsigned int )
    {
        ar & keep_this;
        ar & this_is_also_important;
    }
};
\end{lstlisting}

There are rare situations, in which a saving mechanism differs from its loading counter part.
Then one has to implement the function
\begin{lstlisting}
template< class Archive >
void save( Archive & ar, const unsigned int version ) const;
\end{lstlisting}
and its counter part
\begin{lstlisting}
template< class Archive >
void load( Archive & ar, const unsigned int version );
\end{lstlisting}
and one has to tell {\bf boost} to use these two functions by calling the macro
\begin{lstlisting}
BOOST_SERIALIZATION_SPLIT_MEMBER()
\end{lstlisting}
afterwards. A simple example could look like this.
\begin{lstlisting}
class my_class
{
    ...
    int keep_this;
    int this_is_also_important;
    int useless;
    
    friend class boost::serialization::access;
    template< class Archive >
    void save( Archive & ar, const unsigned int ) const
    {
        ar & keep_this & this_is_also_important;
    }
    template< class Archive >
    void load( Archive & ar, const unsigned int )
    {
        ar & keep_this & this_is_also_important;
        useless=0;
    }
    BOOST_SERIALIZATION_SPLIT_MEMBER()
};
\end{lstlisting}
For more information one should consider the online manual of \cite{boost}.
