\subsection{The Class Tuple}
\label{chapter_program:kappa:tuple}

Recall that we aim to compute the cohomology of the parallel or radial Ehrenfried complex $\Ehrprog$.
A basis for the Ehrenfried complex is given by monotonous cells $\Sigma = (\tau_h \mid \ldots \mid \tau_1)$ satisfiying certain conditions,
and the differential for $\Ehrprog$ can be described by the map $\del_\E$ making the diagram
\[
    \begin{tikzcd}
	\E_p \arrow{r}{\del_\E} \arrow{d}{\kappa}[swap]{\cong}      & \E_{p-1} \\
	\KK_p \arrow{r}{\del_\KK}                                     & \KK_{p-1} \arrow{u}{\cong}[swap]{\pi}
    \end{tikzcd}
\]
commute, compare Definition \ref{cellular_models:Ehrenfried:defi}.
Hence, the core of the program \textbf{kappa} is the class \progclass{Tuple}, 
which represents a tuple of $h$ transpositions $\Sigma = (\tau_h \mid \ldots \mid \tau_1)$
and thus especially the basis elements of $\Ehrprog$.
This class additionally provides several methods which are applied to a basis element during the computation of the differential $\del_\E$.

We are going to give an overview on its data members (Subsubsection \ref{program:kappa:tuple:members}),
the member functions needed to start working with a \progclass{Tuple} (Subsubsection \ref{program:kappa:tuple:get_started}) and 
the functions that represent basic properties of a tuple (Subsubsection \ref{program:kappa:tuple:basics}). 
Afterwards, we explain the class methods computing the orientation sign (Subsubsection \ref{program:kappa:tuple:orientation_sign}), 
the horizontal differential (Subsubsection \ref{program:kappa:tuple:d_hor}) and the implementation of the maps $f$ and $\Phi$ used to compute the isomorphism $\kappa$ (Subsubsection \ref{program:kappa:tuple:prep_for_kappa}) in detail. 

\subsubsection{Data Members}
\label{program:kappa:tuple:members}
Since the class \progclass{Tuple} is supposed to represent cells $\Sigma = (\tau_h \mid \ldots \mid \tau_1)$ of the Ehrenfried complex $\Ehrprog$, 
let us briefly recall what this means. 
If $\Sigma$ is a cell of the parallel Ehrenfried complex 
(compare Definitions \ref{cellular_models:parallel:inhomogeneous_notation}, \ref{cellular_models:ehrenfried}), 
it satisfies the following properties:
\begin{enumerate}
 \item[(i$_P$)] The transpositions $\tau_i$ act on the symbols $1, \dotsc, p$.
 \item[(ii$_P$)] The permutation $\sigma_h$ has exactly $m+1$ cycles.
 \setcounter{enumi}{2}
 \item Each symbol $1, \dotsc, p$ is contained in at least one $\tau_i$.
 \item $\Sigma$ is monotonous, i.e. $\height(\tau_h) \geq \dotsc \geq \height(\tau_1)$.
\end{enumerate}
On the other hand, if $\Sigma$ is a radial cell, it fulfills the same conditions, 
except for the first two, which are replaced by
\begin{enumerate}
 \item[(i$_R$)] The transpositions $\tau_i$ act on the symbols $0, \dotsc, p$,
 \item[(ii$_R$)] The permutation $\sigma_h$ has exactly $m$ cycles.
\end{enumerate}
see also Definitions \ref{cellular_models:radial:cells_in_inhomogenous_notation} and \ref{cellular_models:Ehrenfried:defi}.

Let us now see how the cell $\Sigma$ is represented by a \progclass{Tuple}.
The transpositions $\tau_h, \dotsc, \tau_1$ belonging to $\Sigma$ are stored as the data member
\begin{lstlisting}
std::vector< Transposition > rep;
\end{lstlisting}
where a \progclass{Transposition} is defined as a pair of unsigned integers.
The unsigned integer \progname{p} is stored as another data member of the class \progclass{Tuple},
and there is also a boolean \progname{radial} indicating whether $\Sigma$ is a radial cell or not (and thus a parallel cell).
Depending on this flag, there are other technical parameters to be set, 
e.g. the minimum symbol that may occur in a transposition, see conditions (i$_P$) and (i$_R$).

Our convention for writing a cell $\Sigma = (\tau_h \mid \ldots \mid \tau_1)$ is 
to store $\tau_i$ as \progname{rep[i-1]} for $1 \leq i \leq h$, since by default, a vector starts with the index $0$. 
To make the handling of \progclass{Tuples} more intuitive, 
we offer methods to access the \progclass{Transpositions} in a more canonical way, see Subsubsection \ref{program:kappa:tuple:get_started}. 
Another convention is that we always write \progclass{Transpositions} like $\tau_i = (a\ b)$ with $a > b$ in order to simplify the source code.

Since it is useful for the computation of the differential, 
another data member of the class \progclass{Tuple} is an unsigned integer \progname{id} indicating the index of a \progclass{Tuple} in its basis of the Ehrenfried complex.

\subsubsection{Class Methods To Get Started}
\label{program:kappa:tuple:get_started}

We define two constructors for the class \progclass{Tuple}. 
Firstly, there is a constructor
\begin{lstlisting}
Tuple( size_t h );
\end{lstlisting}
which initializes \progname{p} with $0$ and allocates memory for a \progclass{Tuple} of $h$ \progclass{Transpositions} 
that are supposed to be filled later. 
Secondly, the constructor
\begin{lstlisting}
Tuple( uint32_t symbols, size_t h );
\end{lstlisting}
sets the data member \progmember{p} to be $symbols$ and also allocates memory for the $h$ many \progclass{Transpositions}. 

To create a \progclass{Tuple} $\Sigma = (\tau_h \mid \ldots \mid \tau_1)$, one can call one of these constructors, 
and afterwards initialize for each $i = 1, \dotsc, h$ the $i\Th$ \progclass{Transposition} $\tau_i$ using the non-const operator 
\begin{lstlisting}
Transposition& operator[]( size_t i );
\end{lstlisting}
It is also possible to use the const respectively non-const version of the method
\begin{lstlisting}
Transposition& at( size_t i );
\end{lstlisting}
to access the \progclass{Transposition} $\tau_i$ of the tuple $\Sigma$ for reading respectively writing. 

During the generation of the differential $\del_\E$, we use to mark \progclass{Tuples} as degenerate by
erasing its data member \progname{rep}.
To test whether a \progclass{Tuple} is non-degenerate, we hence define  
\begin{lstlisting}
operator bool();
\end{lstlisting}
which returns true if and only if this \progclass{Tuple} is non-empty.

Using the methods
\begin{lstlisting}
static void parallel_cell(); 
\end{lstlisting}
respectively
\begin{lstlisting}
static void radial_cell();
\end{lstlisting}
one can mark whether a \progclass{Tuple} represents a parallel respectively radial cell.

Furthermore, we define canonical compare operators, 
overload the \progname{operator<<} to print \progname{Tuples} to screen
and offer the possibility to save and load \progclass{Tuples}.

\subsubsection{Basic Properties of a Tuple}\label{program:kappa:tuple:basics}

We provide various class methods for basic calculations with a \progclass{Tuple},
e.g. to verify whether a \progclass{Tuple} represents an element of the Ehrenfried complex. 

Since basis elements of $\Ehrprog$ are required to be monotonous, we provide a method
\begin{lstlisting}
bool monotonous();
\end{lstlisting}
to test whether a \progclass{Tuple} is monotonous, i.\,e. whether, for all $1 \leq i < h$, we have $a_i < a_{i+1}$, 
writing $\tau_i = (a_i, b_i)$ and $\tau_{i+1} = (a_{i+1}, b_{i+1})$ with $a_i > b_i$ and $a_{i+1} > b_{i+1}$.

The class method
\begin{lstlisting}
int32_t norm();
\end{lstlisting}
returns the norm of the given \progclass{Tuple} $\Sigma = (\tau_h \mid \ldots \mid \tau_1)$, 
i.\,e. the sum of the norms of all $\tau_i$, $i = 1, \dotsc, h$. 
But since each $\tau_n$ is a \progclass{Transposition}, the norm of the \progclass{Tuple} is simply the number of its transpositions $h$. 

For the computation of the differential, we sometimes want to switch to the homogeneous notation of \progclass{Tuples}. 
Therefore, we need methods
\begin{lstlisting}
Permutation long_cycle(); 
\end{lstlisting}
respectively
\begin{lstlisting}
Permutation long_cycle_inv();
\end{lstlisting}
returning the \progclass{Permutation} $\sigma_0 = (1\ 2\ \dotsb p-1\ p)$ respectively its inverse, and
\begin{lstlisting}
Permutation sigma_h();
\end{lstlisting}
returning the \progclass{Permutation} $\sigma_h = \tau_h \tau_{h-1} \dotsc \tau_1 \sigma_0$.
Thereby, a \progclass{Permutation} is another class, representing a permutation as a vector
storing for each element its image under the permutation.

Since all $\tau_i$ are \progclass{Transpositions}, 
we can simplify the computation of $\sigma_h$ using the following

\begin{algorithm}[H]
\label{sigma_q}
\DontPrintSemicolon

\KwIn{A tuple $\Sigma = (\tau_h, \dotsc, \tau_1)$ in inhomogeneous notation}
\KwOut{The permutation $\sigma_h$}

\progclass{Permutation} $\sigma_{\text{inv}} := \text{long\_cycle\_inv()}$\;
\For{$i = 1$ \KwTo $h$}
{
	Write $\tau_i = (a, b)$\;
	Swap the elements $\sigma_{\text{inv}}(a)$ and $\sigma_{\text{inv}}(b)$\;
}

\For{$j = 1$ \KwTo $p$}
{
	Write $k = \sigma_{\text{inv}}(j)$\;
	$\sigma_h(k) := j$\;
}

\KwRet{$\sigma_h$}\;

\caption{Computing $\sigma_h$}

\end{algorithm}

\begin{prop}
The given algorithm computes $\sigma_q$ correctly.
\begin{proof}
In line 1, we initialize the \progclass{Permutation} $\sigma_{\text{inv}}$ with $\sigma_0^{-1}$. 
Note that, by definition of $\sigma_i$, we have
\[ \sigma_i^{-1} = \sigma_{i-1}^{-1} \tau_i = \sigma_{i-1}^{-1}  (a, b) \]
for $0 < i \leq h$. 
Thus, the \progclass{Permutation} $\sigma_i^{-1}$ maps $a$ to $\sigma_{i-1}^{-1}(b)$, $b$ to $\sigma_{i-1}^{-1}(a)$ 
and behaves like $\sigma_{i-1}^{-1}$ on all other elements. 
Hence, for each $i = 1, \dotsc, h$, we have $\sigma_{\text{inv}} = \sigma_i^{-1}$ 
after the $i\Th$ iteration of the first for loop. 

The second for loop computes the inverse of $\sigma_{\text{inv}} = \sigma_h^{-1}$, which is $\sigma_h$.
\end{proof}
\end{prop}

The algorithm for the computation of $\sigma_h$ is also used with small adaptions in different parts of the program.

The basis elements of $\Ehrprog$ are supposed to have a distinguished number of punctures, 
i.e. number of cycles of $\sigma_h$ is $m+1$ for parallel and $m$ for radial cells.
Therefore, we give an algorithm to determine the cycle decomposition of a \progclass{Permutation}.

\begin{algorithm}[H]
\label{Cycle Decomp}
\DontPrintSemicolon
\SetKw{KwGoTo}{go to}

\KwIn{A permutation $\pi$ on a subset of $\{0, \dotsc, p\}$}
\KwOut{A decomposition of $\pi$ into disjoint cycles}

$i := \min\{j \in \{0, \dotsc, p\} \colon j \text{ belongs to $\pi$, but not visited yet}\}$ \label{next cycle}\;
cur := $i$\; 
Initialize a new cycle\;
\Repeat(\tcp*[f]{Find the cycle containing $i$}){ cur equals $i$}
{	
  Mark cur as visited\;
	prev := cur\;
	cur := $\pi$(prev)\;
	cycle (prev) := cur\;
}
Store the cycle\;
\If{all symbols in $\{0, \dotsc, p\}$ visited or not belonging to $\pi$}
{
  \KwRet the cycle decomposition\; 
}
\KwGoTo \ref{next cycle}\;

\caption{Cycle Decomposition}

\end{algorithm}

Since each element in $\{0, \dotsc, p\}$ belonging to the permutation $\pi$ is considered as $prev$ exactly once, we can state

\begin{prop}
The algorithm to determine the cycle decomposition of a permutation works correctly.
\end{prop}

Combining this with the previous Algorithm \ref{sigma_q}, we can now define the methods
\begin{lstlisting}
uint32_t num_cycles() const;
\end{lstlisting}
yielding the number of cycles of $\sigma_h$, and
\begin{lstlisting}
bool has_correct_num_cycles(size_t m) const;
\end{lstlisting}
checking whether the number of cycles fits the requirement of the parallel respectively radial Ehrenfried complex.

Since we need to subdivide the cells of $\Ehrprog$ according to their numbers of clusters, 
we introduce the method
\begin{lstlisting}
int32_t Tuple::num_clusters() const;
\end{lstlisting}
This is another part of our computer program where we use the comfortability of the \textbf{boost} library:
Let $\Sigma = (\tau_h \mid \dotsc \mid \tau_1)$ be a cell represented by a \progclass{Tuple}.
We construct a graph on the vertices $0, \dotsc, p$,
where an edge between $a$ and $b$ indicates that there is an $i$ such that $\tau_i = (a, b)$.
Then, the number of connected components of this graph equals the number of clusters of $\Sigma$,
and \textbf{boost} offers a graph data structure and an algorithm to compute this directly.

\subsubsection{The Horizontal Face Operator}
\label{program:kappa:tuple:d_hor}

In order to construct the Ehrenfried complex $\Ehrprog$, 
the computer program has to be able to apply the horizontal differential $\del'' = \del_\KK$ to \progclass{Tuples}, 
compare Definition \ref{cellular_models:Ehrenfried:defi} and Section \ref{cellular_models:orientation}.
So recall that, for a cell $\Sigma$, this differential is given by the alternating sum
\[
 \del''_j(\Sigma) = \sum_{j = 0}^p (-1)^j \eps_j(\Sigma) d_j''(\Sigma)\,,
\]
where $d_j''(\Sigma)$ is the $j\Th$ horizontal face of $\Sigma$, 
i.e. the cell resulting from $\Sigma$ by collapsing the $j\Th$ stripe of the slit domain,
and $\eps_j(\Sigma) = \eps_j''(\Sigma)$ is the additional sign introduced by the orientation system.
Note that, for parallel cells -- but not for radial cells --, the $0\Th$ and $p\Th$ horizontal face is always degenerate.

Here, we will explain our implementation of the face operator $d''$, 
and in Subsubsection \ref{program:kappa:tuple:orientation_sign}, we will present the orientation sign.

In Proposition \ref{cellular_models:parallel:prop_dh}, 
we saw how to express the formula for the $j\Th$ horizontal face in the inhomogeneous notation,
and in Corollary \ref{cellular_models:ehrenfried:cor_d_hor_deg}, we saw how to detect the cases when the resulting face is degenerate.
These statements result in the following

\begin{algorithm}[H]
\label{d_hor}
\DontPrintSemicolon
\SetKw{KwGoTo}{go to}

\KwIn{A tuple $\Sigma = (\tau_h, \dotsc, \tau_1)$ in inhomogeneous notation, a symbol $j \in \{0, \dotsc, p\}$}
\KwOut{The horizontal face $d''_j(\Sigma)$ or the assertion that $d''_j(\Sigma)$ is degenerate}
\For{$i = 1$ \KwTo $h$}
{
	Write $\tau_i = (a, b)$\;
	Write $k = \sigma_{i-1}(j)$\;
	\If{The transpositions $\tau_i$ and $(j, k)$ are disjoint}
	{
	  $\tau'_i := \tau_i$\;
	}
	\ElseIf{$\tau_q = (j, k)$ or $j = k$}
	{
	  \KwRet{$d''_j(\tau)$ is degenerate}\;
	}
	\Else
	{
	  \If{$k = a$ or $k = b$}
	  {
	    $\tau'_i := \tau_i$\;
	  }
	  \Else
	  {
	    \If{$a \neq k$}
	    {
	      $\tau_i' := (a, k)$\; 
	    }
	    \Else
	    {
	      $\tau_i' := (b, k)$\;
	    }
	  }
	}
}
Renormalize $\Sigma' = (\tau_h' \mid \ldots \mid \tau_1')$\;
\KwRet{$\Sigma'$}\;

\caption{Computing the Horizontal Face}

\end{algorithm}

Using the above mentioned theoretical foundations, we obtain

\begin{prop}
The above algorithm computes the $j\Th$ horizontal face of $\Sigma$.
\end{prop}

Note that we can apply Algorithm \ref{sigma_q} to compute $\sigma_i$ for $i = 0, \dotsc, h$.
We choose to handle the case that $\tau_i$ and $(j, k)$ are disjoint before the degenerate case at first
because this is the case that will occure most likely.

The algorithm enables us to define the method
\begin{lstlisting}
Tuple d_hor( uint8_t j );
\end{lstlisting}
computing the $j\Th$ horizontal face of this \progclass{Tuple}.

\subsubsection{The Orientation Sign}\label{program:kappa:tuple:orientation_sign}

Recalling the definition of the orientation sign $\eps_j(\Sigma) = \eps_j''(\Sigma)$ introduced by Mehner (see Section \ref{cellular_models:orientation}), 
we immediately obtain the following

\begin{algorithm}[H]
\label{orientation_sign}
\DontPrintSemicolon
\SetKw{KwGoTo}{go to}

\KwIn{A tuple $\Sigma = (\tau_h, \dotsc, \tau_1)$ in inhomogeneous notation}
\KwOut{Orientation signs $\eps_j(\Sigma)$ for all $j \in \{0, \dotsc, p\}$}

Decompose $\sigma_h$ into $m$ resp. $m+1$ disjoint cycles $(\alpha_0) \alpha_1 \dotsc \alpha_m$ \;
Let $a_i$ be the minimum symbol of the cycle $\alpha_i$\;
Sort the cycles $\alpha_i$ such that $a_i < a_{i+1}$ for all $i$\;
\ForEach{cycle $\alpha_i$}
{
  \If{$\alpha_i = a_i$ is a fixed point}
  {
    Set $\eps_{a_i} = 0$\;
  }
  Let $b$ be the second minimum cycle of $\alpha_i$\;
  Let $k \geq i$ be the minimum integer with $b < a_{k+1}$, or $k = m$, if this does not exist\;
  Set $\eps_{a_i} = (-1)^{k - i}$\;
  \ForEach{$c \in \alpha_i$, $c \neq a_i$}
  {
    Set $\eps_c = 1$\;
  }
}

\caption{Computing the Orientation Sign}

\end{algorithm}

Hereby, we again use Algorithm \ref{sigma_q} to determine $\sigma_h$, 
and Algorithm \ref{Cycle Decomp} to decompose $\sigma_h$ into disjoint cycles.
Note that, in the parallel case, a cell of $\Ehrprog$ has $m+1$ cycles, while in the radial case, it has $m$ cycles.

Storing the cycle decomposition of $\sigma_h$ as a map of cycles stored with their smallest element as a key,
this algorithm is easily implemented.
We obtain the method
\begin{lstlisting}
std::map< uint8_t, int8_t > orientation_sign();
\end{lstlisting}
returning a map of all orientation signs $\eps_j(\sigma_h)$, stored with $j \in \{1, \dotsc, p\}$ as a key.

\subsubsection{Preparations for the Map \texorpdfstring{$\kappa$}{kappa}}
\label{program:kappa:tuple:prep_for_kappa}

Having seen how the differential $\del_\KK$ is implemented, 
recall once more that the differential $\del_\E$ of the Ehrenfried complex is given by the diagram
\[
    \begin{tikzcd}
	\E_p \arrow{r}{\del_\E} \arrow{d}{\kappa}[swap]{\cong}      & \E_{p-1} \\
	\KK_p \arrow{r}{\del_\KK}                                     & \KK_{p-1} \arrow{u}{\cong}[swap]{\pi}
    \end{tikzcd}\,.
\]
Note that the projection $\pi$ onto the monotonous cells can be performed by 
simply checking whether the \progclass{Tuple} is monotonous (see Subsection \ref{program:kappa:tuple:basics})
and marking the \progclass{Tuple} as non-valid, if not.
It remains to define the isomorphism $\kappa$, which is given by
\[
    \kappa = K_h \circ \ldots \circ K_1
\]
with
\[
    K_q = \sum_{j=1}^q (-1)^{q-j} \Phi_{j}^q
\]
and
\[
    \Phi_j^q = f_j \circ \ldots f_{q-1} \,.
\]

Therefore, the class \progclass{Tuple} provides maps
\begin{lstlisting}
bool f( uint32_t j );
\end{lstlisting}
and 
\begin{lstlisting}
bool phi( uint32_t q, uint32_t j );
\end{lstlisting}
which apply the maps $f_j$ respectively $\Phi_j^q$ to this \progclass{Tuple} 
and return true if and only if the resulting \progclass{Tuple} is non-degenerate. 

The method $f$ is implemented as a huge case distinction 
concerning the symbols contained in the transpositions $\tau_j$,
which is similar to the one in the computation of the horizontal boundary. 
This provides the opportunity to handle each of the cases in constant time.

The function $\phi(q, j)$ iteratively calls the function $f$ for $j = 1, \dotsc, q-1$
according to the definition of $\Phi_j^q$.
In each step, we test whether the norm of the \progclass{Tuple} decreases, and if so, 
we abort the computation of $\Phi$ to avoid unneccassary computations.

The map $\kappa$ itself is defined in the class \progclass{ClusterSpectralSequence}, compare Subsection \ref{chapter_program:kappa:css}.