\subsection{The Type CoefficientT}
\label{program:libhomology:CoefficientT}

The coefficient ring must be represented by a class that meets the requirements discussed in Subsubsection \ref{program:libhomology:CoefficientT:obligatory_reqs}.

\subsubsection{Obligatory Operations for CoefficientT}
\label{program:libhomology:CoefficientT:obligatory_reqs}
Clearly, one has to provide the basic ring operations.
\begin{lstlisting}
CoefficientT& operator= ( const CoefficientT& );        // Assignment
bool          operator==( const CoefficientT& ) const;  // Comparison
bool          operator!=( const CoefficientT&, const CoefficientT& );
CoefficientT  operator- () const;                       // Negation
CoefficientT& operator+=( const CoefficientT& );        // Addition
CoefficientT  operator+ ( const CoefficientT&, const CoefficientT& );
CoefficientT& operator-=( const CoefficientT& );        // Subtraction
CoefficientT  operator- ( const CoefficientT&, const CoefficientT& );
CoefficientT& operator*=( const CoefficientT& );        // Multiplication
CoefficientT  operator* ( const CoefficientT&, const CoefficientT& );
\end{lstlisting}
If the coefficients form a field, we suggest to implement the division operators.
\begin{lstlisting}
CoefficientT& operator/=( const CoefficientT& );
CoefficientT  operator/ ( const CoefficientT&, const CoefficientT& );
\end{lstlisting}

The integers are initial in the category of rings, thus it is reasonable to implement the following methods.
\begin{lstlisting}
CoefficientT& operator= ( const int );                      // Assignement
bool          operator==( const int ) const;                // Comparision
CoefficientT  operator* ( const CoefficientT&, const int ); // Multiplication
\end{lstlisting}

In our project {\bf kappa}, we intend to save differentials so you should provide a method that stores a \progclass{CoefficientT} (See Subsection \ref{program:kappa:serialization}).

\subsubsection{Coefficients in the Rationals and the Integers Mod \texorpdfstring{$m$}{m}}
\label{program:libhomology:CoefficientT:our_implementation}
We offer the classes \progclass{Q} respectively \progclass{Zm} that represent coefficients in $\mathbb Q$ respectively $\mathbb Z / m \mathbb Z$:
The class \progclass{Q} is defined via the following \progkeyword{typedef}.
\begin{lstlisting}
typedef mpq_class Q;
\end{lstlisting}
The class \progclass{mpq\_class} itself is the \cpp\ variant of the {\bf GMP} type \progclass{mpq\_t}.
Before using the class \progclass{Zm}, you have to call the static member function
\begin{lstlisting}
static void set_modulus(const uint8_t p, const uint8_t e = 1);
\end{lstlisting}
that defines $m = p^e$.
Omitting the call will throw a segmentation fault which is the result of a division by zero.
The following self-explaining member functions might be useful.
\begin{lstlisting}
static void const print_modulus();    
static void const print_inversetable();
static bool       is_field();
\end{lstlisting}
