\chapter{The Software Project}
\label{program}

Based on the foundations presented in this thesis, 
we provide a computer program for determining the homology of the moduli spaces $\Modspc[1]$ and $\ModspcRad[1]$.
Our software project aims at several goals.
Above all, we put a lot of effort into optimizing the performance of our program
since in Section \ref{complexity:number_of_mono_cells}, we see that our homology computations demand 
an economical use of memory and running time.  
Secondly, our objective is to present software which can easily be adapted to future methods of homology computations. 
Therefore, we designed well documented program code with a modularized structure 
that allows to exchange, improve or extend the different aspects of the homology computation smoothly.
For instance, our program can readily be extended to computing the homology of $\Modspc$ and $\ModspcRad$ for arbitrary $n$.

The project is split into two almost independent units.
The library {\bfseries libhomology} offers an extendible framework for generic homology computations applied to chain complexes and certain types of spectral sequences that collapse at the second page.
We propose coefficient ring, matrix types and algorithms for diagonalizing matrices.

We provide an implementation for rational and finite fields coefficients together with matrices that are diagonalized using parallelized Gaussian elemination.
For coefficients in the field $\mathbb F_2$, we create an own coefficient type based on the observation that they can be stored in a single bit.
This results in vast improvements concerning memory and runtime, see also Section \ref{program:runtime}.

Recall that the homology of $\Modspc[1]$ respectively $\ModspcRad[1]$ equals the cohomology of the parallel respectively radial Ehrenfried complex $\E(h, m;1)$ respectively $\E(h, m, 1)$
(compare Section \ref{cellular_models:ehrenfried}).
Since our program can deal with both cases, we shall denote by $\Ehrprog$ the radial or parallel Ehrenfried complex during this chapter.
Thus, the program {\bfseries kappa} generates $\Ehrprog$ and computes some of its properties.
Most importantly, we use our library {\bfseries libhomology} to derive its cohomology.
Thereby, we filter the Ehrenfried complex by the number of clusters (compare Chapter \ref{chapter_css})
in order to reduce the size of the differentials that have to be diagonalized.
This again causes an enormous reduction of memory usage, 
which especially is useful in the case of rational coefficients that consume a lot of memory. 
The effect upon the running time is also highly positive 
since diagonalizing has cubic complexity and thus several small matrices are faster to diagonalize than a single huge matrix.

In order to get a better feeling for $\Ehrprog$, we also offer the possibility to compute characteristics of its differentials as the number of non-zero entries or the number of blocks.

We chose the programming language \cppeleven\ to realize this project since it is one of the prefered programming languages for mathematical projects, and since the new standard together with some additional libraries suits our needs perfectly.
The \cppeleven\ standard allows us to parallelize all time consuming steps of our project easily
such that we can exhaust the full hardware architecture of the computer.
We make use of the {\bfseries GNU multiple precision arithmetic library} \cite{GMP} for operations on signed integers, rational numbers, and floating-point numbers and
of the {\bfseries boost \cpp\ libraries} \cite{boost}, which is a set of portable libraries that offer high-quality solutions to standard problems as basic linear algebra, graph theory or file compression.

We are starting this chapter in Subsection \ref{program:runtime} with an overview on the performance gain of our programming techniques in order to motivate our work.
Afterwards, we will explain the library {\bfseries libhomology} (see Section \ref{program:libhomology}) and the program {\bfseries kappa} (see Section \ref{chapter_program:kappa}) in detail.
Additionally, we hand out advise about how to compile our computer programs (see Section \ref{program:compiling}).
Finally, we present our results, i.e., the homology groups and cluster spectral sequences we have computed (compare Section \ref{program:results}).

\section{Runtime and Memory Improvements}
\label{program:runtime}

Recall that the enormous size of the Ehrenfried complex forces us to take care of the runtime and memory consumption of our computer program (compare Section \ref{complexity:number_of_mono_cells}).
Via some example calculations, we show how we have improved the performance of our computer program step by step.

Consider the moduli spaces $\mathfrak M_{3, 1}^1$ and $\mathfrak M_3(m, n)$. 
The $E^0$ term of the corresponding cluster spectral sequence on the parallel Ehrenfried complex looks as follows:

\begin{center}
  \begin{tabular}{r||r|r|r|r|r|r|r|}
      \cline{2-8}
      \multicolumn{1}{r|}{} & \multicolumn{7}{c|}{$g = 3$, $m = 1$: Parallel $E^0_{p,l}$ for $\mathbb F_2$} \\ \hline
      \tl{\diagbox[height=1.7em, width=3em]{$p$}{$l$}} & 1 & 2 & 3 & 4 & 5 & 6 & 7\\ \hline\hline
      \tl 2   & 1     &       &       &       &       &       & \\ \hline
      \tl 3   & 252   &       &       &       &       &       & \\ \hline
      \tl 4   & 7563  & 18    &       &       &       &       & \\ \hline
      \tl 5   & 81360 & 2010  &       &       &       &       & \\ \hline
      \tl 6   & 424920& 48855 & 195   &       &       &       & \\ \hline
      \tl 7   &1141056& 469938& 13230 &       &       &       & \\ \hline
      \tl 8   &1305876&2069844& 247898& 1540  &       &       & \\ \hline
      \tl 9   &       &3593880&1810368& 70476 &       &       & \\ \hline
      \tl{10} &       &       &4737360& 915390& 8715  &       & \\ \hline
      \tl{11} &       &       &       &3702820& 258720&       & \\ \hline
      \tl{12} &       &       &       &       &1765335& 31878 & \\ \hline
      \tl{13} &       &       &       &       &       & 477906& \\ \hline
      \tl{14} &       &       &       &       &       &       & 56628 \\ \hline
  \end{tabular}
\end{center}

We were not able to calculate the entire homology of $\mathfrak M_{3, 1}^1$ via this spectral sequence since there are a lot of modules with several million of basis elements,
exeeding our possibilities concerning runtime and memory.
But we can use the radial Ehrenfried complex cluster spectral sequence for our calculations instead due to Proposition \ref{cellular_models:comparision_of_the_models:bundles_are_h_equiv}.
This reduces the dimensions of the modules of the $E^0$ page enourmously:

\begin{center}
  \begin{tabular}{r||r|r|r|r|r|r|r|}
      \cline{2-8}
      \multicolumn{1}{r|}{} & \multicolumn{7}{c|}{$g = 3$, $m = 1$: Radial $E^0_{p,l}$ for $\mathbb F_2$} \\ \hline
      \tl{\diagbox[height=1.7em, width=3em]{$p$}{$l$}} & 1 & 2 & 3 & 4 & 5 & 6 & 7 \\ \hline \hline
      \tl 1   & 1     &       &        &       &       &       &      \\  \hline
      \tl 2   & 82    & 1     &        &       &       &       &      \\  \hline
      \tl 3   & 1212  & 91    &        &       &       &       &      \\  \hline
      \tl 4   & 7200  & 1652  & 9      &       &       &       &      \\  \hline
      \tl 5   & 20400 & 12890 & 500    &       &       &       &      \\  \hline
      \tl 6   & 23760 & 49380 & 7706   & 60    &       &       &      \\  \hline
      \tl 7   &       & 77924 & 48104  & 2310  &       &       &      \\  \hline
      \tl 8   &       &       & 111588 & 25676 & 294   &       &      \\  \hline
      \tl 9   &       &       &        & 91384 & 7497  &       &      \\  \hline
      \tl{10} &       &       &        &       & 44850 & 945   &      \\  \hline
      \tl{11} &       &       &        &       &       & 12375 &      \\  \hline
      \tl{12} &       &       &        &       &       &       & 1485 \\  \hline
  \end{tabular}
\end{center}

The biggest module on the first page for the radial case has dimension $111588$, and not dimension $4737360$ as in the parallel case.
Using all our strategies to improve the performance of our program,
we can now determine the homology of $\mathfrak M_{3, 1}^1$ and $\mathfrak M_3(m, n)$ within less than half an hour.
In each row of the following table, we see running time and maximum memory consumption of one run of our computer program with different improvement techniques. 

\begin{center}
  \begin{tabular}{|c|c|c|l||r|r|}
      \cline{1-6}
      \multicolumn{6}{|c|}{Runtime and Memory Results for $g = 3$, $m = 1$} \\ \hline \hline
      Radial & CSS & Bool & \# Threads            & Runtime [h:min:sec] & Max. Memory Used [MB] \\ \hline\hline
      x      & x   & x    & $t_w = 8$, $t_r = 4$  & $0:27:43$           & $7056$  \\ \hline
      x      & x   & x    & $t_w = 11$, $t_r = 1$ & $0:39:03$           & $7071$  \\ \hline
      x      & x   & x    & $t = 1$               & $1:37:49$           & $7031$  \\ \hline
      x      &     & x    & $t_w = 8$, $t_r = 4$  & $1:25:39$           & $10819$ \\ \hline
      x      & x   &      & $t_w = 8$, $t_r = 4$  & $18:36:22$          & $92857$ \\ \hline
             & x   & x    & $t_w = 8$, $t_r = 4$  & /                   & /       \\ \hline
  \end{tabular}
\end{center}

Thereby, a cross marks whether 
\begin{itemize}
\item we use the radial Ehrenfried complex or the parallel one, 
\item we filter this by cluster sizes,
\item we use our special implementation of boolean coefficients (compare Subsubsection \ref{program:libhomology:MatrixT:MatrixField_for_F_2_and_css}),
\end{itemize}
and we specify whether we run the program single threaded ($t = 1$) or parallel, and if parallel, 
how many threads we use as working threads ($t_w$) and as remaining threads ($t_r$).
For an explanation of the meaning of these threads, see Subsubsection \ref{program:libhomology:DiagonalizerT:DiagonalizerField}.

We see that the most significant effect on both runtime and memory consumption arises by implementing boolean coefficients in a clever way
-- the runtime decreases by a factor $40$ and memory by a factor $13$.
Using this implementation of boolean coefficients, another important step to improve performance is to filter the Ehrenfried complex by cluster sizes,
which yields another factor $3$ of runtime and a factor $1.5$ of memory consumption improvement.
Unfortunately, parallelizing the program with $12$ threads does not gain another factor $12$ concerning runtime 
since diagonalizing is not smoothly parallelizable.
Still, another factor $2.5$ of runtime improvement, when only the jobs of the so-called working threads are distributed,
and even a factor of $3.6$, when also the so-called remaining work is distributed, results in valuable reduction of running time.
\section{The Library Libhomology [B,H]}
\label{program:libhomology}

With {\bf libhomology} we provide an expandable framework for all kinds of homology computations.
In our context, it is used as the foundation of the program kappa (see Section \ref{chapter_program:kappa}).
In the following we explain use and essential details of its classes.

The template class \progclass{ChainComplex} is the core of {\bf libhomology}.
Here we think of a chain complex as a finite sequence of compatible matrices $D_n$ satisfying $D_{n-1} D_n = 0$ which we call differential.
We do not mention bases.
Given a \progclass{ChainComplex}, our goal is to determine its homology.
Therefore one needs an implementation of the coefficient ring \progclass{CoefficientT} (see Subsection \ref{program:libhomology:CoefficientT}) and the matrix type \progclass{MatrixT} (see Subsection \ref{program:libhomology:MatrixT}) of the differentials.
The class \progclass{HomologyT} (see Subsection \ref{program:libhomology:HomologyT}) specifies the scope of homological information one wants to extract.
These should be derived using the class \progclass{DiagonalizerT} (see Subsection \ref{program:libhomology:DiagonalizerT}) which diagonalizes matrices by applying row or column operations.

The implementations of the classes \progclass{ChainComplex}, \progclass{MatrixT}, \progclass{DiagonalizerT} and \progclass{HomologyT} are interdependent,
so we recommend to skim over the details on the first reading.

\subsection{The Class ChainComplex}
\label{program:libhomology:ChainComplex}

We start with the description of the members of the template class
\begin{lstlisting}
template< class CoefficientT,
          class MatrixT,
          class DiagonalizerT,
          class HomologyT >
class ChainComplex;
\end{lstlisting}
A \progclass{ChainComplex} is a finite sequence of differentials which we represent by
\begin{lstlisting}
std::map< int32_t, MatrixT > differential;
\end{lstlisting}
It is reasonable to define the following pass-through methods.
You access the \nth differential by calling
\begin{lstlisting}
MatrixT& operator[]( const int32_t n );
\end{lstlisting}
or its \progkeyword{const} counter part
\begin{lstlisting}
const MatrixT& at( const int32_t n ) const;
\end{lstlisting}
You can test whether the \nth differential is defined by checking whether
\begin{lstlisting}
size_t count( const int32_t n ) const;
\end{lstlisting}
evaluates to zero.
The \nth differential is deleted by the following method.
\begin{lstlisting}
void erase( const int32_t n );
\end{lstlisting}
All homology modules are computed by calling
\begin{lstlisting}
HomologyT homology();
\end{lstlisting}
In order to compute the \nth homology, one calls
\begin{lstlisting}
HomologyT homology( const int32_t n );
\end{lstlisting}
If you are interested in the kernel and torsion parts of the $n\Th$ differential, you should call
\begin{lstlisting}
HomologyT compute_kernel_and_torsion( int32_t n );
\end{lstlisting}
These operations consume by far the most time and 
we recommend using a parallelized diagonalization process (compare Subsection \ref{program:libhomology:DiagonalizerT:DiagonalizerField}).

The \progclass{DiagonalizerT} in use, is accessed via
\begin{lstlisting}
      DiagonalizerT& get_diagonalizer();
const DiagonalizerT& get_diagonalizer() const;
\end{lstlisting}

In order to derive the homology of the moduli spaces, we have to handle very large differentials (compare Section \ref{complexity:number_of_mono_cells}).
Thus we work with a single differential at a time, which is accessed via
\begin{lstlisting}
// Access the current differential.
      MatrixT& get_current_differential();
const MatrixT& get_current_differential() const;

// Erases the current differential.
void erase();

// Access the coefficient of the current differential.
CoefficientT& operator() ( const uint32_t row, const uint32_t col );
    
// Return number of rows resp. columns of the current differential
size_t num_rows() const;
size_t num_cols() const;
\end{lstlisting}


\subsection{The Type CoefficientT}
\label{program:libhomology:CoefficientT}

The coefficient ring must be represented by a class that meets the requirements discussed in Subsubsection \ref{program:libhomology:CoefficientT:obligatory_reqs}.

\subsubsection{Obligatory Operations for CoefficientT}
\label{program:libhomology:CoefficientT:obligatory_reqs}
Clearly, one has to provide the basic ring operations.
\begin{lstlisting}
CoefficientT& operator= ( const CoefficientT& );        // Assignment
bool          operator==( const CoefficientT& ) const;  // Comparison
bool          operator!=( const CoefficientT&, const CoefficientT& );
CoefficientT  operator- () const;                       // Negation
CoefficientT& operator+=( const CoefficientT& );        // Addition
CoefficientT  operator+ ( const CoefficientT&, const CoefficientT& );
CoefficientT& operator-=( const CoefficientT& );        // Subtraction
CoefficientT  operator- ( const CoefficientT&, const CoefficientT& );
CoefficientT& operator*=( const CoefficientT& );        // Multiplication
CoefficientT  operator* ( const CoefficientT&, const CoefficientT& );
\end{lstlisting}
If the coefficients form a field, we suggest to implement the division operators.
\begin{lstlisting}
CoefficientT& operator/=( const CoefficientT& );
CoefficientT  operator/ ( const CoefficientT&, const CoefficientT& );
\end{lstlisting}

The integers are initial in the category of rings, thus it is reasonable to implement the following methods.
\begin{lstlisting}
CoefficientT& operator= ( const int );                      // Assignement
bool          operator==( const int ) const;                // Comparision
CoefficientT  operator* ( const CoefficientT&, const int ); // Multiplication
\end{lstlisting}

In our project {\bf kappa}, we intend to save differentials so you should provide a method that stores a \progclass{CoefficientT} (See Subsection \ref{program:libhomology:serialization}).

\subsubsection{Coefficients in the Rationals and the Integers Mod \texorpdfstring{$m$}{m}}
\label{program:libhomology:CoefficientT:our_implementation}
We offer the classes \progclass{Q} respectively \progclass{Zm} that represent coefficients in $\mathbb Q$ respectively $\mathbb Z / m \mathbb Z$:
The class \progclass{Q} is defined via the following \progkeyword{typedef}.
\begin{lstlisting}
typedef mpq_class Q;
\end{lstlisting}
The class \progclass{mpq\_class} itself is the \cpp\ variant of the {\bf GMP} type \progclass{mpq\_t}.
Before using the class \progclass{Zm}, you have to call the static member function
\begin{lstlisting}
static void set_modulus(const uint8_t p, const uint8_t e = 1);
\end{lstlisting}
that defines $m = p^e$.
Omitting the call will throw a segmentation fault which is the result of a division by zero.
The following self-explaining member functions might be useful.
\begin{lstlisting}
static void const print_modulus();    
static void const print_inversetable();
static bool       is_field();
\end{lstlisting}

\subsection{The Type MatrixT}
\label{program:libhomology:MatrixT}
\subsubsection{Existing Template Classes}
Before writing your own \progclass{MatrixT} you may want to have a look at the {\bf ublas library} provided by {\bf boost}.
They offer several matrix templates for sparse and dense matrices 
as well as BLAS implementations for numerical computations.
Moreover, our {\bf libhomology} provides the template class
\begin{lstlisting}
template < class CoefficientT >
class MatrixField;
\end{lstlisting}
for exact computations with matrix coefficients in a given field.
In the following, we present an overview of the requirements any implementation of \progclass{MatrixT} has to fulfill, whereas
in Subsection \ref{program:libhomology:MatrixT:MatrixField_for_F_2_and_css} we treat more specialized implementations.

\subsubsection{Requirements on MatrixT}
Your implementation of \progclass{MatrixT} has to meet some requirements.
These are inspired by the {\bf boost ublas library} as we make heavy use of it to compactify implementation details.
We denote the coefficients of the matrix by \progclass{CoefficientT}.
A \progclass{MatrixT} is created as follows:
\begin{lstlisting}
MatrixT ( size_t number_rows, size_t number_cols );
\end{lstlisting}
The coefficient in the $i\Th$ row and the $j\Th$ column is accessed by calling
\begin{lstlisting}
CoefficientT& operator()( size_t i, size_t j );
\end{lstlisting}
The number of rows is 
\begin{lstlisting}
size_t size1() const;
\end{lstlisting}
and the number of columns is
\begin{lstlisting}
size_t size2() const;
\end{lstlisting}

As we intend to save differentials in our project {\bf kappa}, you should provide a method that stores a \progclass{MatrixT} (See Subsection \ref{program:kappa:serialization}).

\subsubsection{Optional Requirements on MatrixT}
\label{program:libhomology:MatrixT:optionals}
If you provide your own \progclass{MatrixT} with coefficients in a field, you may want to use our class \progclass{DiagonalizerField} (see Subsection \ref{program:libhomology:DiagonalizerT:DiagonalizerField}) to compute rank and torsion of your matrices.
In order to do so, you have to provide the member function
\begin{lstlisting}
void row_operation( size_t row_1, size_t row_2, size_t col );
\end{lstlisting}
that applies a row operation on the matrix, 
i.e. adds the appropriate multiple of \progname{row\_1} to \progname{row\_2} in order to erase the entry \progname{(row\_2, col)} of the matrix.
Our implementation makes use of multithreading, therefore you have to be careful with race conditions.
You have to ensure that row operations for fixed \progname{row\_1} and \progname{col} with varying \progname{row\_2} can be applied concurrently.

\subsection{Special Implementations of MatrixField}
\label{program:libhomology:MatrixT:MatrixField_for_F_2_and_css}
\subsubsection{MatrixField for Coefficients in \texorpdfstring{$\mathbb F_2$}{Z/2Z}}
For coefficients in the field $\mathbb F_2$, our implementation provides siginificant improvements concerning memory and execution duration, see Section \ref{program:runtime}.
Using well-known techniques, we store multiple entries of a row in a single data entity.
Note that, since the only invertible element in $\mathbb F_2$ is $1$, a row operation corresponds to the bitwise \progname{XOR}-instruction.

Using these insights, we provide an implementation called \progclass{MatrixBool}.
It behaves almost like \progclass{MatrixField} but has a few technical limitations (which are unavoidable as these are direct consequences of the enormous performance gain).
E.g.\ for a matrix of type \progclass{MatrixBool}, it is not possible to access its coefficients by reference.
\begin{lstlisting}
bool operator() ( const size_t i, const size_t j );
bool at         ( const size_t i, const size_t j ) const;
\end{lstlisting}
Observe that for our purpose, it suffices to add $1$ to a given entry which is provided by the method
\begin{lstlisting}
void add_entry( const size_t i, const size_t j );
\end{lstlisting}
It should be easy to equip \progclass{MatrixBool} with more member functions if needed.

\subsubsection{MatrixField for the Cluster Spectral Sequence}
In order to exploit the cluster spectral squence, we provide the adapted version \progclass{MatrixFieldCSS} of \progclass{MatrixField} and also \progclass{MatrixBoolCSS} of \progclass{MatrixBool}.
Here, one should think of a spectral sequence that collapses at the second page as a subdivision of the differentials:
The bases are ordered in a way such that the transposed differential $D$ consists of a diagonal of block matrices $d^0$ which respect the filtration degree and below a single second diagonal of block matrices which decrease the filtration degree by one.
\[
    D = 
        \begin{pmatrix}
            d^1 & d^0 \\
                & d^1   & d^0 \\
                &       & d^1   & d^0 \\
                &       &       &       & \ddots
        \end{pmatrix}
\]
Such a matrix is diagonalized as follows.
We construct the first line given by $d^1$ and $d^0$ in the top left corner.
Then we apply row operations to $d^0$ until its image is determined and then apply row operations to the remaning rows of $d^1$ until the image of the first row is fully understood.
Afterwards, we may forget the matrix $d^1$ in the top left corner, construct the matrix $d^1$ of the next line and apply the needed row operations that are due to the matrix $d^0$ from above.
Now we forget the entire first line, construct the next matrix $d^0$ and iterate this process.

During this procedure, we store at most two submatrices of $D$, namely one of type $d^0$ and one of type $d^1$, so our implementation \progclass{MatrixFieldCSS} and \progclass{MatrixBoolCSS} does exactly the same.
We provide two ways to access the two submatrices $d^0$ and $d^1$.
To use the first approach, the method
\begin{lstlisting}
void define_operations( const OperationType );
\end{lstlisting}
defines on which submatrix we are currently working, where \progclass{OperationType} is an enumeration type and set to be \progclass{main\_and\_secondary} to access $d^0$ or \progclass{secondary} to access $d^1$.
Now one calls member functions of \progclass{MatrixFieldCSS} respectively \progclass{MatrixBoolCSS} which have the same name as the member functions of \progclass{MatrixField} respectively \progclass{MatrixBool}.
Let us give a simple example by printing $d^0$ and $d^1$ to the screen.
\begin{lstlisting}
M.define_operations( main_and_secondary );
std::cout << M; \\ prints d^0 to screen.

M.define_operations( secondary );
std::cout << M; \\ prints d^1 to screen.
\end{lstlisting}


In the second approach, one calls a member method corresponding to $d^0$ by adding the prefix \progclass{main\_}, whereas \progclass{sec\_} applies to $d^1$.
The following listing is an example for the member functions \progname{size1} and \progname{size2}.
\begin{lstlisting}
size_t main_size1() const; // Returns the number of rows of d^0.
size_t main_size2() const; // Returns the number of columns of d^0.
size_t sec_size1() const;  // Returns the number of rows of d^1.
size_t sec_size2() const;  // Returns the number of columns of d^1.
\end{lstlisting}


A row operation on $d^0$ clearly affects the submatrix $d^1$ in the same line.
In the algorithm presented above, we apply only those operations to $d^1$ that leave $d^0$ unchanged.
Therefore we provide the following member functions.
\begin{lstlisting}
void row_operation_main_and_secondary
    ( const size_t row_1, const size_t row_2, const size_t col );
void row_operation_secondary
    ( const size_t row_1, const size_t row_2, const size_t col );
\end{lstlisting}






\subsection{The Type DiagonalizerT [B,H]}
\label{program:libhomology:DiagonalizerT}
{\bf [B]}
Given a differential $C_n \xr{\del} C_{n-1}$ of a chain complex, one wants to derive its kernel and image in order to compute the homology of the chain complex.
In our situation, we are given a differential of the type \progclass{MatrixT}, so we want to apply a range of base changes to end up with a matrix where reading off these informations is easy.
These base changes depend heavily on the coefficient ring.
For field coefficients, one can apply Gaussian elimination, but for integral coefficients, one has to work much harder.
Some state of the art algorithms can be found in \cite{Jaeger2003} and \cite{Jaeger2009}.
We emphasize that this is the most time consuming operation (see Chapter \ref{complexity}) and suggest to carry out an algorithm that makes use of concurrency.

The \progclass{DiagonalizerT} is a function object, so you have to provide the method
\begin{lstlisting}
void operator()( MatrixT& matrix );
\end{lstlisting}
that diagonalizes the given matrix.
Afterwards, kernel and torsion of the matrix should be available by the diagonalizer's member functions
\begin{lstlisting}
HomologyT::KernT kern();
HomologyT::TorsT tors();
\end{lstlisting}
where \progclass{HomologyT} is the class we use to store the homology of a chain complex (compare Subsection \ref{program:libhomology:HomologyT}).

Moreover, we are interested in the defect and the rank of the matrix, so you have to provide the following two member functions.
\begin{lstlisting}
uint32_t dfct();
uint32_t rank();
\end{lstlisting}

\label{program:libhomology:DiagonalizerT:DiagonalizerDummy}

There are situations in which one wants to generate a chain complex without computing homological data:
The size of the differentials of the Ehrenfried complex (see Section \ref{cellular_models:ehrenfried}) is enourmous by Proposition \ref{complexity:number_of_mono_cells},
so it is impossible to compute the number of non-vanishing entries per column by hand.
Therefore we offer the template class
\begin{lstlisting}
template < class MatrixT >
class DiagonalizerDummy;
\end{lstlisting}
that does absolutely nothing, so you can use it together with \progclass{HomologyDummy} (see Subsubsection \ref{program:libhomology:HomologyT:HomologyDummy})
as a template parameter for the template class \progclass{ChainComplexT} (see Subsection \ref{program:libhomology:ChainComplex}).

In the following, we shall describe our implementations of the class \progclass{DiagonalizerField}.

\subsection{The Class DiagonalizerField [B,H]}
\label{program:libhomology:DiagonalizerT:DiagonalizerField}

The \progclass{DiagonalizerField} applies a slightly modified version of the Gaussian elimination to a given matrix.
Note that for computing the homology of a chain complex with field coefficients,
it is sufficient to determine the rank and defect of all its differentials.
Thus, the class \progclass{DiagonalizerField} merely transforms row operations upon the matrix in order to determine its rank,
but does not exchange columns in order to obtain a triangular matrix.
Since computing the rank is equivalent to diagonalizing for our purpose, we refer to this process as diagonalizing nevertheless.
After giving an overview on the usage of this class, we will explain implementation details and runtime results of our parallelized diagonalization algorithm.

\subsubsection{Overview and Usage of DiagonalizerField [H]}

For field coefficients, we offer the following template class.
\begin{lstlisting}
template < class CoefficientT >
class DiagonalizerField;
\end{lstlisting}
Here we assume that \progclass{MatrixT} is given by
\begin{lstlisting}
typedef MatrixField< CoefficientT > MatrixT;
\end{lstlisting}
and it is trivial to alter the class definition in order to allow arbitrary matrices.

The member variable \progname{current\_rank} of the class \progclass{DiagonalizerField} keeps track of the progress of an ongoing computation as
it stores the number of linearly independent rows the algorithm has already found.
Operations on variables of the type \progclass{atomic\_uint} are atomic, i.e. reading, writing, incrementing and so forth is free of race conditions.
We suggest to make use of this feature as follows.
You start two threads, one computes kernel and torsion and the other monitors the progress.
\begin{lstlisting}
ChainComplex< ... > complex;
// Define the differentials of matrix_complex.
// ...

atomic_uint& rank = diagonalizer.current_rank;
measure_duration = Clock(); // Measures duration.

// Set the value of state to 1 if and only if kernel and torsion are computed.
// This is done to terminate the 'monitoring thread'.
atomic_uint state(0);

// Diagonalizing thread.
auto partial_homology_thread = std::async( std::launch::async, [&]()
{
    auto ret = complex.compute_current_kernel_and_torsion( p );
    state = 1;
    return ret;
} );

// Monitoring thread.
auto monitor_thread = std::async( std::launch::async, [&]()
{
    while( state != 1 )
    {
        std::cout << "Diagonalization " << current_rank << "\r";
        std::this_thread::sleep_for( std::chrono::milliseconds( 50 ) );
    }
} );
\end{lstlisting}
The current progress is printed to screen and updated every 50 milliseconds till the computation is done.

\subsubsection{Implementation Details [H]}
\label{diag_field_implementation}
Our key algorithm for computing the rank of a matrix via Gaussian elimination is given by

\begin{algorithm}[H]
\label{rank}
\DontPrintSemicolon
\SetKw{KwCont}{continue}
\KwIn{A matrix $A = (a_{ij})$ with coefficients in a field $\F$}
\KwOut{The rank $\rk(A)$}

Let $R$ be the set of rows of $A$\;
Set $R_r := \emptyset$ \tcp*[f]{Let $R_r$ denote the set of rows contributing to the rank}\;
\ForEach{column $c$}
{
	Let $j \in R \backslash R_r$ be a row index with $a_{jc}$ invertible in $\F$\;
	\If{No such $j$ exists}
	{
		\KwCont\;
	}
	Let $S \subset R \backslash R_r$ be the subset of rows $s \neq j$ with $a_{sc}$ invertible in $\F$\;
	\If{$S \neq \emptyset$}
	{
		Set $R_r := R_r \cup \{j\}$\;
	}
	\ForEach{row $s \in S$}
	{
		Perform \progname{A.row\_operation(j, s, c)}\;
	}
}
\KwRet{$|R|$}

\caption{Rank Computation}

\end{algorithm}

Hereby, \progname{row\_operation} is the member function of the class \progclass{MatrixType} described in Subsubsection
\ref{program:libhomology:MatrixT:optionals}.

A sequential version of this algorithm is implemented as the member function
\begin{lstlisting}
uint32_t diag_field( MatrixType& matrix );
\end{lstlisting}
of \progclass{DiagonalizerField}.
For the parallelized version, the method
\begin{lstlisting}
uint32_t diag_field_parallelized( MatrixType& matrix );
\end{lstlisting}
is used.
Since -- at least for our matrices of type \progclass{MatrixBool} -- a single row operation is performed very fast,
we do not use several threads to parallelize row operations, 
but subdivide the set of row operations such that several row operations are performed simultaneously.

We define two helper classes for parallelizing, which we will explain here roughly,
using the notation from Algorithm \ref{rank}.
The class \progclass{JobQueue} keeps track of all significant data used in Algorithm \ref{rank}.
Obviously, a \progclass{JobQueue} has to know the
\begin{lstlisting}
 MatrixType & matrix;
\end{lstlisting}
which is supposed to be diagonalized, and the column 
\begin{lstlisting}
 size_t col;
\end{lstlisting}
and row
\begin{lstlisting}
 size_t row_1;
\end{lstlisting}
that are currently considered, where, in the notation of Algorithm \ref{rank}, we have \progname{col = $c$} and \progname{row\_1 = $r$}. 
Furthermore, the \progclass{JobQueue} maintains the 
\begin{lstlisting}
 std::vector rows_to_work_at;
\end{lstlisting}
which resembles the set $S$, and the
\begin{lstlisting}
 std::vector remaining_rows;
\end{lstlisting}
consisting of the rows $t$ not yet contributing to the rank for that the entry $a_{tc}$ is not invertible in $\F$, i.e. of $R\backslash (R_r \cup S)$.
Since the \progclass{JobQueue} contains all the information necessary to perform the required row operations for a given column $c$, only two tasks remain:
updating these data members when passing over from one column to the next
and parallelizing the row operations as well as the update.

For each thread used, we create an instantiation of the class \progclass{Worker}, which will not be discussed in this thesis, to perform computations.
Experiments showed that having two different kinds of \progclass{Workers} is more efficient:
Firstly, we define a family of \progclass{Workers} that actually perform the diagonalizing work.
The \progclass{JobQueue} distributes the rows \progname{rows\_to\_work\_at} among these \progclass{Workers} equally.
Afterwards, each of these \progclass{Workers} considers all its assigned rows $s$, 
performs the row operation upon $s$ and marks whether $s$ will also be in the set $S$ for the next column.
This means that the already defined \progclass{Workers} update parts of the arrays \progname{rows\_to\_work\_at} and \progname{remaining\_rows},
and that it remains to update these arrays with respect to the set of \progname{remaining\_rows}.
This is the task the other family of \progclass{Workers} execute,
where the \progname{remaining\_rows} are again distributed equally among the \progclass{Workers} by the \progclass{JobQueue}.

The input parameters \progname{num\_threads} respectively \progname{num\_remaining\_threads} define 
how many threads are occupied for the first respectively second type of \progclass{Workers}, see also Subsection \ref{chapter_program:kappa:compute_css}.

\subsection{The Type HomologyT}
\label{program:libhomology:HomologyT}
In order to derive the homology of a chain complex, we compute all kernels and images of the differentials, given by transposed transformation matrices.
In our situation, we start with a \progclass{ChainComplex} (see Subsection \ref{program:libhomology:ChainComplex}) that is essentially a finite series of matrices of the type \progclass{MatrixT} (see Subsection \ref{program:libhomology:MatrixT}).
The function object \progclass{DiagonalizerT} (see Subsection \ref{program:libhomology:DiagonalizerT}) applies row and column operations until both kernel and image can be read off.
This data should then be communicated to \progclass{HomologyT}.

\subsubsection{Essential Members}
The type \progclass{HomologyT} requires the following members.
You have to provide the types \progclass{HomologyT}\progname{::}\progclass{KernT} respectively \progclass{HomologyT}\progname{::}\progclass{TorsT} that store the kernel respectively the image of a differential.
This can be achieved by including the following two lines in the \progkeyword{public} section of the class definition.
\begin{lstlisting}
class HomologyT{
public:
    typedef /* ... */ KernT;
    typedef /* ... */ TorsT;
};
\end{lstlisting}
You have to define the following two constructors
\begin{lstlisting}
HomologyT ();
HomologyT ( int32_t n, KernT& k, TorsT& t ); // Sets k and t at the spot n.
\end{lstlisting}
and member functions for storing kernels and images.
\begin{lstlisting}
void set_kern ( int32_t , KernT& );
void set_tors ( int32_t , TorsT& );
\end{lstlisting}
Moreover, we want to print the homology to the screen, thus the class definition has to include the following line.
\begin{lstlisting}
friend std::ostream& operator<< ( std::ostream& , const HomologyT& );
\end{lstlisting}

\subsubsection{The Class HomologyDummy}
\label{program:libhomology:HomologyT:HomologyDummy}
As mentioned in Subsection \ref{program:libhomology:DiagonalizerT:DiagonalizerDummy}, there are situations in which one wants to generate a chain complex without computing homological data.
For this purpose, we offer the class
\begin{lstlisting}
class HomologyDummy
\end{lstlisting}
that does absolutely nothing, so you can use it together with \progclass{DiagonalizerDummy} (see Subsection \ref{program:libhomology:DiagonalizerT:DiagonalizerDummy})
as a template parameter for the template class \progclass{ChainComplexT} (see Subsection \ref{program:libhomology:ChainComplex}).

\subsubsection{The Class HomologyField}
Using field coefficients, the homology modules are all vector spaces.
For those who are only interested in the Betti numbers, we offer the class \progclass{HomologyField}.
Here we store only the dimensions of kernel and image.
The class definition is essentially as follows, where the member functions should be self-explaining.
\begin{lstlisting}
class HomologyField{
public:
    typedef int64_t KernT;
    typedef int64_t TorsT;
    
    HomologyField ();
    HomologyField ( int32_t, KernT, TorsT );
    
    void  set_kern   ( int32_t, KernT );
    void  set_tors   ( int32_t, TorsT );
    KernT get_kern   ( int32_t ) const;
    TorsT get_tors   ( int32_t ) const;
    void  erase_kern ( int32_t );
    void  erase_tors ( int32_t );
    friend std::ostream& operator<< ( std::ostream&, const HomologyField& );
private:
    std::map< int32_t, int64_t > kern_rep;
    std::map< int32_t, int64_t > tors_rep;
};
\end{lstlisting}


\section{The Program Kappa}
\label{chapter_program:kappa}

The program {\bfseries kappa} mainly uses the previously introduced {\bfseries libhomology} (compare Section \ref{program:libhomology}) 
and the theory developed in Chapter \ref{cellular_models}
in order to determine the homology of the moduli spaces $\Modspc[1]$ and $\ModspcRad[1]$. 
Thus, it computes the cohomology of the parallel or radial Ehrenfried complex $\Ehrprog$,
filtered by cluster sizes.

Recall that the homology of $\Modspc[1]$ and $\ModspcRad[1]$ coincides for $m > 0$, see Proposition \ref{cellular_models:comparision_of_the_models:bundles_are_h_equiv}.
For computing the homology of $\Modspc[1]$ for fixed $g$ and $m$, 
it is more efficient to use the radial Ehrenfried complex rather than the parallel one, 
since its modules and thus differentials are much smaller, see Section \ref{complexity:number_of_mono_cells}.
However, for $m = 0$, we cannot use the radial model, so we also offer the computation of the homology of $\Modspc[1]$ via the parallel model.
Since we also determine the dimensions of the modules of the cluster spectral sequence, 
computation with the parallel model produces new homological information for $m > 0$.

A central class of our computer program is hence the class \progclass{ClusterSpectralSequence}, see Subsection \ref{chapter_program:kappa:css},
which stores the parallel or radial Ehrenfried complex filtered by cluster sizes.

Since the basis elements of $\Ehrprog$ are monotonous tuples of transpositions,
another important class of the program kappa is the class \progclass{Tuple} (see Subsection \ref{chapter_program:kappa:tuple}),
which represents such a basis element and offers many functions that are applied to it
during the generation of the \progclass{ClusterSpectralSequence}.

With these foundations, we can offer the tool \progname{compute\_css} (see Subsection \ref{chapter_program:kappa:compute_css}) 
for computing the first three pages of the cluster spectral sequences corresponding to the moduli spaces $\Modspc[1]$ and $\ModspcRad[1]$
and especially their homology.

We also provide the tool \progname{compute\_cache},
which computes the bases and the differentials of $\Ehrprog$ and stores them in files via serialization.
For most $g$ and $m$, 
the computation of both takes a lot of time,
and it is thus functional to have the opportunity to store the data of $\Ehrprog$ for later uses.

For example, \progname{compute\_cache} can be used to examine the structure of the Ehrenfried complex via the tools
\progname{compute\_statistics} and \progname{print\_basis}.
After \progname{compute\_cache} has been performed, one can call \progname{compute\_statistics}
to find out various properties of the Ehrenfried complex $\Ehrprog$
like the sizes of the differentials or their largest entry per column.
Alternatively, a call of \progname{print\_basis} outputs the basis of the Ehrenfried complex.

Since all these tools mentioned are organized in a similar way, 
we shall only describe the tool \progname{compute\_css} in detail, see Subsection \ref{chapter_program:kappa:compute_css}.
Thereafter, we illustrate the above mentioned classes \progclass{Tuple} (compare Subsection \ref{chapter_program:kappa:tuple}) and \progclass{ClusterSpectralSequence} (compare Subsection \ref{chapter_program:kappa:css}),
which describe the Ehrenfried complex.

\subsection{The Tool \progname{compute\_css}}
\label{chapter_program:kappa:compute_css}

The tool \progname{compute\_css}, which computes the homology of the moduli spaces $\Modspc[1]$ and $\ModspcRad[1]$
by deriving the second term of the corresponding cluster spectral sequence,
is the most important tool supported by the project {\bf kappa}.
In Subsubsection \ref{chapter_program:kappa:compute_css:usage}, we describe 
how one operates this tool, 
while in Subsubsection \ref{chapter_program:kappa:compute_css:implementation},
we explain its implementation. 

\subsubsection{Usage}
\label{chapter_program:kappa:compute_css:usage}

For the computation of the homology of the moduli space $\ModspcRad[1]$, 
one can use the command
\begin{lstlisting}
.\compute_css -g arg -m arg (-q | -s arg)
\end{lstlisting}
Thereby,
\begin{itemize}
\item the parameter \progname{g} is the genus of the moduli space,
\item the parameter \progname{m} is the number of punctures of the moduli space,
\item one can either use the parameter $q$ or the parameter $s$. 
      If $q$ is chosen -- without any argument -- homology with rational coefficients will be computed.
      If the parameter $s$ is set to some positive prime, we will compute homology with coefficients in $\Z/s\Z$.
\end{itemize}
As the output of this command, one obtains a description of the $E^0$, $E^1$ and $E^2$ term of the cluster spectral sequence associated with $\ModspcRad[1]$,
and can read off the homology from the $E^2$ page.  

Recall that, for $m > 0$, the homology of the moduli space $\Modspc[1]$ coincides with the homology of the moduli space $\Modspc[1]$ 
(compare Proposition \ref{cellular_models:comparision_of_the_models:bundles_are_h_equiv}).
For determining the homology of $\Modspc[1]$ for $m = 0$ 
or the dimensions of the modules in the cluster spectral sequence of $\Modspc[1]$, one can set the optional parameter
\begin{itemize}
 \item \progname{parallel}
\end{itemize}
to true.

For instance, the call
\begin{lstlisting}
 ./compute_css -g 1 -m 3 -s 2
\end{lstlisting}
computes the homolgy of the moduli space $\mathfrak{M}^\bullet_1(3, 1)$ with $\Z/2\Z$ coefficients, 
while the call
\begin{lstlisting}
 ./compute_css -g 1 -m 3 -q --parallel 1
\end{lstlisting}
determines the homolgy of the moduli space $\mathfrak{M}^1_{3, 1}$ with rational coefficients.

There are other optional parameters that improve the performance or handling of the program.
\begin{itemize}
\item The optional parameter \progname{t} is the number of threads 
      that are allowed to be used for parallelization,
      which is $1$ by default. 
\item In addition, one can use the optional parameter \progname{num\_remaining\_threads}
      to determine how exactly computations are parallelized. 
      For a detailed explanation, see \ref{diag_field_implementation}.
      The total number of threads used will then be 
      \[
        \text{\progname{num\_threads + num\_remaining\_threads,}}
      \]
      and we recommend to use one third of the total number of threads as remaining threads.
\item The optional parameters \progname{first\_diff} respectively \progname{last\_diff} are the minimal respectively maximal $p \in \N$ 
      for which the homology $H_p(\Modspc[1])$ is supposed to be computed,
      which are $0$ respectively $2h$ by default.
\end{itemize}

If the command \progname{help} is used or if the input is not valid,
instructions for use will be printed to the console.

During the computations, we offer intermediate results and progress bars as console output. 
When the computation of the homology is finished, 
the file 
\[
\text{\progname{compute\_homology\_(parameters)}}
\]
created by the program contains all the intermediate and final results.
Be aware that this means that calls with the same parameters produce files with the same names and hence old files are overwritten.
The intermediate results give the oportunity to abort the computation
and continue it some other time, 
using the parameters \progname{first\_diff} and \progname{last\_diff} to select certain homology groups.

\subsubsection{Implementation Details}
\label{chapter_program:kappa:compute_css:implementation}

The computation of the cluster spectral sequence and the homology starts in the main function of the file 
\begin{lstlisting}
main_compute_css.cpp
\end{lstlisting}
The input parameters described above are stored in the struct
\begin{lstlisting}
SessionConfig;
\end{lstlisting}
which also tests whether the given configuration of parameters is valid, 
outputting the correct usage to the console if not.
Apart from the data members corresponding to the input parameters, 
the struct \progclass{SessionConfig} contains the data member
\begin{lstlisting}
SignConvention sgn_conv;
\end{lstlisting}
it can set on its own in dependence on the parameters for the coefficients.
The \progclass{SignConvention} parameter can have three different values, 
which indicate which signs have to be respected in the computation of the differentials. 
This way, we avoid sign computations whenever we can.
We set the parameter to \progname{no\_signs} if we only compute the homology up to sign, 
which is the case if we use coefficients in $\Z/2\Z$. 
If we have different coefficients and \progname{m} $\geq 2$, the moduli space $\Modspc[1]$ is non-orientable and the sign convention is set to \progname{all\_signs}. 
Otherwise, it equals \progname{no\_orientation\_sign}. 

At the begin of the computation, the constructor of \progclass{ClusterSpectralSequenceT} is called with respect to the parameters provided above.
Thereby, the bases (see Subsubsection \ref{chapter_program:kappa:css:gen_bases}) are generated, with basis elements sorted by their cluster number.
In particular, printing the $E^0$-term to screen is readily done.

The main task is determining the first and second page of the cluster spectral sequence.
Recall Section \ref{css:section_matrix_version}, which discusses the matrix version of the cluster spectral sequence.
The $p\Th$ transposed transformation matrix of $\del_\E$ is a block matrix of the form
\[ 
    \begin{pmatrix}
        d^1 & d^0 \\
            & d^1   & d^0 \\
            &       & d^1   & d^0 \\
            &       &       &       & \ddots
    \end{pmatrix} \,,
\]
if we sort the basis elements by their number of clusters.
The modules of the $E^1$-term are given by
\[
    \ker(d^0) / \img(d^0)\,,
\]
whereas the modules of the $E^2$-term are given by
\[
    \ker(d^1|_{\ker(d^0)}) / \big( \img(d^0) + \img(d^1|_{\ker(d^0)}) \big) \,.
\]
Hence, it suffices to apply all row operations induced by the sub-matrices $d^0$ and proceed with the diagonalization process of
the parts of the (altered) sub-matrices $d^1$ that are in the kernel of the (diagonalized) sub-matrices $d^0$.
In order to save execution time and memory, our program operates on exactly one $d^0$ and one $d^1$ sub matrix at a time:

\begin{algorithm}[H]
\DontPrintSemicolon
\For{$p = 1$ \KwTo $2h$}
{
    \For{$l=1$ \KwTo $p$}
    {
        Construct $d^1_{p,l}$ and apply row operations of $d^0_{p,l-1}$
        
        Forget $d^0_{p,l-1}$
        
        Generate $d^0_{p,l}$
        
        Compute and save kernel and image of $d^0_{p,l}$
        
        Print the homological results to the screen
        
        Save the diagonal of $d^0_{p,l}$ and detect superflous rows of $d^1_{p,l}$
        
        Compute and save the kernel and image of $d^1_{p,l}$
        
        Print the homological results to the screen
        
        Forget $d^1_{p,l}$
    }
}
Print all three pages of the spectral sequence to the screen.
\caption{Computing $E^1$ and $E^2$}
\end{algorithm}

Here, we have $h = 2g + m$ in the parallel case and $h = 2g + m - 1$ in the radial case.

During all computations, the function \progname{compute\_css} generates the previously mentioned intermediate results 
printed out in the console and into the corresponding output file.
Furthermore, it measures the duration of the important steps of the computations 
and also writes them into the console. 
\subsection{The Class Tuple [H]}
\label{chapter_program:kappa:tuple}

Recall that we aim to compute the cohomology of the parallel or radial Ehrenfried complex $\Ehrprog$.
A basis for the Ehrenfried complex is given by monotonous cells $\Sigma = (\tau_h \mid \ldots \mid \tau_1)$ satisfiying certain conditions,
and the differential for $\Ehrprog$ can be described by the map $\del_\E$ making the diagram
\[
    \begin{tikzcd}
	\E_p \arrow{r}{\del_\E} \arrow{d}{\kappa}[swap]{\cong}      & \E_{p-1} \\
	\KK_p \arrow{r}{\del_\KK}                                     & \KK_{p-1} \arrow{u}{\cong}[swap]{\pi}
    \end{tikzcd}
\]
commute, compare Definition \ref{cellular_models:Ehrenfried:defi}.
Hence, the core of the program \textbf{kappa} is the class \progclass{Tuple}, 
which represents a tuple of $h$ transpositions $\Sigma = (\tau_h \mid \ldots \mid \tau_1)$
and thus especially the basis elements of $\Ehrprog$.
This class additionally provides several methods which are applied to a basis element during the computation of the differential $\del_\E$.

We are going to give an overview on its data members (Subsubsection \ref{program:kappa:tuple:members}),
the member functions needed to start working with a \progclass{Tuple} (Subsubsection \ref{program:kappa:tuple:get_started}) and 
the functions that represent basic properties of a tuple (Subsubsection \ref{program:kappa:tuple:basics}). 
Afterwards, we explain the class methods computing the orientation sign (Subsubsection \ref{program:kappa:tuple:orientation_sign}), 
the horizontal differential (Subsubsection \ref{program:kappa:tuple:d_hor}) and the implementation of the maps $f$ and $\Phi$ used to compute the isomorphism $\kappa$ (Subsubsection \ref{program:kappa:tuple:prep_for_kappa}) in detail. 

\subsubsection{Data Members [H]}
\label{program:kappa:tuple:members}
Since the class \progclass{Tuple} is supposed to represent cells $\Sigma = (\tau_h \mid \ldots \mid \tau_1)$ of the Ehrenfried complex $\Ehrprog$, 
let us briefly recall what this means. 
If $\Sigma$ is a cell of the parallel Ehrenfried complex 
(compare Definitions \ref{cellular_models:parallel:inhomogeneous_notation}, \ref{cellular_models:ehrenfried}), 
it satisfies the following properties:
\begin{enumerate}
 \item[(i$_P$)] The transpositions $\tau_i$ act on the symbols $1, \dotsc, p$.
 \item[(ii$_P$)] The permutation $\sigma_h$ has exactly $m+1$ cycles.
 \setcounter{enumi}{2}
 \item Each symbol $1, \dotsc, p$ is contained in at least one $\tau_i$.
 \item $\Sigma$ is monotonous, i.e. $\height(\tau_h) \geq \dotsc \geq \height(\tau_1)$.
\end{enumerate}
On the other hand, if $\Sigma$ is a radial cell, it fulfills the same conditions, 
except for the first two, which are replaced by
\begin{enumerate}
 \item[(i$_R$)] The transpositions $\tau_i$ act on the symbols $0, \dotsc, p$,
 \item[(ii$_R$)] The permutation $\sigma_h$ has exactly $m$ cycles.
\end{enumerate}
see also Definitions \ref{cellular_models:radial:cells_in_inhomogenous_notation} and \ref{cellular_models:Ehrenfried:defi}.

Let us now see how the cell $\Sigma$ is represented by a \progclass{Tuple}.
The transpositions $\tau_h, \dotsc, \tau_1$ belonging to $\Sigma$ are stored as the data member
\begin{lstlisting}
std::vector< Transposition > rep;
\end{lstlisting}
where a \progclass{Transposition} is defined as a pair of unsigned integers.
The unsigned integer \progname{p} is stored as another data member of the class \progclass{Tuple},
and there is also a boolean \progname{radial} indicating whether $\Sigma$ is a radial cell or not (and thus a parallel cell).
Depending on this flag, there are other technical parameters to be set, 
e.g. the minimum symbol that may occur in a transposition, see conditions (i$_P$) and (i$_R$).

Our convention for writing a cell $\Sigma = (\tau_h \mid \ldots \mid \tau_1)$ is 
to store $\tau_i$ as \progname{rep[i-1]} for $1 \leq i \leq h$, since by default, a vector starts with the index $0$. 
To make the handling of \progclass{Tuples} more intuitive, 
we offer methods to access the \progclass{Transpositions} in a more canonical way, see Subsubsection \ref{program:kappa:tuple:get_started}. 
Another convention is that we always write \progclass{Transpositions} like $\tau_i = (a\ b)$ with $a > b$ in order to simplify the source code.

Since it is useful for the computation of the differential, 
another data member of the class \progclass{Tuple} is an unsigned integer \progname{id} indicating the index of a \progclass{Tuple} in its basis of the Ehrenfried complex.

\subsubsection{Class Methods To Get Started [H]}
\label{program:kappa:tuple:get_started}

We define two constructors for the class \progclass{Tuple}. 
Firstly, there is a constructor
\begin{lstlisting}
Tuple( size_t h );
\end{lstlisting}
which initializes \progname{p} with $0$ and allocates memory for a \progclass{Tuple} of $h$ \progclass{Transpositions} 
that are supposed to be filled later. 
Secondly, the constructor
\begin{lstlisting}
Tuple( uint32_t symbols, size_t h );
\end{lstlisting}
sets the data member \progmember{p} to be $symbols$ and also allocates memory for the $h$ many \progclass{Transpositions}. 

To create a \progclass{Tuple} $\Sigma = (\tau_h \mid \ldots \mid \tau_1)$, one can call one of these constructors, 
and afterwards initialize for each $i = 1, \dotsc, h$ the $i\Th$ \progclass{Transposition} $\tau_i$ using the non-const operator 
\begin{lstlisting}
Transposition& operator[]( size_t i );
\end{lstlisting}
It is also possible to use the const respectively non-const version of the method
\begin{lstlisting}
Transposition& at( size_t i );
\end{lstlisting}
to access the \progclass{Transposition} $\tau_i$ of the tuple $\Sigma$ for reading respectively writing. 

During the generation of the differential $\del_\E$, we use to mark \progclass{Tuples} as degenerate by
erasing its data member \progname{rep}.
To test whether a \progclass{Tuple} is non-degenerate, we hence define  
\begin{lstlisting}
operator bool();
\end{lstlisting}
which returns true if and only if this \progclass{Tuple} is non-empty.

Using the methods
\begin{lstlisting}
static void parallel_cell(); 
\end{lstlisting}
respectively
\begin{lstlisting}
static void radial_cell();
\end{lstlisting}
one can mark whether a \progclass{Tuple} represents a parallel respectively radial cell.

Furthermore, we define canonical compare operators, 
overload the \progname{operator<<} to print \progname{Tuples} to screen
and offer the possibility to save and load \progclass{Tuples}.

\subsubsection{Basic Properties of a Tuple [H]}\label{program:kappa:tuple:basics}

We provide various class methods for basic calculations with a \progclass{Tuple},
e.g. to verify whether a \progclass{Tuple} represents an element of the Ehrenfried complex. 

Since basis elements of $\Ehrprog$ are required to be monotonous, we provide a method
\begin{lstlisting}
bool monotonous();
\end{lstlisting}
to test whether a \progclass{Tuple} is monotonous, i.\,e. whether, for all $1 \leq i < h$, we have $a_i < a_{i+1}$, 
writing $\tau_i = (a_i, b_i)$ and $\tau_{i+1} = (a_{i+1}, b_{i+1})$ with $a_i > b_i$ and $a_{i+1} > b_{i+1}$.

The class method
\begin{lstlisting}
int32_t norm();
\end{lstlisting}
returns the norm of the given \progclass{Tuple} $\Sigma = (\tau_h \mid \ldots \mid \tau_1)$, 
i.\,e. the sum of the norms of all $\tau_i$, $i = 1, \dotsc, h$. 
But since each $\tau_n$ is a \progclass{Transposition}, the norm of the \progclass{Tuple} is simply the number of its transpositions $h$. 

For the computation of the differential, we sometimes want to switch to the homogeneous notation of \progclass{Tuples}. 
Therefore, we need methods
\begin{lstlisting}
Permutation long_cycle(); 
\end{lstlisting}
respectively
\begin{lstlisting}
Permutation long_cycle_inv();
\end{lstlisting}
returning the \progclass{Permutation} $\sigma_0 = (1\ 2\ \dotsb p-1\ p)$ respectively its inverse, and
\begin{lstlisting}
Permutation sigma_h();
\end{lstlisting}
returning the \progclass{Permutation} $\sigma_h = \tau_h \tau_{h-1} \dotsc \tau_1 \sigma_0$.
Thereby, a \progclass{Permutation} is another class, representing a permutation as a vector
storing for each element its image under the permutation.

Since all $\tau_i$ are \progclass{Transpositions}, 
we can simplify the computation of $\sigma_h$ using the following

\begin{algorithm}[H]
\label{sigma_q}
\DontPrintSemicolon

\KwIn{A tuple $\Sigma = (\tau_h, \dotsc, \tau_1)$ in inhomogeneous notation}
\KwOut{The permutation $\sigma_h$}

\progclass{Permutation} $\sigma_{\text{inv}} := \text{long\_cycle\_inv()}$\;
\For{$i = 1$ \KwTo $h$}
{
	Write $\tau_i = (a, b)$\;
	Swap the elements $\sigma_{\text{inv}}(a)$ and $\sigma_{\text{inv}}(b)$\;
}

\For{$j = 1$ \KwTo $p$}
{
	Write $k = \sigma_{\text{inv}}(j)$\;
	$\sigma_h(k) := j$\;
}

\KwRet{$\sigma_h$}\;

\caption{Computing $\sigma_h$}

\end{algorithm}

\begin{prop}
The given algorithm computes $\sigma_q$ correctly.
\begin{proof}
In line 1, we initialize the \progclass{Permutation} $\sigma_{\text{inv}}$ with $\sigma_0^{-1}$. 
Note that, by definition of $\sigma_i$, we have
\[ \sigma_i^{-1} = \sigma_{i-1}^{-1} \tau_i = \sigma_{i-1}^{-1}  (a, b) \]
for $0 < i \leq h$. 
Thus, the \progclass{Permutation} $\sigma_i^{-1}$ maps $a$ to $\sigma_{i-1}^{-1}(b)$, $b$ to $\sigma_{i-1}^{-1}(a)$ 
and behaves like $\sigma_{i-1}^{-1}$ on all other elements. 
Hence, for each $i = 1, \dotsc, h$, we have $\sigma_{\text{inv}} = \sigma_i^{-1}$ 
after the $i\Th$ iteration of the first for loop. 

The second for loop computes the inverse of $\sigma_{\text{inv}} = \sigma_h^{-1}$, which is $\sigma_h$.
\end{proof}
\end{prop}

The algorithm for the computation of $\sigma_h$ is also used with small adaptions in different parts of the program.

The basis elements of $\Ehrprog$ are supposed to have a distinguished number of punctures, 
i.e. number of cycles of $\sigma_h$ is $m+1$ for parallel and $m$ for radial cells.
Therefore, we give an algorithm to determine the cycle decomposition of a \progclass{Permutation}.

\begin{algorithm}[H]
\label{Cycle Decomp}
\DontPrintSemicolon
\SetKw{KwGoTo}{go to}

\KwIn{A permutation $\pi$ on a subset of $\{0, \dotsc, p\}$}
\KwOut{A decomposition of $\pi$ into disjoint cycles}

$i := \min\{j \in \{0, \dotsc, p\} \colon j \text{ belongs to $\pi$, but not visited yet}\}$ \label{next cycle}\;
cur := $i$\; 
Initialize a new cycle\;
\Repeat(\tcp*[f]{Find the cycle containing $i$}){ cur equals $i$}
{	
  Mark cur as visited\;
	prev := cur\;
	cur := $\pi$(prev)\;
	cycle (prev) := cur\;
}
Store the cycle\;
\If{all symbols in $\{0, \dotsc, p\}$ visited or not belonging to $\pi$}
{
  \KwRet the cycle decomposition\; 
}
\KwGoTo \ref{next cycle}\;

\caption{Cycle Decomposition}

\end{algorithm}

Since each element in $\{0, \dotsc, p\}$ belonging to the permutation $\pi$ is considered as $prev$ exactly once, we can state

\begin{prop}
The algorithm to determine the cycle decomposition of a permutation works correctly.
\end{prop}

Combining this with the previous Algorithm \ref{sigma_q}, we can now define the methods
\begin{lstlisting}
uint32_t num_cycles() const;
\end{lstlisting}
yielding the number of cycles of $\sigma_h$, and
\begin{lstlisting}
bool has_correct_num_cycles(size_t m) const;
\end{lstlisting}
checking whether the number of cycles fits the requirement of the parallel respectively radial Ehrenfried complex.

Since we need to subdivide the cells of $\Ehrprog$ according to their numbers of clusters, 
we introduce the method
\begin{lstlisting}
int32_t Tuple::num_clusters() const;
\end{lstlisting}
This is another part of our computer program where we use the comfortability of the \textbf{boost} library:
Let $\Sigma = (\tau_h \mid \dotsc \mid \tau_1)$ be a cell represented by a \progclass{Tuple}.
We construct a graph on the vertices $0, \dotsc, p$,
where an edge between $a$ and $b$ indicates that there is an $i$ such that $\tau_i = (a, b)$.
Then, the number of connected components of this graph equals the number of clusters of $\Sigma$,
and \textbf{boost} offers a graph data structure and an algorithm to compute this directly.

\subsubsection{The Horizontal Face Operator [H]}
\label{program:kappa:tuple:d_hor}

In order to construct the Ehrenfried complex $\Ehrprog$, 
the computer program has to be able to apply the horizontal differential $\del'' = \del_\KK$ to \progclass{Tuples}, 
compare Definition \ref{cellular_models:Ehrenfried:defi} and Section \ref{cellular_models:orientation}.
So recall that, for a cell $\Sigma$, this differential is given by the alternating sum
\[
 \del''_j(\Sigma) = \sum_{j = 0}^p (-1)^j \eps_j(\Sigma) d_j''(\Sigma)\,,
\]
where $d_j''(\Sigma)$ is the $j\Th$ horizontal face of $\Sigma$, 
i.e. the cell resulting from $\Sigma$ by collapsing the $j\Th$ stripe of the slit domain,
and $\eps_j(\Sigma) = \eps_j''(\Sigma)$ is the additional sign introduced by the orientation system.
Note that, for parallel cells -- but not for radial cells --, the $0\Th$ and $p\Th$ horizontal face is always degenerate.

Here, we will explain our implementation of the face operator $d''$, 
and in Subsubsection \ref{program:kappa:tuple:orientation_sign}, we will present the orientation sign.

In Proposition \ref{cellular_models:parallel:prop_dh}, 
we saw how to express the formula for the $j\Th$ horizontal face in the inhomogeneous notation,
and in Corollary \ref{cellular_models:ehrenfried:cor_d_hor_deg}, we saw how to detect the cases when the resulting face is degenerate.
These statements result in the following

\begin{algorithm}[H]
\label{d_hor}
\DontPrintSemicolon
\SetKw{KwGoTo}{go to}

\KwIn{A tuple $\Sigma = (\tau_h, \dotsc, \tau_1)$ in inhomogeneous notation, a symbol $j \in \{0, \dotsc, p\}$}
\KwOut{The horizontal face $d''_j(\Sigma)$ or the assertion that $d''_j(\Sigma)$ is degenerate}
\For{$i = 1$ \KwTo $h$}
{
	Write $\tau_i = (a, b)$\;
	Write $k = \sigma_{i-1}(j)$\;
	\If{The transpositions $\tau_i$ and $(j, k)$ are disjoint}
	{
	  $\tau'_i := \tau_i$\;
	}
	\ElseIf{$\tau_q = (j, k)$ or $j = k$}
	{
	  \KwRet{$d''_j(\tau)$ is degenerate}\;
	}
	\Else
	{
	  \If{$k = a$ or $k = b$}
	  {
	    $\tau'_i := \tau_i$\;
	  }
	  \Else
	  {
	    \If{$a \neq k$}
	    {
	      $\tau_i' := (a, k)$\; 
	    }
	    \Else
	    {
	      $\tau_i' := (b, k)$\;
	    }
	  }
	}
}
Renormalize $\Sigma' = (\tau_h' \mid \ldots \mid \tau_1')$\;
\KwRet{$\Sigma'$}\;

\caption{Computing the Horizontal Face}

\end{algorithm}

Using the above mentioned theoretical foundations, we obtain

\begin{prop}
The above algorithm computes the $j\Th$ horizontal face of $\Sigma$.
\end{prop}

Note that we can apply Algorithm \ref{sigma_q} to compute $\sigma_i$ for $i = 0, \dotsc, h$.
We choose to handle the case that $\tau_i$ and $(j, k)$ are disjoint before the degenerate case at first
because this is the case that will occure most likely.

The algorithm enables us to define the method
\begin{lstlisting}
Tuple d_hor( uint8_t j );
\end{lstlisting}
computing the $j\Th$ horizontal face of this \progclass{Tuple}.

\subsubsection{The Orientation Sign [H]}\label{program:kappa:tuple:orientation_sign}

Recalling the definition of the orientation sign $\eps_j(\Sigma) = \eps_j''(\Sigma)$ introduced by Mehner (see Section \ref{cellular_models:orientation}), 
we immediately obtain the following

\begin{algorithm}[H]
\label{orientation_sign}
\DontPrintSemicolon
\SetKw{KwGoTo}{go to}

\KwIn{A tuple $\Sigma = (\tau_h, \dotsc, \tau_1)$ in inhomogeneous notation}
\KwOut{Orientation signs $\eps_j(\Sigma)$ for all $j \in \{0, \dotsc, p\}$}

Decompose $\sigma_h$ into $m$ resp. $m+1$ disjoint cycles $(\alpha_0) \alpha_1 \dotsc \alpha_m$ \;
Let $a_i$ be the minimum symbol of the cycle $\alpha_i$\;
Sort the cycles $\alpha_i$ such that $a_i < a_{i+1}$ for all $i$\;
\ForEach{cycle $\alpha_i$}
{
  \If{$\alpha_i = a_i$ is a fixed point}
  {
    Set $\eps_{a_i} = 0$\;
  }
  Let $b$ be the second minimum cycle of $\alpha_i$\;
  Let $k \geq i$ be the minimum integer with $b < a_{k+1}$, or $k = m$, if this does not exist\;
  Set $\eps_{a_i} = (-1)^{k - i}$\;
  \ForEach{$c \in \alpha_i$, $c \neq a_i$}
  {
    Set $\eps_c = 1$\;
  }
}

\caption{Computing the Orientation Sign}

\end{algorithm}

Hereby, we again use Algorithm \ref{sigma_q} to determine $\sigma_h$, 
and Algorithm \ref{Cycle Decomp} to decompose $\sigma_h$ into disjoint cycles.
Note that, in the parallel case, a cell of $\Ehrprog$ has $m+1$ cycles, while in the radial case, it has $m$ cycles.

Storing the cycle decomposition of $\sigma_h$ as a map of cycles stored with their smallest element as a key,
this algorithm is easily implemented.
We obtain the method
\begin{lstlisting}
std::map< uint8_t, int8_t > orientation_sign();
\end{lstlisting}
returning a map of all orientation signs $\eps_j(\sigma_h)$, stored with $j \in \{1, \dotsc, p\}$ as a key.

\subsubsection{Preparations for the Map \texorpdfstring{$\kappa$}{kappa} [H]}
\label{program:kappa:tuple:prep_for_kappa}

Having seen how the differential $\del_\KK$ is implemented, 
recall once more that the differential $\del_\E$ of the Ehrenfried complex is given by the diagram
\[
    \begin{tikzcd}
	\E_p \arrow{r}{\del_\E} \arrow{d}{\kappa}[swap]{\cong}      & \E_{p-1} \\
	\KK_p \arrow{r}{\del_\KK}                                     & \KK_{p-1} \arrow{u}{\cong}[swap]{\pi}
    \end{tikzcd}\,.
\]
Note that the projection $\pi$ onto the monotonous cells can be performed by 
simply checking whether the \progclass{Tuple} is monotonous (see Subsection \ref{program:kappa:tuple:basics})
and marking the \progclass{Tuple} as non-valid, if not.
It remains to define the isomorphism $\kappa$, which is given by
\[
    \kappa = K_h \circ \ldots \circ K_1
\]
with
\[
    K_q = \sum_{j=1}^q (-1)^{q-j} \Phi_{j}^q
\]
and
\[
    \Phi_j^q = f_j \circ \ldots f_{q-1} \,.
\]

Therefore, the class \progclass{Tuple} provides maps
\begin{lstlisting}
bool f( uint32_t j );
\end{lstlisting}
and 
\begin{lstlisting}
bool phi( uint32_t q, uint32_t j );
\end{lstlisting}
which apply the maps $f_j$ respectively $\Phi_j^q$ to this \progclass{Tuple} 
and return true if and only if the resulting \progclass{Tuple} is non-degenerate. 

The method $f$ is implemented as a huge case distinction 
concerning the symbols contained in the transpositions $\tau_j$,
which is similar to the one in the computation of the horizontal boundary. 
This provides the opportunity to handle each of the cases in constant time.

The function $\phi(q, j)$ iteratively calls the function $f$ for $j = 1, \dotsc, q-1$
according to the definition of $\Phi_j^q$.
In each step, we test whether the norm of the \progclass{Tuple} decreases, and if so, 
we abort the computation of $\Phi$ to avoid unneccassary computations.

The map $\kappa$ itself is defined in the class \progclass{ClusterSpectralSequence}, compare Subsection \ref{chapter_program:kappa:css}.
\subsection{The Class ClusterSpectralSequence}
\label{chapter_program:kappa:css}

Having described how basis elements of the Ehrenfried complex $\Ehrprog$ are represented by our computer program, 
we will now explain how the cluster spectral sequence associated with $\Ehrprog$ is realized 
and how the methods provided by the class \progclass{Tuple} (compare Subsection \ref{chapter_program:kappa:tuple}) are combined 
to compute its differentials.
For these purposes, we introduce the class \progclass{ClusterSpectralSequence}, 
which is a template class with the class type of the underlying \progclass{ChainComplex} as a template parameter
(compare Subsection \ref{program:libhomology:ChainComplex}).

At first, we are going to describe the class \progclass{CSSBasis} in Subsection \ref{chapter_program:kappa:css:css_basis}, 
which represents the basis of a single module of the cluster spectral sequence. 
Next, we describe the data members of the \progclass{ClusterSpectralSequence} (Subsection \ref{chapter_program:kappa:css:data_members}),
which are most importantly a collection of bases of the modules of the Ehrenfried complex and 
a collection of its differentials.
Thereafter, we explain how the bases (see Subsubsection \ref{chapter_program:kappa:css:gen_bases}) and 
the differentials (see Subsubsection \ref{chapter_program:kappa:css:gen_diff}) are generated, 
and what other methods are provided by the class \progclass{ClusterSpectralSequence} (see Subsubsection \ref{chapter_program:kappa:css:more_methods}). 

\subsubsection{CSSBasis}
\label{chapter_program:kappa:css:css_basis}

Just like the cluster spectral sequence associated with $\Ehrprog$ consists of a finite number of modules, 
our class \progclass{ClusterSpectralSequence} contains a finite number of \progclass{CSSBases}, 
which are structs representing the bases of these modules.

Thus the only data member of the struct \progclass{CSSBasis} is the basis
\begin{lstlisting}
BasisType basis;
\end{lstlisting}
Thereby, \progclass{BasisType} is a map storing all \progclass{Tuples} of this basis, sorted by cluster sizes.
For each cluster size $l$, we organize the corresponding \progclass{Tuples} in an \progname{std::unordered\_set}
with an appropriate hash function that makes it possible to search for basis elements in amortized constant running time.

The struct \progclass{CSSBasis} provides the usual methods for saving and loading as well as the functions
\begin{lstlisting}
uint64_t size( int32_t l ) const;
\end{lstlisting}
returning the number of basis elements with exactly $l$ clusters and 
\begin{lstlisting}
uint64_t total_size() const;
\end{lstlisting}
returning the total number of basis elements.

Since the computation of the differential (compare Subsubsection \ref{chapter_program:kappa:css:gen_diff}) requires 
a unique identification of one basis element among the other basis elements of the same cluster size of one \progclass{CSSBasis}, 
there is a method
\begin{lstlisting}
int64_t id_of( Tuple& t );
\end{lstlisting} 
returning the unique \progname{id} of the given \progclass{Tuple} \progname{t}. 
If \progname{t} is not an element of this \progclass{CSSBasis}, we return \progname{-1} to indicate the failure of the function.

The most important method of the struct \progclass{CSSBasis} is the function
\begin{lstlisting}
uint32_t add_basis_element ( Tuple& t );
\end{lstlisting}

Using the method \progname{num\_clusters} of the class \progclass{Tuple} (see Subsubsection \ref{program:kappa:tuple:basics}),
it inserts the \progclass{Tuple} \progname{t} into the part of the basis corresponding to its number of clusters
and sets the \progname{id} of \progname{t} to the current number of basis elements with exactly $l$ clusters. 
This means that if one builds up a \progclass{MonoBasis} by successively adding basis elements,
all basis elements can be distinguished by their \progname{ids}.

\subsubsection{Data Members}
\label{chapter_program:kappa:css:data_members}

The class \progclass{ClusterSpectralSequence} represents the cluster spectral sequence assoicated with the Ehrenfried complex.
Hence, it contains a collection of \progclass{CSSBases} 
\begin{lstlisting}
std::map< uint32_t, CSSBasis > basis_complex;
\end{lstlisting}
where for each $0 \leq p \leq 2h$, the basis elements of the Ehrenfried complex on the symbols $0, \dotsc, p$ are stored,
and the data member
\begin{lstlisting}
MatrixComplex diff_complex;
\end{lstlisting}
where at each time, the differential needed for computations is stored.

Note that \progclass{MatrixComplex} is a template parameter. 
Depending on the coffficients of the homology one wants to compute,
it can be chosen to be \progclass{ClusterSpectralSequenceQ} or \progclass{ClusterSpectralSequenceZm}.
For coefficients in the field $\mathbb F_2$, we highly recommend to use \progclass{ClusterSpectralSequenceBool} for efficiency reasons.
The genus \progmember{g}, the number of punctures \progmember{m} and the number \progmember{h} of transpositions in a basis tuple 
associated with this \progclass{MonoComplex} are also stored as data members. 
Furthermore, the data member
\begin{lstlisting}
SignConvention sign_conv;
\end{lstlisting}
indicates which sign convention (see also Subsubsection \ref{chapter_program:kappa:compute_css:implementation}) is used to compute the homology of this \progclass{MonoComplex},
and the data member
\begin{lstlisting}
size_t num_threads;
\end{lstlisting}
determines the number of threads using for the construction of the differentials.

\subsubsection{Generating Bases}
\label{chapter_program:kappa:css:gen_bases}

The first step to build up a \progclass{ClusterSpectralSequence} is to call the constructor
\begin{lstlisting}
ClusterSpectralSequence( uint32_t genus, 
                         uint32_t num_punctures, 
                         SignConvention sgn, 
                         uint32_t num_working_threads, 
                         uint32_t num_remaining_threads );
\end{lstlisting}
For an explanation of the parameters \progname{num\_working\_threads} and \progname{num\_remaining\_threads}, 
see Subsubsection \ref{diag_field_implementation}.
In the constructor, \progclass{Diagonalizer} is configured (compare Subsection \ref{program:libhomology:DiagonalizerT}), 
and the bases of all the modules belonging to the \progclass{MonoComplex} are initialized 
via a recursive method we want to explain now. 

Our aim is to enumerate all cells $\Sigma = (\tau_h \mid \ldots \mid \tau_1)$ 
of bidegree $(p, h)$ for all $0 \leq p \leq 2h$ such that
\begin{enumerate}
 \item All $\tau_i$ are non-trivial transpositions on the symbols $min, \dotsc, p$.
 \item Each symbol $min, \dotsc, p$ is permuted non-trivially by at least one $\tau_i$.
 \item $\Sigma$ is monotonous.
\end{enumerate}
Hereby, in the parallel case, $min$ equals $1$, and in the radial case, we want to enumerate all cells for $min = 0$ or $min = 1$.

The enumeration of all these cells works recursively.
Let $\Sigma = (\tau_k \mid \ldots \mid \tau_1)$ be a cell fulfilling the above conditions, 
but with bidegree $(p, k)$ with $k < h$.
Assume we want to find all possibilities to extend such a cell $\Sigma$ to a monotonous tuple $\Sigma'$ of $k+1$ transpositions
by inserting a transposition $\tau_{k + 1}$, perharps using more symbols. 
The following cases occur.

\begin{enumerate}
\item We also use the symbols $min, \dotsc, p$ for $\tau_{k+1}$.
      Since $\Sigma'$ is supposed to be monotonous, 
      $\tau_{k + 1}$ needs to contain the symbol $p$.
      Hence we can set $\tau_{k + 1} := (p, i)$ with $i \in \{min, \dotsc, p-1\}$ chosen arbitrarily.
\item We insert a new row into our parallel slit domain and use the symbols $min, \dotsc, p + 1$ for $\Sigma'$.
      By monotony, $\tau_{k + 1}$ has to contain the highest symbol $p + 1$. 
      But now there are two possibilities:
      \begin{enumerate}
      \item The new row is inserted as a $(p+1)\Th$ row above the old rows. 
            Hence the symbol that is contained in $\tau_{k + 1}$ apart from $p + 1$
            is one of the symbols that were already used for the transpositions $\tau_1, \dotsc, \tau_k$, 
            and we can set $\tau_{k + 1} := (p + 1, i)$ with $i \in \{min, \dotsc, p\}$ chosen arbitrarily.
      \item The new row is inserted as the $i\Th$ row for some $i \in \{min, \dotsc, p\}$  
            and the indices of the old rows $min, \dotsc, p$ are shifted up by one.
            Note that these indices are also shifted up in the transpositions of $\Sigma$.
            By monotony and since all symbols in $\{min, \dotsc, p\}$ have to be covered, 
            the new transposition has to be $\tau_{k + 1} := (p + 1, i)$, 
            meaning that the symbol $p + 1$ is also used by at least one of the transpositions $\tau_1, \dotsc, \tau_k$, 
            and that the symbol $i$ is used by $\tau_{k + 1}$ only.
      \end{enumerate}
\item We insert two rows into our slit domain, using the symbols $\{min, \dotsc, p + 2\}$. 
      Thus the symbols used by the new transposition $\tau_{k + 1}$ do not yet appear in $\Sigma$.
      Therefore $\tau_{k + 1}$ has to contain the symbol $p + 2$ and some other symbol $i \in \{min, \dotsc, p + 1\}$.
      Again, the indices of the rows $i + 1, \dotsc, p$ have to be shifted up by $1$ in the transpositions of $\Sigma$.
\end{enumerate}

We can use these observations to define the recursive method
\begin{lstlisting}
void gen_bases( uint32_t k, uint32_t p, uint32_t min, Tuple& tuple );
\end{lstlisting}
This methods gets as input data a monotonous \progclass{Tuple} consisting of $k$ transpositions 
which contain each of the symbols $min, \dotsc, p$ at least once,
and the minimum symbol the transpositions may contain.
Using the above case distinction, it calls itself recursively with the appropriate parameters
and stores all the transpositions with $h$ transpositions detected thereby in the corresponding basis,
but only if they contain the required number of cycles.

Now we can describe the way the constructor of the \progclass{CSS} sets up its \progmember{basis\_complex}.

\begin{prop}
We can enumerate all basis elements of the parallel Ehrenfried complex
by defining the \progclass{Tuple} $\Sigma = ((2, 1))$ consisting of only one \progclass{Transposition},
and then calling the recursive function \progname{gen\_bases(1, 2, 1, $\Sigma$)}. 

For enumerating all basis elements of the radial Ehrenfried complex,
we additionally need to call the recursive function \progname{gen\_bases(1, 1, 0, $\Sigma'$)} with $\Sigma' = ((1, 0))$.
\begin{proof}

Note that for a given $\Sigma$, the tranpositions arising from the different cases fulfill the above conditions (i) - (iii),
and that they are all distinct.
If we consider two different \progclass{Tuples}, 
the transpositions arising from the case distinction also cannot coincide
since either they have different numbers of transpositions or
their starting sequence of transpositions already differs.
Especially, we don't enumerate \progclass{Tuples} multiple times by the two initial calls in the radial case
since the first family of \progclass{Tuples} does not contain the symbol $0$, and the second family does.

Hence it suffices to show that all monotonous transpositions can be found by our algorithm.
By induction on the number of transpositions, 
we can assume that we have already built up all monotonous tuples with $k$ transpositions.
Let $\Sigma' = (\tau_{k + 1}\mid \dotsc\mid \tau_1)$ be monotonous. 
Then $\tau_{k + 1} = (p+1, i)$ with $i \in \{1, \dotsc, p\}$ by monotony.
The tuple $\Sigma := (\tau_k \mid \dotsc \mid \tau_1)$ is also monotonous, 
but we might have to shift the indices $i, \dotsc, p+1$ down by one
if the symbol $i$ is not contained in the transpositions $\tau_1, \dotsc, \tau_k$.
This yields one of the tuples of $k$ transpositions we have already found.
By reverting the process just described, we can rebuild $\Sigma'$ from $\Sigma$, 
and this case is covered by the above case distinction.
\end{proof}
\end{prop}

\subsubsection{Generating Differentials}
\label{chapter_program:kappa:css:gen_diff}

In Subsubsection \ref{chapter_program:kappa:compute_css:implementation}, we explained how both the $E^1$-term and $E^2$-term are computed.
Recalling Chapter \ref{cellular_models}, the $p\Th$ differential of the Ehrenfried complex is given by the composition $\pi \del''_p \kappa$.
The member function
\begin{lstlisting}
gen_d0( const int32_t p, const int32_t l );
\end{lstlisting}
generates the restriction of the $p\Th$ differential $\del_\E$ to the cells of $\Ehrprog$ with exactly $l$ clusters.
Similarly, the method
\begin{lstlisting}
gen_d1_stage_1( const int32_t p, const int32_t l );
\end{lstlisting}
generates the restriction of the $p\Th$ differential $\del_\E$ from the cells with $l$ clusters to the cells with $l-1$ clusters and
applies all row operations to this restriction that come from the sub-matrix $d^0$ from above.

For runtime reasons, we thereby use yet another formula for the map $\kappa$.
\begin{prop}
    Let $I'$ be the set of tuples of integers $(t_h, \ldots, t_1)$ such that $0 \leq t_q < q$.
    Then the map
    \[
        \{ 0, \ldots, h! - 1\} \to I'
        \mspc{with}{20}
        k \mapsto \left( \left\lfloor \frac{k}{(q-1)!} \right\rfloor \pmod q\right)_q
    \]
    is a bijection.
\end{prop}
\begin{proof}
We show that
\[
    (t_h, \ldots, t_1) \mapsto \sum_{q=2}^h t_q \cdot (q-1)!
\]
defines an inverse of the above map. 
Since both sets have the same cardinality, 
it suffices to show that this map is a right inverse.
For a fixed coordinate $q \in \{1, \dotsc, h\}$, we compute
\[
    \sum_{q=2}^h t_q \cdot (q-1)! = \sum_{q=r}^h t_q \cdot (q-1)! + \sum_{q=2}^{r-1} t_q \cdot (q-1)!\,.
\]
Using induction on $r$, we see that
\[
    \sum_{q=2}^{r-1} t_q \cdot (q-1)! < (r-1)!\,,
\]
since for $r = 1$, the statement is true, and if we assume the statement for $r-1$, we can conclude
\begin{align*}
    \sum_{q=2}^{r} t_q \cdot (q-1)! &= \sum_{q=2}^{r-1} t_q \cdot (q-1)! + t_r (r-1)! \\
                                    &< (r-1)!               + (r-1) (r-1)! \\
                                    &= r!\,.
\end{align*}
From this, we get
\[
    \left\lfloor \frac{\sum_{q=2}^h t_q \cdot (q-1)!}{(r-1)!} \right\rfloor=\left\lfloor \sum_{q=r}^h t_q \cdot \frac{(q-1)!}{(r-1)!} + \sum_{q=2}^{r-1} t_q \cdot \frac{(q-1)!}{(r-1)!} \right\rfloor = \sum_{q=r}^h t_q \cdot \frac{(q-1)!}{(r-1)!}
\]
since the left sum is integral and the right sum vanishes after rounding down.
Taking the remaining term modulo $r$, we get $t_r$ as a result since all summands but the $r\Th$ are zero modulo $r$.
\end{proof}

\subsubsection{More Member Functions}
\label{chapter_program:kappa:css:more_methods}

We also offer class methods to print the \progmember{basis\_complex} and the \progmember{matrix\_complex}, and a method
\begin{lstlisting}
void erase_differential( int32_t p );
\end{lstlisting}
which deletes the \progmember{p}${}\Th$ differential and releases its memory. 
This function is defined because we only want to keep a differential in the memory
as long as we really need it due to storage limitation.

\section{Remarks on Compiling}
\label{program:compiling}

Our software projects were developed and tested on {\bf Debian 7}, {\bf Ubuntu 12.04 LTS} and {\bf Open SUSE 13.1}.
If you use a different operating system, your compiler has to support the full \cppeleven\ standard.
We suggest to use the \cpp\ compiler of the {\bf Gnu Compiler Collection}.

\subsection{Installing the Required Software}
Using an operating system based on {\bf Debian}, it should suffice to install the software and libraries from the official repositories.
\begin{lstlisting}
sudo apt-get install \
    build-essential g++ libboost-all-dev libgmp-dev libbz2-dev
\end{lstlisting}
We document the source code with the {\bf Doxygen-syntax}.
Thus we can generate a documentation using the program {\bf doxygen}.
It is installed as follows.
\begin{lstlisting}
sudo apt-get install doxygen doxygen-gui doxygen-latex
\end{lstlisting}

\subsection{Building the Projects}
Using the provided makefile, the executables are built as follows.
\begin{lstlisting}
make compute_cache
make compute_css
make compute_statistics
make print_basis
\end{lstlisting}
Moreover, you can create your own executables (with a given name say \progname{my\_program})
by creating a corresponding \progname{.cpp} file in the subfolder \progname{./kappa} with the prefix \progname{main\_} (e.g.\ you create \progname{./kappa/main\_my\_program.cpp}).
Calling \progname{make} with the name of your project will create an executable with this name.

\subsection{More Remarks}
Using {\bf doxygen} you can generate a documentation of the source code as follows. 
\begin{lstlisting}
make doc
\end{lstlisting}
The documentation itself can be found in the subdirectory \progname{./doc} in the corresponding project.

The libraries, executables and documentation can be cleaned up as always.
The generated results are kept.
\begin{lstlisting}
make clean
\end{lstlisting}

\section{Results}
\label{program:results}

As discussed above, our program computes the cluster spectral sequence of a given Ehrenfried complex, i.e.\ we compute the cohomology of the associated moduli space.
In this section, we list the results our computer program provides.
Hereby, we distinct the parallel and radial case.
We compute the cluster spectral sequence with respect to coefficients $R$ being either the rationals $\mathbb Q$ or the field $\mathbb F_2$.
Given parameters $g$ and $m$, we provide several tables with entries $\dim_R E^s_{p,l}$ for
\begin{enumerate}
    \item $s$ equals $0$, $1$ or $2$,
    \item $p$ the homological degree and
    \item $l$ the cluster number.
\end{enumerate}
The rightmost column of each table notes $\dim_R \sum_{l} E^s_{p,l}$.
For $s=2$, we thus provide the dimensions of the homology of the moduli spaces $\dim_R H_{2h-p}(\mathfrak M;R)$,
where $h = 2g+m$ in the parallel case and $h = 2g + m -1$ in the radial case.

\subsection{The Parallel Case [B]}
\subsubsection{Genus \texorpdfstring{$g=0$}{g=0} and Punctures \texorpdfstring{$m=0,\ldots,6$}{m=0,...,6}}
For $g=0$ and $n=1$, the moduli space $\Modspc[1]$ is the classifying space of the braid group on $m$ strings.
Its homology is understood.
Moreover, it is easy to see that degree and cluster number agrees for any cell,
so the cluster spectral squence does not give any new insights.
We list our computations anyways.

\paragraph{The case $g=0$, $m=1$ and $h=1$ with coefficients in $\mathbb F_2$ and $\mathbb Q$:}
\begin{center}
    \begin{tabular}{r||r|r||r|}
        \cline{2-3}
        \multicolumn{1}{r|}{} & \multicolumn{2}{c|}{$E^0_{p,l}$ for $\mathbb F_2$ and $\mathbb Q$} \\ \hline
        \tl{\diagbox[height=1.7em, width=3em]{$p$}{$l$}} & 1 & 2& $\dim$ \\ \hline\hline
        \tl 2  & 1     &   & 1\\ \hline
    \end{tabular}
    
    \vspace{1cm}
    
    \begin{tabular}{r||r|r||r|}
        \cline{2-3}
        \multicolumn{1}{r|}{} & \multicolumn{2}{c|}{$E^1_{p,l}$ for $\mathbb F_2$ and $\mathbb Q$} \\ \hline
        \tl{\diagbox[height=1.7em, width=3em]{$p$}{$l$}} & 1 & 2& $\dim$ \\ \hline\hline
        \tl 2  & 1     &   & 1\\ \hline
    \end{tabular}
    
    \vspace{1cm}
    
    \begin{tabular}{r||r|r||r|}
        \cline{2-3}
        \multicolumn{1}{r|}{} & \multicolumn{2}{c|}{$E^2_{p,l}$ for $\mathbb F_2$ and $\mathbb Q$} \\ \hline
        \tl{\diagbox[height=1.7em, width=3em]{$p$}{$l$}} & 1 & 2& $\dimr{0}{1}{2}$ \\ \hline\hline
        \tl 2  & 1     &   & $1$\\ \hline
    \end{tabular}
\end{center}

\paragraph{The case $g=0$, $m=2$ and $h=2$ with coefficients in $\mathbb F_2$ and $\mathbb Q$:}
\begin{center}
    \begin{tabular}{r||r|r||r|}
        \cline{2-3}
        \multicolumn{1}{r|}{} & \multicolumn{2}{c|}{$E^0_{p,l}$ for $\mathbb F_2$ and $\mathbb Q$} \\ \hline
        \tl{\diagbox[height=1.7em, width=3em]{$p$}{$l$}} & 1 & 2& $\dim$ \\ \hline\hline
        \tl 3  & 2     &   & 2\\ \hline
        \tl 4  &       & 2 & 2\\ \hline
    \end{tabular}
        
    \vspace{1cm}
    
    \begin{tabular}{r||r|r||r|}
        \cline{2-3}
        \multicolumn{1}{r|}{} & \multicolumn{2}{c|}{$E^1_{p,l}$ for $\mathbb F_2$ and $\mathbb Q$} \\ \hline
        \tl{\diagbox[height=1.7em, width=3em]{$p$}{$l$}} & 1 & 2& $\dim$ \\ \hline\hline
        \tl 3  & 2     &   & 2\\ \hline
        \tl 4  &       & 2 & 2\\ \hline
    \end{tabular}
        
    \vspace{1cm}
    
    \begin{tabular}{r||r|r||r|}
        \cline{2-3}
        \multicolumn{1}{r|}{} & \multicolumn{2}{c|}{$E^2_{p,l}$ for $\mathbb F_2$ and $\mathbb Q$} \\ \hline
        \tl{\diagbox[height=1.7em, width=3em]{$p$}{$l$}} & 1 & 2& $\dimr{0}{2}{4}$ \\ \hline\hline
        \tl 3  & 1     &   & 1\\ \hline
        \tl 4  &       & 1 & 1\\ \hline
    \end{tabular}
\end{center}

\paragraph{The case $g=0$, $m=3$ and $h=3$ with coefficients in $\mathbb F_2$ and $\mathbb Q$:}
\begin{center}
    \begin{tabular}{r||r|r|r||r|}
        \cline{2-4}
        \multicolumn{1}{r|}{} & \multicolumn{3}{c|}{$E^0_{p,l}$ for $\mathbb F_2$ and $\mathbb Q$} \\ \hline
        \tl{\diagbox[height=1.7em, width=3em]{$p$}{$l$}} & 1 & 2 & 3& $\dim$ \\ \hline\hline
        \tl 4  & 5     &        &   & 5\\ \hline
        \tl 5  &       & 10     &   & 10\\ \hline
        \tl 6  &       &        & 5 & 5\\ \hline
    \end{tabular}
        
    \vspace{1cm}
    
    \begin{tabular}{r||r|r|r||r|}
        \cline{2-4}
        \multicolumn{1}{r|}{} & \multicolumn{3}{c|}{$E^1_{p,l}$ for $\mathbb F_2$ and $\mathbb Q$} \\ \hline
        \tl{\diagbox[height=1.7em, width=3em]{$p$}{$l$}} & 1 & 2 & 3& $\dim$ \\ \hline\hline
        \tl 4  & 5     &        &   & 5\\ \hline
        \tl 5  &       & 10     &   & 10\\ \hline
        \tl 6  &       &        & 5 & 5\\ \hline
    \end{tabular}
        
    \vspace{1cm}
    
    \begin{tabular}{r||r|r|r||r|}
        \cline{2-4}
        \multicolumn{1}{r|}{} & \multicolumn{3}{c|}{$E^2_{p,l}$ for $\mathbb F_2$ and $\mathbb Q$} \\ \hline
        \tl{\diagbox[height=1.7em, width=3em]{$p$}{$l$}} & 1 & 2 & 3& $\dimr{0}{3}{6}$ \\ \hline\hline
        \tl 4  & 0     &        &   & 0\\ \hline
        \tl 5  &       & 1      &   & 1\\ \hline
        \tl 6  &       &        & 1 & 1\\ \hline
    \end{tabular}
\end{center}

\paragraph{The case $g=0$, $m=4$ and $h=4$ with coefficients in $\mathbb F_2$ and $\mathbb Q$:}
\begin{center}
    \begin{tabular}{r||r|r|r|r||r|}
        \cline{2-5}
        \multicolumn{1}{r|}{} & \multicolumn{4}{c|}{$E^0_{p,l}$ for $\mathbb F_2$ and $\mathbb Q$} \\ \hline
        \tl{\diagbox[height=1.7em, width=3em]{$p$}{$l$}} & 1 & 2 & 3 & 4& $\dim$ \\ \hline\hline
        \tl 5   & 14    &       &       &   & 14\\ \hline
        \tl 6   &       & 42    &       &   & 42\\ \hline
        \tl 7   &       &       & 42    &   & 42\\ \hline
        \tl 8   &       &       &       & 14 & 14\\ \hline
    \end{tabular}
        
    \vspace{1cm}
    
    \begin{tabular}{r||r|r|r|r||r|}
        \cline{2-5}
        \multicolumn{1}{r|}{} & \multicolumn{4}{c|}{$E^1_{p,l}$ for $\mathbb F_2$ and $\mathbb Q$} \\ \hline
        \tl{\diagbox[height=1.7em, width=3em]{$p$}{$l$}} & 1 & 2 & 3 & 4& $\dim$ \\ \hline\hline
        \tl 5   & 14    &       &       &   & 14\\ \hline
        \tl 6   &       & 42    &       &   & 42\\ \hline
        \tl 7   &       &       & 42    &   & 42\\ \hline
        \tl 8   &       &       &       & 14 & 14\\ \hline
    \end{tabular}
            
    \vspace{1cm}
    
    \begin{tabular}{r||r|r|r|r||r|}
        \cline{2-5}
        \multicolumn{1}{r|}{} & \multicolumn{4}{c|}{$E^2_{p,l}$ for $\mathbb Q$} \\ \hline
        \tl{\diagbox[height=1.7em, width=3em]{$p$}{$l$}} & 1 & 2 & 3 & 4& $\dimq{0}{4}{8}$ \\ \hline\hline
        \tl 5   & 0     &       &       &   & 0\\ \hline
        \tl 6   &       & 0     &       &   & 0\\ \hline
        \tl 7   &       &       & 1     &   & 1\\ \hline
        \tl 8   &       &       &       & 1 & 1\\ \hline
    \end{tabular}
    
    \vspace{1cm}
    
    \begin{tabular}{r||r|r|r|r||r|}
        \cline{2-5}
        \multicolumn{1}{r|}{} & \multicolumn{4}{c|}{$E^2_{p,l}$ for $\mathbb F_2$} \\ \hline
        \tl{\diagbox[height=1.7em, width=3em]{$p$}{$l$}} & 1 & 2 & 3 & 4& $\dimf{0}{4}{8}$ \\ \hline\hline
        \tl 5   & 1     &       &       &   & 1\\ \hline
        \tl 6   &       & 1     &       &   & 1\\ \hline
        \tl 7   &       &       & 1     &   & 1\\ \hline
        \tl 8   &       &       &       & 1 & 1\\ \hline
    \end{tabular}
\end{center}

\paragraph{The case $g=0$, $m=5$ and $h=5$ with coefficients in $\mathbb F_2$ and $\mathbb Q$:}
\begin{center}
    \begin{tabular}{r||r|r|r|r|r||r|}
        \cline{2-6}
        \multicolumn{1}{r|}{} & \multicolumn{5}{c|}{$E^0_{p,l}$ for $\mathbb F_2$ and $\mathbb Q$} \\ \hline
        \tl{\diagbox[height=1.7em, width=3em]{$p$}{$l$}} & 1 & 2 & 3 & 4 & 5& $\dim$ \\ \hline\hline
        \tl 6   & 42    &       &       &       &  & 42\\ \hline
        \tl 7   &       & 168   &       &       &  & 168\\ \hline
        \tl 8   &       &       & 252   &       &  & 252\\ \hline
        \tl 9   &       &       &       & 168   &  & 168\\ \hline
        \tl{10} &       &       &       &       & 42& 42\\ \hline
    \end{tabular}
        
    \vspace{1cm}
    
    \begin{tabular}{r||r|r|r|r|r||r|}
        \cline{2-6}
        \multicolumn{1}{r|}{} & \multicolumn{5}{c|}{$E^1_{p,l}$ for $\mathbb F_2$ and $\mathbb Q$} \\ \hline
        \tl{\diagbox[height=1.7em, width=3em]{$p$}{$l$}} & 1 & 2 & 3 & 4 & 5& $\dim$ \\ \hline\hline
        \tl 6   & 42    &       &       &       &  & 42\\ \hline
        \tl 7   &       & 168   &       &       &  & 168\\ \hline
        \tl 8   &       &       & 252   &       &  & 252\\ \hline
        \tl 9   &       &       &       & 168   &  & 168\\ \hline
        \tl{10} &       &       &       &       & 42& 42\\ \hline
    \end{tabular}
            
    \vspace{1cm}
    
        \begin{tabular}{r||r|r|r|r|r||r|}
        \cline{2-6}
        \multicolumn{1}{r|}{} & \multicolumn{5}{c|}{$E^2_{p,l}$ for $\mathbb Q$} \\ \hline
        \tl{\diagbox[height=1.7em, width=3em]{$p$}{$l$}} & 1 & 2 & 3 & 4 & 5& $\dimq{0}{5}{10}$ \\ \hline\hline
        \tl 6   & 0     &       &       &       &  & 0\\ \hline
        \tl 7   &       & 0     &       &       &  & 0\\ \hline
        \tl 8   &       &       & 0     &       &  & 0\\ \hline
        \tl 9   &       &       &       & 1     &  & 1\\ \hline
        \tl{10} &       &       &       &       & 1& 1\\ \hline
    \end{tabular}
    
    \vspace{1cm}
    
    \begin{tabular}{r||r|r|r|r|r||r|}
        \cline{2-6}
        \multicolumn{1}{r|}{} & \multicolumn{5}{c|}{$E^2_{p,l}$ for $\mathbb F_2$} \\ \hline
        \tl{\diagbox[height=1.7em, width=3em]{$p$}{$l$}} & 1 & 2 & 3 & 4 & 5& $\dimf{0}{5}{10}$ \\ \hline\hline
        \tl 6   & 0     &       &       &       &  & 0\\ \hline
        \tl 7   &       & 1     &       &       &  & 1\\ \hline
        \tl 8   &       &       & 1     &       &  & 1\\ \hline
        \tl 9   &       &       &       & 1     &  & 1\\ \hline
        \tl{10} &       &       &       &       & 1& 1\\ \hline
    \end{tabular}
\end{center}

\paragraph{The case $g=0$, $m=6$ and $h=6$ with coefficients in $\mathbb F_2$ and $\mathbb Q$:}
\begin{center}
    \begin{tabular}{r||r|r|r|r|r|r||r|}
        \cline{2-7}
        \multicolumn{1}{r|}{} & \multicolumn{6}{c|}{$E^0_{p,l}$ for $\mathbb F_2$ and $\mathbb Q$} \\ \hline
        \tl{\diagbox[height=1.7em, width=3em]{$p$}{$l$}} & 1 & 2 & 3 & 4 & 5 & 6& $\dim$ \\ \hline\hline
        \tl 7   & 132   &       &       &       &       &  & 132\\ \hline
        \tl 8   &       & 660   &       &       &       &  & 660\\ \hline
        \tl 9   &       &       & 1320  &       &       &  & 1320\\ \hline
        \tl{10} &       &       &       & 1320  &       &  & 1320\\ \hline
        \tl{11} &       &       &       &       & 660   &  & 660\\ \hline
        \tl{12} &       &       &       &       &       & 132& 132\\ \hline
    \end{tabular}
    
    \vspace{1cm}
    
    \begin{tabular}{r||r|r|r|r|r|r||r|}
        \cline{2-7}
        \multicolumn{1}{r|}{} & \multicolumn{6}{c|}{$E^1_{p,l}$ for $\mathbb F_2$ and $\mathbb Q$} \\ \hline
        \tl{\diagbox[height=1.7em, width=3em]{$p$}{$l$}} & 1 & 2 & 3 & 4 & 5 & 6& $\dim$ \\ \hline\hline
        \tl 7   & 132   &       &       &       &       &  & 132\\ \hline
        \tl 8   &       & 660   &       &       &       &  & 660\\ \hline
        \tl 9   &       &       & 1320  &       &       &  & 1320\\ \hline
        \tl{10} &       &       &       & 1320  &       &  & 1320\\ \hline
        \tl{11} &       &       &       &       & 660   &  & 660\\ \hline
        \tl{12} &       &       &       &       &       & 132& 132\\ \hline
    \end{tabular}
            
    \vspace{1cm}
    
    \begin{tabular}{r||r|r|r|r|r|r||r|}
        \cline{2-7}
        \multicolumn{1}{r|}{} & \multicolumn{6}{c|}{$E^2_{p,l}$ for $\mathbb Q$} \\ \hline
        \tl{\diagbox[height=1.7em, width=3em]{$p$}{$l$}} & 1 & 2 & 3 & 4 & 5 & 6& $\dimq{0}{6}{12}$ \\ \hline\hline
        \tl 7   & 0     &       &       &       &       &  & 0\\ \hline
        \tl 8   &       & 0     &       &       &       &  & 0\\ \hline
        \tl 9   &       &       & 0     &       &       &  & 0\\ \hline
        \tl{10} &       &       &       & 0     &       &  & 0\\ \hline
        \tl{11} &       &       &       &       & 1     &  & 1\\ \hline
        \tl{12} &       &       &       &       &       & 1& 1\\ \hline
    \end{tabular}
    
    \vspace{1cm}
    
    \begin{tabular}{r||r|r|r|r|r|r||r|}
        \cline{2-7}
        \multicolumn{1}{r|}{} & \multicolumn{6}{c|}{$E^2_{p,l}$ for $\mathbb F_2$} \\ \hline
        \tl{\diagbox[height=1.7em, width=3em]{$p$}{$l$}} & 1 & 2 & 3 & 4 & 5 & 6& $\dimf{0}{6}{12}$ \\ \hline\hline
        \tl 7   & 0     &       &       &       &       &  & 0\\ \hline
        \tl 8   &       & 1     &       &       &       &  & 1\\ \hline
        \tl 9   &       &       & 2     &       &       &  & 2\\ \hline
        \tl{10} &       &       &       & 1     &       &  & 1\\ \hline
        \tl{11} &       &       &       &       & 1     &  & 1\\ \hline
        \tl{12} &       &       &       &       &       & 1& 1\\ \hline
    \end{tabular}
\end{center}

\subsubsection{Genus \texorpdfstring{$g=1$}{g=1} and Punctures \texorpdfstring{$m=0,\ldots,6$}{m=0,...,6}}

\paragraph{The case $g=1$, $m=0$ and $h=2$ with coefficients in $\mathbb F_2$ and $\mathbb Q$:}
\begin{center}
    \begin{tabular}{r||r|r||r|}
        \cline{2-3}
        \multicolumn{1}{r|}{} & \multicolumn{2}{c|}{$E^0_{p,l}$ for $\mathbb F_2$ and $\mathbb Q$} \\ \hline
        \tl{\diagbox[height=1.7em, width=3em]{$p$}{$l$}} & 1 & 2& $\dim$ \\ \hline\hline
        \tl 2   & 1     &   & 1\\ \hline
        \tl 3   & 2     &   & 2\\ \hline
        \tl 4   &       & 1 & 1\\ \hline
    \end{tabular}
        
    \vspace{1cm}
    
    \begin{tabular}{r||r|r||r|}
        \cline{2-3}
        \multicolumn{1}{r|}{} & \multicolumn{2}{c|}{$E^1_{p,l}$ for $\mathbb F_2$ and $\mathbb Q$} \\ \hline
        \tl{\diagbox[height=1.7em, width=3em]{$p$}{$l$}} & 1 & 2& $\dim$ \\ \hline\hline
        \tl 2   & 0     &   & 0\\ \hline
        \tl 3   & 1     &   & 1\\ \hline
        \tl 4   &       & 1 & 1\\ \hline
    \end{tabular}
        
    \vspace{1cm}
    
    \begin{tabular}{r||r|r||r|}
        \cline{2-3}
        \multicolumn{1}{r|}{} & \multicolumn{2}{c|}{$E^2_{p,l}$ for $\mathbb F_2$ and $\mathbb Q$} \\ \hline
        \tl{\diagbox[height=1.7em, width=3em]{$p$}{$l$}} & 1 & 2& $\dimr{1}{0}{4}$ \\ \hline\hline
        \tl 2   & 0     &   & 0\\ \hline
        \tl 3   & 1     &   & 1\\ \hline
        \tl 4   &       & 1 & 1\\ \hline
    \end{tabular}
\end{center}

\paragraph{The case $g=1$, $m=1$ and $h=3$ with coefficients in $\mathbb F_2$ and $\mathbb Q$:}
\begin{center}
    \begin{tabular}{r||r|r|r||r|}
        \cline{2-4}
        \multicolumn{1}{r|}{} & \multicolumn{3}{c|}{$E^0_{p,l}$ for $\mathbb F_2$ and $\mathbb Q$} \\ \hline
        \tl{\diagbox[height=1.7em, width=3em]{$p$}{$l$}} & 1 & 2 & 3& $\dim$ \\ \hline\hline
        \tl 2  & 1     &        &   & 1\\ \hline
        \tl 3  & 12    &        &   & 12\\ \hline
        \tl 4  & 25    & 6      &   & 31\\ \hline
        \tl 5  &       & 30     &   & 30\\ \hline
        \tl 6  &       &        & 10 & 10\\ \hline
    \end{tabular}
        
    \vspace{1cm}
    
    \begin{tabular}{r||r|r|r||r|}
        \cline{2-4}
        \multicolumn{1}{r|}{} & \multicolumn{3}{c|}{$E^1_{p,l}$ for $\mathbb Q$} \\ \hline
        \tl{\diagbox[height=1.7em, width=3em]{$p$}{$l$}} & 1 & 2 & 3& $\dim$ \\ \hline\hline
        \tl 2  & 0     &        &   & 0\\ \hline
        \tl 3  & 0     &        &   & 0\\ \hline
        \tl 4  & 14    & 0      &   & 14\\ \hline
        \tl 5  &       & 24     &   & 24\\ \hline
        \tl 6  &       &        & 10 & 10\\ \hline
    \end{tabular}
        
    \vspace{1cm}
    
    \begin{tabular}{r||r|r|r||r|}
        \cline{2-4}
        \multicolumn{1}{r|}{} & \multicolumn{3}{c|}{$E^2_{p,l}$ for $\mathbb Q$} \\ \hline
        \tl{\diagbox[height=1.7em, width=3em]{$p$}{$l$}} & 1 & 2 & 3& $\dimq{1}{1}{6}$ \\ \hline\hline
        \tl 2  & 0     &        &   & 0\\ \hline
        \tl 3  & 0     &        &   & 0\\ \hline
        \tl 4  & 0     & 0      &   & 0\\ \hline
        \tl 5  &       & 1      &   & 1\\ \hline
        \tl 6  &       &        & 1 & 1\\ \hline
    \end{tabular}
    
    \vspace{1cm}
    
    \begin{tabular}{r||r|r|r||r|}
        \cline{2-4}
        \multicolumn{1}{r|}{} & \multicolumn{3}{c|}{$E^1_{p,l}$ for $\mathbb F_2$} \\ \hline
        \tl{\diagbox[height=1.7em, width=3em]{$p$}{$l$}} & 1 & 2 & 3& $\dim$ \\ \hline\hline
        \tl 2  & 0     &        &   & 0\\ \hline
        \tl 3  & 1     &        &   & 1\\ \hline
        \tl 4  & 15    & 0      &   & 15\\ \hline
        \tl 5  &       & 24     &   & 24\\ \hline
        \tl 6  &       &        & 10 & 10\\ \hline
    \end{tabular}
        
    \vspace{1cm}
    
    \begin{tabular}{r||r|r|r||r|}
        \cline{2-4}
        \multicolumn{1}{r|}{} & \multicolumn{3}{c|}{$E^2_{p,l}$ for $\mathbb F_2$} \\ \hline
        \tl{\diagbox[height=1.7em, width=3em]{$p$}{$l$}} & 1 & 2 & 3& $\dimf{1}{1}{6}$ \\ \hline\hline
        \tl 2  & 0     &        &   & 0\\ \hline
        \tl 3  & 1     &        &   & 1\\ \hline
        \tl 4  & 1     & 0      &   & 1\\ \hline
        \tl 5  &       & 1      &   & 1\\ \hline
        \tl 6  &       &        & 1 & 1\\ \hline
    \end{tabular}
\end{center}

\paragraph{The case $g=1$, $m=2$ and $h=4$ with coefficients in $\mathbb F_2$ and $\mathbb Q$:}
\begin{center}
    \begin{tabular}{r||r|r|r|r||r|}
        \cline{2-5}
        \multicolumn{1}{r|}{} & \multicolumn{4}{c|}{$E^0_{p,l}$ for $\mathbb F_2$ and $\mathbb Q$} \\ \hline
        \tl{\diagbox[height=1.7em, width=3em]{$p$}{$l$}} & 1 & 2 & 3 & 4& $\dim$ \\ \hline\hline
        \tl 3   & 10    &       &       &   & 10\\ \hline
        \tl 4   & 96    & 4     &       &   & 100\\ \hline
        \tl 5   & 210   & 100   &       &   & 310\\ \hline
        \tl 6   &       & 400   & 30    &   & 430\\ \hline
        \tl 7   &       &       & 280   &   & 280\\ \hline
        \tl 8   &       &       &       & 70& 70\\ \hline
    \end{tabular}
        
    \vspace{1cm}
    
    \begin{tabular}{r||r|r|r|r||r|}
        \cline{2-5}
        \multicolumn{1}{r|}{} & \multicolumn{4}{c|}{$E^1_{p,l}$ for $\mathbb Q$} \\ \hline
        \tl{\diagbox[height=1.7em, width=3em]{$p$}{$l$}} & 1 & 2 & 3 & 4& $\dim$ \\ \hline\hline
        \tl 3   & 0     &       &       &   & 0\\ \hline
        \tl 4   & 0     & 0     &       &   & 0\\ \hline
        \tl 5   & 124   & 0     &       &   & 124\\ \hline
        \tl 6   &       & 304   & 0     &   & 304\\ \hline
        \tl 7   &       &       & 250   &   & 250\\ \hline
        \tl 8   &       &       &       & 70& 70\\ \hline
    \end{tabular}
        
    \vspace{1cm}
    
    \begin{tabular}{r||r|r|r|r||r|}
        \cline{2-5}
        \multicolumn{1}{r|}{} & \multicolumn{4}{c|}{$E^2_{p,l}$ for $\mathbb Q$} \\ \hline
        \tl{\diagbox[height=1.7em, width=3em]{$p$}{$l$}} & 1 & 2 & 3 & 4& $\dimq{1}{2}{8}$ \\ \hline\hline
        \tl 3   & 0     &       &       &   & 0\\ \hline
        \tl 4   & 0     & 0     &       &   & 0\\ \hline
        \tl 5   & 0     & 0     &       &   & 0\\ \hline
        \tl 6   &       & 0     & 0     &   & 0\\ \hline
        \tl 7   &       &       & 1     &   & 1\\ \hline
        \tl 8   &       &       &       & 1 & 1\\ \hline
    \end{tabular}

    \vspace{1cm}
    
    \begin{tabular}{r||r|r|r|r||r|}
        \cline{2-5}
        \multicolumn{1}{r|}{} & \multicolumn{4}{c|}{$E^1_{p,l}$ for $\mathbb F_2$} \\ \hline
        \tl{\diagbox[height=1.7em, width=3em]{$p$}{$l$}} & 1 & 2 & 3 & 4& $\dim$ \\ \hline\hline
        \tl 3   & 0     &       &       &   & 0\\ \hline
        \tl 4   & 3     & 0     &       &   & 3\\ \hline
        \tl 5   & 127   & 3     &       &   & 130\\ \hline
        \tl 6   &       & 307   & 0     &   & 307\\ \hline
        \tl 7   &       &       & 250   &   & 250\\ \hline
        \tl 8   &       &       &       & 70& 70\\ \hline
    \end{tabular}
        
    \vspace{1cm}
    
    \begin{tabular}{r||r|r|r|r||r|}
        \cline{2-5}
        \multicolumn{1}{r|}{} & \multicolumn{4}{c|}{$E^2_{p,l}$ for $\mathbb F_2$} \\ \hline
        \tl{\diagbox[height=1.7em, width=3em]{$p$}{$l$}} & 1 & 2 & 3 & 4& $\dimf{1}{2}{8}$ \\ \hline\hline
        \tl 3   & 0     &       &       &   & 0\\ \hline
        \tl 4   & 1     & 0     &       &   & 1\\ \hline
        \tl 5   & 2     & 1     &       &   & 3\\ \hline
        \tl 6   &       & 3     & 0     &   & 3\\ \hline
        \tl 7   &       &       & 2     &   & 2\\ \hline
        \tl 8   &       &       &       & 1 & 1\\ \hline
    \end{tabular}
\end{center}

\paragraph{The case $g=1$, $m=3$ and $h=5$ with coefficients in $\mathbb F_2$ and $\mathbb Q$:}
\begin{center}
    \begin{tabular}{r||r|r|r|r|r||r|}
        \cline{2-6}
        \multicolumn{1}{r|}{} & \multicolumn{5}{c|}{$E^0_{p,l}$ for $\mathbb F_2$ and $\mathbb Q$} \\ \hline
        \tl{\diagbox[height=1.7em, width=3em]{$p$}{$l$}} & 1 & 2 & 3 & 4 & 5& $\dim$ \\ \hline\hline
        \tl 4   & 70    &       &       &       &  & 70\\ \hline
        \tl 5   & 640   & 60    &       &       &  & 700\\ \hline
        \tl 6   & 1470  & 1035  & 15    &       &  & 2520\\ \hline
        \tl 7   &       & 3850  & 360   &       &  & 4210\\ \hline
        \tl 8   &       &       & 4130  & 140   &  & 4270\\ \hline
        \tl 9   &       &       &       & 2100  &  & 2100\\ \hline
        \tl{10} &       &       &       &       & 420& 420\\ \hline
    \end{tabular}
        
    \vspace{1cm}
    
    \begin{tabular}{r||r|r|r|r|r||r|}
        \cline{2-6}
        \multicolumn{1}{r|}{} & \multicolumn{5}{c|}{$E^1_{p,l}$ for $\mathbb Q$} \\ \hline
        \tl{\diagbox[height=1.7em, width=3em]{$p$}{$l$}} & 1 & 2 & 3 & 4 & 5& $\dim$ \\ \hline\hline
        \tl 4   & 0     &       &       &       &  & 0\\ \hline
        \tl 5   & 1     & 0     &       &       &  & 1\\ \hline
        \tl 6   & 901   & 0     & 0     &       &  & 901\\ \hline
        \tl 7   &       & 2875  & 0     &       &  & 2875\\ \hline
        \tl 8   &       &       & 3515  & 0     &  & 3515\\ \hline
        \tl 9   &       &       &       & 1960  &  & 1960\\ \hline
        \tl{10} &       &       &       &       & 420& 420\\ \hline
    \end{tabular}
        
    \vspace{1cm}
    
        \begin{tabular}{r||r|r|r|r|r||r|}
        \cline{2-6}
        \multicolumn{1}{r|}{} & \multicolumn{5}{c|}{$E^2_{p,l}$ for $\mathbb Q$} \\ \hline
        \tl{\diagbox[height=1.7em, width=3em]{$p$}{$l$}} & 1 & 2 & 3 & 4 & 5& $\dimq{1}{3}{10}$ \\ \hline\hline
        \tl 4   & 0     &       &       &       &  & 0\\ \hline
        \tl 5   & 1     & 0     &       &       &  & 1\\ \hline
        \tl 6   & 2     & 0     & 0     &       &  & 2\\ \hline
        \tl 7   &       & 1     & 0     &       &  & 1\\ \hline
        \tl 8   &       &       & 0     & 0     &  & 0\\ \hline
        \tl 9   &       &       &       & 1     &  & 1\\ \hline
        \tl{10} &       &       &       &       & 1& 1\\ \hline
    \end{tabular}

    \vspace{1cm}
    
    \begin{tabular}{r||r|r|r|r|r||r|}
        \cline{2-6}
        \multicolumn{1}{r|}{} & \multicolumn{5}{c|}{$E^1_{p,l}$ for $\mathbb F_2$} \\ \hline
        \tl{\diagbox[height=1.7em, width=3em]{$p$}{$l$}} & 1 & 2 & 3 & 4 & 5& $\dim$ \\ \hline\hline
        \tl 4   & 0     &       &       &       &  & 0\\ \hline
        \tl 5   & 10    & 0     &       &       &  & 10\\ \hline
        \tl 6   & 910   & 20    & 0     &       &  & 930\\ \hline
        \tl 7   &       & 2895  & 10    &       &  & 3905\\ \hline
        \tl 8   &       &       & 3525  & 0     &  & 3525\\ \hline
        \tl 9   &       &       &       & 1960  &  & 1960\\ \hline
        \tl{10} &       &       &       &       & 420& 420\\ \hline
    \end{tabular}
        
    \vspace{1cm}
    
    \begin{tabular}{r||r|r|r|r|r||r|r|}
        \cline{2-6}
        \multicolumn{1}{r|}{} & \multicolumn{5}{c|}{$E^2_{p,l}$ for $\mathbb F_2$} \\ \hline
        \tl{\diagbox[height=1.7em, width=3em]{$p$}{$l$}} & 1 & 2 & 3 & 4 & 5 & $\dimf{1}{3}{10}$ \\ \hline\hline
        \tl 4   & 0     &       &       &       &       & 0\\ \hline
        \tl 5   & 1     & 0     &       &       &       & 1\\ \hline
        \tl 6   & 2     & 2     & 0     &       &       & 4\\ \hline
        \tl 7   &       & 4     & 1     &       &       & 5\\ \hline
        \tl 8   &       &       & 3     & 0     &       & 3\\ \hline
        \tl 9   &       &       &       & 2     &       & 2\\ \hline
        \tl{10} &       &       &       &       & 1     & 1\\ \hline
    \end{tabular}
\end{center}

\paragraph{The case $g=1$, $m=4$ and $h=6$ with coefficients in $\mathbb F_2$:}
\begin{center}
    \begin{tabular}{r||r|r|r|r|r|r||r|}
        \cline{2-7}
        \multicolumn{1}{r|}{} & \multicolumn{6}{c|}{$E^0_{p,l}$ for $\mathbb F_2$} \\ \hline
        \tl{\diagbox[height=1.7em, width=3em]{$p$}{$l$}} & 1 & 2 & 3 & 4 & 5 & 6& $\dim$ \\ \hline\hline
        \tl 5   & 420   &       &       &       &       &  & 420\\ \hline
        \tl 6   & 3840  & 570   &       &       &       &  & 4410\\ \hline
        \tl 7   & 9240  & 8526  & 294   &       &       &  & 18060\\ \hline
        \tl 8   &       & 30898 & 7896  & 56    &       &  & 38850\\ \hline
        \tl 9   &       &       & 44772 & 3528  &       &  & 48300\\ \hline
        \tl{10} &       &       &       & 34440 & 630   &  & 35070\\ \hline
        \tl{11} &       &       &       &       & 13860 &  & 13860\\ \hline
        \tl{12} &       &       &       &       &       & 2310& 2310\\ \hline
    \end{tabular}
    
    \vspace{1cm}
    
    \begin{tabular}{r||r|r|r|r|r|r||r|}
        \cline{2-7}
        \multicolumn{1}{r|}{} & \multicolumn{6}{c|}{$E^1_{p,l}$ for $\mathbb F_2$} \\ \hline
        \tl{\diagbox[height=1.7em, width=3em]{$p$}{$l$}} & 1 & 2 & 3 & 4 & 5 & 6& $\dim$ \\ \hline\hline
        \tl 5   & 0     &       &       &       &       &  & 0\\ \hline
        \tl 6   & 36    & 0     &       &       &       &  & 36\\ \hline
        \tl 7   & 5856  & 105   & 0     &       &       &  & 5961\\ \hline
        \tl 8   &       & 23047 & 105   & 0     &       &  & 23152\\ \hline
        \tl 9   &       &       & 37275 & 35    &       &  & 37310\\ \hline
        \tl{10} &       &       &       & 31003 & 0     &  & 31003\\ \hline
        \tl{11} &       &       &       &       & 13230 &  & 13230\\ \hline
        \tl{12} &       &       &       &       &       & 2310& 2310\\ \hline
    \end{tabular}
        
    \vspace{1cm}
    
    \begin{tabular}{r||r|r|r|r|r|r||r|}
        \cline{2-7}
        \multicolumn{1}{r|}{} & \multicolumn{6}{c|}{$E^2_{p,l}$ for $\mathbb F_2$} \\ \hline
        \tl{\diagbox[height=1.7em, width=3em]{$p$}{$l$}} & 1 & 2 & 3 & 4 & 5 & 6& $\dimf{1}{4}{12}$ \\ \hline\hline
        \tl 5   & 0     &       &       &       &       &  & 0\\ \hline
        \tl 6   & 2     & 0     &       &       &       &  & 2\\ \hline
        \tl 7   & 3     & 2     & 0     &       &       &  & 5\\ \hline
        \tl 8   &       & 6     & 2     & 0     &       &  & 8\\ \hline
        \tl 9   &       &       & 7     & 1     &       &  & 8\\ \hline
        \tl{10} &       &       &       & 4     & 0     &  & 4\\ \hline
        \tl{11} &       &       &       &       & 2     &  & 2\\ \hline
        \tl{12} &       &       &       &       &       & 1& 1\\ \hline
    \end{tabular}
\end{center}

\paragraph{The case $g=1$, $m=5$ and $h=7$ with coefficients in $\mathbb F_2$:}
\begin{center}
    \begin{tabular}{r||r|r|r|r|r|r|r||r|}
        \cline{2-8}
        \multicolumn{1}{r|}{} & \multicolumn{7}{c|}{$E^0_{p,l}$ for $\mathbb F_2$} \\ \hline
        \tl{\diagbox[height=1.7em, width=3em]{$p$}{$l$}} & 1 & 2 & 3 & 4 & 5 & 6 & 7& $\dim$ \\ \hline\hline
        \tl 6   & 2310  &       &       &       &       &       & & 2310\\ \hline
        \tl 7   & 21504 & 4368  &       &       &       &       & & 25872\\ \hline
        \tl 8   & 54054 & 61208 & 3472  &       &       &       & & 118734\\ \hline
        \tl 9   &       & 220500& 76608 & 1344  &       &       & & 298452\\ \hline
        \tl{10} &       &       & 403200& 51660 & 210   &       & & 455070\\ \hline
        \tl{11} &       &       &       & 415800& 18480 &       & & 434280\\ \hline
        \tl{12} &       &       &       &       & 251790& 2772  & & 254562\\ \hline
        \tl{13} &       &       &       &       &       & 84084 & & 84084\\ \hline
        \tl{14} &       &       &       &       &       &       & 12012& 12012\\ \hline
    \end{tabular}
    
    \vspace{1cm}
    
    \begin{tabular}{r||r|r|r|r|r|r|r||r|}
        \cline{2-8}
        \multicolumn{1}{r|}{} & \multicolumn{7}{c|}{$E^1_{p,l}$ for $\mathbb F_2$} \\ \hline
        \tl{\diagbox[height=1.7em, width=3em]{$p$}{$l$}} & 1 & 2 & 3 & 4 & 5 & 6 & 7& $\dim$ \\ \hline\hline
        \tl 6   & 0     &       &       &       &       &       & & 0\\ \hline
        \tl 7   & 129   & 0     &       &       &       &       & & 129\\ \hline
        \tl 8   & 34989 & 507   & 0     &       &       &       & & 35496\\ \hline
        \tl 9   &       & 164167& 756   & 0     &       &       & & 164923\\ \hline
        \tl{10} &       &       & 330820& 504   & 0     &       & & 331324\\ \hline
        \tl{11} &       &       &       & 365988& 126   &       & & 366114\\ \hline
        \tl{12} &       &       &       &       & 233646& 0     & & 233646\\ \hline
        \tl{13} &       &       &       &       &       & 81312 & & 81312\\ \hline
        \tl{14} &       &       &       &       &       &       & 12012& 12012\\ \hline
    \end{tabular}
    
    \vspace{1cm}
    
    \begin{tabular}{r||r|r|r|r|r|r|r||r|}
        \cline{2-8}
        \multicolumn{1}{r|}{} & \multicolumn{7}{c|}{$E^2_{p,l}$ for $\mathbb F_2$} \\ \hline
        \tl{\diagbox[height=1.7em, width=3em]{$p$}{$l$}} & 1 & 2 & 3 & 4 & 5 & 6 & 7& $\dimf{1}{5}{14}$ \\ \hline\hline
        \tl 6   & 0     &       &       &       &       &       & & 0\\ \hline
        \tl 7   & 2     & 0     &       &       &       &       & & 2\\ \hline
        \tl 8   & 3     & 4     & 0     &       &       &       & & 7\\ \hline
        \tl 9   &       & 7     & 3     & 0     &       &       & & 10\\ \hline
        \tl{10} &       &       & 8     & 2     & 0     &       & & 10\\ \hline
        \tl{11} &       &       &       & 7     & 1     &       & & 8\\ \hline
        \tl{12} &       &       &       &       & 4     & 0     & & 4\\ \hline
        \tl{13} &       &       &       &       &       & 2     & & 2\\ \hline
        \tl{14} &       &       &       &       &       &       & 1& 1\\ \hline
    \end{tabular}
\end{center}

\paragraph{The case $g=1$, $m=6$ and $h=8$ with coefficients in $\mathbb F_2$:}
For $p \ge 10$, the differentials of $E_{p,l}$ could not be constructed or diagonalized  due to memory and / or time limitations.
The first and second page is therefore incomplete.
\begin{center}
    \begin{tabular}{r||r|r|r|r|r|r|r|r||r|}
        \cline{2-9}
        \multicolumn{1}{r|}{} & \multicolumn{8}{c|}{$E^0_{p,l}$ for $\mathbb F_2$} \\ \hline
        \tl{\diagbox[height=1.7em, width=3em]{$p$}{$l$}} & 1 & 2 & 3 & 4 & 5 & 6 & 7 & 8& $\dim$ \\ \hline\hline
        \tl 7   & 12012 &       &       &       &       &       &       && 12012\\ \hline
        \tl 8   & 114688& 29456 &       &       &       &       &       && 144144\\ \hline
        \tl 9   & 300300& 400464& 31968 &       &       &       &       && 732732\\ \hline
        \tl{10} &       &1448760& 634500& 18840 &       &       &       && 2102100\\ \hline
        \tl{11} &       &       &3203640& 574200& 5940  &       &       && 3783780\\ \hline
        \tl{12} &       &       &       &4146780& 308880& 792   &       && 4456452\\ \hline
        \tl{13} &       &       &       &       &3354780& 92664 &       && 3447444\\ \hline
        \tl{14} &       &       &       &       &       &1681680& 12012 && 1693692\\ \hline
        \tl{15} &       &       &       &       &       &       & 480480&& 480480\\ \hline
        \tl{16} &       &       &       &       &       &       & 120120&60060& 180180\\ \hline
    \end{tabular}
    
    \vspace{1cm}
    
    \begin{tabular}{r||r|r|r|r|r|r|r|r||r|}
        \cline{2-9}
        \multicolumn{1}{r|}{} & \multicolumn{8}{c|}{$E^1_{p,l}$ for $\mathbb F_2$} \\ \hline
        \tl{\diagbox[height=1.7em, width=3em]{$p$}{$l$}} & 1 & 2 & 3 & 4 & 5 & 6 & 7 & 8& $\dim$ \\ \hline\hline
        \tl 7   & 0     &       &       &       &       &       &       && 0\\ \hline
        \tl 8   & 470   & 0     &       &       &       &       &       && 470\\ \hline
        \tl 9   & 198096& 2330  & 0     &       &       &       &       && 200426\\ \hline
    \end{tabular}
    \\
    $\vdots$
        
    \vspace{1cm}
    
    \begin{tabular}{r||r|r|r|r|r|r|r|r||r|}
        \cline{2-9}
        \multicolumn{1}{r|}{} & \multicolumn{8}{c|}{$E^2_{p,l}$ for $\mathbb F_2$} \\ \hline
        \tl{\diagbox[height=1.7em, width=3em]{$p$}{$l$}} & 1 & 2 & 3 & 4 & 5 & 6 & 7 & 8& $\dimf{1}{6}{16}$ \\ \hline\hline
        \tl 7   & 0     &       &       &       &       &       &       && 0\\ \hline
        \tl 8   & 2     & 0     &       &       &       &       &       && 2\\ \hline
        \tl 9   & 4     & 5     & 0     &       &       &       &       && 9\\ \hline
    \end{tabular}
    \\
    $\vdots$
\end{center}

\subsubsection{Genus \texorpdfstring{$g=2$}{g=2} and Punctures \texorpdfstring{$m=0,1$}{m=0,1}}
\paragraph{The case $g=2$, $m=0$ and $h=4$ with coefficients in $\mathbb F_2$ and $\mathbb Q$:}
\begin{center}
    \begin{tabular}{r||r|r|r|r||r|}
        \cline{2-5}
        \multicolumn{1}{r|}{} & \multicolumn{4}{c|}{$E^0_{p,l}$ for $\mathbb F_2$ and $\mathbb Q$} \\ \hline
        \tl{\diagbox[height=1.7em, width=3em]{$p$}{$l$}} & 1 & 2 & 3 & 4& $\dim$ \\ \hline\hline
        \tl 2   & 1     &       &       &   & 1\\ \hline
        \tl 3   & 18    &       &       &   & 18\\ \hline
        \tl 4   & 78    & 5     &       &   & 83\\ \hline
        \tl 5   & 112   & 60    &       &   & 172\\ \hline
        \tl 6   &       & 168   & 15    &   & 183\\ \hline
        \tl 7   &       &       & 98    &   & 98\\ \hline
        \tl 8   &       &       &       & 21& 21\\ \hline
    \end{tabular}
        
    \vspace{1cm}
    
    \begin{tabular}{r||r|r|r|r||r|}
        \cline{2-5}
        \multicolumn{1}{r|}{} & \multicolumn{4}{c|}{$E^1_{p,l}$ for $\mathbb Q$} \\ \hline
        \tl{\diagbox[height=1.7em, width=3em]{$p$}{$l$}} & 1 & 2 & 3 & 4& $\dim$ \\ \hline\hline
        \tl 2   & 0     &       &       &   & 0\\ \hline
        \tl 3   & 0     &       &       &   & 0\\ \hline
        \tl 4   & 0     & 0     &       &   & 0\\ \hline
        \tl 5   & 51    & 0     &       &   & 51\\ \hline
        \tl 6   &       & 113   & 0     &   & 113\\ \hline
        \tl 7   &       &       & 83    &   & 83\\ \hline
        \tl 8   &       &       &       & 21& 21\\ \hline
    \end{tabular}
        
    \vspace{1cm}
    
    \begin{tabular}{r||r|r|r|r||r|}
        \cline{2-5}
        \multicolumn{1}{r|}{} & \multicolumn{4}{c|}{$E^2_{p,l}$ for $\mathbb Q$} \\ \hline
        \tl{\diagbox[height=1.7em, width=3em]{$p$}{$l$}} & 1 & 2 & 3 & 4& $\dimq{2}{0}{8}$ \\ \hline\hline
        \tl 2   & 0     &       &       &   & 0\\ \hline
        \tl 3   & 0     &       &       &   & 0\\ \hline
        \tl 4   & 0     & 0     &       &   & 0\\ \hline
        \tl 5   & 1     & 0     &       &   & 1\\ \hline
        \tl 6   &       & 0     & 0     &   & 0\\ \hline
        \tl 7   &       &       & 0     &   & 0\\ \hline
        \tl 8   &       &       &       & 1 & 1\\ \hline
    \end{tabular}

    \vspace{1cm}
    
    \begin{tabular}{r||r|r|r|r||r|}
        \cline{2-5}
        \multicolumn{1}{r|}{} & \multicolumn{4}{c|}{$E^1_{p,l}$ for $\mathbb F_2$} \\ \hline
        \tl{\diagbox[height=1.7em, width=3em]{$p$}{$l$}} & 1 & 2 & 3 & 4& $\dim$ \\ \hline\hline
        \tl 2   & 0     &       &       &   & 0\\ \hline
        \tl 3   & 1     &       &       &   & 1\\ \hline
        \tl 4   & 2     & 0     &       &   & 2\\ \hline
        \tl 5   & 52    & 0     &       &   & 52\\ \hline
        \tl 6   &       & 113   & 0     &   & 113\\ \hline
        \tl 7   &       &       & 83    &   & 83\\ \hline
        \tl 8   &       &       &       & 21& 21\\ \hline
    \end{tabular}
        
    \vspace{1cm}
    
    \begin{tabular}{r||r|r|r|r||r|}
        \cline{2-5}
        \multicolumn{1}{r|}{} & \multicolumn{4}{c|}{$E^2_{p,l}$ for $\mathbb F_2$} \\ \hline
        \tl{\diagbox[height=1.7em, width=3em]{$p$}{$l$}} & 1 & 2 & 3 & 4& $\dimf{2}{0}{8}$ \\ \hline\hline
        \tl 2   & 0     &       &       &   & 0\\ \hline
        \tl 3   & 1     &       &       &   & 1\\ \hline
        \tl 4   & 2     & 0     &       &   & 2\\ \hline
        \tl 5   & 3     & 0     &       &   & 3\\ \hline
        \tl 6   &       & 2     & 0     &   & 2\\ \hline
        \tl 7   &       &       & 1     &   & 1\\ \hline
        \tl 8   &       &       &       & 1 & 1\\ \hline
    \end{tabular}
\end{center}

\paragraph{The case $g=2$, $m=1$ and $h=5$ with coefficients in $\mathbb F_2$ and $\mathbb Q$:}
\begin{center}
    \begin{tabular}{r||r|r|r|r|r||r|}
        \cline{2-6}
        \multicolumn{1}{r|}{} & \multicolumn{5}{c|}{$E^0_{p,l}$ for $\mathbb F_2$ and $\mathbb Q$} \\ \hline
        \tl{\diagbox[height=1.7em, width=3em]{$p$}{$l$}} & 1 & 2 & 3 & 4 & 5& $\dim$ \\ \hline\hline
        \tl 2   & 1     &       &       &       &  & 1\\ \hline
        \tl 3   & 60    &       &       &       &  & 60\\ \hline
        \tl 4   & 638   & 12    &       &       &  & 650\\ \hline
        \tl 5   & 2480  & 380   &       &       &  & 2860\\ \hline
        \tl 6   & 3528  & 2985  & 75    &       &  & 6588\\ \hline
        \tl 7   &       & 7238  & 1470  &       &  & 8708\\ \hline
        \tl 8   &       &       & 6398  & 280   &  & 6678\\ \hline
        \tl 9   &       &       &       & 2772  &  & 2772\\ \hline
        \tl{10} &       &       &       &       & 483& 483\\ \hline
    \end{tabular}
        
    \vspace{1cm}
    
    \begin{tabular}{r||r|r|r|r|r||r|}
        \cline{2-6}
        \multicolumn{1}{r|}{} & \multicolumn{5}{c|}{$E^1_{p,l}$ for $\mathbb Q$} \\ \hline
        \tl{\diagbox[height=1.7em, width=3em]{$p$}{$l$}} & 1 & 2 & 3 & 4 & 5& $\dim$ \\ \hline\hline
        \tl 2   & 0     &       &       &       &  & 0\\ \hline
        \tl 3   & 0     &       &       &       &  & 0\\ \hline
        \tl 4   & 1     & 0     &       &       &  & 1\\ \hline
        \tl 5   & 1     & 0     &       &       &  & 1\\ \hline
        \tl 6   & 1627  & 0     & 0     &       &  & 1627\\ \hline
        \tl 7   &       & 4621  & 0     &       &  & 4621\\ \hline
        \tl 8   &       &       & 5003  & 0     &  & 5003\\ \hline
        \tl 9   &       &       &       & 2492  &  & 2492\\ \hline
        \tl{10} &       &       &       &       & 483& 483\\ \hline
    \end{tabular}
        
    \vspace{1cm}
    
        \begin{tabular}{r||r|r|r|r|r||r|}
        \cline{2-6}
        \multicolumn{1}{r|}{} & \multicolumn{5}{c|}{$E^2_{p,l}$ for $\mathbb Q$} \\ \hline
        \tl{\diagbox[height=1.7em, width=3em]{$p$}{$l$}} & 1 & 2 & 3 & 4 & 5& $\dimq{2}{1}{10}$ \\ \hline\hline
        \tl 2   & 0     &       &       &       &  & 0\\ \hline
        \tl 3   & 0     &       &       &       &  & 0\\ \hline
        \tl 4   & 1     & 0     &       &       &  & 1\\ \hline
        \tl 5   & 1     & 0     &       &       &  & 1\\ \hline
        \tl 6   & 0     & 0     & 0     &       &  & 0\\ \hline
        \tl 7   &       & 2     & 0     &       &  & 2\\ \hline
        \tl 8   &       &       & 1     & 0     &  & 1\\ \hline
        \tl 9   &       &       &       & 0     &  & 0\\ \hline
        \tl{10} &       &       &       &       & 1& 1\\ \hline
    \end{tabular}

    \vspace{1cm}
    
    \begin{tabular}{r||r|r|r|r|r||r|}
        \cline{2-6}
        \multicolumn{1}{r|}{} & \multicolumn{5}{c|}{$E^1_{p,l}$ for $\mathbb F_2$} \\ \hline
        \tl{\diagbox[height=1.7em, width=3em]{$p$}{$l$}} & 1 & 2 & 3 & 4 & 5& $\dim$ \\ \hline\hline
        \tl 2   & 0     &       &       &       &  & 0\\ \hline
        \tl 3   & 0     &       &       &       &  & 0\\ \hline
        \tl 4   & 3     & 0     &       &       &  & 3\\ \hline
        \tl 5   & 15    & 3     &       &       &  & 18\\ \hline
        \tl 6   & 1639  & 20    & 0     &       &  & 1659\\ \hline
        \tl 7   &       & 4638  & 5     &       &  & 4643\\ \hline
        \tl 8   &       &       & 5008  &       &  & 5008\\ \hline
        \tl 9   &       &       &       & 2492  &  & 2492\\ \hline
        \tl{10} &       &       &       &       & 483& 483\\ \hline
    \end{tabular}
        
    \vspace{1cm}
    
    \begin{tabular}{r||r|r|r|r|r||r|}
        \cline{2-6}
        \multicolumn{1}{r|}{} & \multicolumn{5}{c|}{$E^2_{p,l}$ for $\mathbb F_2$} \\ \hline
        \tl{\diagbox[height=1.7em, width=3em]{$p$}{$l$}} & 1 & 2 & 3 & 4 & 5& $\dimf{2}{1}{10}$ \\ \hline\hline
        \tl 2   & 0     &       &       &       &  & 0\\ \hline
        \tl 3   & 0     &       &       &       &  & 0\\ \hline
        \tl 4   & 1     & 0     &       &       &  & 1\\ \hline
        \tl 5   & 2     & 1     &       &       &  & 3\\ \hline
        \tl 6   & 1     & 3     & 0     &       &  & 4\\ \hline
        \tl 7   &       & 4     & 1     &       &  & 5\\ \hline
        \tl 8   &       &       & 3     & 0     &  & 3\\ \hline
        \tl 9   &       &       &       & 1     &  & 1\\ \hline
        \tl{10} &       &       &       &       & 1& 1\\ \hline
    \end{tabular}
\end{center}

\subsubsection{Genus \texorpdfstring{$g=3$}{g=3} and Punctures \texorpdfstring{$m=0,1$}{m=0,1}}
\paragraph{The case $g=3$, $m=0$ and $h=6$ with coefficients in $\mathbb F_2$ and $\mathbb Q$:}
\begin{center}
    \begin{tabular}{r||r|r|r|r|r|r||r|}
        \cline{2-7}
        \multicolumn{1}{r|}{} & \multicolumn{6}{c|}{$E^0_{p,l}$ for $\mathbb F_2$ and $\mathbb Q$} \\ \hline
        \tl{\diagbox[height=1.7em, width=3em]{$p$}{$l$}} & 1 & 2 & 3 & 4 & 5 & 6& $\dim$ \\ \hline\hline
        \tl 2   & 1     &       &       &       &       &  & 1\\ \hline
        \tl 3   & 82    &       &       &       &       &  & 82\\ \hline
        \tl 4   & 1212  & 9     &       &       &       &  & 1221\\ \hline
        \tl 5   & 7200  & 440   &       &       &       &  & 7640\\ \hline
        \tl 6   & 20400 & 5690  & 60    &       &       &  & 26150\\ \hline
        \tl 7   & 23760 & 28980 & 2016  &       &       &  & 54756\\ \hline
        \tl 8   &       & 54164 & 19124 & 294   &       &  & 73582\\ \hline
        \tl 9   &       &       & 57424 & 6552  &       &  & 63976\\ \hline
        \tl{10} &       &       &       & 33960 & 945   &  & 34905\\ \hline
        \tl{11} &       &       &       &       & 10890 &  & 10890\\ \hline
        \tl{12} &       &       &       &       &       & 1485& 1485\\ \hline
    \end{tabular}
        
    \vspace{1cm}
    
    \begin{tabular}{r||r|r|r|r|r|r||r|}
        \cline{2-7}
        \multicolumn{1}{r|}{} & \multicolumn{6}{c|}{$E^1_{p,l}$ for $\mathbb Q$} \\ \hline
        \tl{\diagbox[height=1.7em, width=3em]{$p$}{$l$}} & 1 & 2 & 3 & 4 & 5 & 6& $\dim$ \\ \hline\hline
        \tl 2   & 0     &       &       &       &       &  & 0\\ \hline
        \tl 3   & 1     &       &       &       &       &  & 1\\ \hline
        \tl 4   & 0     & 0     &       &       &       &  & 0\\ \hline
        \tl 5   & 0     & 0     &       &       &       &  & 0\\ \hline
        \tl 6   & 1     & 0     &       &       &       &  & 1\\ \hline
        \tl 7   & 9429  & 0     & 0     &       &       &  & 9429\\ \hline
        \tl 8   &       & 30443 & 0     & 0     &       &  & 30443\\ \hline
        \tl 9   &       &       & 40256 & 0     &       &  & 40256\\ \hline
        \tl{10} &       &       &       & 27702 & 0     &  & 27702\\ \hline
        \tl{11} &       &       &       &       & 9945  &  & 9945\\ \hline
        \tl{12} &       &       &       &       &       & 1485& 1485\\ \hline
    \end{tabular}
        
    \vspace{1cm}
    
        \begin{tabular}{r||r|r|r|r|r|r||r|}
        \cline{2-7}
        \multicolumn{1}{r|}{} & \multicolumn{6}{c|}{$E^2_{p,l}$ for $\mathbb Q$} \\ \hline
        \tl{\diagbox[height=1.7em, width=3em]{$p$}{$l$}} & 1 & 2 & 3 & 4 & 5 & 6& $\dimq{3}{0}{12}$ \\ \hline\hline
        \tl 2   & 0     &       &       &       &       &  & 0\\ \hline
        \tl 3   & 1     &       &       &       &       &  & 1\\ \hline
        \tl 4   & 0     & 0     &       &       &       &  & 0\\ \hline
        \tl 5   & 0     & 0     &       &       &       &  & 0\\ \hline
        \tl 6   & 1     & 0     &       &       &       &  & 1\\ \hline
        \tl 7   & 1     & 0     & 0     &       &       &  & 1\\ \hline
        \tl 8   &       & 0     & 0     & 0     &       &  & 0\\ \hline
        \tl 9   &       &       & 1     & 0     &       &  & 1\\ \hline
        \tl{10} &       &       &       & 1     & 0     &  & 1\\ \hline
        \tl{11} &       &       &       &       & 0     &  & 0\\ \hline
        \tl{12} &       &       &       &       &       & 1& 1\\ \hline
    \end{tabular}

    \vspace{1cm}
    
    \begin{tabular}{r||r|r|r|r|r|r||r|}
        \cline{2-7}
        \multicolumn{1}{r|}{} & \multicolumn{6}{c|}{$E^1_{p,l}$ for $\mathbb F_2$} \\ \hline
        \tl{\diagbox[height=1.7em, width=3em]{$p$}{$l$}} & 1 & 2 & 3 & 4 & 5 & 6& $\dim$ \\ \hline\hline
        \tl 2   & 0     &       &       &       &       &  & 0\\ \hline
        \tl 3   & 1     &       &       &       &       &  & 1\\ \hline
        \tl 4   & 1     & 0     &       &       &       &  & 1\\ \hline
        \tl 5   & 10    & 0     &       &       &       &  & 10\\ \hline
        \tl 6   & 33    & 11    &       &       &       &  & 44\\ \hline
        \tl 7   & 9452  & 49    & 5     &       &       &  & 9501\\ \hline
        \tl 8   &       & 30481 & 20    & 0     &       &  & 30501\\ \hline
        \tl 9   &       &       & 40271 & 0     &       &  & 40271\\ \hline
        \tl{10} &       &       &       & 27702 & 0     &  & 27702\\ \hline
        \tl{11} &       &       &       &       & 9945  &  & 9945\\ \hline
        \tl{12} &       &       &       &       &       & 1485& 1485\\ \hline
    \end{tabular}
        
    \vspace{1cm}
    
    \begin{tabular}{r||r|r|r|r|r|r||r|}
        \cline{2-7}
        \multicolumn{1}{r|}{} & \multicolumn{6}{c|}{$E^2_{p,l}$ for $\mathbb F_2$} \\ \hline
        \tl{\diagbox[height=1.7em, width=3em]{$p$}{$l$}} & 1 & 2 & 3 & 4 & 5 & 6& $\dimf{3}{0}{12}$ \\ \hline\hline
        \tl 2   & 0     &       &       &       &       &  & 0\\ \hline
        \tl 3   & 1     &       &       &       &       &  & 1\\ \hline
        \tl 4   & 1     & 0     &       &       &       &  & 1\\ \hline
        \tl 5   & 4     & 0     &       &       &       &  & 4\\ \hline
        \tl 6   & 4     & 1     &       &       &       &  & 5\\ \hline
        \tl 7   & 1     & 2     & 1     &       &       &  & 4\\ \hline
        \tl 8   &       & 2     & 2     & 0     &       &  & 4\\ \hline
        \tl 9   &       &       & 4     & 0     &       &  & 4\\ \hline
        \tl{10} &       &       &       & 2     & 0     &  & 2\\ \hline
        \tl{11} &       &       &       &       & 0     &  & 0\\ \hline
        \tl{12} &       &       &       &       &       & 1& 1\\ \hline
    \end{tabular}
\end{center}

\paragraph{The case $g=3$, $m=1$ and $h=7$ with coefficients in $\mathbb F_2$:}
For $p \ge 7$, the differentials of $E_{p,l}$ could not be constructed or diagonalized  due to memory and / or time limitations.
The first and second page is therefore incomplete.
\begin{center}
    \begin{tabular}{r||r|r|r|r|r|r|r||r|}
        \cline{2-8}
        \multicolumn{1}{r|}{} & \multicolumn{7}{c|}{$E^0_{p,l}$ for $\mathbb F_2$} \\ \hline
        \tl{\diagbox[height=1.7em, width=3em]{$p$}{$l$}} & 1 & 2 & 3 & 4 & 5 & 6 & 7& $\dim$ \\ \hline\hline
        \tl 2   & 1     &       &       &       &       &       & & 1\\ \hline
        \tl 3   & 252   &       &       &       &       &       & & 252\\ \hline
        \tl 4   & 7563  & 18    &       &       &       &       & & 7581\\ \hline
        \tl 5   & 81360 & 2010  &       &       &       &       & & 83370\\ \hline
        \tl 6   & 424920& 48855 & 195   &       &       &       & & 473970\\ \hline
        \tl 7   &1141056& 469938& 13230 &       &       &       & & 1624224\\ \hline
        \tl 8   &1305876&2069844& 247898& 1540  &       &       & & 3625158\\ \hline
        \tl 9   &       &3593880&1810368& 70476 &       &       & & 5474724\\ \hline
        \tl{10} &       &       &4737360& 915390& 8715  &       & & 5661465\\ \hline
        \tl{11} &       &       &       &3702820& 258720&       & & 3961540\\ \hline
        \tl{12} &       &       &       &       &1765335& 31878 & & 1797213\\ \hline
        \tl{13} &       &       &       &       &       & 477906& & 477906\\ \hline
        \tl{14} &       &       &       &       &       &       & 56628& 56628\\ \hline
    \end{tabular}
    
    \vspace{1cm}
    
    \begin{tabular}{r||r|r|r|r|r|r|r||r|}
        \cline{2-8}
        \multicolumn{1}{r|}{} & \multicolumn{7}{c|}{$E^1_{p,l}$ for $\mathbb F_2$} \\ \hline
        \tl{\diagbox[height=1.7em, width=3em]{$p$}{$l$}} & 1 & 2 & 3 & 4 & 5 & 6 & 7& $\dim$ \\ \hline\hline
        \tl 2   & 0     &       &       &       &       &       & & 0\\ \hline
        \tl 3   & 0     &       &       &       &       &       & & 0\\ \hline
        \tl 4   & 3     & 0     &       &       &       &       & & 3\\ \hline
        \tl 5   & 8     & 3     &       &       &       &       & & 11\\ \hline
        \tl 6   & 128   & 5     & 0     &       &       &       & & 133\\ \hline
    \end{tabular}
    \\
    $\vdots$
    
    \vspace{1cm}
    
    \begin{tabular}{r||r|r|r|r|r|r|r||r|}
        \cline{2-8}
        \multicolumn{1}{r|}{} & \multicolumn{7}{c|}{$E^2_{p,l}$ for $\mathbb F_2$} \\ \hline
        \tl{\diagbox[height=1.7em, width=3em]{$p$}{$l$}} & 1 & 2 & 3 & 4 & 5 & 6 & 7& $\dimf{3}{1}{14}$ \\ \hline\hline
        \tl 2   & 0     &       &       &       &       &       & & 0\\ \hline
        \tl 3   & 0     &       &       &       &       &       & & 0\\ \hline
        \tl 4   & 1     & 0     &       &       &       &       & & 1\\ \hline
        \tl 5   & 5     & 3     &       &       &       &       & & 8\\ \hline
        \tl 6   & 6     & 2     & 0     &       &       &       & & 8\\ \hline
    \end{tabular}
    \\
    $\vdots$
\end{center}

\subsection{The Radial Case}

Again note that in the radial case, we have $h = 2g + m - 1$.
For our techniques to work, we need to require $m > 0$.

\subsubsection{Genus \texorpdfstring{$g=0$}{g=0} and Punctures \texorpdfstring{$m=1,\ldots,6$}{m=2,...,8}}

\paragraph{The case $g=0$, $m=1$ and $h=0$ with coefficients in $\mathbb F_2$ and $\mathbb Q$:}
\begin{center}
    \begin{tabular}{r||r|r||r|}
        \cline{2-3}
        \multicolumn{1}{r|}{} & \multicolumn{2}{c|}{$E^0_{p,l}$ for $\mathbb F_2$ and $\mathbb Q$} \\ \hline
        \tl{\diagbox[height=1.7em, width=3em]{$p$}{$l$}} & 1 & 2& $\dim$ \\ \hline\hline
        \tl 0  & 1     &   & 1\\ \hline
    \end{tabular}
    
    \vspace{1cm}
    
    \begin{tabular}{r||r|r||r|}
        \cline{2-3}
        \multicolumn{1}{r|}{} & \multicolumn{2}{c|}{$E^1_{p,l}$ for $\mathbb F_2$ and $\mathbb Q$} \\ \hline
        \tl{\diagbox[height=1.7em, width=3em]{$p$}{$l$}} & 1 & 2& $\dim$ \\ \hline\hline
        \tl 0  & 1     &   & 1\\ \hline
    \end{tabular}
    
    \vspace{1cm}
    
    \begin{tabular}{r||r|r||r|}
        \cline{2-3}
        \multicolumn{1}{r|}{} & \multicolumn{2}{c|}{$E^2_{p,l}$ for $\mathbb F_2$ and $\mathbb Q$} \\ \hline
        \tl{\diagbox[height=1.7em, width=3em]{$p$}{$l$}} & 1 & 2& $\dimrr{0}{1}{2}$ \\ \hline\hline
        \tl 0  & 1     &   & $1$\\ \hline
    \end{tabular}
\end{center}

\paragraph{The case $g=0$, $m=2$ and $h = 1$ with coefficients in $\mathbb F_2$ and $\mathbb Q$:}
\begin{center}
    \begin{tabular}{r||r|r||r|}
        \cline{2-3}
        \multicolumn{1}{r|}{} & \multicolumn{2}{c|}{$E^0_{p,l}$ for $\mathbb F_2$ and $\mathbb Q$} \\ \hline
        \tl{\diagbox[height=1.7em, width=3em]{$p$}{$l$}} & 1 & 2& $\dim$ \\ \hline\hline
        \tl 1  & 1     &   & 1\\ \hline
        \tl 2  &       & 1 & 1\\ \hline
    \end{tabular}
        
    \vspace{1cm}
    
    \begin{tabular}{r||r|r||r|}
        \cline{2-3}
        \multicolumn{1}{r|}{} & \multicolumn{2}{c|}{$E^1_{p,l}$ for $\mathbb F_2$ and $\mathbb Q$} \\ \hline
        \tl{\diagbox[height=1.7em, width=3em]{$p$}{$l$}} & 1 & 2& $\dim$ \\ \hline\hline
        \tl 1  & 1     &   & 1\\ \hline
        \tl 2  &       & 1 & 1\\ \hline
    \end{tabular}
        
    \vspace{1cm}
    
    \begin{tabular}{r||r|r||r|}
        \cline{2-3}
        \multicolumn{1}{r|}{} & \multicolumn{2}{c|}{$E^2_{p,l}$ for $\mathbb F_2$ and $\mathbb Q$} \\ \hline
        \tl{\diagbox[height=1.7em, width=3em]{$p$}{$l$}} & 1 & 2& $\dimrr{0}{2}{2}$ \\ \hline\hline
        \tl 1  & 1     &   & 1\\ \hline
        \tl 2  &       & 1 & 1\\ \hline
    \end{tabular}
\end{center}

\paragraph{The case $g=0$, $m=3$ and $h = 2$ with coefficients in $\mathbb F_2$ and $\mathbb Q$:}
\begin{center}
    \begin{tabular}{r||r|r|r||r|}
        \cline{2-4}
        \multicolumn{1}{r|}{} & \multicolumn{3}{c|}{$E^0_{p,l}$ for $\mathbb F_2$ and $\mathbb Q$} \\ \hline
        \tl{\diagbox[height=1.7em, width=3em]{$p$}{$l$}} & 1 & 2 & 3& $\dim$ \\ \hline\hline
        \tl 2  & 2     &        &   & 2\\ \hline
        \tl 3  &       & 4      &   & 4\\ \hline
        \tl 4  &       &        & 2 & 2\\ \hline
    \end{tabular}
        
    \vspace{1cm}
    
    \begin{tabular}{r||r|r|r||r|}
        \cline{2-4}
        \multicolumn{1}{r|}{} & \multicolumn{3}{c|}{$E^1_{p,l}$ for $\mathbb F_2$ and $\mathbb Q$} \\ \hline
        \tl{\diagbox[height=1.7em, width=3em]{$p$}{$l$}} & 1 & 2 & 3& $\dim$ \\ \hline\hline
        \tl 2  & 2     &        &   & 2\\ \hline
        \tl 3  &       & 4      &   & 4\\ \hline
        \tl 4  &       &        & 2 & 2\\ \hline
    \end{tabular}
        
    \vspace{1cm}
    
    \begin{tabular}{r||r|r|r||r|}
        \cline{2-4}
        \multicolumn{1}{r|}{} & \multicolumn{3}{c|}{$E^2_{p,l}$ for $\mathbb F_2$ and $\mathbb Q$} \\ \hline
        \tl{\diagbox[height=1.7em, width=3em]{$p$}{$l$}} & 1 & 2 & 3& $\dimrr{0}{3}{4}$ \\ \hline\hline
        \tl 2  &       &        &   & 0\\ \hline
        \tl 3  &       & 1      &   & 1\\ \hline
        \tl 4  &       &        & 1 & 1\\ \hline
    \end{tabular}
\end{center}

\paragraph{The case $g=0$, $m=4$ and $h = 3$ with coefficients in $\mathbb F_2$ and $\mathbb Q$:}
\begin{center}
    \begin{tabular}{r||r|r|r|r||r|}
        \cline{2-5}
        \multicolumn{1}{r|}{} & \multicolumn{4}{c|}{$E^0_{p,l}$ for $\mathbb F_2$ and $\mathbb Q$} \\ \hline
        \tl{\diagbox[height=1.7em, width=3em]{$p$}{$l$}} & 1 & 2 & 3 & 4& $\dim$ \\ \hline\hline
        \tl 3   & 5     &       &       &   & 5\\ \hline
        \tl 4   &       & 15    &       &   & 15\\ \hline
        \tl 5   &       &       & 15    &   & 15\\ \hline
        \tl 6   &       &       &       & 5 & 5\\ \hline
    \end{tabular}
        
    \vspace{1cm}
    
    \begin{tabular}{r||r|r|r|r||r|}
        \cline{2-5}
        \multicolumn{1}{r|}{} & \multicolumn{4}{c|}{$E^1_{p,l}$ for $\mathbb F_2$ and $\mathbb Q$} \\ \hline
        \tl{\diagbox[height=1.7em, width=3em]{$p$}{$l$}} & 1 & 2 & 3 & 4& $\dim$ \\ \hline\hline
        \tl 3   & 5     &       &       &   & 5\\ \hline
        \tl 4   &       & 15    &       &   & 15\\ \hline
        \tl 5   &       &       & 15    &   & 15\\ \hline
        \tl 6   &       &       &       & 5 & 5\\ \hline
    \end{tabular}
            
    \vspace{1cm}
    
    \begin{tabular}{r||r|r|r|r||r|}
        \cline{2-5}
        \multicolumn{1}{r|}{} & \multicolumn{4}{c|}{$E^2_{p,l}$ for $\mathbb Q$} \\ \hline
        \tl{\diagbox[height=1.7em, width=3em]{$p$}{$l$}} & 1 & 2 & 3 & 4& $\dimrq{0}{4}{6}$ \\ \hline\hline
        \tl 3   & 0     &       &       &   & 0\\ \hline
        \tl 4   &       & 0     &       &   & 0\\ \hline
        \tl 5   &       &       & 1     &   & 1\\ \hline
        \tl 6   &       &       &       & 1 & 1\\ \hline
    \end{tabular}
    
    \vspace{1cm}
    
    \begin{tabular}{r||r|r|r|r||r|}
        \cline{2-5}
        \multicolumn{1}{r|}{} & \multicolumn{4}{c|}{$E^2_{p,l}$ for $\mathbb F_2$} \\ \hline
        \tl{\diagbox[height=1.7em, width=3em]{$p$}{$l$}} & 1 & 2 & 3 & 4& $\dimrf{0}{4}{6}$ \\ \hline\hline
        \tl 3   & 1     &       &       &   & 1\\ \hline
        \tl 4   &       & 1     &       &   & 1\\ \hline
        \tl 5   &       &       & 1     &   & 1\\ \hline
        \tl 6   &       &       &       & 1 & 1\\ \hline
    \end{tabular}
\end{center}

\paragraph{The case $g=0$, $m=5$ and $h = 4$ with coefficients in $\mathbb F_2$ and $\mathbb Q$:}
\begin{center}
    \begin{tabular}{r||r|r|r|r|r||r|}
        \cline{2-6}
        \multicolumn{1}{r|}{} & \multicolumn{5}{c|}{$E^0_{p,l}$ for $\mathbb F_2$ and $\mathbb Q$} \\ \hline
        \tl{\diagbox[height=1.7em, width=3em]{$p$}{$l$}} & 1 & 2 & 3 & 4 & 5& $\dim$ \\ \hline\hline
        \tl 4   & 14    &       &       &       &   & 14\\ \hline
        \tl 5   &       & 56    &       &       &   & 56\\ \hline
        \tl 6   &       &       & 84    &       &   & 84\\ \hline
        \tl 7   &       &       &       & 56    &   & 56\\ \hline
        \tl{8}  &       &       &       &       & 14& 14\\ \hline
    \end{tabular}
        
    \vspace{1cm}
    
    \begin{tabular}{r||r|r|r|r|r||r|}
        \cline{2-6}
        \multicolumn{1}{r|}{} & \multicolumn{5}{c|}{$E^1_{p,l}$ for $\mathbb F_2$ and $\mathbb Q$} \\ \hline
        \tl{\diagbox[height=1.7em, width=3em]{$p$}{$l$}} & 1 & 2 & 3 & 4 & 5& $\dim$ \\ \hline\hline
        \tl 4   & 14    &       &       &       &   & 14\\ \hline
        \tl 5   &       & 56    &       &       &   & 56\\ \hline
        \tl 6   &       &       & 84    &       &   & 84\\ \hline
        \tl 7   &       &       &       & 56    &   & 56\\ \hline
        \tl{8}  &       &       &       &       & 14& 14\\ \hline
    \end{tabular}
            
    \vspace{1cm}
    
        \begin{tabular}{r||r|r|r|r|r||r|}
        \cline{2-6}
        \multicolumn{1}{r|}{} & \multicolumn{5}{c|}{$E^2_{p,l}$ for $\mathbb Q$} \\ \hline
        \tl{\diagbox[height=1.7em, width=3em]{$p$}{$l$}} & 1 & 2 & 3 & 4 & 5& $\dimrq{0}{5}{8}$ \\ \hline\hline
        \tl 4   & 0     &       &       &       &  & 0\\ \hline
        \tl 5   &       & 0     &       &       &  & 0\\ \hline
        \tl 6   &       &       & 0     &       &  & 0\\ \hline
        \tl 7   &       &       &       & 1     &  & 1\\ \hline
        \tl{8}  &       &       &       &       & 1& 1\\ \hline
    \end{tabular}
    
    \vspace{1cm}
    
    \begin{tabular}{r||r|r|r|r|r||r|}
        \cline{2-6}
        \multicolumn{1}{r|}{} & \multicolumn{5}{c|}{$E^2_{p,l}$ for $\mathbb F_2$} \\ \hline
        \tl{\diagbox[height=1.7em, width=3em]{$p$}{$l$}} & 1 & 2 & 3 & 4 & 5& $\dimrf{0}{5}{8}$ \\ \hline\hline
        \tl 4   & 0     &       &       &       &  & 0\\ \hline
        \tl 5   &       & 1     &       &       &  & 1\\ \hline
        \tl 6   &       &       & 1     &       &  & 1\\ \hline
        \tl 7   &       &       &       & 1     &  & 1\\ \hline
        \tl{8} &       &       &       &       & 1& 1\\ \hline
    \end{tabular}
\end{center}

\paragraph{The case $g=0$, $m=6$ and $h = 5$ with coefficients in $\mathbb F_2$ and $\mathbb Q$:}
\begin{center}
    \begin{tabular}{r||r|r|r|r|r|r||r|}
        \cline{2-7}
        \multicolumn{1}{r|}{} & \multicolumn{6}{c|}{$E^0_{p,l}$ for $\mathbb F_2$ and $\mathbb Q$} \\ \hline
        \tl{\diagbox[height=1.7em, width=3em]{$p$}{$l$}} & 1 & 2 & 3 & 4 & 5 & 6& $\dim$ \\ \hline\hline
        \tl 5   & 42    &       &       &       &       &   & 42\\ \hline
        \tl 6   &       & 210   &       &       &       &   & 210\\ \hline
        \tl 7   &       &       & 420   &       &       &   & 420\\ \hline
        \tl{8}  &       &       &       & 420   &       &   & 420\\ \hline
        \tl{9}  &       &       &       &       & 210   &   & 210\\ \hline
        \tl{10} &       &       &       &       &       & 42& 42\\ \hline
    \end{tabular}
    
    \vspace{1cm}
    
    \begin{tabular}{r||r|r|r|r|r|r||r|}
        \cline{2-7}
        \multicolumn{1}{r|}{} & \multicolumn{6}{c|}{$E^1_{p,l}$ for $\mathbb F_2$ and $\mathbb Q$} \\ \hline
        \tl{\diagbox[height=1.7em, width=3em]{$p$}{$l$}} & 1 & 2 & 3 & 4 & 5 & 6& $\dim$ \\ \hline\hline
        \tl 5   & 42    &       &       &       &       &   & 42\\ \hline
        \tl 6   &       & 210   &       &       &       &   & 210\\ \hline
        \tl 7   &       &       & 420   &       &       &   & 420\\ \hline
        \tl{8}  &       &       &       & 420   &       &   & 420\\ \hline
        \tl{9}  &       &       &       &       & 210   &   & 210\\ \hline
        \tl{10} &       &       &       &       &       & 42& 42\\ \hline
    \end{tabular}
            
    \vspace{1cm}
    
    \begin{tabular}{r||r|r|r|r|r|r||r|}
        \cline{2-7}
        \multicolumn{1}{r|}{} & \multicolumn{6}{c|}{$E^2_{p,l}$ for $\mathbb Q$} \\ \hline
        \tl{\diagbox[height=1.7em, width=3em]{$p$}{$l$}} & 1 & 2 & 3 & 4 & 5 & 6& $\dimrq{0}{6}{10}$ \\ \hline\hline
        \tl  5  & 0     &       &       &       &       &  & 0\\ \hline
        \tl  6  &       & 0     &       &       &       &  & 0\\ \hline
        \tl  7  &       &       & 0     &       &       &  & 0\\ \hline
        \tl{8}  &       &       &       & 0     &       &  & 0\\ \hline
        \tl{9}  &       &       &       &       & 1     &  & 1\\ \hline
        \tl{10} &       &       &       &       &       & 1& 1\\ \hline
    \end{tabular}
    
    \vspace{1cm}
    
    \begin{tabular}{r||r|r|r|r|r|r||r|}
        \cline{2-7}
        \multicolumn{1}{r|}{} & \multicolumn{6}{c|}{$E^2_{p,l}$ for $\mathbb F_2$} \\ \hline
        \tl{\diagbox[height=1.7em, width=3em]{$p$}{$l$}} & 1 & 2 & 3 & 4 & 5 & 6& $\dimrf{0}{6}{10}$ \\ \hline\hline
        \tl 5   & 0     &       &       &       &       &  & 0\\ \hline
        \tl 6   &       & 1     &       &       &       &  & 1\\ \hline
        \tl 7   &       &       & 2     &       &       &  & 2\\ \hline
        \tl{8}  &       &       &       & 1     &       &  & 1\\ \hline
        \tl{9}  &       &       &       &       & 1     &  & 1\\ \hline
        \tl{10} &       &       &       &       &       & 1& 1\\ \hline
    \end{tabular}
\end{center}

\paragraph{The case $g=0$, $m=7$ and $h = 6$ with coefficients in $\mathbb F_2$ and $\mathbb Q$:}
\begin{center}
    \begin{tabular}{r||r|r|r|r|r|r|r||r|}
        \cline{2-8}
        \multicolumn{1}{r|}{} & \multicolumn{7}{c|}{$E^0_{p,l}$ for $\mathbb F_2$ and $\mathbb Q$} \\ \hline
        \tl{\diagbox[height=1.7em, width=3em]{$p$}{$l$}} & 1 & 2 & 3 & 4 & 5 & 6& 7 & $\dim$ \\ \hline\hline
        \tl 6   & 132    &       &       &       &       &      &     & 132\\ \hline
        \tl 7   &        & 792   &       &       &       &      &     & 792\\ \hline
        \tl 8   &        &       & 1980  &       &       &      &     & 1980\\ \hline
        \tl{9}  &        &       &       & 2640  &       &      &     & 2640\\ \hline
        \tl{10} &        &       &       &       & 1980  &      &     & 1980\\ \hline
        \tl{11} &        &       &       &       &       & 792  &     & 792\\ \hline
	\tl{12} &        &       &       &       &       &      & 132 & 132 \\ \hline
    \end{tabular}
    
    \vspace{1cm}
    
    \begin{tabular}{r||r|r|r|r|r|r|r||r|}
        \cline{2-8}
        \multicolumn{1}{r|}{} & \multicolumn{7}{c|}{$E^0_{p,l}$ for $\mathbb F_2$ and $\mathbb Q$} \\ \hline
        \tl{\diagbox[height=1.7em, width=3em]{$p$}{$l$}} & 1 & 2 & 3 & 4 & 5 & 6& 7 & $\dim$ \\ \hline\hline
        \tl 6   & 132    &       &       &       &       &      &     & 132\\ \hline
        \tl 7   &        & 792   &       &       &       &      &     & 792\\ \hline
        \tl 8   &        &       & 1980  &       &       &      &     & 1980\\ \hline
        \tl{9}  &        &       &       & 2640  &       &      &     & 2640\\ \hline
        \tl{10} &        &       &       &       & 1980  &      &     & 1980\\ \hline
        \tl{11} &        &       &       &       &       & 792  &     & 792\\ \hline
	\tl{12} &        &       &       &       &       &      & 132 & 132 \\ \hline
    \end{tabular}
            
    \vspace{1cm}
    
    \begin{tabular}{r||r|r|r|r|r|r|r||r|}
        \cline{2-8}
        \multicolumn{1}{r|}{} & \multicolumn{7}{c|}{$E^2_{p,l}$ for $\mathbb Q$} \\ \hline
        \tl{\diagbox[height=1.7em, width=3em]{$p$}{$l$}} & 1 & 2 & 3 & 4 & 5 & 6& 7 & $\dimrq{0}{7}{12}$ \\ \hline\hline
        \tl 6   &        &       &       &       &       &       &      & \\ \hline
        \tl 7   &        &       &       &       &       &       &      & \\ \hline
        \tl 8   &        &       &       &       &       &       &      & \\ \hline
        \tl{9}  &        &       &       &       &       &       &      & \\ \hline
        \tl{10} &        &       &       &       &       &       &      & \\ \hline
        \tl{11} &        &       &       &       &       &  1    &      & 1\\ \hline
	\tl{12} &        &       &       &       &       &       & 1    & 1\\ \hline
    \end{tabular}
    
    \vspace{1cm}
    
    \begin{tabular}{r||r|r|r|r|r|r|r||r|}
        \cline{2-8}
        \multicolumn{1}{r|}{} & \multicolumn{7}{c|}{$E^2_{p,l}$ for $\mathbb F_2$} \\ \hline
        \tl{\diagbox[height=1.7em, width=3em]{$p$}{$l$}} & 1 & 2 & 3 & 4 & 5 & 6& 7 & $\dimrf{0}{7}{12}$ \\ \hline\hline
        \tl 6   &        &       &       &       &       &       &      & \\ \hline
        \tl 7   &        &       &       &       &       &       &      & \\ \hline
        \tl 8   &        &       & 1     &       &       &       &      & 1\\ \hline
        \tl{9}  &        &       &       & 2     &       &       &      & 2\\ \hline
        \tl{10} &        &       &       &       & 1     &       &      & 1\\ \hline
        \tl{11} &        &       &       &       &       &  1    &      & 1\\ \hline
	\tl{12} &        &       &       &       &       &       & 1    & 1\\ \hline
    \end{tabular}
\end{center}

\paragraph{The case $g=0$, $m=8$ and $h = 7$ with coefficients in $\mathbb F_2$ and $\mathbb Q$:}
\begin{center}
    \begin{tabular}{r||r|r|r|r|r|r|r|r||r|}
        \cline{2-9}
        \multicolumn{1}{r|}{} & \multicolumn{8}{c|}{$E^0_{p,l}$ for $\mathbb F_2$ and $\mathbb Q$} \\ \hline
        \tl{\diagbox[height=1.7em, width=3em]{$p$}{$l$}} & 1 & 2 & 3 & 4 & 5 & 6& 7 & 8 & $\dim$ \\ \hline\hline
        \tl 7   & 429    &       &       &       &       &      &     &     & 429\\ \hline
        \tl 8   &        & 3003  &       &       &       &      &     &     & 3003\\ \hline
        \tl 9   &        &       & 9009  &       &       &      &     &     & 9009\\ \hline
        \tl{10} &        &       &       & 15015 &       &      &     &     & 15015\\ \hline
        \tl{11} &        &       &       &       & 15015 &      &     &     & 15015\\ \hline
        \tl{12} &        &       &       &       &       & 9009 &     &     & 9009\\ \hline
	\tl{13} &        &       &       &       &       &      & 3003&     & 3003\\ \hline
	\tl{14} &        &       &       &       &       &      &     & 429 & 429\\ \hline
    \end{tabular}
    
    \vspace{1cm}
    
    \begin{tabular}{r||r|r|r|r|r|r|r|r||r|}
        \cline{2-9}
        \multicolumn{1}{r|}{} & \multicolumn{8}{c|}{$E^1_{p,l}$ for $\mathbb F_2$ and $\mathbb Q$} \\ \hline
        \tl{\diagbox[height=1.7em, width=3em]{$p$}{$l$}} & 1 & 2 & 3 & 4 & 5 & 6& 7 & 8 & $\dim$ \\ \hline\hline
        \tl 7   & 429    &       &       &       &       &      &     &     & 429\\ \hline
        \tl 8   &        & 3003  &       &       &       &      &     &     & 3003\\ \hline
        \tl 9   &        &       & 9009  &       &       &      &     &     & 9009\\ \hline
        \tl{10} &        &       &       & 15015 &       &      &     &     & 15015\\ \hline
        \tl{11} &        &       &       &       & 15015 &      &     &     & 15015\\ \hline
        \tl{12} &        &       &       &       &       & 9009 &     &     & 9009\\ \hline
	\tl{13} &        &       &       &       &       &      & 3003&     & 3003\\ \hline
	\tl{14} &        &       &       &       &       &      &     & 429 & 429\\ \hline
    \end{tabular}
            
    \vspace{1cm}
    
    \begin{tabular}{r||r|r|r|r|r|r|r|r||r|}
        \cline{2-9}
        \multicolumn{1}{r|}{} & \multicolumn{8}{c|}{$E^2_{p,l}$ for $\mathbb Q$} \\ \hline
        \tl{\diagbox[height=1.7em, width=3em]{$p$}{$l$}} & 1 & 2 & 3 & 4 & 5 & 6& 7 & 8 & $\dimrq{0}{8}{14}$ \\ \hline\hline
        \tl 7   & 0      &       &       &       &       &      &     &     & 0\\ \hline
        \tl 8   &        & 0     &       &       &       &      &     &     & 0\\ \hline
        \tl 9   &        &       & 0     &       &       &      &     &     & 0\\ \hline
        \tl{10} &        &       &       & 0     &       &      &     &     & 0\\ \hline
        \tl{11} &        &       &       &       & 0     &      &     &     & 0\\ \hline
        \tl{12} &        &       &       &       &       & 0    &     &     & 0\\ \hline
	\tl{13} &        &       &       &       &       &      & 1   &     & 1\\ \hline
	\tl{14} &        &       &       &       &       &      &     & 1   & 1\\ \hline
    \end{tabular}
    
    \vspace{1cm}
    
    \begin{tabular}{r||r|r|r|r|r|r|r|r||r|}
        \cline{2-9}
        \multicolumn{1}{r|}{} & \multicolumn{8}{c|}{$E^2_{p,l}$ for $\mathbb F_2$} \\ \hline
        \tl{\diagbox[height=1.7em, width=3em]{$p$}{$l$}} & 1 & 2 & 3 & 4 & 5 & 6& 7 & 8 & $\dimrq{0}{8}{14}$ \\ \hline\hline
        \tl 7   & 1      &       &       &       &       &      &     &     & 1\\ \hline
        \tl 8   &        & 1     &       &       &       &      &     &     & 1\\ \hline
        \tl 9   &        &       & 1     &       &       &      &     &     & 1\\ \hline
        \tl{10} &        &       &       & 2     &       &      &     &     & 2\\ \hline
        \tl{11} &        &       &       &       & 2     &      &     &     & 2\\ \hline
        \tl{12} &        &       &       &       &       & 1    &     &     & 1\\ \hline
	\tl{13} &        &       &       &       &       &      & 1   &     & 1\\ \hline
	\tl{14} &        &       &       &       &       &      &     & 1   & 1\\ \hline
    \end{tabular}
\end{center}

\subsubsection{Genus \texorpdfstring{$g=1$}{g=1} and Punctures \texorpdfstring{$m=1,\ldots,6$}{m=1,...,6}}

\paragraph{The case $g=1$, $m=1$ and $h = 2$ with coefficients in $\mathbb F_2$ and $\mathbb Q$:}
\begin{center}
    \begin{tabular}{r||r|r|r||r|}
        \cline{2-4}
        \multicolumn{1}{r|}{} & \multicolumn{3}{c|}{$E^0_{p,l}$ for $\mathbb F_2$ and $\mathbb Q$} \\ \hline
        \tl{\diagbox[height=1.7em, width=3em]{$p$}{$l$}} & 1 & 2 & 3& $\dim$ \\ \hline\hline
        \tl 1  & 1     &      &   & 1\\ \hline
        \tl 2  & 2     & 1    &   & 3\\ \hline
        \tl 3  &       & 3    &   & 3\\ \hline
        \tl 4  &       &      & 1 & 1\\ \hline
    \end{tabular}
        
    \vspace{1cm}
     
    \begin{tabular}{r||r|r|r||r|}
        \cline{2-4}
        \multicolumn{1}{r|}{} & \multicolumn{3}{c|}{$E^1_{p,l}$ for $\mathbb Q$} \\ \hline
        \tl{\diagbox[height=1.7em, width=3em]{$p$}{$l$}} & 1 & 2 & 3& $\dim$ \\ \hline\hline
        \tl 1  &       &      &   & 0\\ \hline
        \tl 2  & 1     &      &   & 1\\ \hline
        \tl 3  &       & 2    &   & 2\\ \hline
        \tl 4  &       &      & 1 & 1\\ \hline
    \end{tabular}

    \vspace{1cm}

    \begin{tabular}{r||r|r|r||r|}
        \cline{2-4}
        \multicolumn{1}{r|}{} & \multicolumn{3}{c|}{$E^2_{p,l}$ for $\mathbb Q$} \\ \hline
        \tl{\diagbox[height=1.7em, width=3em]{$p$}{$l$}} & 1 & 2 & 3& $\dimrq{1}{1}{4}$ \\ \hline\hline
        \tl 1  &       &      &   & 0\\ \hline
        \tl 2  &       &      &   & 0\\ \hline
        \tl 3  &       & 1    &   & 1\\ \hline
        \tl 4  &       &      & 1 & 1\\ \hline
    \end{tabular}

    \vspace{1cm}
        
    \begin{tabular}{r||r|r|r||r|}
        \cline{2-4}
        \multicolumn{1}{r|}{} & \multicolumn{3}{c|}{$E^1_{p,l}$ for $\mathbb F_2$} \\ \hline
        \tl{\diagbox[height=1.7em, width=3em]{$p$}{$l$}} & 1 & 2 & 3& $\dim$ \\ \hline\hline
        \tl 1  & 1     &      &   & 1\\ \hline
        \tl 2  & 2     &      &   & 2\\ \hline
        \tl 3  &       & 2    &   & 2\\ \hline
        \tl 4  &       &      & 1 & 1\\ \hline
    \end{tabular}
             
    \vspace{1cm}

    \begin{tabular}{r||r|r|r||r|}
        \cline{2-4}
        \multicolumn{1}{r|}{} & \multicolumn{3}{c|}{$E^2_{p,l}$ for $\mathbb F_2$} \\ \hline
        \tl{\diagbox[height=1.7em, width=3em]{$p$}{$l$}} & 1 & 2 & 3& $\dimrf{1}{1}{4}$ \\ \hline\hline
        \tl 1  & 1     &      &   & 1\\ \hline
        \tl 2  & 1     &      &   & 1\\ \hline
        \tl 3  &       & 1    &   & 1\\ \hline
        \tl 4  &       &      & 1 & 1\\ \hline
    \end{tabular}
\end{center}

\paragraph{The case $g=1$, $m=2$ and $h = 3$ with coefficients in $\mathbb F_2$ and $\mathbb Q$:}
\begin{center}
    \begin{tabular}{r||r|r|r|r||r|}
        \cline{2-5}
        \multicolumn{1}{r|}{} & \multicolumn{4}{c|}{$E^0_{p,l}$ for $\mathbb F_2$ and $\mathbb Q$} \\ \hline
        \tl{\diagbox[height=1.7em, width=3em]{$p$}{$l$}} & 1 & 2 & 3 & 4& $\dim$ \\ \hline\hline
        \tl 1   & 1     &       &       &   & 1\\ \hline
        \tl 2   & 12    & 1     &       &   & 13\\ \hline
        \tl 3   & 25    & 18    &       &   & 33\\ \hline
        \tl 4   &       & 55    & 6     &   & 61\\ \hline
        \tl 5   &       &       & 40    &   & 40\\ \hline
        \tl 6   &       &       &       & 10& 10\\ \hline
    \end{tabular}
        
    \vspace{1cm}
    
    \begin{tabular}{r||r|r|r|r||r|}
        \cline{2-5}
        \multicolumn{1}{r|}{} & \multicolumn{4}{c|}{$E^1_{p,l}$ for $\mathbb Q$} \\ \hline
        \tl{\diagbox[height=1.7em, width=3em]{$p$}{$l$}} & 1 & 2 & 3 & 4& $\dim$ \\ \hline\hline
        \tl 1   &       &       &       &   & \\ \hline
        \tl 2   &       &       &       &   & \\ \hline
        \tl 3   & 14    &       &       &   & 14\\ \hline
        \tl 4   &       & 38    &       &   & 38\\ \hline
        \tl 5   &       &       & 34    &   & 34\\ \hline
        \tl 6   &       &       &       & 10& 10\\ \hline
    \end{tabular}
        
    \vspace{1cm}
    
    \begin{tabular}{r||r|r|r|r||r|}
        \cline{2-5}
        \multicolumn{1}{r|}{} & \multicolumn{4}{c|}{$E^2_{p,l}$ for $\mathbb Q$} \\ \hline
        \tl{\diagbox[height=1.7em, width=3em]{$p$}{$l$}} & 1 & 2 & 3 & 4& $\dimrq{1}{2}{6}$ \\ \hline\hline
        \tl 1   & 0     &       &       &   & 0\\ \hline
        \tl 2   & 0     & 0     &       &   & 0\\ \hline
        \tl 3   & 0     & 0     &       &   & 0\\ \hline
        \tl 4   &       & 0     & 0     &   & 0\\ \hline
        \tl 5   &       &       & 1     &   & 1\\ \hline
        \tl 6   &       &       &       & 1 & 1\\ \hline
    \end{tabular}

    \vspace{1cm}
    
    \begin{tabular}{r||r|r|r|r||r|}
        \cline{2-5}
        \multicolumn{1}{r|}{} & \multicolumn{4}{c|}{$E^1_{p,l}$ for $\mathbb F_2$} \\ \hline
        \tl{\diagbox[height=1.7em, width=3em]{$p$}{$l$}} & 1 & 2 & 3 & 4& $\dim$ \\ \hline\hline
        \tl 1   &       &       &       &   & \\ \hline
        \tl 2   & 1     &       &       &   & 1\\ \hline
        \tl 3   & 15    & 1     &       &   & 16\\ \hline
        \tl 4   &       & 39    &       &   & 39\\ \hline
        \tl 5   &       &       & 34    &   & 34\\ \hline
        \tl 6   &       &       &       & 10& 10\\ \hline
    \end{tabular}
        
    \vspace{1cm}
    
    \begin{tabular}{r||r|r|r|r||r|}
        \cline{2-5}
        \multicolumn{1}{r|}{} & \multicolumn{4}{c|}{$E^2_{p,l}$ for $\mathbb F_2$} \\ \hline
        \tl{\diagbox[height=1.7em, width=3em]{$p$}{$l$}} & 1 & 2 & 3 & 4& $\dimrf{1}{2}{6}$ \\ \hline\hline
        \tl 1   & 0     &       &       &   & 0\\ \hline
        \tl 2   & 1     & 0     &       &   & 1\\ \hline
        \tl 3   & 2     & 1     &       &   & 3\\ \hline
        \tl 4   &       & 3     & 0     &   & 3\\ \hline
        \tl 5   &       &       & 2     &   & 2\\ \hline
        \tl 6   &       &       &       & 1 & 1\\ \hline
    \end{tabular}
\end{center}

\paragraph{The case $g=1$, $m=3$ and $h = 4$ with coefficients in $\mathbb F_2$ and $\mathbb Q$:}
\begin{center}
    \begin{tabular}{r||r|r|r|r|r||r|}
        \cline{2-6}
        \multicolumn{1}{r|}{} & \multicolumn{5}{c|}{$E^0_{p,l}$ for $\mathbb F_2$ and $\mathbb Q$} \\ \hline
        \tl{\diagbox[height=1.7em, width=3em]{$p$}{$l$}} & 1 & 2 & 3 & 4 & 5& $\dim$ \\ \hline\hline
        \tl 2   & 10    &       &       &       &   & 10\\ \hline
        \tl 3   & 96    & 14    &       &       &   & 110\\ \hline
        \tl 4   & 210   & 196   & 4     &       &   & 410\\ \hline
        \tl 5   &       & 610   & 130   &       &   & 740\\ \hline
        \tl 6   &       &       & 680   & 30    &   & 710\\ \hline
        \tl 7   &       &       &       & 350   &   & 350\\ \hline
        \tl{8}  &       &       &       &       & 70& 70\\ \hline
    \end{tabular}
        
    \vspace{1cm}
    
    \begin{tabular}{r||r|r|r|r|r||r|}
        \cline{2-6}
        \multicolumn{1}{r|}{} & \multicolumn{5}{c|}{$E^1_{p,l}$ for $\mathbb Q$} \\ \hline
        \tl{\diagbox[height=1.7em, width=3em]{$p$}{$l$}} & 1 & 2 & 3 & 4 & 5& $\dim$ \\ \hline\hline
        \tl 2   & 0     &       &       &       &   & 10\\ \hline
        \tl 3   & 1     & 0     &       &       &   & 1\\ \hline
        \tl 4   & 125   & 0     & 0     &       &   & 125\\ \hline
        \tl 5   &       & 428   & 0     &       &   & 428\\ \hline
        \tl 6   &       &       & 554   & 0     &   & 554\\ \hline
        \tl 7   &       &       &       & 320   &   & 320\\ \hline
        \tl{8}  &       &       &       &       & 70& 70\\ \hline
    \end{tabular}
        
    \vspace{1cm}
    
        \begin{tabular}{r||r|r|r|r|r||r|}
        \cline{2-6}
        \multicolumn{1}{r|}{} & \multicolumn{5}{c|}{$E^2_{p,l}$ for $\mathbb Q$} \\ \hline
        \tl{\diagbox[height=1.7em, width=3em]{$p$}{$l$}} & 1 & 2 & 3 & 4 & 5& $\dimrq{1}{3}{8}$ \\ \hline\hline
        \tl 2   & 0     &       &       &       &  & 0\\ \hline
        \tl 3   & 1     & 0     &       &       &  & 1\\ \hline
        \tl 4   & 2     & 0     & 0     &       &  & 2\\ \hline
        \tl 5   &       & 1     & 0     &       &  & 1\\ \hline
        \tl 6   &       &       & 0     & 0     &  & 0\\ \hline
        \tl 7   &       &       &       & 1     &  & 1\\ \hline
        \tl{8}  &       &       &       &       & 1& 1\\ \hline
    \end{tabular}

    \vspace{1cm}
    
    \begin{tabular}{r||r|r|r|r|r||r|}
        \cline{2-6}
        \multicolumn{1}{r|}{} & \multicolumn{5}{c|}{$E^1_{p,l}$ for $\mathbb F_2$} \\ \hline
        \tl{\diagbox[height=1.7em, width=3em]{$p$}{$l$}} & 1 & 2 & 3 & 4 & 5& $\dim$ \\ \hline\hline
        \tl 2   & 0     &       &       &       &   & 0\\ \hline
        \tl 3   & 3     & 0     &       &       &   & 3\\ \hline
        \tl 4   & 127   & 6     & 0     &       &   & 133\\ \hline
        \tl 5   &       & 434   & 3     &       &   & 437\\ \hline
        \tl 6   &       &       & 557   & 0     &   & 557\\ \hline
        \tl 7   &       &       &       & 320   &   & 320\\ \hline
        \tl{8}  &       &       &       &       & 70& 70\\ \hline
    \end{tabular}
        
    \vspace{1cm}
    
    \begin{tabular}{r||r|r|r|r|r||r|r|}
        \cline{2-6}
        \multicolumn{1}{r|}{} & \multicolumn{5}{c|}{$E^2_{p,l}$ for $\mathbb F_2$} \\ \hline
        \tl{\diagbox[height=1.7em, width=3em]{$p$}{$l$}} & 1 & 2 & 3 & 4 & 5 & $\dimrf{1}{3}{8}$ \\ \hline\hline
        \tl 2   & 0     &       &       &       &       & 0\\ \hline
        \tl 3   & 1     & 0     &       &       &       & 1\\ \hline
        \tl 4   & 2     & 2     & 0     &       &       & 4\\ \hline
        \tl 5   &       & 4     & 1     &       &       & 5\\ \hline
        \tl 6   &       &       & 3     & 0     &       & 3\\ \hline
        \tl 7   &       &       &       & 2     &       & 2\\ \hline
        \tl{8}  &       &       &       &       & 1     & 1\\ \hline
    \end{tabular}
\end{center}

\paragraph{The case $g=1$, $m=4$ and $h = 5$ with coefficients in $\mathbb Q$ and $\mathbb F_2$:}
\begin{center}
    \begin{tabular}{r||r|r|r|r|r|r||r|}
        \cline{2-7}
        \multicolumn{1}{r|}{} & \multicolumn{6}{c|}{$E^0_{p,l}$ for $\mathbb Q$ and $\mathbb F_2$} \\ \hline
        \tl{\diagbox[height=1.7em, width=3em]{$p$}{$l$}} & 1 & 2 & 3 & 4 & 5 & 6& $\dim$ \\ \hline\hline
        \tl 3   & 70    &       &       &       &       &    & 70\\ \hline
        \tl 4   & 640   & 130   &       &       &       &    & 770\\ \hline
        \tl 5   & 1470  & 1675  & 75    &       &       &    & 3220\\ \hline
        \tl 6   &       & 5320  & 1665  & 15    &       &    & 7000\\ \hline
        \tl 7   &       &       & 7980  & 770   &       &    & 8750\\ \hline
        \tl{8}  &       &       &       & 6230  & 140   &    & 6370\\ \hline
        \tl{9}  &       &       &       &       & 2520  &    & 2520\\ \hline
        \tl{10} &       &       &       &       &       & 420& 420\\ \hline
    \end{tabular}
    
    \vspace{1cm}
    
    \begin{tabular}{r||r|r|r|r|r|r||r|}
        \cline{2-7}
        \multicolumn{1}{r|}{} & \multicolumn{6}{c|}{$E^1_{p,l}$ for $\mathbb Q$} \\ \hline
        \tl{\diagbox[height=1.7em, width=3em]{$p$}{$l$}} & 1 & 2 & 3 & 4 & 5 & 6& $\dim$ \\ \hline\hline
        \tl 3   & 0     &       &       &       &       &    & 0\\ \hline
        \tl 4   & 1     & 0     &       &       &       &    & 1\\ \hline
        \tl 5   & 901   & 1     & 0     &       &       &    & 902\\ \hline
        \tl 6   &       & 3776  & 0     & 0     &       &    & 3776\\ \hline
        \tl 7   &       &       & 6390  & 0     &       &    & 6390\\ \hline
        \tl{8}  &       &       &       & 5475  & 0     &    & 5475\\ \hline
        \tl{9}  &       &       &       &       & 2380  &    & 2380\\ \hline
        \tl{10} &       &       &       &       &       & 420& 420\\ \hline
    \end{tabular}
    
    \vspace{1cm}

    \begin{tabular}{r||r|r|r|r|r|r||r|}
        \cline{2-7}
        \multicolumn{1}{r|}{} & \multicolumn{6}{c|}{$E^2_{p,l}$ for $\mathbb Q$} \\ \hline
        \tl{\diagbox[height=1.7em, width=3em]{$p$}{$l$}} & 1 & 2 & 3 & 4 & 5 & 6& $\dimrq{1}{4}{10}$ \\ \hline\hline
        \tl 3   & 0     &       &       &       &       &    & 0\\ \hline
        \tl 4   & 1     & 0     &       &       &       &    & 1\\ \hline
        \tl 5   & 1     & 1     & 0     &       &       &    & 2\\ \hline
        \tl 6   &       & 3     & 0     & 0     &       &    & 3\\ \hline
        \tl 7   &       &       & 2     & 0     &       &    & 2\\ \hline
        \tl{8}  &       &       &       & 0     & 0     &    & 0\\ \hline
        \tl{9}  &       &       &       &       & 1     &    & 1\\ \hline
        \tl{10} &       &       &       &       &       & 1  & 1\\ \hline
    \end{tabular}

    \vspace{1cm}

    \begin{tabular}{r||r|r|r|r|r|r||r|}
        \cline{2-7}
        \multicolumn{1}{r|}{} & \multicolumn{6}{c|}{$E^1_{p,l}$ for $\mathbb F_2$} \\ \hline
        \tl{\diagbox[height=1.7em, width=3em]{$p$}{$l$}} & 1 & 2 & 3 & 4 & 5 & 6& $\dim$ \\ \hline\hline
        \tl 3   & 0     &       &       &       &       &    & 0\\ \hline
        \tl 4   & 11    & 0     &       &       &       &    & 11\\ \hline
        \tl 5   & 911   & 30    & 0     &       &       &    & 941\\ \hline
        \tl 6   &       & 3805  & 30    & 0     &       &    & 3835\\ \hline
        \tl 7   &       &       & 6420  & 10    &       &    & 6430\\ \hline
        \tl{8}  &       &       &       & 5485  & 0     &    & 5485\\ \hline
        \tl{9}  &       &       &       &       & 2380  &    & 2380\\ \hline
        \tl{10} &       &       &       &       &       & 420& 420\\ \hline
    \end{tabular}
        
    \vspace{1cm}
    
    \begin{tabular}{r||r|r|r|r|r|r||r|}
        \cline{2-7}
        \multicolumn{1}{r|}{} & \multicolumn{6}{c|}{$E^2_{p,l}$ for $\mathbb F_2$} \\ \hline
        \tl{\diagbox[height=1.7em, width=3em]{$p$}{$l$}} & 1 & 2 & 3 & 4 & 5 & 6& $\dimrf{1}{4}{10}$ \\ \hline\hline
        \tl 3   & 0     &       &       &       &       &  & 0\\ \hline
        \tl 4   & 2     & 0     &       &       &       &  & 2\\ \hline
        \tl 5   & 3     & 2     & 0     &       &       &  & 5\\ \hline
        \tl 6   &       & 6     & 2     & 0     &       &  & 8\\ \hline
        \tl 7   &       &       & 7     & 1     &       &  & 8\\ \hline
        \tl{8}  &       &       &       & 4     & 0     &  & 4\\ \hline
        \tl{9}  &       &       &       &       & 2     &  & 2\\ \hline
        \tl{10} &       &       &       &       &       & 1& 1\\ \hline
    \end{tabular}
\end{center}

\paragraph{The case $g=1$, $m=5$ and $h = 6$ with coefficients in $\mathbb Q$ and $\mathbb F_2$:}
For rational coefficients and $p \ge 9$, the differentials of $E_{p,l}$ could not be constructed or diagonalized due to memory and / or time limitations.
The first and second page with rational coefficients is therefore incomplete.
\begin{center}
    \begin{tabular}{r||r|r|r|r|r|r|r||r|}
        \cline{2-8}
        \multicolumn{1}{r|}{} & \multicolumn{7}{c|}{$E^0_{p,l}$ for $\mathbb Q$ and $\mathbb F_2$} \\ \hline
        \tl{\diagbox[height=1.7em, width=3em]{$p$}{$l$}} & 1 & 2 & 3 & 4 & 5 & 6 & 7& $\dim$ \\ \hline\hline
        \tl 4   & 420  &       &       &       &       &       &      & 420\\ \hline
        \tl 5   & 3840 & 990   &       &       &       &       &      & 4830\\ \hline
        \tl 6   & 9240 & 12366 & 864   &       &       &       &      & 22470\\ \hline
        \tl 7   &      & 40138 & 16422 & 350   &       &       &      & 56910\\ \hline
        \tl{8}  &      &       & 75670 & 11424 & 56    &       &      & 87150\\ \hline
        \tl{9}  &      &       &       & 79212 & 4158  &       &      & 83370\\ \hline
        \tl{10} &      &       &       &       & 48300 & 630   &      & 48930\\ \hline
        \tl{11} &      &       &       &       &       & 16170 &      & 16170\\ \hline
        \tl{12} &      &       &       &       &       &       & 2310 & 2310\\ \hline
    \end{tabular}
    
    \vspace{1cm}
    
    \begin{tabular}{r||r|r|r|r|r|r|r||r|}
        \cline{2-8}
        \multicolumn{1}{r|}{} & \multicolumn{7}{c|}{$E^1_{p,l}$ for $\mathbb Q$} \\ \hline
        \tl{\diagbox[height=1.7em, width=3em]{$p$}{$l$}} & 1 & 2 & 3 & 4 & 5 & 6 & 7& $\dim$ \\ \hline\hline
        \tl 4   & 0    &       &       &       &       &       &      & 0\\ \hline
        \tl 5   & 2    & 0     &       &       &       &       &      & 2\\ \hline
        \tl 6   & 5822 & 6     & 0     &       &       &       &      & 5828\\ \hline
        \tl 7   &      & 40138 & 3     & 0     &       &       &      & 40141\\ \hline
        \tl{8}  &      &       & 60115 & 0     & ?     &       &      & 60115 + ?\\ \hline
    \end{tabular}
    
    \vspace{1cm}
    
    \begin{tabular}{r||r|r|r|r|r|r|r||r|}
        \cline{2-8}
        \multicolumn{1}{r|}{} & \multicolumn{7}{c|}{$E^2_{p,l}$ for $\mathbb Q$} \\ \hline
        \tl{\diagbox[height=1.7em, width=3em]{$p$}{$l$}} & 1 & 2 & 3 & 4 & 5 & 6 & 7& $\dimrq{1}{5}{12}$ \\ \hline\hline
        \tl 4   & 0    &       &       &       &       &       &      & 0\\ \hline
        \tl 5   & 0    & 0     &       &       &       &       &      & 0\\ \hline
        \tl 6   & 0    & 2     & 0     &       &       &       &      & 2\\ \hline
        \tl 7   &      & 3     & 1     & 0     &       &       &      & 4\\ \hline
        \tl{8}  &      &       & 0     & 0     & ?     &       &      & ?\\ \hline
    \end{tabular}
    \vspace{1cm}
        
    \begin{tabular}{r||r|r|r|r|r|r|r||r|}
        \cline{2-8}
        \multicolumn{1}{r|}{} & \multicolumn{7}{c|}{$E^1_{p,l}$ for $\mathbb F_2$} \\ \hline
        \tl{\diagbox[height=1.7em, width=3em]{$p$}{$l$}} & 1 & 2 & 3 & 4 & 5 & 6 & 7& $\dim$ \\ \hline\hline
        \tl 4   & 0    &       &       &       &       &       &      & 420\\ \hline
        \tl 5   & 36   & 0     &       &       &       &       &      & 4830\\ \hline
        \tl 6   & 5856 & 141   & 0     &       &       &       &      & 22470\\ \hline
        \tl 7   &      & 28903 & 210   & 0     &       &       &      & 56910\\ \hline
        \tl{8}  &      &       & 60322 &  140  & 56    &       &      & 87150\\ \hline
        \tl{9}  &      &       &       & 68278 & 4158  &       &      & 83370\\ \hline
        \tl{10} &      &       &       &       & 48300 & 630   &      & 48930\\ \hline
        \tl{11} &      &       &       &       &       & 16170 &      & 16170\\ \hline
        \tl{12} &      &       &       &       &       &       & 2310 & 2310\\ \hline
    \end{tabular}
    
    \vspace{1cm}
    
    \begin{tabular}{r||r|r|r|r|r|r|r||r|}
        \cline{2-8}
        \multicolumn{1}{r|}{} & \multicolumn{7}{c|}{$E^2_{p,l}$ for $\mathbb F_2$} \\ \hline
        \tl{\diagbox[height=1.7em, width=3em]{$p$}{$l$}} & 1 & 2 & 3 & 4 & 5 & 6 & 7& $\dimrq{1}{5}{12}$ \\ \hline\hline
        \tl 4   & 0     &       &       &       &       &       & & 0\\ \hline
        \tl 5   & 2     & 0     &       &       &       &       & & 2\\ \hline
        \tl 6   & 3     & 4     & 0     &       &       &       & & 7\\ \hline
        \tl 7   &       & 7     & 3     & 0     &       &       & & 10\\ \hline
        \tl{8}  &       &       & 8     & 2     & 0     &       & & 10\\ \hline
        \tl{9}  &       &       &       & 7     & 1     &       & & 8\\ \hline
        \tl{10} &       &       &       &       & 4     & 0     & & 4\\ \hline
        \tl{11} &       &       &       &       &       & 2     & & 2\\ \hline
        \tl{12} &       &       &       &       &       &       & 1& 1\\ \hline
    \end{tabular}
\end{center}

\paragraph{The case $g=1$, $m=6$ and $h = 7$ with coefficients in $\mathbb F_2$:}
\begin{center}
    \begin{tabular}{r||r|r|r|r|r|r|r|r||r|}
        \cline{2-9}
        \multicolumn{1}{r|}{} & \multicolumn{8}{c|}{$E^0_{p,l}$ for $\mathbb F_2$} \\ \hline
        \tl{\diagbox[height=1.7em, width=3em]{$p$}{$l$}} & 1 & 2 & 3 & 4 & 5 & 6 & 7 & 8& $\dim$ \\ \hline\hline
        \tl 5   & 2310  &       &       &       &       &       &       &     & 2310\\ \hline
        \tl 6   & 21504 & 6678  &       &       &       &       &       &     & 28182\\ \hline
        \tl 7   & 54054 & 82712 & 7840  &       &       &       &       &     & 144606\\ \hline
        \tl{8}  &       & 274554& 137816& 4816  &       &       &       &     & 417186\\ \hline
        \tl{9}  &       &       & 623700& 128268& 1554  &       &       &     & 753522\\ \hline
        \tl{10} &       &       &       & 819000& 70140 & 210   &       &     & 889350\\ \hline
        \tl{11} &       &       &       &       & 667590& 21252 &       &     & 688842\\ \hline
        \tl{12} &       &       &       &       &       & 335874& 2772  &     & 338646\\ \hline
        \tl{13} &       &       &       &       &       &       & 96096 &     & 96096\\ \hline
        \tl{14} &       &       &       &       &       &       & 0     &12012& 12012\\ \hline
    \end{tabular}
    
    \vspace{1cm}
    
    \begin{tabular}{r||r|r|r|r|r|r|r|r||r|}
        \cline{2-9}
        \multicolumn{1}{r|}{} & \multicolumn{8}{c|}{$E^1_{p,l}$ for $\mathbb F_2$} \\ \hline
        \tl{\diagbox[height=1.7em, width=3em]{$p$}{$l$}} & 1 & 2 & 3 & 4 & 5 & 6 & 7 & 8& $\dim$ \\ \hline\hline
        \tl 5   & 0     &       &       &       &       &       &       &     & 0\\ \hline
        \tl 6   & 129   & 0     &       &       &       &       &       &     & 129\\ \hline
        \tl 7   & 34989 & 636   & 0     &       &       &       &       &     & 35625\\ \hline
        \tl{8}  &       & 199156& 1263  & 0     &       &       &       &     & 200419\\ \hline
        \tl{9}  &       &       & 494987& 1260  & 0     &       &       &     & 496247\\ \hline
        \tl{10} &       &       &       & 696808& 630   & 0     &       &     & 697438\\ \hline
        \tl{11} &       &       &       &       & 599634& 126   &       &     & 599760\\ \hline
        \tl{12} &       &       &       &       &       & 314958& 0     &     & 314958\\ \hline
        \tl{13} &       &       &       &       &       &       & 93324 &     & 93324\\ \hline
        \tl{14} &       &       &       &       &       &       & 0     &12012& 12012\\ \hline
    \end{tabular}
        
    \vspace{1cm}
    
        \begin{tabular}{r||r|r|r|r|r|r|r|r||r|}
        \cline{2-9}
        \multicolumn{1}{r|}{} & \multicolumn{8}{c|}{$E^2_{p,l}$ for $\mathbb F_2$} \\ \hline
        \tl{\diagbox[height=1.7em, width=3em]{$p$}{$l$}} & 1 & 2 & 3 & 4 & 5 & 6 & 7 & 8& $\dimrf{1}{6}{14}$ \\ \hline\hline
        \tl 5   & 0     &       &       &       &       &       &       &     & 0\\ \hline
        \tl 6   & 2     & 0     &       &       &       &       &       &     & 2\\ \hline
        \tl 7   & 4     & 5     & 0     &       &       &       &       &     & 9\\ \hline
        \tl{8}  &       & 10    & 5     & 0     &       &       &       &     & 15\\ \hline
        \tl{9}  &       &       & 12    & 3     & 0     &       &       &     & 15\\ \hline
        \tl{10} &       &       &       & 11    & 2     & 0     &       &     & 13\\ \hline
        \tl{11} &       &       &       &       & 8     & 1     &       &     & 9\\ \hline
        \tl{12} &       &       &       &       &       & 4     & 0     &     & 4\\ \hline
        \tl{13} &       &       &       &       &       &       & 2     &     & 2\\ \hline
        \tl{14} &       &       &       &       &       &       & 0     & 1   & 1\\ \hline
    \end{tabular}

\end{center}

\subsubsection{Genus \texorpdfstring{$g=2$}{g=2} and Punctures \texorpdfstring{$m=1,\dotsc, 4$}{m=1,..., 4}}

\paragraph{The case $g=2$, $m=1$ and $h = 4$ with coefficients in $\mathbb F_2$ and $\mathbb Q$:}
\begin{center}
    \begin{tabular}{r||r|r|r|r|r||r|}
        \cline{2-6}
        \multicolumn{1}{r|}{} & \multicolumn{5}{c|}{$E^0_{p,l}$ for $\mathbb F_2$ and $\mathbb Q$} \\ \hline
        \tl{\diagbox[height=1.7em, width=3em]{$p$}{$l$}} & 1 & 2 & 3 & 4 & 5& $\dim$ \\ \hline\hline
        \tl 1   & 1     &       &       &       &    & 1\\ \hline
        \tl 2   & 18    & 1     &       &       &    & 19\\ \hline
        \tl 3   & 78    & 23    &       &       &    & 101\\ \hline
        \tl 4   & 112   & 138   & 5     &       &    & 255\\ \hline
        \tl 5   &       & 280   & 75    &       &    & 355\\ \hline
        \tl 6   &       &       & 266   & 15    &    & 281\\ \hline
        \tl 7   &       &       &       & 119   &    & 119\\ \hline
        \tl 8   &       &       &       &       & 21 & 21\\ \hline
        \end{tabular}
        
    \vspace{1cm}
    
    \begin{tabular}{r||r|r|r|r|r||r|}
        \cline{2-6}
        \multicolumn{1}{r|}{} & \multicolumn{5}{c|}{$E^1_{p,l}$ for $\mathbb F_2$ and $\mathbb Q$} \\ \hline
        \tl{\diagbox[height=1.7em, width=3em]{$p$}{$l$}} & 1 & 2 & 3 & 4 & 5& $\dim$ \\ \hline\hline
        \tl 1   & 0     &       &       &       &    & 0\\ \hline
        \tl 2   & 1     & 0     &       &       &    & 1\\ \hline
        \tl 3   & 1     & 0     &       &       &    & 1\\ \hline
        \tl 4   & 51    & 0     & 0     &       &    & 51\\ \hline
        \tl 5   &       & 164   & 0     &       &    & 164\\ \hline
        \tl 6   &       &       & 196   & 0     &    & 196\\ \hline
        \tl 7   &       &       &       & 104   &    & 104\\ \hline
        \tl 8   &       &       &       &       & 21 & 21\\ \hline
        \end{tabular}
        
    \vspace{1cm}
    
        \begin{tabular}{r||r|r|r|r|r||r|}
        \cline{2-6}
        \multicolumn{1}{r|}{} & \multicolumn{5}{c|}{$E^2_{p,l}$ for $\mathbb Q$} \\ \hline
        \tl{\diagbox[height=1.7em, width=3em]{$p$}{$l$}} & 1 & 2 & 3 & 4 & 5& $\dimrq{2}{1}{8}$ \\ \hline\hline
        \tl 1   & 0     &       &       &       &  & 0\\ \hline
        \tl 2   & 1     & 0     &       &       &  & 1\\ \hline
        \tl 3   & 1     & 0     &       &       &  & 1\\ \hline
        \tl 4   & 0     & 0     & 0     &       &  & 0\\ \hline
        \tl 5   &       & 2     & 0     &       &  & 2\\ \hline
        \tl 6   &       &       & 1     & 0     &  & 1\\ \hline
        \tl 7   &       &       &       & 0     &  & 0\\ \hline
        \tl{8}  &       &       &       &       & 1& 1\\ \hline
    \end{tabular}

    \vspace{1cm}
    
    \begin{tabular}{r||r|r|r|r|r||r|}
        \cline{2-6}
        \multicolumn{1}{r|}{} & \multicolumn{5}{c|}{$E^1_{p,l}$ for $\mathbb F_2$} \\ \hline
        \tl{\diagbox[height=1.7em, width=3em]{$p$}{$l$}} & 1 & 2 & 3 & 4 & 5& $\dim$ \\ \hline\hline
        \tl 1   & 0     &       &       &       &     & 0\\ \hline
        \tl 2   & 1     & 0     &       &       &     & 1\\ \hline
        \tl 3   & 2     & 1     & 0     &       &     & 3\\ \hline
        \tl 4   & 52    & 2     & 0     &       &     & 54\\ \hline
        \tl 5   &       & 165   & 0     &       &     & 165\\ \hline
        \tl 6   &       &       & 196   & 0     &     & 196\\ \hline
        \tl 7   &       &       &       & 104   &     & 104\\ \hline
        \tl 8   &       &       &       &       &  21 & 21\\ \hline
    \end{tabular}
        
    \vspace{1cm}
    
    \begin{tabular}{r||r|r|r|r|r||r|}
        \cline{2-6}
        \multicolumn{1}{r|}{} & \multicolumn{5}{c|}{$E^2_{p,l}$ for $\mathbb F_2$} \\ \hline
        \tl{\diagbox[height=1.7em, width=3em]{$p$}{$l$}} & 1 & 2 & 3 & 4 & 5& $\dimrf{2}{1}{8}$ \\ \hline\hline
        \tl 1   & 0     &       &       &       &   & 0\\ \hline
        \tl 2   & 1     & 0     &       &       &   & 1\\ \hline
        \tl 3   & 2     & 1     & 0     &       &   & 3\\ \hline
        \tl 4   & 2     & 2     & 0     &       &   & 4\\ \hline
        \tl 5   &       & 5     & 0     &       &   & 5\\ \hline
        \tl 6   &       &       & 3     & 0     &   & 3\\ \hline
        \tl 7   &       &       &       & 1     &   & 1\\ \hline
        \tl 8   &       &       &       &       & 1 & 1\\ \hline
    \end{tabular}
\end{center}

\paragraph{The case $g=2$, $m=2$ and $h = 5$ with coefficients in $\mathbb F_2$ and $\mathbb Q$:}
\begin{center}
\begin{tabular}{r||r|r|r|r|r|r||r|}
        \cline{2-7}
        \multicolumn{1}{r|}{} & \multicolumn{6}{c|}{$E^0_{p,l}$ for $\mathbb F_2$ and $\mathbb Q$} \\ \hline
        \tl{\diagbox[height=1.7em, width=3em]{$p$}{$l$}} & 1 & 2 & 3 & 4 & 5 &6 & $\dim$ \\ \hline\hline
        \tl 1   & 1     &       &       &       &      &     & 1\\ \hline
        \tl 2   & 60    & 1     &       &       &      &     & 61\\ \hline
        \tl 3   & 638   & 72    &       &       &      &     & 710\\ \hline
        \tl 4   & 2480  & 1018  & 12    &       &      &     & 3510\\ \hline
        \tl 5   & 3528  & 5465  & 455   &       &      &     & 9448\\ \hline
        \tl 6   &       & 10766 & 4455  & 75    &      &     & 15296\\ \hline
        \tl 7   &       &       & 13636 & 1750  &      &     & 15386\\ \hline
        \tl 8   &       &       &       & 9170  & 280  &     & 9450\\ \hline
        \tl 9   &       &       &       &       & 3255 &     & 3255\\ \hline
        \tl{10} &       &       &       &       &      & 483 & 483\\ \hline
    \end{tabular}

\vspace{1cm}
    
\begin{tabular}{r||r|r|r|r|r|r||r|}
        \cline{2-7}
        \multicolumn{1}{r|}{} & \multicolumn{6}{c|}{$E^1_{p,l}$ for $\mathbb Q$} \\ \hline
        \tl{\diagbox[height=1.7em, width=3em]{$p$}{$l$}} & 1 & 2 & 3 & 4 & 5 &6 & $\dim$ \\ \hline\hline
        \tl 1   & 0     &       &       &       &      &     & 0\\ \hline
        \tl 2   & 0     & 0     &       &       &      &     & 0\\ \hline
        \tl 3   & 0     & 0     &       &       &      &     & 0\\ \hline
        \tl 4   & 1     & 1     & 0     &       &      &     & 2\\ \hline
        \tl 5   & 1628  & 1     & 0     &       &      &     & 1629\\ \hline
        \tl 6   &       & 6248  & 0     & 0     &      &     & 6248\\ \hline
        \tl 7   &       &       & 9624  & 0     &      &     & 9624\\ \hline
        \tl 8   &       &       &       & 7495  & 0    &     & 7495\\ \hline
        \tl 9   &       &       &       &       & 2975 &     & 2975\\ \hline
        \tl{10} &       &       &       &       &      & 483 & 483\\ \hline
    \end{tabular}

\vspace{1cm}
    
\begin{tabular}{r||r|r|r|r|r|r||r|}
        \cline{2-7}
        \multicolumn{1}{r|}{} & \multicolumn{6}{c|}{$E^2_{p,l}$ for $\mathbb Q$} \\ \hline
        \tl{\diagbox[height=1.7em, width=3em]{$p$}{$l$}} & 1 & 2 & 3 & 4 & 5 &6 & $\dimrq{2}{2}{10}$ \\ \hline\hline
        \tl 1   & 0     &       &       &       &      &     & 0\\ \hline
        \tl 2   & 0     & 0     &       &       &      &     & 0\\ \hline
        \tl 3   & 0     & 0     &       &       &      &     & 0\\ \hline
        \tl 4   & 1     & 1     & 0     &       &      &     & 2\\ \hline
        \tl 5   & 1     & 1     & 0     &       &      &     & 2\\ \hline
        \tl 6   &       & 1     & 0     & 0     &      &     & 1\\ \hline
        \tl 7   &       &       & 3     & 0     &      &     & 3\\ \hline
        \tl 8   &       &       &       & 1     & 0    &     & 1\\ \hline
        \tl 9   &       &       &       &       & 0    &     & 0\\ \hline
        \tl{10} &       &       &       &       &      & 1   & 1\\ \hline
    \end{tabular}

\vspace{1cm}
    
\begin{tabular}{r||r|r|r|r|r|r||r|}
        \cline{2-7}
        \multicolumn{1}{r|}{} & \multicolumn{6}{c|}{$E^1_{p,l}$ for $\mathbb F_2$} \\ \hline
        \tl{\diagbox[height=1.7em, width=3em]{$p$}{$l$}} & 1 & 2 & 3 & 4 & 5 &6 & $\dim$ \\ \hline\hline
        \tl 1   & 0     &       &       &       &      &     & 0\\ \hline
        \tl 2   & 1     & 0     &       &       &      &     & 1\\ \hline
        \tl 3   & 6     & 0     &       &       &      &     & 6\\ \hline
        \tl 4   & 19    & 6     & 0     &       &      &     & 25\\ \hline
        \tl 5   & 1641  & 35    & 3     &       &      &     & 1679\\ \hline
        \tl 6   &       & 6277  & 25    & 0     &      &     & 6302\\ \hline
        \tl 7   &       &       & 9646  & 5     &      &     & 9646\\ \hline
        \tl 8   &       &       &       & 7500  & 0    &     & 7500\\ \hline
        \tl 9   &       &       &       &       & 2975 &     & 2975\\ \hline
        \tl{10} &       &       &       &       &      & 483 & 483\\ \hline
    \end{tabular}

\vspace{1cm}
    
\begin{tabular}{r||r|r|r|r|r|r||r|}
        \cline{2-7}
        \multicolumn{1}{r|}{} & \multicolumn{6}{c|}{$E^2_{p,l}$ for $\mathbb F_2$} \\ \hline
        \tl{\diagbox[height=1.7em, width=3em]{$p$}{$l$}} & 1 & 2 & 3 & 4 & 5 &6 & $\dimrf{2}{2}{10}$ \\ \hline\hline
        \tl 1   & 0     &       &       &       &      &     & 0\\ \hline
        \tl 2   & 1     & 0     &       &       &      &     & 1\\ \hline
        \tl 3   & 4     & 0     &       &       &      &     & 4\\ \hline
        \tl 4   & 7     & 2     & 0     &       &      &     & 9\\ \hline
        \tl 5   & 5     & 5     & 1     &       &      &     & 11\\ \hline
        \tl 6   &       & 7     & 3     & 0     &      &     & 10\\ \hline
        \tl 7   &       &       & 8     & 1     &      &     & 9\\ \hline
        \tl 8   &       &       &       & 5     & 0    &     & 5\\ \hline
        \tl 9   &       &       &       &       & 2    &     & 2\\ \hline
        \tl{10} &       &       &       &       &      & 1   & 1\\ \hline
    \end{tabular}
\end{center}

\paragraph{The case $g=2$, $m=3$ and $h = 6$ with coefficients in $\mathbb F_2$:}
\begin{center}
    \begin{tabular}{r||r|r|r|r|r|r|r||r|}
        \cline{2-8}
        \multicolumn{1}{r|}{} & \multicolumn{7}{c|}{$E^0_{p,l}$ for $\mathbb F_2$} \\ \hline
        \tl{\diagbox[height=1.7em, width=3em]{$p$}{$l$}} & 1 & 2 & 3 & 4 & 5 & 6 & 7 & $\dim$ \\ \hline\hline
        \tl 2   & 42    &        &        &        &        &       &      & 42\\ \hline
        \tl 3   & 1296  & 48     &        &        &        &       &      & 1344\\ \hline
        \tl 4   & 11580 & 1896   & 6      &        &        &       &      & 13482\\ \hline
        \tl 5   & 43200 & 22680  & 690    &        &        &       &      & 66570\\ \hline
        \tl 6   & 61908 & 115230 & 15510  & 90     &        &       &      & 192738\\ \hline
        \tl 7   &       & 224070 & 125510 & 5110   &        &       &      & 354690\\ \hline
        \tl 8   &       &        & 354774 & 73640  & 700    &       &      & 429114\\ \hline
        \tl 9   &       &        &        & 318192 & 23310  &       &      & 341502\\ \hline
        \tl{10} &       &        &        &        & 169470 & 3150  &      & 172620\\ \hline
        \tl{11} &       &        &        &        &        & 50358 &      & 50358\\ \hline
        \tl{12} &       &        &        &        &        &       & 6468 & 6468\\ \hline
    \end{tabular}
\vspace{1cm}
    
    \begin{tabular}{r||r|r|r|r|r|r|r||r|}
        \cline{2-8}
        \multicolumn{1}{r|}{} & \multicolumn{7}{c|}{$E^1_{p,l}$ for $\mathbb F_2$} \\ \hline
        \tl{\diagbox[height=1.7em, width=3em]{$p$}{$l$}} & 1 & 2 & 3 & 4 & 5 & 6& 7 & $\dim$ \\ \hline\hline
        \tl 2   & 0     &        &        &        &        &       &      & 0\\ \hline
        \tl 3   & 1     & 0      &        &        &        &       &      & 1\\ \hline
        \tl 4   & 14    & 1      & 0      &        &        &       &      & 15\\ \hline
        \tl 5   & 155   & 34     & 0      &        &        &       &      & 189\\ \hline
        \tl 6   & 29176 & 516    & 31     & 0      &        &       &      & 29725\\ \hline
        \tl 7   &       & 130155 & 642    & 10     &        &       &      & 130807\\ \hline
        \tl 8   &       &        & 244701 & 350    & 0      &       &      & 245051\\ \hline
        \tl 9   &       &        &        & 249912 & 70     &       &      & 249982\\ \hline
        \tl{10} &       &        &        &        & 146930 & 0     &      & 146930\\ \hline
        \tl{11} &       &        &        &        &        & 47208 &      & 47208\\ \hline
        \tl{12} &       &        &        &        &        &       & 6468 & 6468\\ \hline
    \end{tabular}

\vspace{1cm}
    
    \begin{tabular}{r||r|r|r|r|r|r|r||r|}
        \cline{2-8}
        \multicolumn{1}{r|}{} & \multicolumn{7}{c|}{$E^2_{p,l}$ for $\mathbb F_2$} \\ \hline
        \tl{\diagbox[height=1.7em, width=3em]{$p$}{$l$}} & 1 & 2 & 3 & 4 & 5 & 6& 7 &$\dimrf{2}{3}{12}$ \\ \hline\hline
        \tl 2   & 0     &        &        &        &        &       &      & 0\\ \hline
        \tl 3   & 1     & 0      &        &        &        &       &      & 1\\ \hline
        \tl 4   & 4     & 1      & 0      &        &        &       &      & 5\\ \hline
        \tl 5   & 8     & 5      & 0      &        &        &       &      & 13\\ \hline
        \tl 6   & 5     & 12     & 3      & 0      &        &       &      & 20\\ \hline
        \tl 7   &       & 8      & 9      & 1      &        &       &      & 18\\ \hline
        \tl 8   &       &        & 8      & 5      & 0      &       &      & 13\\ \hline
        \tl 9   &       &        &        & 9      & 1      &       &      & 10\\ \hline
        \tl{10} &       &        &        &        & 5      & 0     &      & 5\\ \hline
        \tl{11} &       &        &        &        &        & 2     &      & 2\\ \hline
        \tl{12} &       &        &        &        &        &       & 1    & 1\\ \hline
    \end{tabular}
\end{center}

\paragraph{The case $g=2$, $m=4$ and $h = 7$ with coefficients in $\mathbb F_2$:}
For $p \ge 8$, the differentials of $E_{p,l}$ could not be constructed or diagonalized due to memory and / or time limitations.
The first and second page is therefore incomplete.

\begin{center}
    \hskip-1.5cm\begin{tabular}{r||r|r|r|r|r|r|r|r||r|}
        \cline{2-9}
        \multicolumn{1}{r|}{} & \multicolumn{8}{c|}{$E^0_{p,l}$ for $\mathbb F_2$} \\ \hline
        \tl{\diagbox[height=1.7em, width=3em]{$p$}{$l$}} & 1 & 2 & 3 & 4 & 5 & 6& 7 & 8 & $\dim$ \\ \hline\hline
        \tl 3   & 735    &         &         &         &         &         &        &         & 735\\ \hline
        \tl 4   & 18000  & 1005    &         &         &         &         &        &         & 19005\\ \hline
        \tl 5   & 149490 & 31965   & 300     &         &         &         &        &         & 181755\\ \hline
        \tl 6   & 551040 & 352560  & 18375   & 30      &         &         &        &         & 922005\\ \hline
        \tl 7   & 801801 & 1739500 & 326760  & 4970    &         &         &        &         & 2873031\\ \hline
        \tl 8   &        & 3369807 & 2377900 & 162750  & 560     &         &        &         & 5911017\\ \hline
        \tl 9   &        &         & 6408591 & 1860390 & 44310   &         &        &         & 8313291\\ \hline
        \tl{10} &        &         &         & 7185465 & 878850  & 5250    &        &         & 8069565\\ \hline
        \tl{11} &        &         &         &         & 5098170 & 235620  &        &         & 5333790\\ \hline
        \tl{12} &        &         &         &         &         & 2269806 & 27720  &         & 2297526\\ \hline
        \tl{13} &        &         &         &         &         &         & 582582 &         & 582582\\ \hline
        \tl{14} &        &         &         &         &         &         &        & 66066   & 66066\\ \hline
    \end{tabular}

\vspace{1cm}
    
    \begin{tabular}{r||r|r|r|r|r|r|r|r||r|}
        \cline{2-9}
        \multicolumn{1}{r|}{} & \multicolumn{8}{c|}{$E^1_{p,l}$ for $\mathbb F_2$} \\ \hline
        \tl{\diagbox[height=1.7em, width=3em]{$p$}{$l$}} & 1 & 2 & 3 & 4 & 5 & 6& 7 & 8 & $\dim$ \\ \hline\hline
        \tl 3   & 0    &      &     &        &         &         &        &         & 0\\ \hline
        \tl 4   & 3    & 0    &     &        &         &         &        &         & 3\\ \hline
        \tl 5   & 55   & 6    & 0   &        &         &         &        &         & 61\\ \hline
        \tl 6   & 1426 & 170  & 3   & 0      &         &         &        &         & 1599\\ \hline
    \end{tabular}

\vspace{1cm}
    
    \begin{tabular}{r||r|r|r|r|r|r|r|r||r|}
        \cline{2-9}
        \multicolumn{1}{r|}{} & \multicolumn{8}{c|}{$E^2_{p,l}$ for $\mathbb F_2$} \\ \hline
        \tl{\diagbox[height=1.7em, width=3em]{$p$}{$l$}} & 1 & 2 & 3 & 4 & 5 & 6& 7 & 8 & $\dimrf{2}{4}{14}$ \\ \hline\hline
        \tl 3   & 0    &      &     &        &         &         &        &         & 0\\ \hline
        \tl 4   & 1    & 0    &     &        &         &         &        &         & 1\\ \hline
        \tl 5   & 11   & 2    & 0   &        &         &         &        &         & 13\\ \hline
        \tl 6   & 19   & 9    & 1   & 0      &         &         &        &         & 29\\ \hline
    \end{tabular}
\end{center}

\subsubsection{Genus \texorpdfstring{$g=3$}{g=3} and Punctures \texorpdfstring{$m=1$}{m=1}}

\paragraph{The case $g=3$, $m=1$ and $h = 6$ with coefficients in $\mathbb F_2$:}
\begin{center}
    \begin{tabular}{r||r|r|r|r|r|r|r||r|}
        \cline{2-8}
        \multicolumn{1}{r|}{} & \multicolumn{7}{c|}{$E^0_{p,l}$ for $\mathbb F_2$} \\ \hline
        \tl{\diagbox[height=1.7em, width=3em]{$p$}{$l$}} & 1 & 2 & 3 & 4 & 5 & 6 & 7& $\dim$ \\ \hline\hline
        \tl 1   & 1     &       &       &       &       &       &     & 1\\ \hline
        \tl 2   & 82    & 1     &       &       &       &       &     & 83\\ \hline
        \tl 3   & 1212  & 91    &       &       &       &       &     & 1303\\ \hline
        \tl 4   & 7200  & 1652  & 9     &       &       &       &     & 8861\\ \hline
        \tl 5   & 20400 & 12890 & 500   &       &       &       &     & 33790\\ \hline
        \tl 6   & 23760 & 49380 & 7706  & 60    &       &       &     & 80906\\ \hline
        \tl 7   &       & 77927 & 48104 & 2310  &       &       &     & 128341\\ \hline
        \tl 8   &       &       & 111588& 25676 & 294   &       &     & 137558\\ \hline
        \tl 9   &       &       &       & 91384 & 7497  &       &     & 98881\\ \hline
        \tl{10} &       &       &       &       & 44850 & 945   &     & 45795\\ \hline
        \tl{11} &       &       &       &       &       & 12375 &     & 12375\\ \hline
        \tl{12} &       &       &       &       &       &       & 1485& 1485\\ \hline
    \end{tabular}

    \vspace{1cm}
    
        \begin{tabular}{r||r|r|r|r|r|r|r||r|}
        \cline{2-8}
        \multicolumn{1}{r|}{} & \multicolumn{7}{c|}{$E^1_{p,l}$ for $\mathbb F_2$} \\ \hline
        \tl{\diagbox[height=1.7em, width=3em]{$p$}{$l$}} & 1 & 2 & 3 & 4 & 5 & 6 & 7& $\dim$ \\ \hline\hline
        \tl 1   & 0     &       &       &       &       &       &     & 0\\ \hline
        \tl 2   & 1     & 0     &       &       &       &       &     & 1\\ \hline
        \tl 3   & 5     & 1     &       &       &       &       &     & 6\\ \hline
        \tl 4   & 12    & 1     & 0     &       &       &       &     & 13\\ \hline
        \tl 5   & 34    & 21    & 0     &       &       &       &     & 55\\ \hline
        \tl 6   & 9455  & 82    & 16    & 0     &       &       &     & 9553\\ \hline
        \tl 7   &       & 39933 & 69    & 5     &       &       &     & 40007\\ \hline
        \tl 8   &       &       & 70752 & 20    & 0     &       &     & 70772\\ \hline
        \tl 9   &       &       &       & 67973 & 0     &       &     & 67973\\ \hline
        \tl{10} &       &       &       &       & 37647 & 0     &     & 37647\\ \hline
        \tl{11} &       &       &       &       &       & 11430 &     & 11430\\ \hline
        \tl{12} &       &       &       &       &       &       & 1485& 1485\\ \hline
    \end{tabular}
    
    \vspace{1cm}
    
        \begin{tabular}{r||r|r|r|r|r|r|r||r|}
        \cline{2-8}
        \multicolumn{1}{r|}{} & \multicolumn{7}{c|}{$E^2_{p,l}$ for $\mathbb F_2$} \\ \hline
        \tl{\diagbox[height=1.7em, width=3em]{$p$}{$l$}} & 1 & 2 & 3 & 4 & 5 & 6 & 7& $\dimrf{3}{1}{12}$ \\ \hline\hline
        \tl 1   & 0     &       &       &       &       &       &     & 0\\ \hline
        \tl 2   & 1     & 0     &       &       &       &       &     & 1\\ \hline
        \tl 3   & 5     & 1     &       &       &       &       &     & 6\\ \hline
        \tl 4   & 7     & 1     & 0     &       &       &       &     & 8\\ \hline
        \tl 5   & 7     & 5     & 0     &       &       &       &     & 12\\ \hline
        \tl 6   & 7     & 6     & 1     & 0     &       &       &     & 14\\ \hline
        \tl 7   &       & 7     & 2     & 1     &       &       &     & 10\\ \hline
        \tl 8   &       &       & 3     & 2     & 0     &       &     & 5\\ \hline
        \tl 9   &       &       &       & 4     & 0     &       &     & 4\\ \hline
        \tl{10} &       &       &       &       & 3     & 0     &     & 3\\ \hline
        \tl{11} &       &       &       &       &       & 0     &     & 0\\ \hline
        \tl{12} &       &       &       &       &       &       & 1   & 1\\ \hline
    \end{tabular}
\end{center}


\nocite{Austern2003}
\nocite{Josuttis2012}
\nocite{Stroustrup2013}
