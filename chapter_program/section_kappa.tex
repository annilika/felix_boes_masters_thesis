\section{The Program Kappa}
\label{chapter_program:kappa}

The program {\bf kappa} mainly uses the previously introduced {\bf libhomology} (compare Section \ref{program:libhomology}) 
and the theory developed in Chapter \ref{cellular_models}
in order to determine the homology of the moduli spaces $\Modspc[1]$ and $\ModspcRad[1]$. 
Thus, it computes the cohomology of the parallel or radial Ehrenfried complex $\Ehrprog$,
filtered by cluster sizes.

Recall that the homology of $\Modspc[1]$ and $\ModspcRad[1]$ coincides for $m > 0$, see Proposition \ref{cellular_models:comparision_of_the_models:bundles_are_h_equiv}.
For computing the homology of $\Modspc[1]$ for fixed $g$ and $m$, 
it is more efficient to use the radial Ehrenfried complex rather than the parallel one, 
since its modules and thus differentials are much smaller, see Section \ref{complexity:number_of_mono_cells}.
However, for $m = 0$, we cannot use the radial model, so we also offer the computation of the homology of $\Modspc[1]$ via the parallel model.
Since we also determine the dimensions of the modules of the cluster spectral sequence, 
computation with the parallel model produces new homological information for $m > 0$.

A central class of our computer program is hence the class \progclass{ClusterSpectralSequence}, see Subsection \ref{chapter_program:kappa:css},
which stores the parallel or radial Ehrenfried complex filtered by cluster sizes.

Since the basis elements of $\Ehrprog$ are monotonous tuples of transpositions,
another important class of the program kappa is the class \progclass{Tuple} (see Subsection \ref{chapter_program:kappa:tuple}),
which represents such a basis element and offers many functions that are applied to it
during the generation of the \progclass{ClusterSpectralSequence}.

With these foundations, we can offer the tool \progname{compute\_css} (see Subsection \ref{chapter_program:kappa:compute_css}) 
for computing the first three pages of the cluster spectral sequences corresponding to the moduli spaces $\Modspc[1]$ and $\ModspcRad[1]$
and especially their homology.

We also provide the tool \progname{compute\_cache},
which computes the bases and the differentials of $\Ehrprog$ and stores them in files via serialization.
For most $g$ and $m$, 
the computation of both takes a lot of time,
and it is thus functional to have the opportunity to store the data of $\Ehrprog$ for later uses.

For example, \progname{compute\_cache} can be used to examine the structure of the Ehrenfried complex via the tools
\progname{compute\_statistics} and \progname{print\_basis}.
After \progname{compute\_cache} has been performed, one can call \progname{compute\_statistics}
to find out various properties of the Ehrenfried complex $\Ehrprog$
like the sizes of the differentials or their largest entry per column.
Alternatively, a call of \progname{print\_basis} outputs the basis of the Ehrenfried complex.

Since all these tools mentioned are organized in a similar way, 
we shall only describe the tool \progname{compute\_css} in detail, see Subsection \ref{chapter_program:kappa:compute_css}.
Thereafter, we illustrate the above mentioned classes \progclass{Tuple} (compare Subsection \ref{chapter_program:kappa:tuple}) and \progclass{ClusterSpectralSequence} (compare Subsection \ref{chapter_program:kappa:css}),
which describe the Ehrenfried complex.

\subsection{The Tool \progname{compute\_css}}
\label{chapter_program:kappa:compute_css}

The tool \progname{compute\_css}, which computes the homology of the moduli spaces $\Modspc[1]$ and $\ModspcRad[1]$
by deriving the second term of the corresponding cluster spectral sequence,
is the most important tool supported by the project {\bf kappa}.
In Subsubsection \ref{chapter_program:kappa:compute_css:usage}, we describe 
how one operates this tool, 
while in Subsubsection \ref{chapter_program:kappa:compute_css:implementation},
we explain its implementation. 

\subsubsection{Usage}
\label{chapter_program:kappa:compute_css:usage}

For the computation of the homology of the moduli space $\ModspcRad[1]$, 
one can use the command
\begin{lstlisting}
.\compute_css -g arg -m arg (-q | -s arg)
\end{lstlisting}
Thereby,
\begin{itemize}
\item the parameter \progname{g} is the genus of the moduli space,
\item the parameter \progname{m} is the number of punctures of the moduli space,
\item one can either use the parameter $q$ or the parameter $s$. 
      If $q$ is chosen -- without any argument -- homology with rational coefficients will be computed.
      If the parameter $s$ is set to some positive prime, we will compute homology with coefficients in $\Z/s\Z$.
\end{itemize}
As the output of this command, one obtains a description of the $E^0$, $E^1$ and $E^2$ term of the cluster spectral sequence associated with $\ModspcRad[1]$,
and can read off the homology from the $E^2$ page.  

Recall that, for $m > 0$, the homology of the moduli space $\Modspc[1]$ coincides with the homology of the moduli space $\Modspc[1]$ 
(compare Proposition \ref{cellular_models:comparision_of_the_models:bundles_are_h_equiv}).
For determining the homology of $\Modspc[1]$ for $m = 0$ 
or the dimensions of the modules in the cluster spectral sequence of $\Modspc[1]$, one can set the optional parameter
\begin{itemize}
 \item \progname{parallel}
\end{itemize}
to true.

For instance, the call
\begin{lstlisting}
 ./compute_css -g 1 -m 3 -s 2
\end{lstlisting}
computes the homolgy of the moduli space $\mathfrak{M}^\bullet_1(3, 1)$ with $\Z/2\Z$ coefficients, 
while the call
\begin{lstlisting}
 ./compute_css -g 1 -m 3 -q --parallel 1
\end{lstlisting}
determines the homolgy of the moduli space $\mathfrak{M}^1_{3, 1}$ with rational coefficients.

There are other optional parameters that improve the performance or handling of the program.
\begin{itemize}
\item The optional parameter \progname{t} is the number of threads 
      that are allowed to be used for parallelization,
      which is $1$ by default. 
\item In addition, one can use the optional parameter \progname{num\_remaining\_threads}
      to determine how exactly computations are parallelized. 
      For a detailed explanation, see \ref{diag_field_implementation}.
      The total number of threads used will then be 
      \[
        \text{\progname{num\_threads + num\_remaining\_threads,}}
      \]
      and we recommend to use one third of the total number of threads as remaining threads.
\item The optional parameters \progname{first\_diff} respectively \progname{last\_diff} are the minimal respectively maximal $p \in \N$ 
      for which the homology $H_p(\Modspc[1])$ is supposed to be computed,
      which are $0$ respectively $2h$ by default.
\end{itemize}

If the command \progname{help} is used or if the input is not valid,
instructions for use will be printed to the console.

During the computations, we offer intermediate results and progress bars as console output. 
When the computation of the homology is finished, 
the file 
\[
\text{\progname{compute\_homology\_(parameters)}}
\]
created by the program contains all the intermediate and final results.
Be aware that this means that calls with the same parameters produce files with the same names and hence old files are overwritten.
The intermediate results give the oportunity to abort the computation
and continue it some other time, 
using the parameters \progname{first\_diff} and \progname{last\_diff} to select certain homology groups.

\subsubsection{Implementation Details}
\label{chapter_program:kappa:compute_css:implementation}

The computation of the cluster spectral sequence and the homology starts in the main function of the file 
\begin{lstlisting}
main_compute_css.cpp
\end{lstlisting}
The input parameters described above are stored in the struct
\begin{lstlisting}
SessionConfig;
\end{lstlisting}
which also tests whether the given configuration of parameters is valid, 
outputting the correct usage to the console if not.
Apart from the data members corresponding to the input parameters, 
the struct \progclass{SessionConfig} contains the data member
\begin{lstlisting}
SignConvention sgn_conv;
\end{lstlisting}
it can set on its own in dependence on the parameters for the coefficients.
The \progclass{SignConvention} parameter can have three different values, 
which indicate which signs have to be respected in the computation of the differentials. 
This way, we avoid sign computations whenever we can.
We set the parameter to \progname{no\_signs} if we only compute the homology up to sign, 
which is the case if we use coefficients in $\Z/2\Z$. 
If we have different coefficients and \progname{m} $\geq 2$, the moduli space $\Modspc[1]$ is non-orientable and the sign convention is set to \progname{all\_signs}. 
Otherwise, it equals \progname{no\_orientation\_sign}. 

At the begin of the computation, the constructor of \progclass{ClusterSpectralSequenceT} is called with respect to the parameters provided above.
Thereby, the bases (see Subsubsection \ref{chapter_program:kappa:css:gen_bases}) are generated, with basis elements sorted by their cluster number.
In particular, printing the $E^0$-term to screen is readily done.

The main task is determining the first and second page of the cluster spectral sequence.
Recall Section \ref{css:section_matrix_version}, which discusses the matrix version of the cluster spectral sequence.
The $p\Th$ transposed transformation matrix of $\del_\E$ is a block matrix of the form
\[ 
    \begin{pmatrix}
        d^1 & d^0 \\
            & d^1   & d^0 \\
            &       & d^1   & d^0 \\
            &       &       &       & \ddots
    \end{pmatrix} \,,
\]
if we sort the basis elements by their number of clusters.
The modules of the $E^1$-term are given by
\[
    \ker(d^0) / \img(d^0)\,,
\]
whereas the modules of the $E^2$-term are given by
\[
    \ker(d^1|_{\ker(d^0)}) / \big( \img(d^0) + \img(d^1|_{\ker(d^0)}) \big) \,.
\]
Hence, it suffices to apply all row operations induced by the sub-matrices $d^0$ and proceed with the diagonalization process of
the parts of the (altered) sub-matrices $d^1$ that are in the kernel of the (diagonalized) sub-matrices $d^0$.
In order to save execution time and memory, our program operates on exactly one $d^0$ and one $d^1$ sub matrix at a time:

\begin{algorithm}[H]
\DontPrintSemicolon
\For{$p = 1$ \KwTo $2h$}
{
    \For{$l=1$ \KwTo $p$}
    {
        Construct $d^1_{p,l}$ and apply row operations of $d^0_{p,l-1}$
        
        Forget $d^0_{p,l-1}$
        
        Generate $d^0_{p,l}$
        
        Compute and save kernel and image of $d^0_{p,l}$
        
        Print the homological results to the screen
        
        Save the diagonal of $d^0_{p,l}$ and detect superflous rows of $d^1_{p,l}$
        
        Compute and save the kernel and image of $d^1_{p,l}$
        
        Print the homological results to the screen
        
        Forget $d^1_{p,l}$
    }
}
Print all three pages of the spectral sequence to the screen.
\caption{Computing $E^1$ and $E^2$}
\end{algorithm}

Here, we have $h = 2g + m$ in the parallel case and $h = 2g + m - 1$ in the radial case.

During all computations, the function \progname{compute\_css} generates the previously mentioned intermediate results 
printed out in the console and into the corresponding output file.
Furthermore, it measures the duration of the important steps of the computations 
and also writes them into the console. 
\subsection{The Class Tuple [H]}
\label{chapter_program:kappa:tuple}

Recall that we aim to compute the cohomology of the parallel or radial Ehrenfried complex $\Ehrprog$.
A basis for the Ehrenfried complex is given by monotonous cells $\Sigma = (\tau_h \mid \ldots \mid \tau_1)$ satisfiying certain conditions,
and the differential for $\Ehrprog$ can be described by the map $\del_\E$ making the diagram
\[
    \begin{tikzcd}
	\E_p \arrow{r}{\del_\E} \arrow{d}{\kappa}[swap]{\cong}      & \E_{p-1} \\
	\KK_p \arrow{r}{\del_\KK}                                     & \KK_{p-1} \arrow{u}{\cong}[swap]{\pi}
    \end{tikzcd}
\]
commute, compare Definition \ref{cellular_models:Ehrenfried:defi}.
Hence, the core of the program \textbf{kappa} is the class \progclass{Tuple}, 
which represents a tuple of $h$ transpositions $\Sigma = (\tau_h \mid \ldots \mid \tau_1)$
and thus especially the basis elements of $\Ehrprog$.
This class additionally provides several methods which are applied to a basis element during the computation of the differential $\del_\E$.

We are going to give an overview on its data members (Subsubsection \ref{program:kappa:tuple:members}),
the member functions needed to start working with a \progclass{Tuple} (Subsubsection \ref{program:kappa:tuple:get_started}) and 
the functions that represent basic properties of a tuple (Subsubsection \ref{program:kappa:tuple:basics}). 
Afterwards, we explain the class methods computing the orientation sign (Subsubsection \ref{program:kappa:tuple:orientation_sign}), 
the horizontal differential (Subsubsection \ref{program:kappa:tuple:d_hor}) and the implementation of the maps $f$ and $\Phi$ used to compute the isomorphism $\kappa$ (Subsubsection \ref{program:kappa:tuple:prep_for_kappa}) in detail. 

\subsubsection{Data Members [H]}
\label{program:kappa:tuple:members}
Since the class \progclass{Tuple} is supposed to represent cells $\Sigma = (\tau_h \mid \ldots \mid \tau_1)$ of the Ehrenfried complex $\Ehrprog$, 
let us briefly recall what this means. 
If $\Sigma$ is a cell of the parallel Ehrenfried complex 
(compare Definitions \ref{cellular_models:parallel:inhomogeneous_notation}, \ref{cellular_models:ehrenfried}), 
it satisfies the following properties:
\begin{enumerate}
 \item[(i$_P$)] The transpositions $\tau_i$ act on the symbols $1, \dotsc, p$.
 \item[(ii$_P$)] The permutation $\sigma_h$ has exactly $m+1$ cycles.
 \setcounter{enumi}{2}
 \item Each symbol $1, \dotsc, p$ is contained in at least one $\tau_i$.
 \item $\Sigma$ is monotonous, i.e. $\height(\tau_h) \geq \dotsc \geq \height(\tau_1)$.
\end{enumerate}
On the other hand, if $\Sigma$ is a radial cell, it fulfills the same conditions, 
except for the first two, which are replaced by
\begin{enumerate}
 \item[(i$_R$)] The transpositions $\tau_i$ act on the symbols $0, \dotsc, p$,
 \item[(ii$_R$)] The permutation $\sigma_h$ has exactly $m$ cycles.
\end{enumerate}
see also Definitions \ref{cellular_models:radial:cells_in_inhomogenous_notation} and \ref{cellular_models:Ehrenfried:defi}.

Let us now see how the cell $\Sigma$ is represented by a \progclass{Tuple}.
The transpositions $\tau_h, \dotsc, \tau_1$ belonging to $\Sigma$ are stored as the data member
\begin{lstlisting}
std::vector< Transposition > rep;
\end{lstlisting}
where a \progclass{Transposition} is defined as a pair of unsigned integers.
The unsigned integer \progname{p} is stored as another data member of the class \progclass{Tuple},
and there is also a boolean \progname{radial} indicating whether $\Sigma$ is a radial cell or not (and thus a parallel cell).
Depending on this flag, there are other technical parameters to be set, 
e.g. the minimum symbol that may occur in a transposition, see conditions (i$_P$) and (i$_R$).

Our convention for writing a cell $\Sigma = (\tau_h \mid \ldots \mid \tau_1)$ is 
to store $\tau_i$ as \progname{rep[i-1]} for $1 \leq i \leq h$, since by default, a vector starts with the index $0$. 
To make the handling of \progclass{Tuples} more intuitive, 
we offer methods to access the \progclass{Transpositions} in a more canonical way, see Subsubsection \ref{program:kappa:tuple:get_started}. 
Another convention is that we always write \progclass{Transpositions} like $\tau_i = (a\ b)$ with $a > b$ in order to simplify the source code.

Since it is useful for the computation of the differential, 
another data member of the class \progclass{Tuple} is an unsigned integer \progname{id} indicating the index of a \progclass{Tuple} in its basis of the Ehrenfried complex.

\subsubsection{Class Methods To Get Started [H]}
\label{program:kappa:tuple:get_started}

We define two constructors for the class \progclass{Tuple}. 
Firstly, there is a constructor
\begin{lstlisting}
Tuple( size_t h );
\end{lstlisting}
which initializes \progname{p} with $0$ and allocates memory for a \progclass{Tuple} of $h$ \progclass{Transpositions} 
that are supposed to be filled later. 
Secondly, the constructor
\begin{lstlisting}
Tuple( uint32_t symbols, size_t h );
\end{lstlisting}
sets the data member \progmember{p} to be $symbols$ and also allocates memory for the $h$ many \progclass{Transpositions}. 

To create a \progclass{Tuple} $\Sigma = (\tau_h \mid \ldots \mid \tau_1)$, one can call one of these constructors, 
and afterwards initialize for each $i = 1, \dotsc, h$ the $i\Th$ \progclass{Transposition} $\tau_i$ using the non-const operator 
\begin{lstlisting}
Transposition& operator[]( size_t i );
\end{lstlisting}
It is also possible to use the const respectively non-const version of the method
\begin{lstlisting}
Transposition& at( size_t i );
\end{lstlisting}
to access the \progclass{Transposition} $\tau_i$ of the tuple $\Sigma$ for reading respectively writing. 

During the generation of the differential $\del_\E$, we use to mark \progclass{Tuples} as degenerate by
erasing its data member \progname{rep}.
To test whether a \progclass{Tuple} is non-degenerate, we hence define  
\begin{lstlisting}
operator bool();
\end{lstlisting}
which returns true if and only if this \progclass{Tuple} is non-empty.

Using the methods
\begin{lstlisting}
static void parallel_cell(); 
\end{lstlisting}
respectively
\begin{lstlisting}
static void radial_cell();
\end{lstlisting}
one can mark whether a \progclass{Tuple} represents a parallel respectively radial cell.

Furthermore, we define canonical compare operators, 
overload the \progname{operator<<} to print \progname{Tuples} to screen
and offer the possibility to save and load \progclass{Tuples}.

\subsubsection{Basic Properties of a Tuple [H]}\label{program:kappa:tuple:basics}

We provide various class methods for basic calculations with a \progclass{Tuple},
e.g. to verify whether a \progclass{Tuple} represents an element of the Ehrenfried complex. 

Since basis elements of $\Ehrprog$ are required to be monotonous, we provide a method
\begin{lstlisting}
bool monotonous();
\end{lstlisting}
to test whether a \progclass{Tuple} is monotonous, i.\,e. whether, for all $1 \leq i < h$, we have $a_i < a_{i+1}$, 
writing $\tau_i = (a_i, b_i)$ and $\tau_{i+1} = (a_{i+1}, b_{i+1})$ with $a_i > b_i$ and $a_{i+1} > b_{i+1}$.

The class method
\begin{lstlisting}
int32_t norm();
\end{lstlisting}
returns the norm of the given \progclass{Tuple} $\Sigma = (\tau_h \mid \ldots \mid \tau_1)$, 
i.\,e. the sum of the norms of all $\tau_i$, $i = 1, \dotsc, h$. 
But since each $\tau_n$ is a \progclass{Transposition}, the norm of the \progclass{Tuple} is simply the number of its transpositions $h$. 

For the computation of the differential, we sometimes want to switch to the homogeneous notation of \progclass{Tuples}. 
Therefore, we need methods
\begin{lstlisting}
Permutation long_cycle(); 
\end{lstlisting}
respectively
\begin{lstlisting}
Permutation long_cycle_inv();
\end{lstlisting}
returning the \progclass{Permutation} $\sigma_0 = (1\ 2\ \dotsb p-1\ p)$ respectively its inverse, and
\begin{lstlisting}
Permutation sigma_h();
\end{lstlisting}
returning the \progclass{Permutation} $\sigma_h = \tau_h \tau_{h-1} \dotsc \tau_1 \sigma_0$.
Thereby, a \progclass{Permutation} is another class, representing a permutation as a vector
storing for each element its image under the permutation.

Since all $\tau_i$ are \progclass{Transpositions}, 
we can simplify the computation of $\sigma_h$ using the following

\begin{algorithm}[H]
\label{sigma_q}
\DontPrintSemicolon

\KwIn{A tuple $\Sigma = (\tau_h, \dotsc, \tau_1)$ in inhomogeneous notation}
\KwOut{The permutation $\sigma_h$}

\progclass{Permutation} $\sigma_{\text{inv}} := \text{long\_cycle\_inv()}$\;
\For{$i = 1$ \KwTo $h$}
{
	Write $\tau_i = (a, b)$\;
	Swap the elements $\sigma_{\text{inv}}(a)$ and $\sigma_{\text{inv}}(b)$\;
}

\For{$j = 1$ \KwTo $p$}
{
	Write $k = \sigma_{\text{inv}}(j)$\;
	$\sigma_h(k) := j$\;
}

\KwRet{$\sigma_h$}\;

\caption{Computing $\sigma_h$}

\end{algorithm}

\begin{prop}
The given algorithm computes $\sigma_q$ correctly.
\begin{proof}
In line 1, we initialize the \progclass{Permutation} $\sigma_{\text{inv}}$ with $\sigma_0^{-1}$. 
Note that, by definition of $\sigma_i$, we have
\[ \sigma_i^{-1} = \sigma_{i-1}^{-1} \tau_i = \sigma_{i-1}^{-1}  (a, b) \]
for $0 < i \leq h$. 
Thus, the \progclass{Permutation} $\sigma_i^{-1}$ maps $a$ to $\sigma_{i-1}^{-1}(b)$, $b$ to $\sigma_{i-1}^{-1}(a)$ 
and behaves like $\sigma_{i-1}^{-1}$ on all other elements. 
Hence, for each $i = 1, \dotsc, h$, we have $\sigma_{\text{inv}} = \sigma_i^{-1}$ 
after the $i\Th$ iteration of the first for loop. 

The second for loop computes the inverse of $\sigma_{\text{inv}} = \sigma_h^{-1}$, which is $\sigma_h$.
\end{proof}
\end{prop}

The algorithm for the computation of $\sigma_h$ is also used with small adaptions in different parts of the program.

The basis elements of $\Ehrprog$ are supposed to have a distinguished number of punctures, 
i.e. number of cycles of $\sigma_h$ is $m+1$ for parallel and $m$ for radial cells.
Therefore, we give an algorithm to determine the cycle decomposition of a \progclass{Permutation}.

\begin{algorithm}[H]
\label{Cycle Decomp}
\DontPrintSemicolon
\SetKw{KwGoTo}{go to}

\KwIn{A permutation $\pi$ on a subset of $\{0, \dotsc, p\}$}
\KwOut{A decomposition of $\pi$ into disjoint cycles}

$i := \min\{j \in \{0, \dotsc, p\} \colon j \text{ belongs to $\pi$, but not visited yet}\}$ \label{next cycle}\;
cur := $i$\; 
Initialize a new cycle\;
\Repeat(\tcp*[f]{Find the cycle containing $i$}){ cur equals $i$}
{	
  Mark cur as visited\;
	prev := cur\;
	cur := $\pi$(prev)\;
	cycle (prev) := cur\;
}
Store the cycle\;
\If{all symbols in $\{0, \dotsc, p\}$ visited or not belonging to $\pi$}
{
  \KwRet the cycle decomposition\; 
}
\KwGoTo \ref{next cycle}\;

\caption{Cycle Decomposition}

\end{algorithm}

Since each element in $\{0, \dotsc, p\}$ belonging to the permutation $\pi$ is considered as $prev$ exactly once, we can state

\begin{prop}
The algorithm to determine the cycle decomposition of a permutation works correctly.
\end{prop}

Combining this with the previous Algorithm \ref{sigma_q}, we can now define the methods
\begin{lstlisting}
uint32_t num_cycles() const;
\end{lstlisting}
yielding the number of cycles of $\sigma_h$, and
\begin{lstlisting}
bool has_correct_num_cycles(size_t m) const;
\end{lstlisting}
checking whether the number of cycles fits the requirement of the parallel respectively radial Ehrenfried complex.

Since we need to subdivide the cells of $\Ehrprog$ according to their numbers of clusters, 
we introduce the method
\begin{lstlisting}
int32_t Tuple::num_clusters() const;
\end{lstlisting}
This is another part of our computer program where we use the comfortability of the \textbf{boost} library:
Let $\Sigma = (\tau_h \mid \dotsc \mid \tau_1)$ be a cell represented by a \progclass{Tuple}.
We construct a graph on the vertices $0, \dotsc, p$,
where an edge between $a$ and $b$ indicates that there is an $i$ such that $\tau_i = (a, b)$.
Then, the number of connected components of this graph equals the number of clusters of $\Sigma$,
and \textbf{boost} offers a graph data structure and an algorithm to compute this directly.

\subsubsection{The Horizontal Face Operator [H]}
\label{program:kappa:tuple:d_hor}

In order to construct the Ehrenfried complex $\Ehrprog$, 
the computer program has to be able to apply the horizontal differential $\del'' = \del_\KK$ to \progclass{Tuples}, 
compare Definition \ref{cellular_models:Ehrenfried:defi} and Section \ref{cellular_models:orientation}.
So recall that, for a cell $\Sigma$, this differential is given by the alternating sum
\[
 \del''_j(\Sigma) = \sum_{j = 0}^p (-1)^j \eps_j(\Sigma) d_j''(\Sigma)\,,
\]
where $d_j''(\Sigma)$ is the $j\Th$ horizontal face of $\Sigma$, 
i.e. the cell resulting from $\Sigma$ by collapsing the $j\Th$ stripe of the slit domain,
and $\eps_j(\Sigma) = \eps_j''(\Sigma)$ is the additional sign introduced by the orientation system.
Note that, for parallel cells -- but not for radial cells --, the $0\Th$ and $p\Th$ horizontal face is always degenerate.

Here, we will explain our implementation of the face operator $d''$, 
and in Subsubsection \ref{program:kappa:tuple:orientation_sign}, we will present the orientation sign.

In Proposition \ref{cellular_models:parallel:prop_dh}, 
we saw how to express the formula for the $j\Th$ horizontal face in the inhomogeneous notation,
and in Corollary \ref{cellular_models:ehrenfried:cor_d_hor_deg}, we saw how to detect the cases when the resulting face is degenerate.
These statements result in the following

\begin{algorithm}[H]
\label{d_hor}
\DontPrintSemicolon
\SetKw{KwGoTo}{go to}

\KwIn{A tuple $\Sigma = (\tau_h, \dotsc, \tau_1)$ in inhomogeneous notation, a symbol $j \in \{0, \dotsc, p\}$}
\KwOut{The horizontal face $d''_j(\Sigma)$ or the assertion that $d''_j(\Sigma)$ is degenerate}
\For{$i = 1$ \KwTo $h$}
{
	Write $\tau_i = (a, b)$\;
	Write $k = \sigma_{i-1}(j)$\;
	\If{The transpositions $\tau_i$ and $(j, k)$ are disjoint}
	{
	  $\tau'_i := \tau_i$\;
	}
	\ElseIf{$\tau_q = (j, k)$ or $j = k$}
	{
	  \KwRet{$d''_j(\tau)$ is degenerate}\;
	}
	\Else
	{
	  \If{$k = a$ or $k = b$}
	  {
	    $\tau'_i := \tau_i$\;
	  }
	  \Else
	  {
	    \If{$a \neq k$}
	    {
	      $\tau_i' := (a, k)$\; 
	    }
	    \Else
	    {
	      $\tau_i' := (b, k)$\;
	    }
	  }
	}
}
Renormalize $\Sigma' = (\tau_h' \mid \ldots \mid \tau_1')$\;
\KwRet{$\Sigma'$}\;

\caption{Computing the Horizontal Face}

\end{algorithm}

Using the above mentioned theoretical foundations, we obtain

\begin{prop}
The above algorithm computes the $j\Th$ horizontal face of $\Sigma$.
\end{prop}

Note that we can apply Algorithm \ref{sigma_q} to compute $\sigma_i$ for $i = 0, \dotsc, h$.
We choose to handle the case that $\tau_i$ and $(j, k)$ are disjoint before the degenerate case at first
because this is the case that will occure most likely.

The algorithm enables us to define the method
\begin{lstlisting}
Tuple d_hor( uint8_t j );
\end{lstlisting}
computing the $j\Th$ horizontal face of this \progclass{Tuple}.

\subsubsection{The Orientation Sign [H]}\label{program:kappa:tuple:orientation_sign}

Recalling the definition of the orientation sign $\eps_j(\Sigma) = \eps_j''(\Sigma)$ introduced by Mehner (see Section \ref{cellular_models:orientation}), 
we immediately obtain the following

\begin{algorithm}[H]
\label{orientation_sign}
\DontPrintSemicolon
\SetKw{KwGoTo}{go to}

\KwIn{A tuple $\Sigma = (\tau_h, \dotsc, \tau_1)$ in inhomogeneous notation}
\KwOut{Orientation signs $\eps_j(\Sigma)$ for all $j \in \{0, \dotsc, p\}$}

Decompose $\sigma_h$ into $m$ resp. $m+1$ disjoint cycles $(\alpha_0) \alpha_1 \dotsc \alpha_m$ \;
Let $a_i$ be the minimum symbol of the cycle $\alpha_i$\;
Sort the cycles $\alpha_i$ such that $a_i < a_{i+1}$ for all $i$\;
\ForEach{cycle $\alpha_i$}
{
  \If{$\alpha_i = a_i$ is a fixed point}
  {
    Set $\eps_{a_i} = 0$\;
  }
  Let $b$ be the second minimum cycle of $\alpha_i$\;
  Let $k \geq i$ be the minimum integer with $b < a_{k+1}$, or $k = m$, if this does not exist\;
  Set $\eps_{a_i} = (-1)^{k - i}$\;
  \ForEach{$c \in \alpha_i$, $c \neq a_i$}
  {
    Set $\eps_c = 1$\;
  }
}

\caption{Computing the Orientation Sign}

\end{algorithm}

Hereby, we again use Algorithm \ref{sigma_q} to determine $\sigma_h$, 
and Algorithm \ref{Cycle Decomp} to decompose $\sigma_h$ into disjoint cycles.
Note that, in the parallel case, a cell of $\Ehrprog$ has $m+1$ cycles, while in the radial case, it has $m$ cycles.

Storing the cycle decomposition of $\sigma_h$ as a map of cycles stored with their smallest element as a key,
this algorithm is easily implemented.
We obtain the method
\begin{lstlisting}
std::map< uint8_t, int8_t > orientation_sign();
\end{lstlisting}
returning a map of all orientation signs $\eps_j(\sigma_h)$, stored with $j \in \{1, \dotsc, p\}$ as a key.

\subsubsection{Preparations for the Map \texorpdfstring{$\kappa$}{kappa} [H]}
\label{program:kappa:tuple:prep_for_kappa}

Having seen how the differential $\del_\KK$ is implemented, 
recall once more that the differential $\del_\E$ of the Ehrenfried complex is given by the diagram
\[
    \begin{tikzcd}
	\E_p \arrow{r}{\del_\E} \arrow{d}{\kappa}[swap]{\cong}      & \E_{p-1} \\
	\KK_p \arrow{r}{\del_\KK}                                     & \KK_{p-1} \arrow{u}{\cong}[swap]{\pi}
    \end{tikzcd}\,.
\]
Note that the projection $\pi$ onto the monotonous cells can be performed by 
simply checking whether the \progclass{Tuple} is monotonous (see Subsection \ref{program:kappa:tuple:basics})
and marking the \progclass{Tuple} as non-valid, if not.
It remains to define the isomorphism $\kappa$, which is given by
\[
    \kappa = K_h \circ \ldots \circ K_1
\]
with
\[
    K_q = \sum_{j=1}^q (-1)^{q-j} \Phi_{j}^q
\]
and
\[
    \Phi_j^q = f_j \circ \ldots f_{q-1} \,.
\]

Therefore, the class \progclass{Tuple} provides maps
\begin{lstlisting}
bool f( uint32_t j );
\end{lstlisting}
and 
\begin{lstlisting}
bool phi( uint32_t q, uint32_t j );
\end{lstlisting}
which apply the maps $f_j$ respectively $\Phi_j^q$ to this \progclass{Tuple} 
and return true if and only if the resulting \progclass{Tuple} is non-degenerate. 

The method $f$ is implemented as a huge case distinction 
concerning the symbols contained in the transpositions $\tau_j$,
which is similar to the one in the computation of the horizontal boundary. 
This provides the opportunity to handle each of the cases in constant time.

The function $\phi(q, j)$ iteratively calls the function $f$ for $j = 1, \dotsc, q-1$
according to the definition of $\Phi_j^q$.
In each step, we test whether the norm of the \progclass{Tuple} decreases, and if so, 
we abort the computation of $\Phi$ to avoid unneccassary computations.

The map $\kappa$ itself is defined in the class \progclass{ClusterSpectralSequence}, compare Subsection \ref{chapter_program:kappa:css}.
\subsection{The Class ClusterSpectralSequence}
\label{chapter_program:kappa:css}

Having described how basis elements of the Ehrenfried complex $\Ehrprog$ are represented by our computer program, 
we will now explain how the cluster spectral sequence associated with $\Ehrprog$ is realized 
and how the methods provided by the class \progclass{Tuple} (compare Subsection \ref{chapter_program:kappa:tuple}) are combined 
to compute its differentials.
For these purposes, we introduce the class \progclass{ClusterSpectralSequence}, 
which is a template class with the class type of the underlying \progclass{ChainComplex} as a template parameter
(compare Subsection \ref{program:libhomology:ChainComplex}).

At first, we are going to describe the class \progclass{CSSBasis} in Subsection \ref{chapter_program:kappa:css:css_basis}, 
which represents the basis of a single module of the cluster spectral sequence. 
Next, we describe the data members of the \progclass{ClusterSpectralSequence} (Subsection \ref{chapter_program:kappa:css:data_members}),
which are most importantly a collection of bases of the modules of the Ehrenfried complex and 
a collection of its differentials.
Thereafter, we explain how the bases (see Subsubsection \ref{chapter_program:kappa:css:gen_bases}) and 
the differentials (see Subsubsection \ref{chapter_program:kappa:css:gen_diff}) are generated, 
and what other methods are provided by the class \progclass{ClusterSpectralSequence} (see Subsubsection \ref{chapter_program:kappa:css:more_methods}). 

\subsubsection{CSSBasis}
\label{chapter_program:kappa:css:css_basis}

Just like the cluster spectral sequence associated with $\Ehrprog$ consists of a finite number of modules, 
our class \progclass{ClusterSpectralSequence} contains a finite number of \progclass{CSSBases}, 
which are structs representing the bases of these modules.

Thus the only data member of the struct \progclass{CSSBasis} is the basis
\begin{lstlisting}
BasisType basis;
\end{lstlisting}
Thereby, \progclass{BasisType} is a map storing all \progclass{Tuples} of this basis, sorted by cluster sizes.
For each cluster size $l$, we organize the corresponding \progclass{Tuples} in an \progname{std::unordered\_set}
with an appropriate hash function that makes it possible to search for basis elements in amortized constant running time.

The struct \progclass{CSSBasis} provides the usual methods for saving and loading as well as the functions
\begin{lstlisting}
uint64_t size( int32_t l ) const;
\end{lstlisting}
returning the number of basis elements with exactly $l$ clusters and 
\begin{lstlisting}
uint64_t total_size() const;
\end{lstlisting}
returning the total number of basis elements.

Since the computation of the differential (compare Subsubsection \ref{chapter_program:kappa:css:gen_diff}) requires 
a unique identification of one basis element among the other basis elements of the same cluster size of one \progclass{CSSBasis}, 
there is a method
\begin{lstlisting}
int64_t id_of( Tuple& t );
\end{lstlisting} 
returning the unique \progname{id} of the given \progclass{Tuple} \progname{t}. 
If \progname{t} is not an element of this \progclass{CSSBasis}, we return \progname{-1} to indicate the failure of the function.

The most important method of the struct \progclass{CSSBasis} is the function
\begin{lstlisting}
uint32_t add_basis_element ( Tuple& t );
\end{lstlisting}

Using the method \progname{num\_clusters} of the class \progclass{Tuple} (see Subsubsection \ref{program:kappa:tuple:basics}),
it inserts the \progclass{Tuple} \progname{t} into the part of the basis corresponding to its number of clusters
and sets the \progname{id} of \progname{t} to the current number of basis elements with exactly $l$ clusters. 
This means that if one builds up a \progclass{MonoBasis} by successively adding basis elements,
all basis elements can be distinguished by their \progname{ids}.

\subsubsection{Data Members}
\label{chapter_program:kappa:css:data_members}

The class \progclass{ClusterSpectralSequence} represents the cluster spectral sequence assoicated with the Ehrenfried complex.
Hence, it contains a collection of \progclass{CSSBases} 
\begin{lstlisting}
std::map< uint32_t, CSSBasis > basis_complex;
\end{lstlisting}
where for each $0 \leq p \leq 2h$, the basis elements of the Ehrenfried complex on the symbols $0, \dotsc, p$ are stored,
and the data member
\begin{lstlisting}
MatrixComplex diff_complex;
\end{lstlisting}
where at each time, the differential needed for computations is stored.

Note that \progclass{MatrixComplex} is a template parameter. 
Depending on the coffficients of the homology one wants to compute,
it can be chosen to be \progclass{ClusterSpectralSequenceQ} or \progclass{ClusterSpectralSequenceZm}.
For coefficients in the field $\mathbb F_2$, we highly recommend to use \progclass{ClusterSpectralSequenceBool} for efficiency reasons.
The genus \progmember{g}, the number of punctures \progmember{m} and the number \progmember{h} of transpositions in a basis tuple 
associated with this \progclass{MonoComplex} are also stored as data members. 
Furthermore, the data member
\begin{lstlisting}
SignConvention sign_conv;
\end{lstlisting}
indicates which sign convention (see also Subsubsection \ref{chapter_program:kappa:compute_css:implementation}) is used to compute the homology of this \progclass{MonoComplex},
and the data member
\begin{lstlisting}
size_t num_threads;
\end{lstlisting}
determines the number of threads using for the construction of the differentials.

\subsubsection{Generating Bases}
\label{chapter_program:kappa:css:gen_bases}

The first step to build up a \progclass{ClusterSpectralSequence} is to call the constructor
\begin{lstlisting}
ClusterSpectralSequence( uint32_t genus, 
                         uint32_t num_punctures, 
                         SignConvention sgn, 
                         uint32_t num_working_threads, 
                         uint32_t num_remaining_threads );
\end{lstlisting}
For an explanation of the parameters \progname{num\_working\_threads} and \progname{num\_remaining\_threads}, 
see Subsubsection \ref{diag_field_implementation}.
In the constructor, \progclass{Diagonalizer} is configured (compare Subsection \ref{program:libhomology:DiagonalizerT}), 
and the bases of all the modules belonging to the \progclass{MonoComplex} are initialized 
via a recursive method we want to explain now. 

Our aim is to enumerate all cells $\Sigma = (\tau_h \mid \ldots \mid \tau_1)$ 
of bidegree $(p, h)$ for all $0 \leq p \leq 2h$ such that
\begin{enumerate}
 \item All $\tau_i$ are non-trivial transpositions on the symbols $min, \dotsc, p$.
 \item Each symbol $min, \dotsc, p$ is permuted non-trivially by at least one $\tau_i$.
 \item $\Sigma$ is monotonous.
\end{enumerate}
Hereby, in the parallel case, $min$ equals $1$, and in the radial case, we want to enumerate all cells for $min = 0$ or $min = 1$.

The enumeration of all these cells works recursively.
Let $\Sigma = (\tau_k \mid \ldots \mid \tau_1)$ be a cell fulfilling the above conditions, 
but with bidegree $(p, k)$ with $k < h$.
Assume we want to find all possibilities to extend such a cell $\Sigma$ to a monotonous tuple $\Sigma'$ of $k+1$ transpositions
by inserting a transposition $\tau_{k + 1}$, perharps using more symbols. 
The following cases occur.

\begin{enumerate}
\item We also use the symbols $min, \dotsc, p$ for $\tau_{k+1}$.
      Since $\Sigma'$ is supposed to be monotonous, 
      $\tau_{k + 1}$ needs to contain the symbol $p$.
      Hence we can set $\tau_{k + 1} := (p, i)$ with $i \in \{min, \dotsc, p-1\}$ chosen arbitrarily.
\item We insert a new row into our parallel slit domain and use the symbols $min, \dotsc, p + 1$ for $\Sigma'$.
      By monotony, $\tau_{k + 1}$ has to contain the highest symbol $p + 1$. 
      But now there are two possibilities:
      \begin{enumerate}
      \item The new row is inserted as a $(p+1)\Th$ row above the old rows. 
            Hence the symbol that is contained in $\tau_{k + 1}$ apart from $p + 1$
            is one of the symbols that were already used for the transpositions $\tau_1, \dotsc, \tau_k$, 
            and we can set $\tau_{k + 1} := (p + 1, i)$ with $i \in \{min, \dotsc, p\}$ chosen arbitrarily.
      \item The new row is inserted as the $i\Th$ row for some $i \in \{min, \dotsc, p\}$  
            and the indices of the old rows $min, \dotsc, p$ are shifted up by one.
            Note that these indices are also shifted up in the transpositions of $\Sigma$.
            By monotony and since all symbols in $\{min, \dotsc, p\}$ have to be covered, 
            the new transposition has to be $\tau_{k + 1} := (p + 1, i)$, 
            meaning that the symbol $p + 1$ is also used by at least one of the transpositions $\tau_1, \dotsc, \tau_k$, 
            and that the symbol $i$ is used by $\tau_{k + 1}$ only.
      \end{enumerate}
\item We insert two rows into our slit domain, using the symbols $\{min, \dotsc, p + 2\}$. 
      Thus the symbols used by the new transposition $\tau_{k + 1}$ do not yet appear in $\Sigma$.
      Therefore $\tau_{k + 1}$ has to contain the symbol $p + 2$ and some other symbol $i \in \{min, \dotsc, p + 1\}$.
      Again, the indices of the rows $i + 1, \dotsc, p$ have to be shifted up by $1$ in the transpositions of $\Sigma$.
\end{enumerate}

We can use these observations to define the recursive method
\begin{lstlisting}
void gen_bases( uint32_t k, uint32_t p, uint32_t min, Tuple& tuple );
\end{lstlisting}
This methods gets as input data a monotonous \progclass{Tuple} consisting of $k$ transpositions 
which contain each of the symbols $min, \dotsc, p$ at least once,
and the minimum symbol the transpositions may contain.
Using the above case distinction, it calls itself recursively with the appropriate parameters
and stores all the transpositions with $h$ transpositions detected thereby in the corresponding basis,
but only if they contain the required number of cycles.

Now we can describe the way the constructor of the \progclass{CSS} sets up its \progmember{basis\_complex}.

\begin{prop}
We can enumerate all basis elements of the parallel Ehrenfried complex
by defining the \progclass{Tuple} $\Sigma = ((2, 1))$ consisting of only one \progclass{Transposition},
and then calling the recursive function \progname{gen\_bases(1, 2, 1, $\Sigma$)}. 

For enumerating all basis elements of the radial Ehrenfried complex,
we additionally need to call the recursive function \progname{gen\_bases(1, 1, 0, $\Sigma'$)} with $\Sigma' = ((1, 0))$.
\begin{proof}

Note that for a given $\Sigma$, the tranpositions arising from the different cases fulfill the above conditions (i) - (iii),
and that they are all distinct.
If we consider two different \progclass{Tuples}, 
the transpositions arising from the case distinction also cannot coincide
since either they have different numbers of transpositions or
their starting sequence of transpositions already differs.
Especially, we don't enumerate \progclass{Tuples} multiple times by the two initial calls in the radial case
since the first family of \progclass{Tuples} does not contain the symbol $0$, and the second family does.

Hence it suffices to show that all monotonous transpositions can be found by our algorithm.
By induction on the number of transpositions, 
we can assume that we have already built up all monotonous tuples with $k$ transpositions.
Let $\Sigma' = (\tau_{k + 1}\mid \dotsc\mid \tau_1)$ be monotonous. 
Then $\tau_{k + 1} = (p+1, i)$ with $i \in \{1, \dotsc, p\}$ by monotony.
The tuple $\Sigma := (\tau_k \mid \dotsc \mid \tau_1)$ is also monotonous, 
but we might have to shift the indices $i, \dotsc, p+1$ down by one
if the symbol $i$ is not contained in the transpositions $\tau_1, \dotsc, \tau_k$.
This yields one of the tuples of $k$ transpositions we have already found.
By reverting the process just described, we can rebuild $\Sigma'$ from $\Sigma$, 
and this case is covered by the above case distinction.
\end{proof}
\end{prop}

\subsubsection{Generating Differentials}
\label{chapter_program:kappa:css:gen_diff}

In Subsubsection \ref{chapter_program:kappa:compute_css:implementation}, we explained how both the $E^1$-term and $E^2$-term are computed.
Recalling Chapter \ref{cellular_models}, the $p\Th$ differential of the Ehrenfried complex is given by the composition $\pi \del''_p \kappa$.
The member function
\begin{lstlisting}
gen_d0( const int32_t p, const int32_t l );
\end{lstlisting}
generates the restriction of the $p\Th$ differential $\del_\E$ to the cells of $\Ehrprog$ with exactly $l$ clusters.
Similarly, the method
\begin{lstlisting}
gen_d1_stage_1( const int32_t p, const int32_t l );
\end{lstlisting}
generates the restriction of the $p\Th$ differential $\del_\E$ from the cells with $l$ clusters to the cells with $l-1$ clusters and
applies all row operations to this restriction that come from the sub-matrix $d^0$ from above.

For runtime reasons, we thereby use yet another formula for the map $\kappa$.
\begin{prop}
    Let $I'$ be the set of tuples of integers $(t_h, \ldots, t_1)$ such that $0 \leq t_q < q$.
    Then the map
    \[
        \{ 0, \ldots, h! - 1\} \to I'
        \mspc{with}{20}
        k \mapsto \left( \left\lfloor \frac{k}{(q-1)!} \right\rfloor \pmod q\right)_q
    \]
    is a bijection.
\end{prop}
\begin{proof}
We show that
\[
    (t_h, \ldots, t_1) \mapsto \sum_{q=2}^h t_q \cdot (q-1)!
\]
defines an inverse of the above map. 
Since both sets have the same cardinality, 
it suffices to show that this map is a right inverse.
For a fixed coordinate $q \in \{1, \dotsc, h\}$, we compute
\[
    \sum_{q=2}^h t_q \cdot (q-1)! = \sum_{q=r}^h t_q \cdot (q-1)! + \sum_{q=2}^{r-1} t_q \cdot (q-1)!\,.
\]
Using induction on $r$, we see that
\[
    \sum_{q=2}^{r-1} t_q \cdot (q-1)! < (r-1)!\,,
\]
since for $r = 1$, the statement is true, and if we assume the statement for $r-1$, we can conclude
\begin{align*}
    \sum_{q=2}^{r} t_q \cdot (q-1)! &= \sum_{q=2}^{r-1} t_q \cdot (q-1)! + t_r (r-1)! \\
                                    &< (r-1)!               + (r-1) (r-1)! \\
                                    &= r!\,.
\end{align*}
From this, we get
\[
    \left\lfloor \frac{\sum_{q=2}^h t_q \cdot (q-1)!}{(r-1)!} \right\rfloor=\left\lfloor \sum_{q=r}^h t_q \cdot \frac{(q-1)!}{(r-1)!} + \sum_{q=2}^{r-1} t_q \cdot \frac{(q-1)!}{(r-1)!} \right\rfloor = \sum_{q=r}^h t_q \cdot \frac{(q-1)!}{(r-1)!}
\]
since the left sum is integral and the right sum vanishes after rounding down.
Taking the remaining term modulo $r$, we get $t_r$ as a result since all summands but the $r\Th$ are zero modulo $r$.
\end{proof}

\subsubsection{More Member Functions}
\label{chapter_program:kappa:css:more_methods}

We also offer class methods to print the \progmember{basis\_complex} and the \progmember{matrix\_complex}, and a method
\begin{lstlisting}
void erase_differential( int32_t p );
\end{lstlisting}
which deletes the \progmember{p}${}\Th$ differential and releases its memory. 
This function is defined because we only want to keep a differential in the memory
as long as we really need it due to storage limitation.
\subsection{Serialization}
\label{program:kappa:serialization}

The transformation of data and objects of a running computer program into storable information (which can be saved on a hard disk) is called serialization.
In our project, we want to generate basis elements and differentials to save them for later use, i.e.\ we want to serialize them.
Therefore, we provide the general purpose template functions
\begin{lstlisting}
template < class StorableType >
void save_to_file_bz2
    ( StorableType& t, std::string filename, bool print_duration=true );
\end{lstlisting}
and
\begin{lstlisting}
template < class StorableType >
void load_from_file_bz2
    ( StorableType& t, std::string filename, bool print_duration=true);
\end{lstlisting}
Both functions, use the \progname{bzip2} algorithm for fast file compresseion, so we produce much smaller files.

\subsubsection{Storable Types}
A type can be handled by the above template functions, if
it is defined by the \cppeleven\ standard (such as \progclass{int} or \progclass{std}::\progclass{vector}\progname{<} \progclass{char} \progname{>}) or the {\bf boost \cpp library}.
Otherwise your class has to meet the following conditions.
First of all, it has to be a friend of \progclass{boost}\progname{::}\progclass{serialization}\progname{::}\progclass{access}, i.e.\ its class description includes the following line.
\begin{lstlisting}
class my_class
{
    ...
    friend class boost::serialization::access;
    ...
};
\end{lstlisting}
There are two ways of proceeding from here.
Most commonly, the process of saving and loading is the same.
In this situation one defines the template member function
\begin{lstlisting}
template < class Archive >
void serialize( Archive &ar, const unsigned int version );
\end{lstlisting}
whereas the template parameter \progclass{Archive} is filled in by {\bf boost} (and you do not need to know what it is).
Now saving and loading is achieved via the binary operator \progname{\&} applied to \progname{ar} and every member variable we want to save and load.
In the example below, we tell the serialization function, to save and load two out of three member variables.
\begin{lstlisting}
class my_class
{
    ...
    int keep_this;
    int this_is_also_important;
    int useless;
    
    friend class boost::serialization::access;
    template < class Archive >
    void serialize( Archive &ar, const unsigned int )
    {
        ar & keep_this;
        ar & this_is_also_important;
    }
};
\end{lstlisting}

There are rare situations, in which a saving mechanism differs from its loading counter part.
Then one has to implement the function
\begin{lstlisting}
template< class Archive >
void save( Archive & ar, const unsigned int version ) const;
\end{lstlisting}
and its counter part
\begin{lstlisting}
template< class Archive >
void load( Archive & ar, const unsigned int version );
\end{lstlisting}
and one has to tell {\bf boost} to use these two functions by calling the macro
\begin{lstlisting}
BOOST_SERIALIZATION_SPLIT_MEMBER()
\end{lstlisting}
afterwards. A simple example could look like this.
\begin{lstlisting}
class my_class
{
    ...
    int keep_this;
    int this_is_also_important;
    int useless;
    
    friend class boost::serialization::access;
    template< class Archive >
    void save( Archive & ar, const unsigned int ) const
    {
        ar & keep_this & this_is_also_important;
    }
    template< class Archive >
    void load( Archive & ar, const unsigned int )
    {
        ar & keep_this & this_is_also_important;
        useless=0;
    }
    BOOST_SERIALIZATION_SPLIT_MEMBER()
};
\end{lstlisting}
For more information one should consider the online manual of \cite{boost}.

