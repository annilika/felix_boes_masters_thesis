\section{Remarks on Compiling}
\label{program:compiling}

Our software projects were developed and tested on {\bf Debian 7}, {\bf Ubuntu 12.04 LTS} and {\bf Open SUSE 13.1}.
If you use a different operating system, your compiler has to support the full \cppeleven\ standard.
We suggest to use the \cpp\ compiler of the {\bf Gnu Compiler Collection}.

\subsection{Installing the Required Software}
Using an operating system based on {\bf Debian}, it should suffice to install the software and libraries from the official repositories.
\begin{lstlisting}
sudo apt-get install \
    build-essential g++ libboost-all-dev libgmp-dev libbz2-dev
\end{lstlisting}
We document the source code with the {\bf Doxygen-syntax}.
Thus we can generate a documentation using the program {\bf doxygen}.
It is installed as follows.
\begin{lstlisting}
sudo apt-get install doxygen doxygen-gui doxygen-latex
\end{lstlisting}

\subsection{Building the Projects}
Using the provided makefile, the executables are built as follows.
\begin{lstlisting}
make compute_cache
make compute_css
make compute_statistics
make print_basis
\end{lstlisting}
Moreover, you can create your own executables (with a given name say \progname{my\_program})
by creating a corresponding \progname{.cpp} file in the subfolder \progname{./kappa} with the prefix \progname{main\_} (e.g.\ you create \progname{./kappa/main\_my\_program.cpp}).
Calling \progname{make} with the name of your project will create an executable with this name.

\subsection{More Remarks}
Using {\bf doxygen} you can generate a documentation of the source code as follows. 
\begin{lstlisting}
make doc
\end{lstlisting}
The documentation itself can be found in the subdirectory \progname{./doc} in the corresponding project.

The libraries, executables and documentation can be cleaned up as always.
The generated results are kept.
\begin{lstlisting}
make clean
\end{lstlisting}
