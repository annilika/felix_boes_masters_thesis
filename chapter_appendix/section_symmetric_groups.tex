\section{The Symmetric Groups \texorpdfstring{$\Symgrp^\Delta$}{SDelta} as Semisimplicial Set [B]}
\label{notation:semi_simpl_sym_grp}
In order to provide a compact notation, we introduce the symmetric groups as semisimplicial set $\SymD$.
\label{page:simplicial_category}%
\index{simplicial!simplicial category}
\symbolindex[d]{$\Delta$}{The simplicial category, with face maps $d^\Delta_i$ and degeneracy maps $s^\Delta_i$.}{Page \pageref{page:simplicial_category}}
The {\bf simplicial category} (which is used to define simplicial sets) is $\Delta$.
\label{page:simplicial_face}%
\index{simplicial!simplicial face map}
\symbolindex[d]{$d^\Delta_i$}{The $i\Th$ simplicial face map.}{Page \pageref{page:simplicial_face}}
Its {\bf face maps} are denoted by
\[
    d^\Delta_i \colon \{0, \ldots, n\} \to \{0, \ldots, n+1\} \mspc{i.e.}{20} d^\Delta_i(c) = \begin{cases} c & c < i \\ c+1 & c \ge i \end{cases}\,,
\]
\label{page:simplicial_degeneracy}%
\index{simplicial!simplicial degeneracy map}
\symbolindex[s]{$s^\Delta_i$}{The $i\Th$ simplicial degeneracy map.}{Page \pageref{page:simplicial_degeneracy}}
and its {\bf degeneracy maps} by
\[
    s^\Delta_i \colon \{0, \ldots, n+1\} \to \{0, \ldots, n\} \mspc{i.e.}{20} s^\Delta_i(c) = \begin{cases} c & c \le i \\ c-1 & c > i \end{cases} \,.
\]
\label{page:semisimplicial_category}%
\index{simplicial!semisimplicial category}
\symbolindex[d]{$\Delta_{semi}$}{The semisimplicial category with face maps $d^\Delta_i$.}{Page \pageref{page:semisimplicial_category}}
The {\bf semisimplicial category} $\Delta_{semi} \subset \Delta$ is a faithful subcategory.
It consists of the same objects, contains all face maps $d^\Delta$ and misses all degeneracy maps $s^\Delta$.

\begin{defi}
    \label{notation:sym_grp}
    \index{symmetric group}
    \symbolindex[s]{$\Symgrp_n$}{The $n\Th$ symmetric group is $\Symgrp_n = Aut(\{0, \ldots, n\})$.}{Definition \ref{notation:sym_grp}}
    \symbolindex[s]{$\Symgrp_n^\times$}{A shorthand for $\Symgrp_n^\times = Aut(\{1, \ldots, n\}) \subset \SymGr_n$.}{Definition \ref{notation:sym_grp}}
    The group of bijections of a set $S$ is the {\bf symmetric group} with respect to $S$ and is denoted by $\Symgrp_{S} = Aut(S)$.
    For $n$ a non-negative integer, it is convenient to identify $n$ with $\{0, \ldots, n\}$.
    Consequently, the $n\Th$ {\bf symmetric group} $\SymGr_n$ is the group of all bijections of the set $\{ 0, \ldots, n\}$.
    The subgroup $Aut(\{ 1, \ldots, n\})$ is denoted by $\SymGr_n^\times$.
\end{defi}

\begin{defi}
    \label{notation:support}
    \index{symmetric group!support of permutations}
    \symbolindex[s]{$\supp(\alpha)$}{The support of a permutation $\alpha$.}{Definition \ref{notation:support}}
    The {\bf support} of a permutation $\alpha$ is the set of non-fixed points, i.e.\ 
    \[
        \supp(\alpha) = \{ k \mid \alpha(k) \neq k \}.
    \]
    The {\bf support} of permutations $\alpha_1, \ldots, \alpha_n$ is
    \[
        \supp(\alpha_1, \ldots, \alpha_n) = \supp(\alpha_1) \cup \ldots \cup \supp(\alpha_n) \,.
    \]
\end{defi}

\begin{defi}
    \label{notation:sym_grp:face}
    \index{symmetric group!face map}
    \symbolindex[d]{$D_i$}{The $i\Th$ face map of the symmetric groups $\SymD$.}{Definition \ref{notation:sym_grp:face}}
    Let $n > 0$ and $0 \le i \le n$.
    The $i\Th$ {\bf face}
    \[
        D_i \colon \SymGr_n \xdr{} \SymGr_{n-1}
    \]
    is defined as follows.
    Consider a permutation $\alpha \in \SymGr_n$ and alter it by skipping $i$, which results in a permutation on $\{0, \ldots, n-1\}$ up to renormalization:
    \[
        D_i\alpha = s^\Delta_i \circ \big( \alpha \cdot (i\ \alpha^{-1}(i)) \big) \circ d^\Delta_i \,.
    \]
\end{defi}

\begin{rem}
    Each $D_i$ is a surjective map of sets, but not a homomorphism of groups.
    The semisimplicial identities
    \[
        D_i D_j = D_{j-1}D_i \mspc{for}{20} i < j
    \]
    are readily verified.
\end{rem}

\begin{defi}
    \label{notation:sym_grp:semisimplicial_set}
    \index{symmetric group!symmetric groups as semisimplicial set}
    \symbolindex[S]{$\SymD$}{The symmetric groups $\SymD$.}{Definition \ref{notation:sym_grp:semisimplicial_set}}
    The symmetric groups define a semisimplicial set $\SymD$ with $\SymD_n = \Symgrp_n$ and face maps $D_i$ as above.
\end{defi}

\begin{defi}
    \label{notation:sym_grp:degeneracy}
    \index{symmetric group!degeneracy map}
    \symbolindex[s]{$S_i$}{The $i\Th$ pseudo degeneracy map of the symmetric groups $\SymD$.}{Definition \ref{notation:sym_grp:degeneracy}}
    Let $n \ge 0$ and $0 \le i \le n$.
    The $i\Th$ {\bf pseudo degeneracy}
    \[
        S_i \colon \SymGr_n \xhr{} \SymGr_{n+1}
    \]
    is defined as follows.
    Consider a permutation $\alpha \in \SymGr_n$ and recognize it as permutation on $\{0, \ldots, n+1\}$ via shifting all $j > i$ up by one:
    \[
        (S_i\alpha)(c) = \begin{cases} i & c = i \\ d^\Delta_i \circ \alpha \circ s^\Delta_i & c \neq i \end{cases} \,.
    \]
\end{defi}

\begin{rem}
    Each $S_i$ is a monomorphism of groups because $s^\Delta_i d^\Delta_i = \id_{\{0, \ldots, n\}}$
\end{rem}

The face and pseudo degeneracy maps fulfill all but one simplicial identity.
In particular they do not make $\SymD$ a simplicial set.

\begin{prop}
    \label{notation:DiSj_identities}
    The following identities are fulfilled
    \begin{align*}
        D_i D_j &= D_{j-1}D_i    \mspc{for}{20} i < j \\
        S_i S_j &= S_j S_{i-1}   \mspc{for}{20} i > j \\
        D_i S_j &=
            \begin{cases}
                S_{j-1} D_i         &\text{ for } i < j \\
                \id             &\text{ for } i = j \\
                S_j D_{i-1}     &\text{ for } i > j+1 
            \end{cases}
        \intertext{but}
        D_iS_j &\neq \id \mspc{for}{20} i = j+1\,.
    \end{align*}
\end{prop}

\begin{proof}
    The identities are readily verified and $(D_{i+1} S_i)(\tau) = 1_{\SymGr_n}$ holds for any transposition $\tau = (c\ i)$.
\end{proof}

\begin{lem}
Let $\pi$ be a permutation on $\pi \in \Symgrp_p$ and $0 \le j \le p$.
Then we have
\[ 
    N(D_j(\pi)) =
        \begin{cases}
            N(\pi)      & j \text{ fix point of } \pi \\
            N(\pi) - 1  & \text{otherwise}
        \end{cases}\,,
\] 
where $N$ denotes the word length norm.

\begin{proof}
Recall that we can express the norm of $\pi$ as $N(\pi) = p - \mathrm{cyc}(\pi)$, 
where $\mathrm{cyc}(\pi)$ denotes the number of cycles of $\pi$, also considering fixed points as cycles. 
Now, if $j$ is a fixed point of $\pi$, we have 
\begin{align*}
N(D_j(\pi))&= (p - 1) - \mathrm{cyc}(D_j(\pi)) \\
           &= (p - 1) - (\mathrm{cyc}(\pi) - 1) \\
           &= N(\pi)
\end{align*}
since $D_j(\pi)$ consists of the same cycles as $\pi$ (up to renormalization), 
except that the fixed point $j$ is not contained in $D_j(\pi)$ anymore.
If $j$ is not a fixed point of $\pi$, it is removed from its cycle by $D_j$, 
but the number of cycles remains the same. 
Thus, in this case, we have
\begin{align*}
N(D_j(\pi)) &= (p - 1) - \mathrm{cyc}(D_j(\pi)) \\
           &= (p - 1) - \mathrm{cyc}(\pi) \\
           &= N(\pi) - 1\,.
\end{align*}
\end{proof}
\end{lem}