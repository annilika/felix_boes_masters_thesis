\section{Radial Multiplication}
\label{homology_operations:radial_multiplication}
We will now define a multiplication for radial slit pictures,
which looks very analogous to the multiplication for parallel slit pictures.
Nevertheless, it will turn out that the radialization map defined in Subsection \ref{cellular_radialization} is not multiplicative with respect to these multiplications, compare Remark \ref{radialization_not_mult}.

So let $A$ respectively $A'$ be two radial slit domains, each on $n$ annuli.
We merge each corresponding pair of annuli of the radial slit pictures of $A$ and $A'$ into one as in Picture \ref{homology_op:radial_mult}.
\begin{figure}[ht]
\centering
\incgfx{pictures/radial_mult}
\centering
\incgfx{pictures/radial_mult_result}
\caption{\label{homology_op:radial_mult} Two radial slit pictures $A$ and $A'$ and their product $A \radmult A'$.}
\end{figure}
We subdivide the new annulus equally into an outer and an inner ring, putting $A$ into the outer ring and $A'$ into the inner ring,
whereby $A'$ uses the lower half of the slits and $A'$ the upper half.
Be aware that the slits of $A$ also start at the outer boundary of the annulus.
Thus we introduce a new slit pair on each annulus in order to ensure that the outer boundary curves of $A$ and $A'$ do not interfere.
We obtain

\begin{defi}
\label{rad_mult_defi}
\symbolindex[m]{$\radmult$}{Radial multiplication map}{Definition \ref{rad_mult_defi}}
\index{radial multiplication}
   Let $A \in \Rad$ and $A' \in \Radt_{g'}(m', n)$ be two radial slit domains
   with the same number of incoming boundary curves.
   The \textbf{product} 
   \[ 
      A \radmult A'
   \]
   of $A$ and $A'$ is given by the radial slit domain obtained by the process described above.
   We shall refer to this multiplication as \textbf{radial multiplication}. 
\end{defi}

\begin{prop}
   The radial multiplication is a map
   \[
       \radmult \colon \Rad \times \Radt_{g'}(m', n) \to \Radt_{\tilde g}(\tilde m, n)\,,
   \]
   where $\tilde g = g + g' + n - 1$ and $\tilde m = m + m'$.
\begin{proof}
   By construction, we obtain $n(A \radmult A') = n$.
   Due to the insertion of the new slits, we assert 
   \[
      m(A \radmult A') = m(A) + m(A')\,.
   \]
   Using the formula $N(A) = 2 g(A) - 2 + m(A) + n(A)$ for $A$, $A'$ and $A \radmult A'$,
   we obtain 
   \[
      g(A \radmult \A') = g + g' + n - 1\,.    
   \]
\end{proof}
\end{prop}

In order to develop a deeper understanding why these formulas are correct,  
let us see how radial multiplication looks on surfaces, compare Figure \ref{homology_op:radial_mult_on_modspc}.

\begin{figure}[ht]
\centering
\def\svgwidth{0.8\columnwidth}
\input{pictures/homology_op_radial_mult_on_modspc.pdf_tex}
\caption{\label{homology_op:radial_mult_on_modspc} Radial multiplication applied to surfaces $F$ and $F'$.}
\end{figure}

The $k\Th$ inner boundary curves of $F$ and $F'$ are connected by a tube, upon which a new $k\Th$ inner boundary curve arises.
Thereby, each two neighboring tubes contribute to the genus of the new surface.
The newly inserted slits make sure that the outer boundary curves of the new surface are simply the outer boundary curves of $F$ and $F'$. 

Now we come to several properties we expect to be fulfilled by a multiplication,
and see which of them are indeed satisfied by this radial multiplication.

\begin{prop}
Radial multiplication is associative up to homotopy.
\begin{proof}
   Let $A, B, C$ be three radial slit pictures. 
   In Figure \ref{homology_op:radial_mult_associative}, we see how $A \radmult (B \radmult C)$ can be homotoped into $(A \radmult B) \radmult C$
   by successively applying slit jumps and changing the lengths of slits.
   \begin{figure}[hp]
   \centering
   \incgfx{pictures/radial_mult_associative_in_steps}
   \vspace{1cm}
   \centering
   \incgfx{pictures/radial_mult_associative_in_steps_2}
   \vspace{1cm}
   \centering
   \incgfx{pictures/radial_mult_associative_in_steps_3}
   \caption{\label{homology_op:radial_mult_associative} Transforming $A \radmult (B \radmult C)$ into $(A \radmult B) \radmult C$, to be read from the left to the right and then from up to down.}
   \end{figure}
\end{proof}
\end{prop}

If we consider the definition of the radial multiplication on surfaces, we immediately see

\begin{prop}
   Radial multiplication is commutative up to homotopy.
\end{prop}

Note that the insertion of a new pair of slits on each annulus nicely maintains all glueing information, 
but also cause several disadvantages.

\begin{rem}
  Radial multiplication does not have a unit (even up to homotopy):
  Consider a radial slit domain $A$ and try to imagine another radial slit domain $B$, for which $A \radmult B$ is homotopic to $A$. 
  But since we always insert a new pair of slits isolating $A$ from the rest of the annulus, we will always obtain an additional outgoing boundary curve,
  no matter how $B$ looks like (even if it is empty).
\end{rem}

\begin{rem}
   Since the radial multiplication of two slit pictures involves inserting new slits, 
   we cannot directly describe it in terms of operads. 
   See Section \ref{radial_composition} for an associative operation that does fulfill this property, but that is only defined under limited conditions.
\end{rem}

\begin{rem}\label{radialization_not_mult}
   The radialization map 
   \[
      \radmap \colon \Modspc \to \mathfrak M_g(m + n, n)
   \]
   defined in Subsection \ref{cellular_radialization} is not multiplicative with respect to the radial multiplication and any of the multiplications $\mu^{\upuparrows}$, $\mu^{\updownarrows}$ or $\mu^{cs}$
   from Subsections \ref{homology_operations:parallel_patching_slit_pics:construction_of_mu_upuparrows}, 
   \ref{homology_operations:parallel_patching_slit_pics:construction_of_mu_updownarrows}
   and \ref{homology_operations:parallel_patching_slit_pics:construction_of_mu_cs}.
   To see this, let $F$ respectively $F'$ be two surfaces with $m$ respectively $m'$ punctures, and with $n$ boundary curves each.
   Compare $\radmap(F) \radmult \radmap(F')$ and $\radmap(F \cdot F')$ with $\_\cdot\_$ denoting any of these parallel multiplications.
   Recall that the radialization map applied to $F$ transforms the punctures of $F$ in outgoing boundaries and creates one additional outgoing boundary for each boundary curve of $F$.
   Hence, the number of outgoing boundaries of $\radmap(F) \radmult \radmap(F')$ equals $m + n + m' + n$.
   On the other hand, the product $F \cdot F'$, however the parallel multiplication is chosen among $\mu^{\upuparrows}$, $\mu^{\updownarrows}$ or $\mu^{cs}$, 
   only keeps the punctures from $F$ and $F'$ and does not create any new ones.
   Thus, we have
   \[
    m(\radmap(F \cdot F')) = m + m' + n \neq m + m' + 2n = m(\radmap(F) \radmult \radmap(F'))
   \]
   for $n >  0$, which is always presumed.
\end{rem}

Summarizing the positive results, we obtain

\begin{cor}
  Radial multiplication yields an associative and commutative homology operation
  \[
     \radmult_* \colon H_*(\mathfrak M_g(m, n)) \otimes H_*(\mathfrak M_{g'}(m', n)) \to H_*(\mathfrak M_{g+g'+n - 1}(m + m', n))\,.
  \]
\end{cor}