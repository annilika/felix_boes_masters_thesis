\section{Operations for Parallel Slit Domains on Several Levels [B]}
\label{homology_operations:glueing_construction}
In this section, we propose a generalization of the above operations to parallel slit domains on several levels.
Hereby, we imagine the parallel slit domains in question as slit pictures and surfaces with boundaries simultaneously.
Consequently, we picture the product of two parallel slit domains $L_1 \in \Par_{g_1,1}^{m_1}[(1)]$ and $L_2 \in \Par_{g_2,1}^{m_2}[(1)]$ as follows.
We view $L_1$ and $L_2$ as disjoint paraxial rectangles that miss several slits and identify them with the associated surfaces $F_1$ and $F_2$ which have exactly one boundary curve,
see Figure \ref{homology_operations:parallel_patching_slit_pics:general_glueing_two_slit_pics_and_surfaces}.
\begin{figure}[ht]
    \centering
    \def\svgwidth{.7\columnwidth}
    \input{pictures/general_glueing_two_slit_pics_and_surfaces.pdf_tex}
    \caption{\label{homology_operations:parallel_patching_slit_pics:general_glueing_two_slit_pics_and_surfaces}Two surfaces with boundary and their associated parallel slit domains.}
\end{figure}

The slit picture $\mu(L_1, L_2)$ is obtained by glueing in a stripe which joins the top of $L_2$ with the bottom of $L_1$.
\label{page:glueing_construction_arcs_c_plus_and_c_minus}%
\symbolindex[c]{$c^+$ resp.\ $c^-$}{The arcs of a boundary curve of a surface, which correspond to the top respectively bottom of a slit picture}{Page \pageref{page:glueing_construction_arcs_c_plus_and_c_minus}}
The boundary of the surfaces $F_1$ respectively $F_2$ admits a distinguished arc $c_2^+$ respectively $c_1^-$,
which corresponds to the top respectively bottom of the associated slit picture.
Joining $c_1^-$ with $c_2^+$ by glueing a stripe inbetween gives rise to the surface associated with $\mu(L_1, L_2)$,
compare Figure \ref{homology_operations:parallel_patching_slit_pics:general_glueing_two_slit_pics_and_surfaces_glued}.
\begin{figure}[ht]
    \centering
    \def\svgwidth{.8\columnwidth}
    \input{pictures/general_glueing_two_slit_pics_and_surfaces_glued.pdf_tex}
    \caption{\label{homology_operations:parallel_patching_slit_pics:general_glueing_two_slit_pics_and_surfaces_glued}The surface associated with the parallel slit domain $\mu(L_1, L_2)$.}
\end{figure}
In order to define the glueing construction, we will think of two surfaces to stand opposite to each other,
compare Figure \ref{homology_operations:parallel_patching_slit_pics:general_glueing_two_surfaces_glued_untwisted}.
\begin{figure}[ht]
    \centering
    \def\svgwidth{.8\columnwidth}
    \input{pictures/general_glueing_two_surfaces_glued_untwisted.pdf_tex}
    \caption{\label{homology_operations:parallel_patching_slit_pics:general_glueing_two_surfaces_glued_untwisted}The surface associated with the parallel slit domain $\mu(L_1, L_2)$.}
\end{figure}

\subsection{The Glueing Construction [B]}
In order to treat the general case, consider a parallel slit domain $L \in \Parr$.
The associated surface $F$ has exactly $n$ boundary curves $C_1, \ldots, C_n$,
each $C_i$ is subdivided into $r_i$ increasingly enumerated regions and each region admits two arcs $c^+_{ij}$ and $c^-_{ij}$ which correspond to the top and bottom
of the incidential level of the parallel slit domain $L$.

Fix parallel slit domains $L_1 \in \Par_{g_1, n_1}^{m_1}[(r_1^{(1)}, \ldots, r_{n_1}^{(1)})]$ and $L_2 \in \Par_{g_2, n_2}^{m_2}[(r_1^{(2)}, \ldots, r_{n_2}^{(2)})]$.
We discuss the glueing construction for the associated surfaces $F_1$ and $F_2$ at first.
Imagine $F_1$ to stand left of $F_2$.
Moreover, the boundary curves of $F_1$ form the boundaries of tubes that tend to $F_2$ and analogously 
the boundary curves of $F_2$ form the boundaries of tubes that tend to $F_1$ 
as is sketched in Figure \ref{homology_operations:parallel_patching_slit_pics:general_glueing_two_surfaces_look_at_each_other}.
\begin{figure}[ht]
    \centering
    \def\svgwidth{\columnwidth}
    \input{pictures/general_glueing_two_surfaces_look_at_each_other.pdf_tex}
    \caption{\label{homology_operations:parallel_patching_slit_pics:general_glueing_two_surfaces_look_at_each_other}Two surfaces looking at each other.}
\end{figure}

In the first step, we match some of the arcs $c^-$ of $F_1$ with some of the arcs $c^+$ of $F_2$.
The matching is not allowed to be empty (this would produce a disconnected surface).
In the second step, we join the matched arcs by glueing (untwisted) stripes inbetween.
The resulting surface $F$ has $m= m_1+m_2$ punctures, but both the genus $g$ and the number of boundary curves $n$ is subject to the chosen matching.
In the third step, we choose an enumeration of the boundaries of $F$ and for each boundary $C_i$ of $F$
there are $r_i$ arcs $c^-$ which were not used in the glueing process (e.g.\ all arcs $c^-$ in $F_2$).
The resulting (ordered) partition is therefore $r = r_1 + \ldots +r_n$.
Moreover, we choose an arc $c^-_i$ for each boundary curve $C_i$ of $F$ and
the levels of the associated parallel slit domain are ordered with respect to the occurence of the incidential arcs $c^-$
by wandering through the corresponding boundary curve starting at the arc $c^-_i$ chosen above.

In terms of parallel slit domains, we start with the choice of a partial, non-empty matching of the respective levels of $L_1$ and $L_2$.
Using the action of the little cubes operad for each pair of levels, we emplace each pair into its own complex plane.
The number of punctures is $m = m_1 + m_2$, the norm is $h = h_1 + h_2$, but both the number $n$ of boundary curves and the number $r_i$ of levels per boundary curve have to be computed
--- note that the number of levels is not determined by $r^{(1)}$, $r^{(2)}$ and the size of the matching ---
and the genus is given by $g = \frac{h+2-m-n-r}{2}$.
Having this done, we choose an enumeration of the $n$ boundary curves by declaring a first level for each boundary.
The remaining levels are ordered by their occurence of the permutation induced by $\sigma_{h}$.
This ends the construction of a non-degenerate parallel slit domain in $\Par_{g,n}^m[(r_1, \ldots, r_n)]$.

We reflect the construction discussed above in the next
\begin{defi}
    The {\bf combinatorial type} $G$ which specifies the glueing construction {\bf depends on} the parameters
    \[
        \mathfrak P(G) = (g_1, g_2, n_1, n_2, m_1, m_2, (r_1^{(1)}, \ldots, r_{n_1}^{(1)}), (r_1^{(2)}, \ldots, r_{n_2}^{(2)}))
    \]
    and {\bf consists of} the following two data.
    \begin{enumerate}
        \item A partial, non-empty matching of the levels.
    \end{enumerate}
    The size of the matching is denoted by $s(G)$.
    The corresponding surface of genus $g(G)$ has $m(G) = m_1 + m_2$ punctures and $n(G)$ (yet unordered) boundary curves each consisting of several levels.
    \begin{enumerate}
        \setcounter{enumi}{1}
        \item A partial enumeration of the levels such that each boundary curve belongs to exactly one selected level.
    \end{enumerate}
    The corresponding ordered configuration is $(r^{(G)}_1, \ldots, r^{(G)}_{n(G)})$.
    The {\bf set of combinatorial types} that specify a glueing construction is denoted by $\mc G$.
\end{defi}

Using the introduced notation we have proven the following
\begin{prop}
    \label{homology_operations:parallel_patching_slit_pics:glueing_construction_defines_operations}
    For every combinatorial type $G \in \mc G$ with parameters
    \[
        \mathfrak P(G) = (g_1, g_2, n_1, n_2, m_1, m_2, (r_1^{(1)}, \ldots, r_{n_1}^{(1)}), (r_1^{(2)}, \ldots, r_{n_2}^{(2)}))
    \]
    there are operations
    \begin{multline*}
        \tilde\vartheta_G \colon 
            \left( \coprod_{s(G)} \cspc 2(\C) \right) \times
            \Par_{g_1, n_1}^{m_1}(r_1^{(1)}, \ldots, r_{n_1}^{(1)}) \times
            \Par_{g_2, n_2}^{m_2}(r_1^{(2)}, \ldots, r_{n_2}^{(2)}) \to\\
            \Par_{g(G), n(G)}^{m(G)}(r^{(G)}_1, \ldots, r^{(G)}_{n(G)}) \,,
    \end{multline*}
    where each $\cspc 2(\C)$ acts on exactly one predescribed pair of matched levels.
    Using the homology cross product, we obtain homology operations
    \begin{multline*}
        (\tilde\vartheta_G)_\ast \colon 
            H_i(\cspc 2(\C))^{\oplus s(G)} \otimes
            H_j(\Par_{g_1, n_1}^{m_1}(r_1^{(1)}, \ldots, r_{n_1}^{(1)})) \otimes
            H_k(\Par_{g_2, n_2}^{m_2}(r_1^{(2)}, \ldots, r_{n_2}^{(2)})) \to\\[10pt]
            H_{i+j+k}(\Par_{g(G), n(G)}^{m(G)}(r^{(G)}_1, \ldots, r^{(G)}_{n(G)})) \,.
    \end{multline*}
\end{prop}

Now that we have established the general framework, let us discuss three special cases of the glueing construction.
The first operation $\mu^{\upuparrows}$ is discussed in terms of parallel slit domains,
the second operation $\mu^{\updownarrows}$ is discussed in terms of surfaces with boundaries
and the third operation $\mu^{cs}$ is discussed in terms of the dual Ehrenfried complex.

\subsection{The Operation \texorpdfstring{$\mu^{\upuparrows}_\ast$}{muupuparrows*} [B]}
\label{homology_operations:parallel_patching_slit_pics:construction_of_mu_upuparrows}
\symbolindex[m]{$\mu^\upuparrows$}{A selected glueing construction}{Subsubsection \ref{homology_operations:parallel_patching_slit_pics:construction_of_mu_upuparrows}}
The first homology operation is induced by a product called $\mu^{\upuparrows}$
\label{page:shorthand_par_n_r}%
\symbolindex[p]{$\Par_n[(r_1, \ldots, r_n)]$}{A shorthand for $\Par_n[(r_1, \ldots, r_n)] = \coprod_{g,m}\Parr$.}{Page \pageref{page:shorthand_par_n_r}}
and is defined for all parallel slit domains in $\Par_n[(r_1, \ldots, r_n)] = \coprod_{g,m} \Par_{g,n}^m[(r_1, \ldots, r_n)]$ with $n$ and $r=r_1 + \ldots + r_n$ fixed.
For two parallel slit domains $L_1 \in \Par_{g_1,n}^{m_1}[(r_1, \ldots, r_n)]$ and $L_2 \in \Par_{g_2,n}^{m_2}[(r_1, \ldots, r_n)]$,
we match the levels with the same index and insert each pair into a complex plane via $\mu$.
The resulting parallel slit domains are sketched in Figure \ref{homology_operations:parallel_patching_slit_pics:mu_upuparrows} where $L_1$ is colored green and $L_2$ is colored blue.
\begin{figure}[ht]
    \centering
    \def\svgwidth{.85\columnwidth}
    \input{pictures/mu_upuparrows.pdf_tex}
    \caption{\label{homology_operations:parallel_patching_slit_pics:mu_upuparrows}The parallel slit domain $\mu^\upuparrows(L_1, L_2)$.}
\end{figure}
Note that for each pair of boundary curves $C_i$ of $L_1$ and $L_2$ with $r_i \equiv_{2} 1$ the resulting parallel slit domain $L$ receives exactly one boundary curve,
but for $r_i \equiv_{2} 0$ we receive two boundaries.
This is due to the fact that the induced ordering of the levels of $L$ is
\[
    \big( ( \ul0_1, \ldots, \ul0_{r_1} )( \ul0_{r_1+1}, \ldots, \ul0_{r_2} ) \ldots \big)^2 = ( \ul0_1, \ul0_3, \ldots ) \ldots \,.
\]
We order the boundaries and levels ascendingly, i.e.\ the lowest level corresponds to the first level of the first boundary.
The subsequent levels are ordered with respect to their occurence in $( \ul0_1, \ldots, \ul0_{r_1} )^2$.
On the remaining levels we repeat this process until all levels are ordered.
The resulting parallel slit domain $L$ has $\tilde m = m_1 + m_2$ punctures,
$\tilde n = n + \#\{ r_i \equiv_2 0\}$ boundary components,
the ordered partition $(\tilde r_1, \ldots \tilde r_{\tilde n})$ arises from $(r_1, \ldots, r_n)$ by replacing every even $(\ldots, r_i, \ldots )$ by $(\ldots, \frac{r_i}{2}, \frac{r_i}{2}, \ldots)$
and the genus is $\tilde g = g_1 + g_2 + \frac{n+r-\#\{r_i \equiv_2 0\}}{2} - 1$.
We write $\mu^\upuparrows(L_1, L_2)$ to remind us that the levels of both parallel slit domains occured ascendingly.
This ends the discussion of the first selected homology operation.
Summing up, we have
\begin{defprop}
    \label{homology_operations:parallel_patching_slit_pics:operation_mu_upuparrows}
    \symbolindex[m]{$\mu^\upuparrows_\ast$}{A selected homology operation}{Definition / Proposition \ref{homology_operations:parallel_patching_slit_pics:operation_mu_upuparrows}}
    Using the homology cross product we obtain a family of homology operations
    \[
        \mu^\upuparrows_\ast \colon H_s(\Par_{g_1,n}^{m_1}[(r_1, \ldots, r_n)])  \otimes H_t(\Par_{g_2,n}^{m_2}[(r_1, \ldots, r_n)]) \to H_{s+t}(\Par_{\tilde g,\tilde n}^{\tilde m}[(\tilde r_1, \ldots, \tilde r_n)])
    \]
    by
    \[
        x \otimes y \mapsto \mu^\upuparrows_\star( x \otimes y ) \,,
    \]
    with $\tilde g$, $\tilde m$, $\tilde n$ and $\tilde r$ as above and $\mu^\upuparrows_\star$ the induced map in homology.
\end{defprop}

\subsection{The Operation \texorpdfstring{$\mu^{\updownarrows}_\ast$}{muupdownarrows*} [B]}
\label{homology_operations:parallel_patching_slit_pics:construction_of_mu_updownarrows}
\symbolindex[m]{$\mu^\updownarrows$}{A selected glueing construction}{Subsubsection \ref{homology_operations:parallel_patching_slit_pics:construction_of_mu_updownarrows}}
The second homology operation is induced by a product called $\mu^{\updownarrows}$
and is defined for all parallel slit domains in $\Par_n[(r_1, \ldots, r_n)] = \coprod_{g,m} \Par_{g,n}^m[(r_1, \ldots, r_n)]$ with $n$ and $r=r_1 + \ldots + r_n$ fixed.
Consider two parallel slit domains $L_1 \in \Par_{g_1,n}^{m_1}[(r_1, \ldots, r_n)]$ and $L_2 \in \Par_{g_2,n}^{m_2}[(r_1, \ldots, r_n)]$
and imagine their associated surfaces $F_1$ and $F_2$ such that the boundary components with the same numbering are in the face of each other.
For each boundary curve we join the arcs $c^-$ in $F_1$ with the arcs $c^+$ on the opposite side in $F_2$.
In Figure \ref{homology_operations:parallel_patching_slit_pics:parallel_mutliplication_three_slit_pictures}, we sketch this for parallel slit domains $L_1 = L_2$
with combinatorial type $\Sigma_1 = \Sigma_2 = ((\ul1_3\ \ul1_2) | (\ul2_3\ \ul1_1))$ and
\begin{figure}[ht]
    \centering
    
    \def\svgwidth{.5\columnwidth}
    \raisebox{-0.5\height}{\input{pictures/parallel_mutliplication_glueing_naive.pdf_tex}}
    \hspace{10pt}
    \def\svgwidth{.3\columnwidth}
    \raisebox{-0.5\height}{\input{pictures/parallel_mutliplication_three_slit_pictures.pdf_tex}}
    \caption{
        \label{homology_operations:parallel_patching_slit_pics:parallel_mutliplication_three_slit_pictures}
        The multiplication of two closed discs on three levels, where $g=m=0$, $n=1$ and $r=3$.}
\end{figure}
we obtain the same surface by connecting the boundary component of $L_1$ with the boundary component of $L_2$ by a tube which has three additional, enumerated boundary curves,
see Figure \ref{homology_operations:parallel_patching_slit_pics:parallel_mutliplication_glueing_with_handle}.
\begin{figure}[ht]
    \centering
    \def\svgwidth{.5\columnwidth}
    \input{pictures/parallel_mutliplication_glueing_with_handle.pdf_tex}
    \caption{
        \label{homology_operations:parallel_patching_slit_pics:parallel_mutliplication_glueing_with_handle}
        A better picture for the multiplication of two closed discs on three levels, where $g=m=0$, $n=1$ and $r=3$.}
\end{figure}
In general, every two boundary components with the same numbering say $i$ are connected by a tube with $r_i$ additional boundary curves enumerated by the arcs $c^-$ in $F_1$.
The resulting surface has $\tilde n = r$ enumerated boundary components and it is clear that the associated partition is $(1 + \ldots + 1)$.
Moreover, the number of punctures of the resulting parallel slit domain is $\tilde m = m_1 + m_2$ and by taking a glance at Figure \ref{homology_operations:parallel_patching_slit_pics:parallel_mutliplication_global_glueing}
it is clear that its genus is $g_1+ g_2+ n-1$.
\begin{figure}[ht]
    \centering
    \def\svgwidth{.6\columnwidth}
    \input{pictures/parallel_mutliplication_global_glueing.pdf_tex}
    \caption{
        \label{homology_operations:parallel_patching_slit_pics:parallel_mutliplication_global_glueing}
        The genus of the resulting surface is $g_1+g_2 + n - 1$, it has $m_1 + m_2$ punctures and $r$ enumerated boundary components.
        The associated partition is $(1,\ldots,1)$.}
\end{figure}

As operation on parallel slit domains, we match the ascending levels of $L_1$ with the descending levels of $L_2$.
This is denoted by the symbol $\mu^{\updownarrows}$.
This ends the discussion of the second selected homology operation.
Summing up, we provided
\begin{defprop}
    \label{homology_operations:parallel_patching_slit_pics:operation_mu_updownarrows}
    \symbolindex[m]{$\mu^\updownarrows_\ast$}{A selected homology operation}{Definition / Proposition \ref{homology_operations:parallel_patching_slit_pics:operation_mu_updownarrows}}
    Using the homology cross product we obtain a family of homology operations
    \[
        \mu^\updownarrows_\ast \colon H_s(\Par_{g_1,n}^{m_1}[(r_1, \ldots, r_n)])  \otimes H_t(\Par_{g_2,n}^{m_2}[(r_1, \ldots, r_n)]) \to H_{s+t}(\Par_{\tilde g,\tilde n}^{\tilde m}[(\tilde r_1, \ldots, \tilde r_n)])
    \]
    by
    \[
        x \otimes y \mapsto \mu^\updownarrows_\star( x \otimes y ) \,,
    \]
    with $\tilde g = g_1 + g_2 + n-1$, $\tilde m = m_1 + m_2$, $\tilde n = r$ and partition $(1, \ldots, 1)$, and $\mu^\updownarrows_\star$ the induced map in homology.
\end{defprop}

\subsection{The Operation \texorpdfstring{$\mu^{cs}_\ast$}{muconnected_sum*} [B]}
\label{homology_operations:parallel_patching_slit_pics:construction_of_mu_cs}
\symbolindex[m]{$\mu^{cs}$}{A selected glueing construction}{Subsubsection \ref{homology_operations:parallel_patching_slit_pics:construction_of_mu_cs}}
The third homology operation is induced by a product called $\mu^{cs}$
\label{page:shorthand_par}%
\symbolindex[p]{$\Par$}{A shorthand for $\Par = \coprod_{g,m,(r_1,\ldots, r_n)}\Parr$.}{Page \pageref{page:shorthand_par}}
and is defined for all parallel slit domains in $\Par = \coprod_{g,m,(r_1,\ldots, r_n)}\Parr$.
If we think of parallel slit domains as surfaces with poles, the well-known product $\mu$, which was defined above for $n=r=1$,
is understood as the connected sum at the distinguished dipole.
The product $\mu^{cs}$ (which we are about to define) does the same and its superscript should remind us of the connected sum operation.
For two parallel slit domains $L_1 \in \Par_{g_1,n_1}^{m_1}[(r^{(1)}_1, \ldots, r^{(1)}_{n_2})]$ and $L_2 \in \Par_{g_2,n_2}^{m_2}[(r^{(2)}_1, \ldots, r^{(2)}_{n_2})]$,
we consider the connected sum of the two associated surfaces with respect to the first level of $L_1$ and the last level of $L_2$, see Figure \ref{homology_operations:parallel_patching_slit_pics:mu_cs}.
\begin{figure}[ht]
    \centering
    \def\svgwidth{.3\columnwidth}
    \input{pictures/mu_cs.pdf_tex}
    \caption{\label{homology_operations:parallel_patching_slit_pics:mu_cs}The parallel slit domain $\mu^{cs}(L_1, L_2)$.}
\end{figure}

Let us describe this operation in terms of the dual Ehrenfried complex.
\begin{defi}
    \label{homology_operations:parallel_patching_slit_pics:connected_sum}
    \index{connected sum of parallel slit domains}
    Consider cells $\Sigma = \inhomq[\tau] \in P^\ast(h,m; r_1, \ldots, r_n)$ of bidegree $(p,q)$ and $\Sigma' = \inhomq[\tau'] \in P^\ast(h',m';r'_1, \ldots, r'_n)$ of bidegree $(s, t)$
    and denote the $s\Th$ iterated pseudo degeneracy $S = S_0 \circ \cdots \circ S_0 \colon \SymGr_{p} \to \SymGr_{s+p}$.
    The {\bf connected sum of} $\Sigma$ and $\Sigma'$ is
    \[
        \mu^{cs}(\Sigma, \Sigma') = (-1)^{pq} (S\tau_q \mid \ldots \mid S\tau_1 \mid \tau'_t \mid \ldots \mid \tau'_1)
    \]
    as cell in $P(h+h',m+m'; r_1', \ldots, r_{n'-1}', r_{n'}' + r_1, r_2, \ldots, r_n)$.
\end{defi}

\begin{prop}
    \label{homology_operations:parallel_multiplication:multiplication_is_a_cochainmap}
    The connected sum defines a cochain map
    \begin{multline*}
        \mu^{cs} \colon \E^\ast(h,m;r_1, \ldots, r_n) \otimes \E^\ast(h',m';r_1', \ldots, r_n') \to\\
        \E^\ast(h+h', m+m'; r_1', \ldots, r_{n'-1}', r_{n'}' + r_1, r_2, \ldots, r_n)
    \end{multline*}
    and therefore a homology operation on the associated moduli spaces.
\end{prop}

\begin{notation}
    In order to simplify notation, we write $\Sigma \cdot \Sigma'$ instead of $\mu^{cs}_\ast(\Sigma, \Sigma')$.
    Moreover, we assume $r = [(1)] = r'$, since the general case is proven in the same way.
\end{notation}

\begin{lem}
    \label{homology_operations:parallel_multiplication:multiplication_is_well_def}
    If $\Sigma$ and $\Sigma'$ are non-degenerate cells, then the same holds true for $\Sigma \cdot \Sigma'$.
\end{lem}

\begin{proof}
    Consider non-degenerate cells $x = (x_q\mid \ldots\mid x_1) \in \E^\ast(q,m;1)_p$ and $y = (y_t\mid\ldots\mid y_1) \in \E^\ast(t,m';1)_s$ and denote their product by
    \[
        z = (z_{t+q} \mid \ldots \mid z_{1}) = (Sx_q \mid \ldots \mid Sx_q \mid y_t \mid \ldots \mid y_1) \,.
    \]
    By assumption, there is neither $1 = x_i$ respectively $1 = y_i$ nor a common fixed point of $x_q, \ldots, x_1$ respectively $y_t, \ldots y_1$, so the same holds true for $z$.
    Moreover, $N(z) = N(x) + N(y)$.
    Thus, $z$ is non-degenerate in $\E^\ast(t+q,m+m';1)_{s+p}$ if the number of boundary curves is
    \begin{align}
        \label{homology_operations:parallel_multiplication:multiplication_is_well_def:boundary_curves} n(z) &= n(x) + n(y) - 1
        \intertext{and the number of punctures is}
        \label{homology_operations:parallel_multiplication:multiplication_is_well_def:punctures} m(z) &= m(x) + m(y) \,.
    \end{align}
    We compare the cycles of
    \[
        \alpha = z_{t+q} \cdots z_{1} \cdot(\ul 0\  \ldots\  \ul{s+p} )
    \]
    with the cycles of
    \[
       \sigma = x_q \cdots x_1 \cdot(\ul 0\  \ldots\  \ul p ) \mspc{and}{20} \rho = y_t \cdots y_1 \cdot(\ul 0\  \ldots\  \ul s ) \,.
    \]
    By construction $z_{t+q} \cdots z_{t+1}$ respectively$z_t \cdots z_1$ is an automorphism of the set $\{\ul{s+1}, \ldots, \ul{s+p}\}$ respectively$\{\ul 1, \ldots, \ul s\}$, so
    \[
        \alpha|_{\{\ul 0, \ldots, \ul{s-1}\}} = \rho|_{\{\ul 0, \ldots, \ul{s-1}\}}
    \]
    and
    \[
        \alpha|_{\{\ul s, \ldots, \ul{s+p-1}\}} = S(\sigma)|_{\{\ul s, \ldots, \ul{s+p-1}\}} \,.
    \]
    Therefore, $\ul 0$ and $\ul s$ are in the same orbit of $\alpha$ whereas every other cycle corresponds to exactly one cycle of either $\sigma$ or $\rho$.
    Both \eqref{homology_operations:parallel_multiplication:multiplication_is_well_def:boundary_curves} and \eqref{homology_operations:parallel_multiplication:multiplication_is_well_def:punctures} are immediate consequences.
    
    In order to prove the general case, observe that $\mu^{cs}_\ast(\Sigma, \Sigma')$ is connected.
    Observe that its levels are ordered appropriately.
\end{proof}

\begin{lem}
    \label{homology_operations:parallel_multiplication:leibniz_rule}
    The connected sum is subject to the Leibniz rule
    \[
         (\del_\KK^\ast \pi^\ast x) \cdot y + (-1)^{|x|}x \cdot (\del_\KK^\ast \pi^\ast y)= \del_\KK^\ast \pi^\ast(x \cdot y)
    \]
\end{lem}

\begin{proof}
    The signs are readily checked.
    It suffices to show that the following relation,
    \[
        \cof_{i-p'}(x) \sqcup \cof_i(y) \sim \cof_i(x \cdot y)
    \]
    with
    \[
        \cof_{i-p'}(x) \ni \tilde x \sim \tilde x \cdot y \mspc{and}{20} \cof_i(y) \ni \tilde y \sim x \cdot \tilde y
    \]
    is a bijection.
    By Proposition \ref{cellular_models:dual_ehrenfried:cob_tr_equals_cob} it suffices to compare the corresponding coboundary traces.
    But a coboundary trace of $x \cdot y$ is a sequence in either $\{ \ul 0, \ldots, \ul s\}$ or $\{ \ul{s+1}, \ldots, \ul{s+p}\}$
    and therefore corresponds to a coboundary trace of either $x$ or $y$.
    The converse is true by the same argument.
\end{proof}

\begin{lem}
    \label{homology_operations:parallel_multiplication:kappa_commutes_with_products}
    The map $\kappa^\ast$ commutes with connected sums, i.e.\ 
    \[
        \kappa^\ast(\Sigma \cdot \Sigma') = (\kappa^\ast \Sigma) \cdot (\kappa^\ast \Sigma') \,.
    \]
\end{lem}

\begin{proof}
    This is an immediate consequence of Lemma \ref{cellular_models:dual_ehrenfried:relevant_kappa_dual_sequences_of_blocks}.
\end{proof}

\begin{proof}[Proof of Proposition \ref{homology_operations:parallel_multiplication:multiplication_is_a_cochainmap}]
    The product of two monotoneous cells is clearly monotoneous.
    
    We have 
    \[
        (\del_\KK^\ast \pi^\ast x) \cdot y + (-1)^{|x|}x \cdot (\del_\KK^\ast \pi^\ast y) = \del_\KK^\ast \pi^\ast(x \cdot y)
    \]
    by Lemma \ref{homology_operations:parallel_multiplication:leibniz_rule} and $\kappa^\ast$ commutes with products by Lemma \ref{homology_operations:parallel_multiplication:kappa_commutes_with_products}.
    \[
        (\kappa^\ast\del_\KK^\ast \pi^\ast x) \cdot \kappa^\ast y + (-1)^{|x|}\kappa^\ast x \cdot (\kappa^\ast\del_\KK^\ast \pi^\ast y) = \kappa^\ast\del_\KK^\ast \pi^\ast(x \cdot y)
    \]
    The cells $x$ and $y$ are monotoneous, thus by Lemma \ref{cellular_models:dual_ehrenfried:f_dual_vanishes_at_monotonous_spot}
    \[
        \kappa^\ast(x) = x \mspc{and}{20} \kappa^\ast(y) = y \,.
    \]
\end{proof}

This ends the discussion of the third selected homology operation.
Summing up, we provided
\begin{defprop}
    \label{homology_operations:parallel_patching_slit_pics:operation_mu_cs}
    \symbolindex[m]{$\mu^{cs}_\ast$}{A selected homology operation}{Definition / Proposition \ref{homology_operations:parallel_patching_slit_pics:operation_mu_cs}}
    By Poincaré duality the operation on the dual Ehrenfried complex defines an operation
    \[
        \mu^{cs}_\ast \colon H_{s}( \mathfrak M_{g_1,n_1}^{m_1} ) \otimes H_t ( \mathfrak M_{g_2,n_2}^{m_2} ) \to H_{s+t}(\mathfrak M_{g_1+g_2,n_1+n_2}^{m_1 + m_2}) \,.
    \]
\end{defprop}