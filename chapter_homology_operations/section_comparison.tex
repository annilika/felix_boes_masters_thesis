\section{Correlation of Parallel and Radial Homology Operations [H]}
\label{homology_operations:comparision_of_par_and_rad}

In this section, we will use the radialization and parallelization maps to compare parallel and radial homology operations,
or obtain operations of the space of parallel slit domains on the space of radial slit domains.
Thereby, let always $r = n = 1 + \dotsb + 1$ be the trivial partition, and denote $\mathfrak{Par}_{g, 1}^{m_1} = \mathfrak{Par}_{g, 1}^{m_1}[(1, \dotsc, 1)]$.
Thus, the two multiplications $\mu^{\upuparrows}$ and $\mu^{\updownarrows}$ on parallel slit domains coincide 
and will be denoted by a simple dot during this chapter.
In our case, the parallel multiplication is hence a map
\[
   \_\cdot\_ \colon \mathfrak{Par}_{g_1, n}^{m_1} \times \mathfrak{Par}_{g_2, n}^{m_2} \to \mathfrak{Par}_{\tilde g, n}^{m_1} 
\]
with $\tilde g = g_1 + g_2 + n - 1$.

\subsection{Placing Parallel Slit Pictures into Annuli via Operads [H]}
\label{par_into_annulus}
\symbolindex[r]{$\Radt(n)$}{A shorthand for $\Radt(n) = \coprod_{g, m} \Rad$.}{Page \pageref{par_into_annulus}}

First, we would like to let $\Par_1 = \coprod{g, m} \Par_{g, 1}^m$ act on $\Radt(n) = \coprod_{g, m} \Rad$ via little cubes operads, for fixed $n > 0$.

Therefore, let $\A \subset \C$ be an annulus in the complex plane, 
and let
\label{conf_space_annuli}
\symbolindex[c]{$\tilde C^k(\A)$}{The ordered configuration space of the annulus $\A \subset \C$.}{Page \pageref{conf_space_annuli}}
\[
   \tilde C^k(\A) = \{(z_1, \dotsc, z_n) \in \A \mid z_i \neq z_j \text{ for } i \neq j\}
\]
denote the $k\Th$ ordered configuration space of $\A$. 
We want to use this configuration space to emplace $k$ parallel slit pictures into $\A$,
resulting in a radial slit picture.

Assume we are given a configuration $(z_1, \dotsc, z_k) \in \tilde C^k(\A)$ and $k$ parallel slit pictures $L_1, \dotsc, L_k$,
where slit picture $L_i$ fulfills $m(L_i) = m_i$, $h(L_i) = h_i$ and $n(L_i) = 1$.
Since the points $z_1, \dotsc, z_k$ are pairwise distinct, we can choose pairwise disjoint regions around them.
Here, a region is the intersection of a radial segment and a concentrical stripe of the annulus, see also Figure \ref{radial_threading}.
Into the $i\Th$ such region, we insert the radialization of the parallel slit picture $L_i$.

Similar as in Figures \ref{homology_operations:parallel_patching_slit_pics:action_of_the_little_cube_operad_naive} and \ref{homology_operations:parallel_patching_slit_pics:action_of_the_little_cube_operad},
we thereby have to be careful that we insert the slits in the right order and way:
We have to put the slit pictures $L_i$ into the annulus $\A$ by increasing distance of the point $z_i$ from the center of the annulus. 
Whenever, during the insertion of some slit picture $L_i$, there are already some slits in the region we chose for $L_i$ to reside in,
we have to thread in the new slit picture through the old slits.
For a better understanding of this process, compare Figure \ref{radial_threading} and the exact definition of threading in Section \ref{radial_composition}.

\begin{figure}[ht]
  \centering
  \incgfx{pictures/radial_threading}
  \caption{\label{radial_threading} The operation $\rho$ on slit pictures At first, a blue parallel slit domain is inserted into the inner ring of the annulus via radialization. 
                                    Afterwards, the green radial slit picture is threaded in into the outer ring.}
\end{figure}

Note that, when the first slit picture, say this is $L_1$, is inserted into $\A$, the number of punctures of the resulting radial slit picture is $m_1 + 1$
since this process is simply the radialization map, compare Subsection \ref{cellular_parallelization}.
Each further inserted slit picture $L_i$ causes the number of punctures to increase by $m_i$ since it is threaded in into one of the cycles of the slit picture, which has been built up so far.
Since the number $h$ of slit pairs of the final slit picture is the sum of the slit pairs of $L_1, \dotsc, L_k$, 
we can compute that the genus of the resulting slit picture will be the sum of all the genuses.
Thus, this process results in

\begin{prop}
\label{little_cubes_radial}
   There is a map 
   \[
      \tilde \vartheta \colon \tilde C^k(\A) \times \mathfrak{Par}_{g_1, 1}^{m_1} \times \dotsb \times \mathfrak{Par}_{g_k, 1}^{m_k} \to \Radt_{\tilde g}(\tilde m + 1, 1) 
   \]
   defined by an action of the little cubes operade.
   The map is given by choosing $k$ distinct points on the annulus $\A$ and emplacing each of the $k$ parallel slit pictures into disjoint regions as described above.
   Hereby, we have $\tilde g = \sum_{i = 1}^k g_i$ and $\tilde m = \sum_{i = 1}^k m_i$.
\end{prop}

Note that the map $\tilde \vartheta$ defined here restricts to the map $\tilde \vartheta$ defined in Theorem \ref{homology_operations:thm_action_of_the_little_cube_operad}.
To give the precise statement, let $\tilde C^k(\C) \xhr{\iota} \tilde C^k(\A)$ denote the inclusion, where the complex plane is wrapped around the annulus $\A$.
Now, the definitions of the maps $\tilde \vartheta$ and $\radmap$ immediately yield

\begin{prop}
\label{rad_par_operad}
    The map $\tilde \vartheta$ restricts to the action of the little cubes operad on parallel slit domains (compare Theorem \ref{homology_operations:thm_action_of_the_little_cube_operad}).
    To be precise, the diagram
    \[
    \begin{tikzcd}
       \tilde C^k(\A) \times \mathfrak{Par}_{g_1, 1}^{m_1} \times \dotsb \times \mathfrak{Par}_{g_k, 1}^{m_k} \arrow{r}{\tilde \vartheta} & \Radt_{\tilde g}(\tilde m + 1, 1) \\
       \tilde C^k(\C) \times \mathfrak{Par}_{g_1, 1}^{m_1} \times \dotsb \times \mathfrak{Par}_{g_k, 1}^{m_k} \arrow{r}{\tilde \vartheta} \arrow[hookrightarrow]{u}{\iota \times \text{id}} & \mathfrak{Par}_{\tilde g, 1}^{\tilde m} \arrow[hookrightarrow]{u}{\radmap}
    \end{tikzcd}
   \]
   commutes, where, as above, $\tilde g = \sum_{i = 1}^{k} g_i$ and $\tilde m = \sum_{i = 1}^k m_i$.
\end{prop}
   
With the homology cross product, we obtain

\begin{prop}
   The map $\tilde \vartheta$ yields a family of homology operations
   \[\begin{tikzcd}
      \tilde \vartheta_* \colon H_s(\tilde C^k(\A)) \otimes H_{t_1}(\mathfrak{Par}_{g_1, 1}^{m_1}) \otimes \dotsb \otimes H_{t_k}(\mathfrak{Par}_{g_k, 1}^{m_k}) \to H_{s+\tilde t}(\Radt_{\tilde g}(\tilde m + 1, 1))\,, 
   \end{tikzcd}\]
   where $\tilde g = \sum_{i = 1}^{k} g_i$, $\tilde m = \sum_{i = 1}^k m_i$ and $\tilde t = \sum_{i = 1}^k t_i$.
\end{prop}

For later uses, it is a good idea to imagine what the map $\tilde \vartheta$ looks like on surfaces, compare Figure \ref{rad_par_operad_on_surfaces}.
\begin{figure}[ht]
\centering
\def\svgwidth{.9\columnwidth}
\input{pictures/rad_par_operad_on_surfaces.pdf_tex}
\caption{\label{rad_par_operad_on_surfaces} An excerpt of the map $\tilde \vartheta$ applied to three surfaces.}
\end{figure}
At the bottom of the picture, we see the boundary curves of three surfaces with punctures and each one boundary curve.
Since the map $\tilde \vartheta$ at first applies the radialization map to each of these surfaces, 
there are pairs of pants glued to each boundary curve, with one leg an outgoing and one leg an incoming boundary curve.
Secondly, we need to carry over the meaning of the configuration space for the surfaces.
Recalling the definition of $\tilde \vartheta$ on slit pictures, we see that a configuration in $\tilde C^k(\A)$ determines an order in which to insert the $k$ parallel slit pictures into the annulus.
During the insertion of a new parallel slit picture $L$ , it also determines an already placed slit domain $L'$ and inserts the new picture into one of its outgoing boundary curves.
On the corresponding surfaces $F$ and $F'$, this outgoing boundary curve of $F'$ is hence glued to the incoming boundary curve of $F$. 
So Figure \ref{rad_par_operad_on_surfaces} shows an excerpt of one possibility how $\tilde \vartheta$ acts with the surfaces. 
It is not neccessarily the new outgoing boundary curve arising by radialization that is glued with some incoming boundary curve.

We can easily generalize the map $\tilde \vartheta$ by inserting parallel slit pictures into $n$ annuli instead of one. 

\begin{defi}
\label{pars_into_n_annuli}
\symbolindex[a]{$n\A$}{The disjoint union of $n$ annuli in the complex plane.}{Page \pageref{pars_into_n_annuli}}
Write $n \A = \A_1 \sqcup \dotsc \sqcup \A_n$ for the disjoint union of $n$ annuli in the complex plane.
There is a map
\[
   \tilde \vartheta \colon \tilde C^k(n\A) \times \mathfrak{Par}_{g_1, 1}^{m_1} \times \dotsb \times \mathfrak{Par}_{g_k, 1}^{m_k} \to \Radt_{\tilde g}(\tilde m + n, n) 
\]
given by placing $k$ parallel slit pictures into $n$ annuli using the same method as above.
By this, we have $\tilde g = \sum_{k = 1}^k g_i$ and $\tilde m = \sum_{i = 1}^k m_i$.
\end{defi}

Here, the regions where parallel slit pictures are inserted can lie on different annuli, but each slit picture is placed completely into a single annulus.
Thus, the resulting radial slit picture is not connected for $n > 1$, and it is even possible that one annulus stays empty.
The parameters of the target space of this generalized map $\tilde \vartheta$ are obvious.
In Proposition \ref{cor_all}, we will use this map to describe how parallel slit pictures can also be placed into radial slit pictures and not only into empty annuli.
Thus, we also need to generalize the above proposition to

\begin{prop}
\label{rad_par_operad_gen}
    Let $\A = \A_1 \sqcup \dotsc \sqcup \A_n$ be the disjoint of $n$ complex annuli.
    Then, the generalized map $\tilde \vartheta$ restricts to a similarly generalized action of the little cubes operad on parallel slit domains.
    To be precise, the diagram
    \[
    \begin{tikzcd}
       \tilde C^k(n\A) \times \mathfrak{Par}_{g_1, 1}^{m_1} \times \dotsb \times \mathfrak{Par}_{g_k, 1}^{m_k} \arrow{r}{\tilde \vartheta} & \Radt_{\tilde g}(\tilde m + n, 1) \\
       \tilde C^k(n\C) \times \mathfrak{Par}_{g_1, 1}^{m_1} \times \dotsb \times \mathfrak{Par}_{g_k, 1}^{m_k} \arrow{r}{\tilde \vartheta} \arrow[hookrightarrow]{u}{\iota \times \text{id}} & \mathfrak{Par}_{\tilde g, n}^{\tilde m} \arrow[hookrightarrow]{u}{\radmap}
    \end{tikzcd}
   \]
   commutes, where $\tilde g = \sum_{i = 1}^{k} g_i$ and $\tilde m = \sum_{i = 1}^k m_i$. 
\end{prop}

\subsection{\texorpdfstring{$\Part$}{Par} as a Module over \texorpdfstring{$\Radt$}{Rad} [H]}
\label{par_as_rad_module}

We will now develop an operation that makes the homology of $\Part(n) = \coprod_{g, m} \Par$ a module over the homology of $\Radt(n) = \coprod_{g, m} \Rad$.

Therefore, let $L \in \Part$ and $A \in \Radt$ be two slit pictures with $n(L) = n(A)$.
We want to merge $L$ and $A$ into a radial slit picture on $n$ new annuli $\tilde \A_1, \dotsc, \tilde \A_n$.
For a visualization of the following description, see Figure \ref{par_operates_on_rad}.
Consider a fixed annulus $\tilde \A_k$ and seperate it equally into an inner and an outer ring.
\begin{figure}[ht]
  \centering
  \incgfx{pictures/par_operates_on_rad}
  \caption{\label{par_operates_on_rad} Three regions of an annulus, into which three parallel slit pictures are placed, and how slits have to be threaded in.}
\end{figure}
Put the $k\Th$ level of the parallel slit picture $x$ into the inner ring of $\tilde \A_k$ like via the radialization map, extending the slits to the outer boundary of the annulus. 
We obtain a new distinguished outgoing boundary curve that arises during radialization, marked red in the picture.
Starting at the real horizontal line, we insert the $k\Th$ level of the radial slit picture $y$ into the outer ring of the annulus,
threading in the slits into the distinguished cycle similarly as in Subsection \ref{par_into_annulus}.

\begin{defi}
   The above procedure defines a a map
   \[
      \rho \colon \Radt(n) \times \Part(n) \to \Radt(n)\,, (A, L) \mapsto \rho(A, L) = A.L\,.
   \]
\end{defi}

We also denote the map $\rho$ with a low dot since it will turn out that it is a module operation in the sense of operads.
Before we show that, we take a closer look at the map $\rho$ itself.

\begin{prop}
   For $A \in \Radt_{g_1}(m_1, n)$ and $L \in \Part_{g_2, n}^{m_2}$, we have $A.L \in \Radt_{\tilde g}(\tilde m, n)$ with $\tilde g = g_1 + g_2 + n - 1$ and $\tilde m = m_1 + m_2$.
\begin{proof}
   By construction, the radial slit picture $A.L$ has $n$ incoming boundary curves.
   Since the $n$ incoming boundary curves of $A$ are glued to the $n$ outgoing boundary curves that arise due to radialization, we have 
   \[
      \tilde m = (m_2 + n) - n + m_1 = m_1 + m_2\,.
   \]
   Using the formulas for $h$ for each of the three spaces involved, we obtain
   \[
      \tilde g = g_1 + g_2 + n - 1\,.
   \]
\end{proof}
\end{prop}

In Figure \ref{rad_is_par_module}, one can see that the genus increases by one for each two neighboring incoming boundary curves of $A$.
\begin{figure}[ht]
\centering
\def\svgwidth{\columnwidth}
\input{pictures/rad_is_par_module.pdf_tex}
\caption{\label{rad_is_par_module} The operation $\rho$ on surfaces.}
\end{figure}
Note that here, it is always the new outgoing boundary curves of $L$ arising from radialization that are glued together with the incoming boundary curves of $A$.

\begin{prop}
   Let $n > 0$.
   There is a right module structure 
   \[
      H_*(\Radt(n)) \otimes H_*(\Part(n)) \to H_*(\Radt(n))
   \]
   induced by an action of the litte cubes operade.
\begin{proof}
   We define the right module structure by the composition
   \[
      H_*(\Radt(n)) \otimes H_*(\Part(n)) \xr{\otimes} H_*(\Radt(n) \times \Part(n)) \xr{\rho_*} H_*(\Radt(n))\,.
   \]
   The map $\rho$ is induced by a little cubes operade in the following way.
   We can restrict the map $\tilde \theta$ of Proposition \ref{little_cubes_radial} for $k = 1$ to the point $z = -1$ in the configuration space $\tilde C^1(\A)$ in order to choose
   the position for inserting one level of a parallel slit picture into one annulus $\A$.
   We can apply this to a parallel slit picture on $n$ levels by treating the levels seperately.
   It is also possible to do this if there is already a radial slit picture residing on the annuli since we can thread in the slits of the parallel slit picture.
   
   It remains to verify that, for $L_1, L_2 \in \Part(n)$ and $A \in \Radt(n)$, we have $A.(L_1 \cdot L_2) \simeq (A.L_1).L_2$.
   Using the definition of $\rho$ on the surfaces resulting from glueing the slit pictures, it is not difficult to see that this formula if fulfilled.
   Now we can compose the induced map $\rho_*$ with the homology cross product in order to obtain the desired operation.
\end{proof}
\end{prop}

\subsection{Formulas [H]}

In this subsection, we will see several formulas relating all the maps and operations we have seen so far.

First, we obtain another property of the radialization map from Section \ref{cellular_radialization}.
The radialization map may not be multiplicative with respect to the radial multiplication (see \ref{radialization_not_mult}).
However, it is compatible with the operation $\rho$ defined in Subsection \ref{par_as_rad_module}.

\begin{prop}
\label{prop_rad_compatible}
   The radialization map is compatible with the operation $\rho$ and the parallel multiplication, i.e., for $L_1$, $L_2 \in \Part(n)$, we have
   \[
      \radmap(L_1 \cdot L_2) \simeq \radmap(L_1).L_2\,. 
   \]
\begin{proof}
  See Figure \ref{rad_compatible} for a proof.
  Here, $L_1$ is colored green and $L_2$ blue.
  The first picture shows the radial slit picture $\radmap(L_1).L_2$.
  Note that the slits of $L_1$ are threaded in into a single outgoing boundary curve of $\radmap(L_2)$.
  We can reverse this threading process in order to move all slits of $L_1$ into a connected part of the annulus.
  This happens in the second picture.
  From the radial slit picture shown there, the slits only have to be rotated around the annulus, and their lengths have to be changed,
  and then we arrive at the third radial slit picture, $\radmap(L_1 \cdot L_2)$.  
  \begin{figure}[ht]
  \centering
  \incgfx{pictures/rad_compatible}
  \caption{\label{rad_compatible} Proof of Proposition \ref{prop_rad_compatible}.}
  \end{figure}
   
\end{proof}
\end{prop}

Next, we obtain another little formula that relates radialization and the operation $\rho$ with the composition of radial slit pictures defined in Section \ref{radial_composition}.

\begin{prop}
\label{rad_rho_comp}
   Denote the map swapping the two factors of a product by $t$.
   The diagram
   \[
    \begin{tikzcd}
	\Radt^{\bullet \bullet}_{g_1}(n, n) \times \Part_{g_2, n}^m \arrow{d}{\id \times \radmap} \arrow{r}{\rho} & \Rad^{\bullet \bullet}_{g_1 + g_2 + n - 1}(m + n, n)\\
	\Radt^{\bullet \bullet}_{g_1}(n, n) \times \Radt^{\bullet \bullet}_{g_2}(m+n, n) \arrow{r}{t} & \Radt^{\bullet \bullet}_{g_2}(m+n, n) \times \Radt^{\bullet \bullet}_{g_1}(n, n) \arrow{u}{\odot_{\pi_n}}
    \end{tikzcd}
    \]
    commutes,
    where the partial pairing $\pi_n$ pairs the $k\Th$ incoming boundary curve of the second factor with the outgoing boundary curves of the first factor arising from the $k\Th$ boundary curve by radialization.
    \begin{proof}
       Recall how the operation $\rho$ is defined on surfaces $F \in \mathfrak M^{\bullet \bullet}_{g_1}(n, n)$ and $F' \in \mathfrak M_{g_2, n}^m$, see Figure \ref{rad_is_par_module}.
       The surface $F'$ is at first radialized, i.e. each puncture is transformed into an outgoing boundary, 
       and onto each boundary curve, a pair of pants with one incoming and one outgoing boundary curve is glued.
       Now the $k\Th$ new outgoing boundary curve arising from the $k\Th$ boundary curve of $F'$ is glued together with the $k\Th$ incoming boundary curve of $F$.
       But this is exactly what the map $\odot_{\pi_n} \circ t \circ (\id \times \radmap)$ does.
    \end{proof}
\end{prop}

Note that we need to restrict to these parameters in order to state the preceeding formula.
The number of incoming boundary curves of the radial and the parallel slit picture in this formula have to coincide since this is required by the operation $\rho$.
Furthermore, the number of outgoing boundary curves of the radial slit picture needs to equal the same number due to the composition map.

We can use this proposition to show that, in some special cases, we can relate the operad structure on $\Part$ with the radial composition map by the operation of $\Part$ on $\Radt$.
The next corollary will state that the diagram
\[
 \begin{tikzcd}
    \Radt^{\bullet \bullet} \times \tilde C^k(n\C) \times \Part^k \arrow{r}{\id \times \tilde \vartheta} \arrow{dd}{\id \times \iota \times \id} & \Radt^{\bullet \bullet} \times \Part \arrow{dr}{\rho} \arrow[dashed]{dd}{\id \times \radmap} & \\
                                                                                                                              &                                      & \Radt^{\bullet \bullet}\\  
    \Radt^{\bullet \bullet} \times \tilde C^k(n\A) \times \Part^k \arrow{r}{\id \times \tilde \vartheta}                                        & \Radt^{\bullet \bullet} \times \Radt^{\bullet \bullet} \arrow{ur}{\odot \circ t} & 
 \end{tikzcd}
\]
commutes up to homotopy, whenever it makes sense to write down the compositions of the participating maps.
We will come to state a more formal version of this diagram.
But without all the indices, it is easier to see that the left square consists of the commutative diagram in Proposition \ref{rad_par_operad_gen}
and the triangle of the commutative diagram in Proposition \ref{rad_rho_comp}.

Furthermore, we can already interpret the diagram in this simple version.
There are two canonical ways to emplace parallel slit pictures into a given radial slit picture in order to obtain another radial slit picture via little cubes operads.
Firstly, by composing the parallel map $\tilde \theta$ given by operads on the parallel slit pictures with the operation $\rho$ of $\Part$ on $\Radt$;
secondly, by using the radial map $\tilde \theta$ given by operads followed by the radial composition.
According to the diagram, these processes coincide whenever they are comparable.
We now formalize the diagram in 

\begin{cor}
\label{cor_all}
Let $\pi_k$ and $t$ as in the preceeding proposition.
The diagram 
\[
 \begin{tikzcd}
    \Radt^{\bullet \bullet}_{g}(n, n) \times \tilde C^k(n\C) \times (\Part_{g_1, 1}^{m_1} \times \dotsc \times \Part_{g_k, 1}^{m_k}) \arrow{r}{\id \times \tilde \vartheta} \arrow{dd}{\id \times \iota \times \id} & \Radt^{\bullet \bullet}_g(n, n) \times \Part_{g', n}^{m'} \arrow{d}{\rho} \\
																			                                          & \Radt^{\bullet \bullet}_{g + g' + n - 1}(m' + n, n) \\
    \Radt^{\bullet \bullet}_{g}(n, n) \times \tilde C^k(n\A) \times (\Part_{g_1, 1}^{m_1} \times \dotsc \times \Part_{g_k, 1}^{m_k}) \arrow{r}{\id \times \tilde \vartheta}                                         & \Radt^{\bullet \bullet}_g(n, n) \times \Radt^{\bullet \bullet}_{g'}(m' + n, n) \arrow{u}{\odot_{\pi_k \circ t}} 
 \end{tikzcd}
\]   
commutes up to homotopy.
Here, we have $m' = \sum_{i = 1}^k m_i$ and $g' = \sum_{i = 1}^k g_i$.
\end{cor}

\begin{prop}
   Let $A$ be a parallel slit picture on $n$ levels, $B$ a radial cell with $n(A) = n(B) = n$. 
   Then the parallel slit pictures $\alpha (A \cdot \parmap(B))$ and $\parmap(\radmap(A) \radmult B)$ coincide.
\begin{proof}
   Consider the product $\radmap(A) \radmult B$ and its parallelization $\parmap(\radmap(A) \radmult B)$, see Figure \ref{homology_op:comparison_par_rad_formula_part_1}.
  \begin{figure}[ht]
  \centering
  \incgfx{pictures/comparison_par_rad_formula_part_1}
  \caption{\label{homology_op:comparison_par_rad_formula_part_1} The radial / parallel slit pictures $\radmap(A) \radmult B$ and $\parmap(\radmap(A) \radmult B)$.}
  \end{figure}
  Comparing this with Figure \ref{homology_op:comparison_par_rad_formula_part_2},
  we see that $\parmap(\radmap(A) \radmult B)$ results from $\alpha (A \cdot \parmap(B))$ by moving the lowest green slit upwards a little,
  which means that the two slit pictures agree.
  \begin{figure}[ht]
  \centering
  \incgfx{pictures/comparison_par_rad_formula_part_2}
  \caption{\label{homology_op:comparison_par_rad_formula_part_2} The parallel slit pictures $A \cdot \parmap(B)$ and $\alpha(A \cdot \parmap(B))$.}
  \end{figure}
\end{proof}
\end{prop}

\begin{prop}
   Let $A$ be a parallel slit domain on $n$ levels, $B$ a radial slit domain with $n(A) = n(B) = n$. 
   Then the radial slit domains $\radmap(A) \radmult B$ and $\radmap(A \cdot \parmap(B))$ coincide.
\begin{proof}
   Comparing Figures \ref{homology_op:comparison_par_rad_formula2_part_1} and \ref{homology_op:comparison_par_rad_formula2_part_2},
   we see that we can transform $\radmap(A \cdot \parmap(B))$ into $\radmap(A) \radmult B$ by rotating all slits a little.

  \begin{figure}[ht]
  \centering
  \incgfx{pictures/comparison_par_rad_formula2_part_1}
  \caption{\label{homology_op:comparison_par_rad_formula2_part_1} The radial slit pictures $\radmap(A)$ and $\radmap(A) \radmult B$.}
  \end{figure}
  
  \begin{figure}[ht]
  \centering
  \incgfx{pictures/comparison_par_rad_formula2_part_2}
  \caption{\label{homology_op:comparison_par_rad_formula2_part_2} The slit pictures $A \cdot \parmap B$ and $\radmap(A \cdot \parmap B)$.}
  \end{figure}

\end{proof}

\end{prop}
