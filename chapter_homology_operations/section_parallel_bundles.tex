\section{Operations on \texorpdfstring{$\Par$}{Par} via Bundles}
\label{homology_operations:operations_par_via_bundles}
We assume that the reader is familiar with spectral sequences.
There are several introductions to the theory of spectral sequences and we recommend working through \cite[Chapter 5]{Weibel1995} or \cite[Chapter 9]{Spanier199412}.

In this section, we sketch three homology operations that are induced by certain bundle maps.
For a more detailed discussion we refere the reader to \cite[Chapter 4]{Mehner201112}.
Moreover, we realize the homology operation $T$ via the dual Ehrenfried complex using our description of the coboundary operator.

For trivial bundles $F \to X \to B$ with $F$ a $q$-dimensional, connected, oriented, closed manifold,
taking the cross product with the fundamental class defines a homomorphism in integral homology
\[
    H_p(B) \to H_{p+q}(X) \,.
\]
For oriented bundles a similar construction is possible and there are many ways to state naturality of this construction.
The next proposition is an algebraic solution.
For a given bundle with fibre $F$, denote by $\dim(F)$ the largest homological degree with $H_{\dim(F)}(F) \neq 0$.
\label{page:q_dim_bundle}%
\index{fibre bundle!$q$-dimensional fibre bundle}
In this case, we call $X$ a $\dim(F)$-dimensional bundle over $B$.
\begin{prop}
    \label{homology_operations:parallel_bundles:generalized_cross_product}
    \index{orientation!with respect to a coefficient ring}
    \symbolindex[o]{$\otimes$}{The natural map $\otimes \colon H_p(B;R) \otimes H_q(F;R) \to H_{p+q}(X;R)$ being the homology cross product for trivial bundles.}{Definition \ref{homology_operations:parallel_bundles:generalized_cross_product}}
    Fix a coefficient ring $R$ and a homological degree $q$.
    For the category of $R$-oriented%
    \footnote{
        We assume that the induced action of the fundamental groupoid on $H_\ast(F;R)$ is trivial.
        Geometrically speaking, the tour along a closed path in $X$ is always orientation preserving (with respect to the fibres and the coefficient ring).
    },
    $q$-dimensional bundles $F \to X \to B$ there is a natural map
    \[
        \otimes \colon H_p(B;R) \otimes H_q(F;R) \to H_{p+q}(X;R)
    \]
    which agrees with the homological cross product in case $X$ is a trivial bundle and $H_\ast(F)$ is of finite rank and torsion free.
\end{prop}
\begin{proof}
    Recall the construction of the homological Leray--Serre spectral sequence (\cite[Chapter 9]{Spanier199412}).
    The base $B$ is (up to CW-replacement) a CW-complex and the preimage of the cellular filtration $F_pB$ of $B$ defines a filtration of $X$.
    The associated spectral sequence is the Leray--Serre spectral squence with second page
    \[
        E^2_{p,q} = H_p(B; H_q(F;R)) \Rightarrow H_{p+q}(X;R) \,.
    \]
    The local coefficient system is trivial by assumption.
    In particular, we have natural, exact sequences
    \[
        H_p(B;R) \otimes H_q(F;R) = E^2_{p,q} \xr{\alpha} E^\infty_{p,q} \to 0
    \]
    and
    \[
        0 \to E^\infty_{p,q} \xr{\beta} H_{p+q}(X;R)
    \]
    since $E^\infty_{p,q} = F_pH_{p+q}(X;R) / F_{p-1}H_{p+q}(X;R)$ and $E^\infty_{p-k,q+k} = 0$ for $k > 0$.
    Thus
    \[
        \otimes = \beta \alpha \colon H_p(B;R) \otimes H_q(F;R) \to H_{p+q}(X;R)
    \]
    is a natural homomorphism.
    
    For $X$ the trivial bundle $B \times F \to B$ the filtration of $X$ is just $F_pX = F_pB \times F$.
    By the Eilenberg-Zilber theorem, we have a natural chain homotopy equivalence
    \[
        C(X) \simeq C(B) \otimes C(F)\,,
    \]
    so the induced filtration on the right hand side is $(F_pC(B)) \otimes C(F)$.
    But this filtration induces the Tor spectral sequence which collapses at the second page (as $H_\ast(F)$ is torsion free and of finite rank).
    In particular, the Leray--Serre spectral squence of $B \times F \to F$ is natural isomorphic to this Tor spectral sequence, 
    so the map $\otimes$ agrees (up to natural isomorphism) with the homology cross product.
\end{proof}

\label{page:transfer_map}%
\index{transfer}
\symbolindex[t]{$tr$}{The transfer map.}{Page \pageref{page:transfer_map}}
Consider the space of parallel slit domains with a distinguished punctures $\Par_{g,n}^{m-1,1}[(r_1, \ldots, r_n)]$.
This is a non-trivial $m$-fold covering $\Par_{g,n}^{m-1,1} \xr{\pi} \Par_{g,n}^m[(r_1, \ldots, r_n)]$ and
we have the transfer map%
\footnote{%
    The transfer map is already defined on the singular complexes by summing over the $m$ choices of lifting singular chains.
    Note that $\pi_\ast \circ tr$ is just the multiplication by $m$.
}%
\[
    tr \colon H_\ast(\Par_{g,n}^m[(r_1, \ldots, r_n)]) \to H_\ast(\Par_{g,n}^{m-1,1}[(r_1, \ldots, r_n)])\,. 
\]
For varying slit domains, we continuously insert a small circle around the distinguished puncture.
This defines a non-trivial orientable circle bundle $\mathbb S^1 \to \mathfrak I_T \to \Par_{g,n}^{m-1,1}[(r_1, \ldots, r_n)]$.
Adding a pair of slits, one slit ending in a given point $z$ of the small circle and its partner above all other slits (take a glance at Figure \ref{homology_operations:parallel_bundles:operation_t})
defines a continuous map
\label{page:operation_map_t}%
\symbolindex[t]{$\tilde\vartheta_T$}{A map inserting a pair of slits.}{Page \pageref{page:operation_map_t}}
\[
    \tilde\vartheta_T \colon \mathfrak I_T \to \Par_{g+1,n}^{m-1}[(r_1, \ldots, r_n)] \,.
\]
\begin{figure}[ht]
    \centering
    \incgfx{pictures/operation_t.pdf}
    \caption{\label{homology_operations:parallel_bundles:operation_t}%
        The operation $(\tilde\vartheta_T)_\ast$ inserts a new pair of slits while rotating the slit sitting in the distinguished puncture.}
\end{figure}
Topologically, we forget the distinguished puncture, remove small discs around the two points and glue in a handle, increasing the genus of the surface by one.
\begin{defi}
    \label{homology_operations:parallel_bundles:def_operation_t}
    \symbolindex[t]{$(\tilde\vartheta_T)_\ast$}{The homology operation induced by $\tilde\vartheta_T$.}{Definition \ref{homology_operations:parallel_bundles:def_operation_t}}
    The map $\tilde\vartheta_T$ induces the homology operation
    \[
        T \colon H_s(\Par_{g,n}^m[(r_1, \ldots, r_n)]) \to H_{s+1}(\Par_{g+1,n}^{m-1}[(r_1, \ldots, r_n)])
    \]
    by
    \[
        x \mapsto (\tilde\vartheta_T)_\star( tr(x) \otimes [\mathbb S^1]) \,,
    \]
    where $[\mathbb S^1]$ is the fundamental class of the circle and $\otimes$ as in Proposition \ref{homology_operations:parallel_bundles:generalized_cross_product}
    and $(\tilde\vartheta_T)_\star$ the induced homomorphism in homology.
    We imagine $(\tilde\vartheta_T)_\ast$ as seen in Figure \ref{homology_operations:parallel_bundles:operation_t}.
\end{defi}

Similar to the construction above, we continuously embed a circle $C$ near a distinguished puncture and consider two distinct points on $C$
This is, up to homotopy, a non-trivial orientable circle bundle $\mathbb S^1 \to \mathfrak I_F \to \Par_{g,n}^{m-1,1}[(r_1, \ldots, r_n)]$.
Inserting a pair of slits which end in the two marked points on the circle (take a glance at Figure \ref{homology_operations:parallel_bundles:operation_f}) defines a continuous map
\label{page:operation_map_f}%
\symbolindex[t]{$\tilde\vartheta_F$}{A map inserting a pair of slits.}{Page \pageref{page:operation_map_f}}
\[
    \tilde\vartheta_F \colon \mathfrak I_F \to \Par_{g,n}^{m+1}[(r_1, \ldots, r_n)] \,.
\]
\begin{figure}[ht]
    \centering
    \incgfx{pictures/operation_f.pdf}
    \caption{\label{homology_operations:parallel_bundles:operation_f}%
        The operation $(\tilde\vartheta_F)_\ast$ inserts a new pair of rotating slits sitting in the distinguished puncture.}
\end{figure}
Topologically, we introduce a new puncture near the distinguished one.
\begin{defi}
    \label{homology_operations:parallel_bundles:def_operation_f}
    \symbolindex[t]{$(\tilde\vartheta_F)_\ast$}{The homology operation induced by $\tilde\vartheta_F$.}{Definition \ref{homology_operations:parallel_bundles:def_operation_f}}
    For $m \ge 1$, the map $\tilde\vartheta_F$ induces the homology operation
    \[
        F \colon H_s(\Par_{g,n}^m[(r_1, \ldots, r_n)]) \to H_{s+1}(\Par_{g,n}^{m+1}[(r_1, \ldots, r_n)])
    \]
    by
    \[
        x \mapsto (\tilde\vartheta_F)_\star( tr(x) \otimes [\mathbb S^1])
    \]
    where $[\mathbb S^1]$ is the fundamental class of the circle and $\otimes$ as in Proposition \ref{homology_operations:parallel_bundles:generalized_cross_product}
    and $(\tilde\vartheta_F)_\star$ the induced homomorphism in homology, see Figure \ref{homology_operations:parallel_bundles:operation_f}.
\end{defi}

Similarly, for the space of parallel slit domains with two distinguished, enumerated punctures $\Par_{g,n}^{m-2,1,1}[(r_1, \ldots, r_n)]$,
there is a non-trivial orientable torus bundle $\tilde{\mathfrak I}_E \to \Par_{g,n}^{m-2,1,1}[(r_1, \ldots, r_n)]$ by considering two enumerated circles on the surface, each near one of the two enumerated punctures.
Dividing out the obvious action of $\Symgrp_2^\times = Aut(\{1,2\})$ (defined by interchanging the coordinates in the torus respectively the enumerated punctures)
defines a $\mathbb F_2$-orientable torus bundle
\[
    \mathbb S^1 \times \mathbb S^1 \to \mathfrak I_E \to \Par_{g,n}^{m-2,2}[(r_1, \ldots, r_n)]
\]
with $\Par_{g,n}^{m-2,2}[(r_1, \ldots, r_n)]$ the space of parallel slit domains with two distinguished, unordered punctures.
Inserting a pair of slits which end in the two marked points on the two unordered circles (take a glance at Figure \ref{homology_operations:parallel_bundles:operation_e})
defines a continuous map
\label{page:operation_map_e}%
\symbolindex[t]{$\tilde\vartheta_T$}{A map inserting a pair of slits.}{Page \pageref{page:operation_map_e}}
\[
    \tilde\vartheta_E \colon \mathfrak I_E \to \Par_{g+1,n}^{m-2}[(r_1, \ldots, r_n)] \,.
\]
Topologically, we forget the two distinguished punctures, remove small discs around the two points and glue in a handle, increasing the genus of the surface by one.
In order to define the homology operation, observe that the forgetful map
\[
    \Par_{g,n}^{m-2,2}[(r_1, \ldots, r_n)] \to \Par_{g,n}^m[(r_1, \ldots, r_n)]
\]
is a $\binom{m}{2}$-fold non-trivial covering, so we have again a transfer map $tr$.
\begin{defi}
    \label{homology_operations:parallel_bundles:def_operation_e}
    \symbolindex[t]{$(\tilde\vartheta_E)_\ast$}{The homology operation induced by $\tilde\vartheta_E$.}{Definition \ref{homology_operations:parallel_bundles:def_operation_e}}
    For $m \ge 2$, the map $\tilde\vartheta_E$ induces the homology operation
    \[
        E \colon H_s(\Par_{g,n}^m[(r_1, \ldots, r_n)]) \to H_{s+2}(\Par_{g+1,n}^{m-2}[(r_1, \ldots, r_n)])
    \]
    by
    \[
        x \mapsto (\tilde\vartheta_E)_\star( tr (x) \otimes [\mathbb S^1 \times \mathbb S^1])
    \]
    where $[\mathbb S^1 \times \mathbb S^1]$ is the fundamental class of the torus and $\otimes$ as in Proposition \ref{homology_operations:parallel_bundles:generalized_cross_product}
    and $(\tilde\vartheta_E)_\star$ the induced homomorphism in homology.
    We sketch our geometric interpretation of $(\tilde\vartheta_E)_\ast$ in Figure \ref{homology_operations:parallel_bundles:operation_e}.
\end{defi}
\begin{figure}[ht]
    \centering
    \incgfx{pictures/operation_e.pdf}
    \caption{\label{homology_operations:parallel_bundles:operation_e}%
        The operation $(\tilde\vartheta_E)_\ast$ inserts a new pair of independently rotating slits sitting in the distinguished punctures.}
\end{figure}

\subsection{The Operation \texorpdfstring{$T$}{T} via the Dual Ehrenfried Complex [B]}
In this subsection, we construct the operation $T$ in terms of the dual Ehrenfried complex.
Geometrically speaking, we start with a combinatorial cell $\Sigma$ and have to introduce two new slits,
one rotating in a puncture and the other on top of all other slits.
In the algebraic model, there is no notion of rotating slits but it is easy to come up with the right definition.
In order to reduce the cohomological degree of the resulting cell by one, we append every cell by a transposition $(\ul{p_r+1}_r\ c)$ with $c$ a symbol in a puncture.
Using the geometric intuition of jumping slits and relevant $\kappa^\ast$-sequences, it is easy to see that we defined a coboundary map.

\begin{defi}
    \label{homology_operations:parallel_T:symbols_of_a_puncture}
    \index{symbols of a puncture}
    \symbolindex[p]{$\punc(\Sigma)$}{The symbols of the $m$ cycles of $\Sigma$ corresponding to the punctures.}{Definition \ref{homology_operations:parallel_T:symbols_of_a_puncture}}
    Let $\Sigma = \inhom = \homog$ be a top dimensional, non-degenerate cell in $P(h,m; r_1, \ldots, r_n)$.
    Denote the cycles of $\sigma_h$ that correspond to the $m$ punctures by $\alpha_1, \ldots, \alpha_m$.
    The {\bf symbols corresponding to the punctures of $\Sigma$} are
    \[
        \punc(\Sigma) = \supp(\alpha_1, \ldots, \alpha_m) \,.
    \]
\end{defi}
\begin{notation}
    Since $T$ adds a new slit above all other slits, we obtain the ordered partition $p+1 = p_1 + \ldots + p_{r-1} + (p_r+1)$.
    In particular, the largest symbol is $\ul{p_r+1}_r = p_r+1$.
\end{notation}

\begin{defi}
    \label{homology_operations:parallel_T:defn_on_cells}
    \symbolindex[t]{$T(\Sigma)$}{The homology operation $T$ of $\Sigma$ in terms of the dual Ehrenfried complex.}{Definition \ref{homology_operations:parallel_T:defn_on_cells}}
    Then the operation $T$ is defined on generators $\Sigma \in \E^\ast$ by
    \[
        T(\Sigma) = \sum_{c \in \punc(\Sigma)} \Sigma_c 
    \]
    where
    \[
        \Sigma_c = \big(\ (\ul{p_r+1}_r\ c) \mid \tau_q \mid \ldots \mid \tau_1 \big) \,.
    \]
\end{defi}

\begin{prop}
    \label{homology_operations:parallel_T:T_is_a_cochain_map}
    The operation $T$ defines a cochain map
    \begin{multline*}
        T \colon \E^{\ast+1}(h,m; r_1, \ldots, r_n) = \E^\ast(h,m; r_1, \ldots, r_n) \otimes \Z[1] \to\\
        \E^\ast(h+1,m-1;r_1, \ldots, r_n) \,.
    \end{multline*}
\end{prop}

\begin{lem}
    \label{homology_operations:parallel_T:T_is_well_defined}
    If $\Sigma$ is top dimensional, non-degenerate cell of bidegree $(p,h)$ in $P_{g,n}^m[(r_1, \ldots, r_n)]$, then
    every term $\Sigma_c$ of $T(\Sigma)$ is a top dimensional, non-degenerate cell of bidegree $(p+1,h+1)$ in $P(h+1,m-1; r_1, \ldots, r_n)$.
\end{lem}

\begin{proof}
    Consider $\Sigma = \inhom = \homog$ and let $\Sigma_c = (x_{h+1} \mid \ldots \mid x_1) = ( \sigma_c : \sigma_h : \ldots : \sigma_0)$ be a term of $T(\Sigma)$, i.e.\ 
    $\Sigma_c = \big( (\ul{p_r+1}_r\ c)\mid \tau_q \mid \ldots \mid \tau_1 \big)$ and $\sigma_c = (\ul{p_r+1}_r\ c)\sigma_h$, with $c \in \punc(\Sigma)$.
    
    The following is evident:
    $N(\Sigma_c) = N(\Sigma) + 1 = h+1$, $\Sigma_c$ is connected, the levels are ordered and there is neither $1 = x_i$ nor a common fixed point of $x_{q+1}, \ldots, x_1$.
    From $\sigma_c = (\ul{p_r+1}_r\ c) \sigma_h$ and $\ul{p_r+1}_r \not\in \punc(\Sigma) \ni c$ we deduce
    \[
        n(\Sigma_c) = n(\Sigma) \mspc{and}{20} m(\Sigma_c) = m(\Sigma) - 1\,.
    \]
\end{proof}

\begin{lem}
    \label{homology_operations:parallel_T:T_commutes_with_del_pi}
    We have
    \[
        T\del_\KK^\ast\pi^\ast = \del_\KK^\ast\pi^\ast T \,.
    \]
\end{lem}

In order to prove the lemma, we use Proposition \ref{cellular_models:dual_ehrenfried:cob_tr_equals_cob} to show that every term in $T\del_\KK^\ast\pi^\ast(\Sigma)$ occures in $\del_\KK^\ast\pi^\ast T(\Sigma)$.
Then, using Proposition \ref{cellular_models:dual_ehrenfried:cob_tr_equals_cob} again, the difference of both sums is zero.

\begin{proof}
    By Proposition \ref{cellular_models:dual_ehrenfried:cob_tr_equals_cob}, the terms of $\del_\KK^\ast \pi^\ast(\Sigma)$ correspond bijectively to all coboundary traces of $\Sigma$.
    Applying $T$, every term $x$ of $T\del_\KK^\ast \pi^\ast(\Sigma)$ is identified with an $i\Th$ coboundary trace $a = a(x)$ and a symbol $c = c(x)$ corresponding to one of the punctures of $a.\Sigma$.
    
    If $c \neq i$, we identify $x$ with the term $\tilde a .\Sigma_{\tilde c}$, where $\tilde c = d_i^\Delta(c)$ and $\tilde a$ is the $i\Th$ coboundary trace of $\Sigma_{\tilde c}$ with
    \begin{align}
        \label{homology_operations:parallel_T:T_commutes_with_del_pi:c_neq_i}
        \tilde a_j = \begin{cases} a_j & j \le h \\ (\ul{p_r+1}_r\ c)(a_h) & j = h+1 \end{cases}
    \end{align}
    as both $\tilde a \in T_i(\Sigma_{\tilde c})$ and $x = \tilde a.\Sigma_{\tilde c}$ are readily verified.
    
    Otherwise, i.e\ if $c = i$, we identify $x$ with the term $a'.\Sigma_{c'}$ where $c' = d_i^\Delta(a_h)$ and  $a'$ is the $i\Th$ coboundary trace of $\Sigma_{c'}$ with
    \begin{align}
        \label{homology_operations:parallel_T:T_commutes_with_del_pi:c_eq_i}
        a_j' = \begin{cases} a_j & j \le h \\ a_h & j = h+1 \end{cases}
    \end{align}
    as both $a' \in T_i(\Sigma_{c'})$ and $x = a'.\Sigma_{c'}$ are again readily verified.
    
    Observe that in case \eqref{homology_operations:parallel_T:T_commutes_with_del_pi:c_neq_i} we have
    \[
        \tilde a_{h+1} = S_i(\ul{p_r+1}_r\ \tilde c)(a_j)
    \]
    and in case \eqref{homology_operations:parallel_T:T_commutes_with_del_pi:c_eq_i} we have
    \[
        \tilde a_{h+1} \neq S_i(\ul{p_r+1}_r\ c')(a_j) \,.
    \]
    We identify the terms of $T\del_\KK^\ast\pi^\ast(\Sigma)$ with all coboundary traces $a = (a_{h+1} : \ldots : a_0)$ of all terms of $T(\Sigma)$ that satisfy both
    \[
        a_j \neq a_{j-1} \mspc{for some}{10} j \le h \mspc{and}{30} a_j \neq (S_i\tau_j)(a_{j-1}) \mspc{for some}{10} j \le h \,.
    \]
    The remaining terms of $\del_\KK^\ast\pi^\ast T(\Sigma) - T\del_\KK^\ast\pi^\ast(\Sigma)$
    are identified with the coboundary traces of all terms $\Sigma_c = (x_{h+1} \mid \ldots \mid x_1)$ of $T(\Sigma)$ that satisfy
    \begin{align}
        \label{homology_operations:parallel_T:T_commutes_with_del_pi:a_j_equal_a_j_1}
        a_j = a_{j-1} = i+1 \mspc{for all}{10} j \le h \mspc{and}{30} a_{h+1} \neq a_h
        \intertext{or}
        \label{homology_operations:parallel_T:T_commutes_with_del_pi:a_j_equal_S_i_x_j}
        a_j = (S_i x_j)(a_{j-1})  \mspc{for all}{10} j \le h \mspc{and}{30} a_{h+1} \neq (S_i x_{h+1})(a_h) \,.
    \end{align}
    Let us reformulate \eqref{homology_operations:parallel_T:T_commutes_with_del_pi:a_j_equal_a_j_1} and \eqref{homology_operations:parallel_T:T_commutes_with_del_pi:a_j_equal_S_i_x_j}.
    Clearly
    \begin{align}
        a \in T_i(\Sigma_c) \text{ satisfies \eqref{homology_operations:parallel_T:T_commutes_with_del_pi:a_j_equal_a_j_1}}
            &\iff S_i( \ul{p_r+1}_r\ c )(i+1) \neq i+1 \\
            &\iff c = i \\
            &\iff a = (\ul{p_r+1}_r: i+1 : \ldots : i+1) \mspc{and}{10} i \in \punc(\Sigma) \\
            \label{homology_operations:parallel_T:T_commutes_with_del_pi:a_j_equals_a_j_1_equiv}
            &\iff a = (\ul{p_r+1}_r: i+1 : \ldots : i+1) \mspc{and}{10} \sigma_h(i) \in \punc(\Sigma)
    \end{align}
    and
    \begin{align}
        a \in T_{i+1}(\Sigma_c) \text{ satisfies \eqref{homology_operations:parallel_T:T_commutes_with_del_pi:a_j_equal_S_i_x_j}}
            &\iff S_{i+1}( \ul{p_r+1}_r\ c )(a_h) \neq a_h
            \intertext{using $a_j = S_{i+1}(\tau_j\cdots\tau_1)(i+2) = s_{i+1}^\Delta(\sigma_j(i))$ yields}
            &\iff c = a_h = \sigma_h(i) \\
            &\iff a = (s_{i+1}^\Delta(\sigma_h(i)) : s_{i+1}^\Delta(\sigma_h(i)) : \ldots : s_{i+1}^\Delta(\sigma_0(i))) \notag \\
            \label{homology_operations:parallel_T:T_commutes_with_del_pi:a_j_equal_S_i_x_j_equiv}
            &\phantom{{}\iff} \mspc{and}{10} \sigma_h(i) \in \punc(\Sigma) \,.
    \end{align}
    By \eqref{homology_operations:parallel_T:T_commutes_with_del_pi:a_j_equals_a_j_1_equiv} and \eqref{homology_operations:parallel_T:T_commutes_with_del_pi:a_j_equal_S_i_x_j_equiv}
    the remaining terms of $\del_\KK^\ast\pi^\ast T(\Sigma) - T\del_\KK^\ast\pi^\ast(\Sigma)$ come in pairs, where
    \[
        a \in T_i(\Sigma_i) \mspc{is paired with}{30} a' \in T_{i+1}(\Sigma_{\sigma_h(i)})
    \]
    and a direct computation shows
    \[
        a.\Sigma_i = \big(\ (\ul{p_r+1}_r\ i+1) \mid S_{i+1}\tau_q \mid \ldots \mid S_{i+1}\tau_1 \big) = a'.\Sigma_{\sigma_h(i)}
    \]
    which finishes the proof:
    \[
        \del_\KK^\ast\pi^\ast T(\Sigma) - T\del_\KK^\ast\pi^\ast(\Sigma)
            = \sum_{a \in T_i(\Sigma_i)} (-1)^i a.\Sigma_i + (-1)^{i+1} a'.\Sigma_{\sigma_h(i)} = 0 \,.
    \]
\end{proof}

\begin{proof}[Proof of Proposition \ref{homology_operations:parallel_T:T_is_a_cochain_map}]
    The map $T$ is well defined by Lemma \ref{homology_operations:parallel_T:T_is_well_defined},
    it commutes up to $\kappa^\ast$ with $\del_\E^\ast = \kappa^\ast \del_\KK^\ast \pi^\ast$ by Lemma \ref{homology_operations:parallel_T:T_commutes_with_del_pi},
    so the comparision of relevant $\kappa^\ast$-sequences using Lemma \ref{cellular_models:dual_ehrenfried:f_dual_vanishes_at_monotonous_spot} ends the proof.
\end{proof}
